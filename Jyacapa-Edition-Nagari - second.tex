\documentclass[12pt]{article}

%%%Font related packages
\usepackage[no-math]{fontspec}
\defaultfontfeatures{Ligatures=TeX} 
%\setmainfont{Linux Libertine O}
\newfontfamily\s[Scale=0.92, Script=Devanagari]{Shobhika-Regular}
\usepackage{color, dtk-logos}
 
 
 %%%Formatting related packages
\usepackage{graphicx}
\usepackage[hang,flushmargin]{footmisc} % For no indentation of footnotes.
\usepackage{array}
\usepackage{tabularx, multicol, vwcol}
\usepackage{fullpage}
\usepackage{marginnote}


%Math related packages
\usepackage{amsmath} % āmerican Mathematical Society packages for metamathematical symbols.
%āpril 2017 āditya says install libertine math font package specially from website.

%Page formatting commands
%\newdimen\stdbaseline
%\stdbaseline = 13pt 
%\baselineskip = 2\baselineskip
\parskip = \baselineskip
\parindent = 0pt


%%%Critical editing packages
\usepackage[series={A,B,C,D},noend,noeledsec,nofamiliar,noledgroup]{reledmac}
\Xarrangement[B]{twocol}
\Xarrangement[C]{threecol}
\Xarrangement[D]{paragraph}

%=============================================================%


%% MāCROS for Diacriticals, symbols, math and text features
%%
\let\*=\d
\def\elp{$\ldots\,$}
\def\degrees{$^\circ$}
\def\degree{$^\circ$}
\def\signs{$^s$}
\def\Sin{\mathop{\rm Sin}\nolimits}
\def\Cos{\mathop{\rm Cos}\nolimits}
\def\Versin{\mathop{\rm Vers}\nolimits}
\def\Coversin{\mathop{\rm Coversin}\nolimits}
\def\Crd{\mathop{\rm Crd}\nolimits}
\def\crd{\rm{crd}}

%%%%%%%%%%%%%%%%%%%%%%%%


\begin{document}

{\s  
श्रीगणेशाय नमः || \marginnote{f. 1r $J_2$}
अथैकांशजीवाविषये
उलुग्वेगीजीकस्य
शरहविर्जन्दीस्य\footnote{{\s @विर्जंदीस्थ }${\rm J_1}$}
व्याख्या लिख्यते |
तत्रैकांशजी\-वान\-यने नव्यतरं प्रकारद्वयमस्ति |
एकं यमशैदकाशीसंज्ञेन कृतम् |
द्वितीयं मिर्योलुग्वेगेन कृतम् |
परंचोलुग्वेगस्य यमसैदकाशीवेधप्रक्रियायां
साहाय्यकार्यस्ति \footnote{{\s साहाय्यकार्यस्ति}${\rm SB_J}$}| 
तत्र प्रथमप्रकारः कथ्यते| सा\footnote{{\s सं}${\rm J_1}$} यथा |
अबजदं षडंश ६ चापं कल्पितम् |
पुनरस्य समानं भागत्रयं बचिह्नजचिह्नयोः कृतं । 
तत्र अबमजमहदं बजं बदं जदं चैताः पूर्णज्यारेखाः
योज्याः |
अथात्र पूर्वोक्तप्रकारेणांशत्रयस्य\footnote{{\s अथात्रपूर्वाक्त@}${\rm J_1}$}
जीवा ज्ञातास्ति । सा चेयं ३| ८| २४| ३३| ५९| ३४| २८| ५४| ५० ।
इयं द्विगुणा ६| १६| ४९| ७| ५९| ८| ५६| २९| ४० जाता अदपूर्णज्या \footnote{{\s @पर्णज्या}${\rm SB_J}$, {\s मदपूर्णज्या}${\rm J_1}$}||

अत्रांशद्वयस्याबपूर्णज्या ज्ञानमिष्टमस्ति |
तत्र मिजिस्तीग्रन्थस्य
प्रथमाध्यायस्य द्वितीयक्षेत्रे\footnote{{\s @क्षेत्र }${\rm J_1}$}
इदमुपपन्नम् |\footnote{{\s @द्वितीयक्षेत्रयिदमुप@ }${\rm J_1}$}
यच्चतुर्भुजवृत्तान्तः पतति
तत्सन्मुखस्थभुजानां\footnote{{\s तत्संन्मुख@ } ${\rm J_1}$}
घातयोगस्तच्चतुर्भुजान्तः पतितकर्णद्वयघातसमो भवति |
पुनस्तत्रैव प्रथमाध्यायस्य
चतुर्थक्षेत्रे\footnote{{\s @
षेत्र }${\rm J_1}$} इदमुपपन्नम् |\footnote{{\s चतुर्थक्षेत्रे इदमुप@ }${\rm J_1}$}
इष्टचापार्धपूर्णज्यावर्गस्तेनाङ्केनसमो \footnote{{\s @बर्ग@}${\rm SB_J}$} भवति |
योऽङ्कः\footnote{{\s यो कः }${\rm J_1}$} %\\(|\\)
 तच्चापोनभार्धांशानां \footnote{{\s @नं }${\rm J_1}$} या पूर्णज्या भवत्यस्या \footnote{{\s भवति । अस्याः}${\rm SB_J}$}
व्यासेन यदन्तरं तेन गुणित यो
व्यासार्धस्तदङ्कतुल्यो\footnote{{\s व्यासार्धः स्तदंक@ }${\rm J_1}$}
भवति |

एवमत्र अबचापजदचापे मिथः समाने स्तः\footnote{{\s स्थः }${\rm J_1}$} |
एवं अजबदचापे च समाने स्तः |
तस्मात् अबजदघातो \footnote{{\s अवजदघातो@}${\rm SB_J}$} वर्गो भविष्यति |
अत्र बजमज्ञातं तद्यावत्तावन्मितं कल्पितमिदमदेनगुणितं
जातं या \footnote{{\s वा }${\rm J_1}$} ६| १६|४९| ७| ५९| ८| ५६| २९| ४०|
एवं अबजदघातरूपवर्गराशेर्यावत्तावतश्च \footnote{{\s अवजद@}${\rm SB_J}$} योगः
अजवर्गतुल्य अनबदथा \footnote{{\s अजबदघा@}${\rm SB_J}$} तेन समानो ष्ति \footnote{{\s @स्ति}${\rm SB_J}$}|

अतो .अयं \footnote{{\s अतोयं}${\rm SB_J}$} अजवर्गतुल्यो जातः |
तस्मात्
अजवर्गराशिरेकोवर्गराशिरेभिर्यावत्तावद्भिरधिको
\footnote{{\s अजवर्गराशी एको@ }${\rm J_1}$}
\footnote{{\s अजवर्गराशौ एको@}${\rm SB_J}$}
जातः|
याव १ या ६| १६| ४९| ७| ५९| ८| ५६| २९| ४०

अथ
प्रकारान्तरेण अजवर्गः\footnote{{\s प्रकारांतरेणाजवर्गः }${\rm J_1}$}
साध्यते |
तत्र अजचापं भार्धांशेभ्यः शोध्यं शेषस्य
पूर्णज्याया \footnote{{\s पूर्णज्यायाः}${\rm SB_J}$} व्यासस्य चान्तरं कार्यं तेन घ्न षष्टितुल्यव्यासार्धं
अववर्गतुल्यं भविष्यति |
यतः अजचापं अवचापाद्द्विगुणमस्ति |
तस्माद्यदि अववर्गो\footnote{{\s अवबर्गः} ${\rm J_1}$}
याव १ षष्टि ६० तुल्यव्यासार्धेन भाज्यते तदा यावद्वर्गस्य
षष्ट्यंशो लभ्यते तच्च अजचापानभार्धानां \footnote{{\s अजचोपोन@}${\rm SB_J}$} या पूर्णज्या
तदूनो यो व्यासस्तस्य प्रमाणमस्ति |
पुनरिदमन्तरं संपूर्णव्यासे १२० शोधितं
याव \upbefore{०}१ \downafter{ ६०} रू १२०
इदं अजचापोनभार्धांशानां पूर्णज्या जाता |
अस्या वर्गः %\\(|\\)
यावव \upbefore{१}३६०० याव \upbefore{०}२४०\downafter{६०}
रू १४४००
अथ च यदीष्ष्तचापपूर्णज्यावर्गो \footnote{{\s यदीष्टचापपूर्णज्यावर्गः}${\rm SB_J}$} व्यासवर्गाच्छोद्यते
तत्र शेषं
$|$\marginnote{f. 1v $J_2$}
तच्चापोनभार्धांशपूर्णज्यावर्गो
भवति ||\footnote{{\s भावति }${\rm J_1}$}
यतो व्यासेनेष्टचापे \footnote{{\s व्यासेन इ्ष्टचापे}${\rm SB_J}$} पूर्णज्यया इष्टचापोनभार्धांशपूर्णज्यया
चैकसमकोणत्रि\-भुजं भवति यतो वृत्तार्धे व्यासप्रान्तादुत्पन्नत्रिभुजस्य
पालिकोणः समकोणो
भवतीत्युक्लीदसस्य \footnote{OH MY GOD IT'S EUCLID}
तृतीयाध्याय\footnote{{\s तृतीयाध्याये उप@ }${\rm J_1}$}
उपपन्न\-मस्ति |
तत्त्रिभुजे
समकोणसन्मुखभुजो व्यासो .अस्ति \footnote{{\s व्यासोस्ति}${\rm SB_J}$}|
स च कर्णरूपः |
पुनः कर्णवर्गो \footnote{{\s कर्णवर्गः}${\rm SB_J}$} भुजकोट्योर्वर्गयोगेन समो
भवति|\footnote{{\s भवती| }${\rm J_1}$}\footnote{{there is no second भवती}${\rm SB_J}$}
भवतीत्युक्लीदसस्य
प्रथमाध्याय\footnote{{\s प्रथमध्याये उप@ }${\rm J_1}$}
उपपन्नम् |

अतो .अत्र \footnote{{\s अतोत्र}${\rm SB_J}$} व्यासवर्गः १४४०० कर्णवर्गरूपः |
अस्मिनिष्टचापोनभार्धांशानां पूर्णज्यावर्गः
यावव \upbefore{१}३६०० याव \upbefore{०}२४०\downafter{६०}
रू १४४००
शोधितः शेषं अजपूर्णज्यावर्गो .अवशिष्टः \footnote{{\s वर्गोवशिष्टः}${\rm SB_J}$} |
यावव %\\wहितेस्पचे\\wहितेस्पचे\\wहितेस्पचे
\upbefore{०}१ \downafter{ ३६००} या \upbefore{२४०} \downafter{६०}
अयं
द्वितीयप्रकारेण अजवर्गः\footnote{{\s @प्रकारेणा .अजवर्गः }${\rm J_1}$}
सिद्धः |
अस्यप्रथमप्रकारागत अजवर्गः याव १ या ६| १६| ४९| ७| ५९| ८| ५६| २९| ४०
एतौ समावितिसमशोधनार्थं न्यासः
%"\nl"
\Column{यावव \whitespace\whitespace\upbefore{०}१ \downafter{३६००}}
%"{८एम्}"
\Column{याव \upbefore{२४०}६०}%"{६एम्}"
\Column{या ०}%"{३एम्}"
%"\\न्ल्"
\Column{यावव ०}%"{८एम्}"
\Column{याव १}%"{६एम्}"
\Column{या ६| १६| ४९| ७| ५९| ८| ५६| २९| ४०}%"{१६एम्}"

अथ यावनीयबीजगणिते समीकरणसंप्रदायस्त्वनया रीत्यास्ति
सा यथा ||
यदि समयोः पक्षयोर्मध्य एकाराशिरृण\-श्चेत्तद्राशितुल्यं
\footnote{{\s पक्षयोर्मध्ये एकाराशिः ऋणश्चेत् तद्राशि@ }${\rm J_1}$} \footnote{{\s पक्षयोर्मध्ये एकोराशिः ऋणश्चेत् तद्राशि@}${\rm SB_J}$}
पक्षद्वये योज्यं\footnote{{\s पक्षद्वयो जोज्यं }${\rm J_1}$}
तदापि पक्षद्वयं सममेव भवति ||
तस्मादत्र प्रथमपक्षे यावव %\\whitespace\\wहितेस्पचे\\wहितेस्पचे
\upbefore{०}१ \downafter{ ३६००}
इदं पक्षद्वये योज्यं तत्र प्रथमपक्षे
धनर्णयोस्तुल्यत्वाद्यावद्वर्गवर्गस्य नाशो .अवशिष्टो \footnote{{\s नाशोः अवशिष्टः}${\rm SB_J}$}
याव \upbefore{२४०}६०
अयं छेदभक्तो \footnote{{\s छेदभक्तः}${\rm SB_J}$} याव ४\\(|\\) अयं प्रथमपक्षः पुनर्द्वितीयपक्षे
यावव \upbefore{१}३६०० योजितः यावव \upbefore{१}३६००
याव १ या ६| १६| ४९| ७| ५९| ८| ५६| २९| ४०
एतौ पक्षौ पुनरपि समौ
%"\\न्ल्"
\Column{याव ४}%"{८एम्}"
\Column{या ०}%"{४एम्}"
%"\\न्ल्"
\Column{यावव \upbefore{१}३६००}%"{८एम्}"
\Column{याव १ या ६| १६| ४९| ७| ८| ५६| २९| ४०} \footnote{{\s याव १ या ६| १६| ४९| ७| ५९| ८| ५६| २९| ४०}${\rm SB_J}$}%"{२८एम्}"

अथयावनीयसंप्रदाये समपक्षयोर्मध्ये यौ
राश्येकजाती footnote{{\s राशी एकजाती}${\rm SB_J}$} भवतः ||
तत्र लघुराशिर्महद्राशौ शोध्यते तदापि पक्षौ समावेदावशिष्टौ \footnote{{\s समावेवावशिष्टौ}${\rm SB_J}$}
भवतः ||
तस्मादत्र यावर्गराशिरेकजाती \footnote{{\s यावर्गराशिः एकजाती}${\rm SB_J}$} यो .अस्त्यतो \footnote{{\s योस्त्यतो}${\rm SB_J}$} लघुराशिर्याव \footnote{{\s लघुराशिः याव}${\rm SB_J}$} १
महद्राशौ याव ४ शोधितः शेषं प्रथमपक्षो \footnote{{\s प्रथमपक्षः}${\rm SB_J}$} याव ३
द्वितीयपक्षे च यावव \upbefore{१}३६००
या ६| १६| ४९| ७| ५९| ८| ५६| २९| ४०
एतावपि समौ ||
पुनः\footnote{{\s पुन } ${\rm J_1}$}
पक्षद्वयमध्ये यद्येकोपक्षः खण्डितो \footnote{{\s खण्डितः}${\rm SB_J}$} द्वितीयपक्षः
संपूर्णश्चेत्तदा खण्डितराशिं प्रपूर्यं
तावद्गुणितं \footnote{{\s तावद्गुणं}${\rm SB_J}$} द्वितीयपक्षमपि
कुर्वति ||\footnote{{\s कुर्वंति }${\rm J_1}$}
अत्र सछेदराशिः खण्डितशब्देनोच्यते|
छेदरहितः संपूर्णोच्यते ||\footnote{{\s संपूर्ण उच्यते }${\rm J_1}$}
एवं कृते सति पक्षयोः समछेदत्वं\footnote{{\s पक्षयोसम@ }${\rm J_1}$}
विधाये छेदाय गम \footnote{{\s विधाय छेदापगम }${\rm J_1}$}
 एवोपपद्यते ||
तस्मादत्र द्वितीयप [क्षे \marginnote{f. 2r $J_2$}
खण्डितराशिर्यावव \footnote{{\s खण्डितराशिः यावव }${\rm J_1}$} \upbefore{१}३६००
अयमर्थः %\\(|\\)

अत्र यावद्वर्गवर्गराशेषट् शताधिकसहस्रत्रयमितोंशो
यावद्वर्गवर्गराशिरस्ति\\(|\\)
तस्मादयं हरगुणितं क्रियते ताव १ "{\\र्म् ?}" द्यावद्वर्गवर्गराशिरेको
भवतीति\footnote{{\s भवती }"{\\र्म् wइथ्}" ति "{\\र्म् इन्सेर्तेद् इन् लेfत् मर्गिन् J१}"}
द्वितीयपक्षमप्येतावता गुणनीयम् ||
एवं कृते
प्रथम\\(प\\)क्षे\footnote{{\s प्रथमक्षे }${\rm J_1}$}
याव १०८०० द्वितीयपक्षे यावव १ अस्य प्रथमपक्षो
वारद्वयं षष्टोध्व "{\\र्म् ?}" कृतो याव ३\\(|\\)
अथ द्वितीयपक्षस्थयावद्राशिस्तेनैव छेदेन ३६००
संगुण्य वारद्वयं षष्ट्यो ३ र्ध्व १ "{\\र्म् ?}" कृतो
या ६| १६| ४९| ७| ५९| ८| ५६| २९| ४०
एवम् जातौ पक्षौ समौ
"\\न्ल्"
\Column{याव ३ द्विपरिवर्त्ताः}"{१२एम्}"
"\\न्ल्"
\Column{यावव १ या ६| १६| ४९| ७| ५९| ८| ५६| २९| ४०}"{३०एम्}"
एतौ यावत्तावता पन्नत्ति "{\\र्म् ?}" द्विपरिवर्त्ता तौ
"\\न्ल्"
\Column{याव ३ द्विपरिव@}"{१२एम्}"
"\\न्ल्"
\Column{याद्य "{\\र्म् ?}" १ रू ६| १६| ४९| ७| ५९| ८| ५६| २९| ४० द्विपरिव@}"{३६एम्}"

अथात्रयावद्द्वयेन सरूपस्थराशिना
चेद्यावत्तावत्त्रयं\footnote{{\s चेद्यावत्तावत्रयं }${\rm J_1}$}
३ भाज्यते
तदा यावत्तावत्प्रमाणं लभ्यते|
अस्य भागग्रहणरीतिः पूर्वाचार्पैरे "{\\र्म् ?}" तावत्कालपर्यन्तं
न लब्धा|
अत्र यमशैदेन रीतिः प्रदर्शिता ||
सा यथा ||
रूपस्य प्रथमाङ्को यावत्तावता भाज्यः\\(|\\) लब्धेरेकान्ते
स्थाप्या ||
पुनर्लाब्धिघनं शेषाङ्के योज्यं पुनरत्र
द्वितीयाङ्को\footnote{{\s  द्वितीयांकौ }${\rm J_1}$}
यावत्तावता भाज्यः|
लब्धिपूर्वलब्धेरधस्\-थाप्यः ||
पुनर्लब्धिद्वययोगस्य घनः कार्यः\\(|\\)
एवं तत्र प्रथमलब्धिघनः शोध्यः ||
शेषं द्वितीयलब्धिशेषे योज्यम् ||\footnote{{\s ज्योज्यं }${\rm J_1}$}
पुनस्तृतीयाङ्को यावत्तावता भाज्यः ||
इयं लब्धिः पूर्वलब्धिद्वयाधस्थाप्यः ||
पुनरस्य लब्धित्रययोगस्य घनः कार्यः\\(|\\)
तत्र लब्धिद्वययोगस्य घनः शोध्यः ||
शेषं तृतीयलब्धिशेषे योज्यम् ||
पुनस्तत्रचतुर्थाङ्को यावत्तावता भाज्यः ||
एवमेते "{\\र्म् ?}" ष्टभागलब्धिपर्यन्तं विधिः कार्यः ||
एवं यमशैदेन\footnote{{\s यमशदैन }${\rm J_1}$}
यावत्तावत्प्रमाणो निष्काशितः २| ५| ३९| २६| २२| २९| २८| ३२| ५२| ३३
इयमंशद्वयस्य\footnote{{\s इयंमंश@ }${\rm J_1}$}
पूर्णज्यास्ति ||
अस्यार्धमेकांशज्या जाता १| २| ४९| ४३| ११| १४| ४५| १६| १६| १७
अस्य भागहरणोपपत्तिः ||
तत्र पक्षद्वयमध्य एकपक्षे\footnote{{\s पक्षद्वयमध्ये एक@ }${\rm J_1}$}
यावत्तावदस्ति ||
द्विपक्षे यावद्यतः $?$ रूपराशिश्च ||
एवं तत्र यावत्तावद्ज्ञानं चेत्तदा यावत्तावद्घनं कृत्वा
रूपराशौ प्रक्षिप्य यावत्तावता भाज्यते तत्र लब्धिर्यावत्तावन्मानं
स्यात् ||
अत्र तु यावत्तावद्ज्ञानं\footnote{{\s यावत्त्रवद्ज्ञानं }${\rm J_1}$}
नास्त्यतो रूपराशिरेव यावत्तावता भाज्यः\\(|\\)
यल्लब्धं तत्सरूपयावघनस्य को .अप्यंशो लब्धः
स चैकान्ते धृतः ||
पुनरस्य घनं कृत्वा $|$ \marginnote{f. 2v $J_2$}
शेषे जोतितुं पुनरत्र यावत्तावता द्वितीयां को भाजितः!!
एवं यल्लब्धं तत्तु\footnote{{\s त्तत्तु }${\rm J_1}$}
पूर्वलब्धेरथ स्थापितम् ||
एवं लघयावत्तावद्व्यनस्यासन्नो भागो
\footnote{{\s @स्वासन्नौ भोगो } with strokes stuck out $J_1$} %"{\\र्म् wइथ् स्त्रोकेस् स्त्रुच्क् ओउत्, J१}"}
लब्धः %\\(|\\)
अत्रासन्नता शेषावयवसत्वाद्निरवयवत्वे स एव सूक्ष्मा
लब्धिः ||
अथलब्धिघने पूर्वघनं शोध्यं यतस्तदधिकं जातम् ||
एवमिष्टभागपर्यन्तं मुहुर्मुहुः कार्यम् ||

अथास्योद \\qउएर्य्
तत्र द्वितीयपक्षे रूपाणि ६| १६| ४९| ७| ५९| ८| ५६| २९| ४० ||
अत्र षट् ६ द्विपरिवर्ताः षष्टेरूर्ध्वमस्ति ||
यावत्तावत्त्रयं द्विपरिवर्ताः षष्टेरूर्ध्वमस्ति ||
अत एकजातौ\footnote{{\s अतो एकजातौ }${\rm J_1}$}
भागे गृहीते लब्धमंशाः ||
पुनरस्य घने .अंशाः ८ शेषां १६ शेषु ५९ योजितं
१६| ५७ पुनः पूर्वलब्धे शेषं २८ तेनैव या ३
भक्तं लब्धकलाः ५ लब्धिः पूर्वलब्धेरदस्थोपित अं २
क ५ पुनरत्र शिष्ठं १| ५७ पुनर्लब्धिद्वयस्यघनः\\(|\\)
अं ९| २| ३२| ५ अत्र प्रथमलब्धिघनः\\(|\\)
अं ८ शोधितः शेषं अं १| २| ३२| ५७|
इदं शेषे अं १| ५७| क ७| वि ५९| ८| ५६| २९| ४०\footnote{{\s १| ५७| %\\उपfतेर्{अं} ७|\\उपfतेर्{क} ५९\\उपfतेर्{वि}|
८| ५६| २९| ४० } ${\rm J_1}$}
अंशस्थाने योजितं \\(|\\) अं १| ५८| १०| ३१| १३\footnote{{\s १| ५८| %\\उपfतेर्{अं}
 १०| ३१| १३} ${\rm J_1}$}
पुनो हरेण\footnote{{\s पुनः हरेण} ${\rm J_1}$}
या ३ अंशे ५८ भोगो\footnote{{\s अंशे \upbefore{५८}भोगो  }${\rm J_1}$}
गृहीतः \\(|\\)
लब्धिर्विकलात्मिका ३९\footnote{{\s लब्धिर्विकालात्मिका ३९ } ${\rm J_1}$}
शेषं अं १| १०| ३१| १३ पुनरियं लब्धिः पूर्वलब्धेरधःस्थाप्यः
अं २| ५| ३९
अस्य व्यनोशादि}
 %\query



 



\end{document}
