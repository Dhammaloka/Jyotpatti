\documentclass[11pt,a5paper]{book}

%%%Font related packages
\usepackage[no-math]{fontspec}
\defaultfontfeatures{Ligatures=TeX} 
%\setmainfont{Linux Libertine O}
\newfontfamily\s[Scale=0.92, Script=Devanagari]{Shobhika-Regular}
\usepackage{color, dtk-logos}
 
 
 %%%Formatting related packages
\usepackage{graphicx}
\usepackage[hang,flushmargin]{footmisc} % For no indentation of footnotes.
\usepackage{array}
\usepackage{tabularx, multicol, vwcol}
\usepackage{fullpage}
\usepackage{marginnote}
\usepackage{bm} %making accents thicker

%Math related packages
\usepackage{amsmath} % āmerican Mathematical Society packages for metamathematical symbols.
%āpril 2017 āditya says install libertine math font package specially from website.

%Page formatting commands

%\newdimen\stdbaseline
%\stdbaseline = 13pt 
%\baselineskip = 2\baselineskip
\parskip = \baselineskip
\parindent = 0pt


%%%Critical editing packages
\usepackage[series={A,B,C,D},noend,noeledsec,nofamiliar,noledgroup]{reledmac}
\Xarrangement[B]{twocol}
\Xarrangement[C]{threecol}
\Xarrangement[D]{paragraph}

% Internal editing packages
\usepackage{todonotes}

%% MāCROS for Diacriticals, symbols, math and text features

\def\elp{$\ldots\,$}
\def\degrees{$^\circ$}
\def\degree{$^\circ$}
\def\signs{$^s$}
\def\Sin{\mathop{\rm Sin}\nolimits}
\def\Cos{\mathop{\rm Cos}\nolimits}
\def\Versin{\mathop{\rm Vers}\nolimits}
\def\Coversin{\mathop{\rm Coversin}\nolimits}
\def\Crd{\mathop{\rm Crd}\nolimits}
\def\crd{\rm{crd}}
\newcommand\myatop[2]{\genfrac{}{}{0pt}{}{#1}{#2}}
%%%%%%%%%%%%%%%%%%%%%%%%

%% MāCROS for Sanskrit Names

\def\Ganesa{Ga\-\*ne\-\'sa}
\def\Bhaskara{Bh\=a\-ska\-ra}
%
\def\KKT{\textit{Kara\-\*na\-kut\=u\-hala\-\*t\={\i}\-k\=a}}
\def\KKe{\textit{Kara\-\*na\-ke\-\'sari}}
\def\KheSi{\textit{Khe\-cara\-siddhi}}
\def\JCU{\textit{Jy\=a\-c\=a\-potpa\-tti}}
\def\PhCC{\textit{Phira\.ngi\-candra\-cchedyayopa\-yogika}}
\def\Mahadevi{\textit{Mah\=a\-dev\={\i}}}
\def\RG{\textit{Rekh\=a\-ga\*ni\-ta}}
\def\Lil{\textit{L\={\i}l\=a\-va\-t\={\i}}}
%
\def\kalpa{\textit{ka\-lpa}}
\def\kala{\textit{ka\-l\=a}}
\def\kalas{\textit{ka\-l\=as}}
\def\ghatika{\textit{gha\*tik\=a}}
\def\jyotisa{\textit{jyo\-ti\-\*sa}}
\def\pratikalas{\textit{prati\-ka\-l\=as}}
\def\prativikalas{\textit{prati\-vi\-ka\-l\=as}}
\def\bijaganita{\textit{b\={\i}ja\-ga\*ni\-ta}}
\def\yavattavat{\textit{y\=avatt\=avat}}
\def\ya{\textit{y\=a}}
\def\yava{\textit{y\=ava}}
\def\yagha{\textit{y\=agha}}
\def\yavava{\textit{y\=avava}}
\def\rasi{\textit{rāśi}}
\def\ru{\textit{r\=upa}}
\def\vikala{\textit{vi\-ka\-l\=a}}
\def\vikalas{\textit{vi\-ka\-l\=as}}
%
\def\zij{\textit{z\={\i}j}}
\def\yavaniya{\textit{y\=avan\=iya}}
\def\bijaganita{\textit{b\=ijaga\*nita}}
%
\def\opcit{\textit{op.\ cit.}}
\def\loccit{\textit{loc.\ cit.}}
\def\elp{$\ldots$}
\def\danda{$|$}
\def\nl{\hfill\break}
%
\def\sri{\textit{\'sr\={\i}}}
%

% \newcommand{\JOne}[1]{\colchunk{{#1}}}
% \newcommand{\JTwo}[1]{\colchunk{{#1}%
%    }\colplacechunks}

%=============================================================%



\begin{document}

{\s  
श्रीगणेशाय नमः | \marginnote{f. 1r $J_2$}
अथैकांशजीवाविषये
उलुग्वेगीजीकस्य \\
शरहविर्जन्दीस्य\footnote{{\s @विर्जंदीस्थ }${\rm J_1}$}
व्याख्या लिख्यते |
तत्रैकांशजी\-वान\-यने नव्यतरं \\प्रकारद्वयमस्ति |
एकं यमशैदकाशीसंज्ञेन कृतम् |
द्वितीयं मिर्योलुग्वेगेन कृतम् |
परं चोलुग्वेगस्य यमसैदकाशीवेधप्रक्रियायां
साहाय्यकार्यस्ति\footnote{{\s सा हाथकार्यस्ति} $J$} |
तत्र प्रथमप्रकारः कथ्यते | सा\footnote{{\s सं} $J$} यथा |
अबजदं षडंश ६ चापं कल्पितम् |
पुनरस्य समानं भागत्रयं बचिह्नजचिह्नयोः कृतम् ।
तत्र अबमजमदं बजं बदं जदं चैताः पूर्णज्यारेखाः
योज्याः |
अथात्र पूर्वोक्तप्रकारेणांशत्रयस्य\footnote{{\s अथात्रपूर्वाक्त@}${\rm J_1}$}
जीवा ज्ञातास्ति | सा चेयं ३ | ८ | २४ | ३३ | ५९ | ३४ | २८ | ५४ | ५० |
इयं द्विगुणा ६ | १६ | ४९ | ७ | ५९ | ८ | ५६ | २९ | ४० | जाता अदपूर्णज्या\footnote{{\s मदपर्णज्या}${\rm J}$}|

अत्रांशद्वयस्याबपूर्णज्या ज्ञानमिष्टमस्ति |
तत्र मिजिस्तीग्रन्थस्य \\
प्रथमाध्यायस्थद्वितीयक्षेत्रे\footnote{{\s क्षेत्र} $J$} 
इदमुपपन्नम् \footnote{{\s @द्वितीयक्षेत्रयिदमुप@ }${\rm J_1}$}|
यच्चतुर्भुजवृत्तान्तः पतति \\
तत्सन्मुखस्थभुजानां\footnote{{\s तत्संन्मुख@ } ${\rm J_1}$}
घातयोगस्तच्चतुर्भुजान्तः पतितकर्णद्वयघातसमो भवति |
पुनस्तत्रैव प्रथमाध्यायस्य
चतुर्थक्षेत्रे\footnote{{\s क्षेत्र} $J$} इदमुपपन्नम् \footnote{{\s चतुर्थक्षेत्रे इदमुप@ }${\rm J_1}$}|| \\
इष्टचापार्धपूर्णज्यावर्गस्तेनाङ्केनसमो \footnote{{\s @बर्ग@}${\rm SB_J}$} भवति |
योऽङ्कः %\\(|\\)
 तच्चापोनभार्धांशानां या पूर्णज्या भवत्यस्या \footnote{{\s भवति । अस्याः}${\rm SB_J}$}
व्यासेन यदन्तरं तेन गुणित यो
व्यासार्धस्तदङ्कतुल्यो\footnote{{\s व्यासार्धः स्तदंक@ }${\rm J_1}$}
भवति |}


\newpage

Homage to Lord \Ganesa. \marginnote{f. 1r $J_2$} Now, with regard to the Sine of one degree, the commentary
\textit{\'Sarahavirja\*md\={\i}} of \textit{Ulugveg\={\i} j\={\i}ka} is written.
There in the determination of the Sine of one degree there is a newer pair of methods. 
One is made by [the one] named Yama\'saida K\=a\'s\={\i}; the second is made by
Miryolugvega. The latter of Ulugvega is assisted [by] the expounding chapter of Yamasaida K\=a\'s\={\i}. 

There the first method \todo{theorem?} is stated. It is as follows: 
ABJD is considered an arc of six degrees. Again, three equal parts of this [arc] are made at points $B$ and $J$. 
There, $AB$, $AJ$, $AD$, $BJ$, $BD$, and $JD$: they are to be drawn as Chord lines. 
Now, here by the previously-stated method the Sine of three degrees is known and 
it is this: $3|8|24|33|59|34|28|14|50$ \footnote{Has 54|50}.
This is multiplied by two: $6|16|49|7|59|8|56|29|40$, results the Chord of AD. 
Here the knowledge of the Chord AB of two degrees is desired. 
Then in the second figure of the first chapter of the book Mijisti this is demonstrated (\textit{upapanna}). 

Whatever quadrilateral falls inside a circle, the sum of the 
products of the sides standing opposite becomes equal to the
product of the two diagonals falling inside that quadrilateral. 

Then again, in the fourth figure of the first chapter, this is demonstrated (\textit{upapanna}). \todo{Not sure about this}The square
of the Chord of half the desired arc [? the half-Chord of the desired
arc?] becomes equal with that number. Whatever is the number, whatever becomes
the Chord of the degrees of half a circle diminished by that arc, whatever is the difference 
of that with the diameter, whatever is the half-diameter multiplied by that, becomes 
equal to that number. [?] 

\newpage
{\s एवमत्र अबचापजदचापे मिथः समाने स्तः\footnote{{\s स्थः} $J$} |
एवं अजबदचापे च समाने स्तः ||
तस्मात् अबजदघातो \footnote{{\s अवजदघातो@}${\rm SB_J}$} वर्गो भविष्यति ||
अत्र बजमज्ञातं तद्यावत्तावन्मितं कल्पितमिदमदेनगुणितं
जातं या\footnote{{\s वा} $J$} ६| १६|४९| ७| ५९| ८| ५६| २९| ४०|
एवं अबजदघातरूपवर्गराशेर्यावत्तावतश्च \footnote{{\s अवजद@}${\rm SB_J}$} योगः
अजवर्गतुल्य अजबदघा\footnote{{\s अनबदथ@}${\rm J}$} तेन समानोऽस्ति \footnote{{\s @ष्ति}${\rm J}$}|

अतोऽयं अजवर्गतुल्यो जातः ||
तस्मात् \\
अजवर्गराशौ एकोवर्गराशिरेभिर्यावत्तावद्भिरधिको
\footnote{{\s अजवर्गराशी एको@ }${\rm J_1}$}
\footnote{{\s अजवर्गराशौ एको@}${\rm SB_J}$}
जातः|
याव १ या ६| १६| ४९| ७| ५९| ८| ५६| २९| ४०}


\newpage
In this way here arc $AB$ [and] arc $JD$ pairwise become equal. 
And in this way arcs $AJ,$  $BD$ become equal.
From that, the product of $AB$, $JD$ will be a square. Here, $BJ$ is not known;
that is considered the measure of the \ya. This is multiplied by $AD$; producing \ya\
6\danda 16\danda 49\danda 7\danda 59\danda 8\danda 56\danda 29\danda 40.
In this way the sum of the \ya\
and of the amount of the square of the number of the product of $AB$, $JD$
[is] equal to the square of $AJ$, moreover $AJ$ $BD$, is equal with that. [?]

%  \begin{figure}[h]
% \includegraphics[width=2in] %% 
% {JCUFig1bis}
% \centering
% \label{JCUFig1bis}
% \end{figure}

Here, this result is equal to the square of $AJ$. Therefore, one [side] is equal to the amount of the square
$AJ$, the [other] is equal to the amount of the square increased by those \yavattavat s. \yava\ 1 \ya\ 6\danda 16\danda 49\danda 7\danda 59\danda 8\danda
56\danda 29\danda 40. 

\newpage
{\s अथ
प्रकारान्तरेण अजवर्गः\footnote{{\s प्रकारांतरेणाजवर्गः }${\rm J_1}$}
साध्यते |
तत्र अजचापं भार्धांशेभ्यः शोध्यं शेषस्य
पूर्णज्याया\footnote{{\s पूर्णज्यायाः}${\rm SB_J}$} व्यासस्य चान्तरं कार्यम् । तेन घ्नं षष्टितुल्यव्यासार्धं
अबवर्गतुल्यं भविष्यति
यतः अजचापं अबचापाद्द्विगुणमस्ति |
तस्माद्यदि अवबर्गः
याव १ षष्टि ६० तुल्यव्यासार्धेन [यदा] भाज्यते तदा यावद्वर्गस्य
षष्ट्यंशो लभ्यते तच्च अजचोपोनभार्धानां या पूर्णज्या
तदूनो यो व्यासस्तस्य प्रमाणमस्ति |
पुनरिदमन्तरं संपूर्णव्यासे १२० शोधितं
याव $\bm{\frac{\dot {1}}{60}}$ रू १२०
इदं अजचापोनभार्धांशानां पूर्णज्या जाता |
अस्याः वर्गः %\\(|\\)
यावव $\bm{\frac{1}{3600}}$ याव $\bm{\frac{\dot{240}}{60}}$
रू १४४००}


\newpage
Now by the other theorem, the square of $AJ$ is determined. There, arc $AJ$ is to be subtracted
from the degrees of half a circle; the difference of the Chord of the remainder 
and of the diameter is to be made. The half-diameter equal to sixty multiplied by that
will be equal to the square of $AB$ since arc $AB$ is doubled from arc $AB$. 
From that, if the square of $AB$ \yava\ 1, [when] it is divided by the half-diameter equal to sixty, 60, 
then a sixtieth part of the square of the \yava\ is obtained, that is, whatever is the Chord of half a 
circle diminished by arc $AJ$, whatever diameter is decreased by that, it is the amount of that.
Again, this difference is subtracted from the full diameter 120:
\yava\ $-\frac{1}{60}$, un.\ 120. This is the Chord of the
degrees of half a circle diminished by arc $AJ$. Its square: 
\yavava\ $\frac{1}{3600}$ \yava\ $\frac{\dot{240}}{60}$ un.\ 14400.

\newpage
{\s अथ च यदीष्टचापपूर्णज्यावर्गो \footnote{{\s यदीष्टचापपूर्णज्यावर्गः}${\rm SB_J}$} व्यासवर्गाच्छोध्यते
तत्र शेषं
$|$\marginnote{f. 1v $J_2$} \\
तच्चापोनभार्धांशपूर्णज्यावर्गो
भवति\footnote{{\s भावति }${\rm J_1}$}
यतो व्यासेन इ्ष्टचापे पूर्णज्यया इष्टचापोनभार्धांशपूर्णज्यया
चैकं समकोणत्रि\-भुजं भवति यतो वृत्तार्धे व्यासप्रान्तादुत्पन्नत्रिभुजस्य
पालिकोणः समकोणो
भवतीत्युक्लीदसस्य \footnote{OH MY GOD IT'S EUCLID}
तृतीयाध्याये
उपपन्न\-मस्ति 
तत्त्रिभुजे
समकोणसन्मुखभुजो व्यासोऽस्ति |
स च कर्णरूपः |
पुनः कर्णवर्गो \footnote{{\s कर्णवर्गः}${\rm SB_J}$} भुजकोट्योर्वर्गयोगेन समो
भवतीत्युक्लीदसस्य
प्रथमध्याये
उपपन्नम् |

अतोऽत्र व्यासवर्गः १४४०० कर्णवर्गरूपः |
अस्मिन्निष्टचापोनभार्धांशानां\footnote{{\s अस्मिनिष्टचा@} $J$} पूर्णज्यावर्गः
यावव $\bm{\frac{1}{3600}}$ याव $\bm{\frac{\dot{240}}{60}}$
रू १४४००
शोधितः | शेषं अजपूर्णज्यावर्गोऽवशिष्टः 
यावव %\\wहितेस्पचे\\wहितेस्पचे\\wहितेस्पचे
$\frac{\dot{1}}{3600}$ याव $\frac{240}{60}$ अयं
द्वितीयप्रकारेण अजवर्गः\footnote{{\s @प्रकारेणा .अजवर्गः }${\rm J_1}$}
सिद्धः |}

\newpage
And now, when the square of the Chord of the desired arc is subtracted from the square of the
diameter, there the remainder \marginnote{f. 1v $J_2$} is the square of the Chord of the degrees 
of half a circle diminished by that arc, since one right-angled triangle is [formed] by the diameter
and the Chord of the desired arc and the Chord of the degrees of half a circle diminished by the desired arc; 
and since ``the circumference-angle of a triangle produced from the end [actually ends] of the diameter 
in a half-circle is a right angle'' is demonstrated in the third book of Euclid (\textit{uklīdasa}), 
the side facing the right angle in that triangle is the diameter. And it 
has the form of a hypotenuse. Again ``the square of the hypotenuse
is equal to the sum of the squares of the arm and the upright'' is demonstrated in the first book of Euclid (\textit{uklīdasa}). 
So here the square of the diameter,14400, has the form of the square of a hypotenuse. From it, the square of the Chord of the 
degrees of half a circle diminished by the desired arc \yavava\ $\frac{1}{3600}$ \yava\ $\frac{\dot{240}}{60}$ un. 14400 is subtracted. 
The remainder is the square of the Chord AJ: 
remainder \yavava\ $\frac{\dot{1}}{3600}$ \yava\ $\frac{240}{60}$.
This by the second method is the established square of $AJ$.

\newpage

{\s अथ प्रथमप्रकारागत अजवर्गः याव १ या ६| १६| ४९| ७| ५९| ८| ५६| २९ | ४० |
एतौ समावितिसमशोधनार्थं न्यासः
%"\nl"
%\Column{यावव \whitespace\whitespace\upbefore{०}१ \downafter{३६००}}
%"{८एम्}"
%\Column{याव \upbefore{२४०}६०}%"{६एम्}"
%\Column{या ०}%"{३एम्}"
%"\\न्ल्"
%\Column{यावव ०}%"{८एम्}"
%\Column{याव १}%"{६एम्}"
%\Column{या ६| १६| ४९| ७| ५९| ८| ५६| २९| ४०}%"{१६एम्}"

अथ यावनीयबीजगणिते समीकरणसंप्रदायस्त्वनया रीत्यास्ति |
सा यथा |
यदि समयोः पक्षयोर्मध्येा राशिरृण\-श्चेत्तद्राशितुल्यं 
पक्षद्वये योज्यं\footnote{{\s पक्षद्वयो जोज्यं }${\rm J_1}$} |
तदापि पक्षद्वयं सममेव भवति |
तस्मादत्र प्रथमपक्षे यावव %\\whitespace\\wहितेस्पचे\\wहितेस्पचे
$\frac{\dot{1}}{3600}$
इदं पक्षद्वये योज्यं तत्र प्रथमपक्षे
धनर्णयोस्तुल्यत्वाद्यावद्वर्गवर्गस्य नाशः | अवशिष्टः \footnote{{\s नाशोः अवशिष्टः}${\rm SB_J}$}
याव $\frac{240}{60}$ |
अयं छेदभक्तः \footnote{{\s छेदभक्तः}${\rm SB_J}$} याव ४ | अयं प्रथमपक्षः | पुनर्द्वितीयपक्षे
यावव $\frac{1}{3600}$ योजितः यावव $\frac{1}{3600}$
याव १ या ६| १६| ४९| ७| ५९| ८| ५६| २९| ४०
एतौ पक्षौ पुनरपि समौ |
%"\\न्ल्"
%\Column{याव ४}%"{८एम्}"
%\Column{या ०}%"{४एम्}"
%"\\न्ल्"
%\Column{यावव \upbefore{१}३६००}%"{८एम्}"
%\Column{याव १ या ६| १६| ४९| ७| ८| ५६| २९| ४०} \footnote{{\s याव १ या ६| १६| ४९| ७| ५९| ८| ५६| २९| ४०}${\rm SB_J}$}%"{२८एम्}"  
}
\newpage

Now, the square of $AJ$ established by the first theorem: \yava\ 1 \ya\ 6|16|49|7|59|8|56|29|40. These two are equal. Thus for the
sake of subtraction of equals, the setting-out:
\yavava\ $\frac{-1}{3600}$ \yava\ $\frac{240}{60}$ \ya\ 0\\
\yavava\ 0 \yava\ 1 \ya\  6|16|49|7|59|8|56|29|40\\

Now, in western (\yavaniya) algebra (\bijaganita) the balancing (sam.pradaya) of equations is however by this method. That is as follows: If one quantity (\rasi) among the two equal sides, if it is negative, then an equal quantity is to be added to both sides. And then in this way, the pair of sides becomes equal. From
that, here in the first side \yavava\ $\frac{-1}{3600}$ this is to be added to both sides. There in the first side,
because of equality of positive and negative,there is elimination of the \yavava; remainder \yava\ $\frac{240}{60}$. This is divided by the divisor: \yava\ 4. This is the first side. Again in the second side \yavava\ $\frac{1}{3600}$ is added: \yavava\  $\frac{1}{3600}$
\yava\ 1 \ya\ 6|16|49|7|59|8|56|29|40. The two sides are again equal:\\
\yava\ 4 \ya\ 0
\yavava\  $\frac{1}{3600}$ \ya\ 6|16|49|7|59|8|56|29|40.

\newpage
{\s अथ यावनीयसंप्रदाये समपक्षयोर्मध्ये यौ
राश्येकजाती \footnote{{\s राशी एकजाती}${\rm SB_J}$} भवतः |
तत्र लघुराशिर्महद्राशौ शोध्यते तदापि पक्षौ समावेावशिष्टौ 
भवतः |
तस्मादत्र या वर्गराशिरेकजाती | योऽस्त्यतो लघुराशिः याव १
महद्राशौ याव ४ शोधितः | शेषं प्रथमपक्षः याव ३
द्वितीयपक्षे च यावव $\frac{1}{3600}$
या ६ | १६ | ४९ | ७ | ५९ | ८ | ५६ | २९ | ४० |
एतावपि समौ |
पुनः\footnote{{\s पुन } ${\rm J_1}$}
पक्षद्वयमध्ये यद्येकः पक्षः खण्डितो\footnote{{\s खण्डितः}${\rm SB_J}$} द्वितीयपक्षः
संपूर्णश्चेत्तदा खण्डितराशिं प्रपूर्य\footnote{{\s प्रपूर्यं}${\rm J}$}
तावद्गुणितं \footnote{{\s तावद्गुणं}${\rm SB_J}$} द्वितीयपक्षमपि
कुर्वन्ति |
अत्र सछेदराशिः खण्डितशब्देनोच्यते |
छेदरहितः संपूर्णोच्यते |\footnote{{\s संपूर्ण उच्यते }${\rm J_1}$}
एवं कृते सति पक्षयोः समछेदत्वं\footnote{{\s पक्षयोसम@ }${\rm J_1}$}
विधाय छेदापगम \footnote{{\s विधाय छेदापगम }${\rm J_1}$}
 एवोपपद्यते |
तस्मादत्र द्वितीयपक्षे \marginnote{f. 2r $J_2$}
खण्डितराशिः यावव $\frac{1}{3600}$} |

\newpage
Now in the western balancing among the two equal sides there are two quantities of the same type. There the lesser quantity is subtracted from the greater quantity; and then the two sides
remain just the same. From that, here whatever is the square quantity that is the same type (ekajāti). 
Whatever it is, then the lesser quantity \yava\ 1 is subtracted from the greater quantity \yava\
4. The remainder is the first side, \yava\ 3 and on the second side \yavava\  $\frac{1}{3600}$  \ya\ 6|16|49|7|59|8|56|29|40. 
And these two are equal. 

Again, among the two sides, if one side is `broken', if the second side is full, then the broken quantity is filled by multiplying so much. They make [multiply?] the second side also. [???] Here a quantity with a divisor is called
“broken” (\textit{kha\d n\d dita}), devoid of divisor is called “filled”. Computing thus, having established same-divisorness of the two sides, the removal of divisors is in this way achieved. From that, here on the second side \marginnote{f. 2r $J_2$} 
the broken quantity is \yavava\ $\frac{1}{3600}$

\newpage 
{\s अयमर्थः 
अत्र यावद्वर्गवर्गराशे षट्शताधिकसहस्रत्रयमितोंऽशो
यावद्वर्गवर्गराशिरस्ति । 
तस्मादयं हरगुणितः\footnote{{\s हरगुणितं}$J$} क्रियते यावत्तावद्वर्गवर्गवर्गरािरेको\footnote{{\s ताव १ द्यावद्वर्गवर्गराशिरेको} $J$} 
भवति । 
द्वितीयपक्षमप्येतावता गुणनीयम् | 
एवं कृते
प्रथमपक्षे\footnote{{\s प्प्रथमक्षे}${\rm K}$}
याव १०८०० द्वितीयपक्षे यावव १ । अथ प्रथमपक्षः\footnote{{\s प्रथमपक्षो }${\rm K}$}
वारद्वयं षष्टोर्ध्वीकृतः कृतो\footnote{{\s कृतो}${\rm K}$} याव ३ 
अथ द्वितीयपक्षस्य\footnote{{\s प्द्वितीयपक्षस्थ}${\rm K}$} यावद्राशिस्तेनैवच्छेदेन\footnote{{\s यावद्राशिस्तेनैव छेदेन}$J$} ३६००
संगुण्य वारद्वयं षष्ट्योर्ध्वी\footnote{{\s षष्ट्यो उर्ध्व}${\rm J}$} कृतः\footnote{{\s कृतो}${\rm K}$}
या ६ | १६ | ४९ | ७ | ५९ | ८ | ५६ | २९ | ४० । 
एवं जातौ पक्षौ समौ ।
\begin{tabular}{c | c }
{\s याव ३ द्विपरिवर्त्ताः }						
{\s यावव १ या ६ | १६ | ४९ | ७ | ५९ | ८ | ५६ | २९ | ४० |}	
\end{tabular}
एतौ यावत्तावतापवर्त्तितौ\footnote{{\s यावत्तावता पवत्ति}${\rm J}$} द्विपरिवर्त्ता तौ
\begin{tabular}{c}
{\s या\footnote{{\s याव}${\rm K}$} ३ द्विपरिव@}"{{12em}"}				
{\s याघ १ रू ६| १६| ४९| ७| ५९| ८| ५६| २९| ४० द्विपरिव@}"{{36em}"} 	
\end{tabular} 
}

\newpage 
This is the meaning.  Here the quantity \todo{coefficient?} of the square of the square of the \yavattavat: the coefficient of the square of the square of the \yavattavat\ is commensurate with three-thousand increased by six-hundred (3600) parts. Thereafter, this is multiplied by the denominator. The coefficient of square of the square of the \yavattavat\ is one. The second side is also to be multiplied by so much.  Computing in this way, on the first side there is \yava\ 10800 [and] on the second side \yavava\ 1. Now the 
first side is made upwards-quantity-pair-60, \yava\ 3.  Now, the quantity of \yavattavat s 
of the second side, multiplying by that divisor, 3600, it is made upwards-quantity-pair-60. 
\ya\ 6 \danda 16 \danda 49 \danda 7 \danda 59 \danda 8 \danda 57 \danda 49 \danda 40.  In this way these two sides  produced equal: \yava 3 upwards-twice. \yavava 1 \ya\ 6 \danda 16 \danda 49 \danda 7 \danda 59 \danda 
8 \danda 57 \danda 49 \danda 40 upwards-twice. These two are reduced 
by a \yavattavat  \yava\ 1 r\=u 6 \danda 
16 \danda 49 \danda 7 \danda 59 \danda 8 \danda 57 \danda 49 \danda 40.




\newpage 
{\s अथात्र यावद्द्वयेन सरूपस्थराशिना
चेद्यावत्तावत्रयं\footnote{{\s चेद्यावत्तावत्त्रयं}${\rm K}$}
३ भाज्यते
तदा यावत्तावत्प्रमाणं लभ्यते |
अस्य भागग्रहणरीतिः पूर्वाचार्यैरेतावत्कालपर्यन्तं\footnote{{\s पूर्वाचार्पैरे तावत्कालपर्यन्तं}${\rm K}$}
न लब्धा |
अत्र यमशैदेन रीतिः प्रदर्शिता | 
सा यथा | 
रूपस्य प्रथमाङ्को यावत्तावता भाज्यः । लब्धिरेकान्ते\footnote{{\s लब्धेरेकान्ते}$J$}
स्थाप्या | 
पुनर्लाब्धिघनं\footnote{{\s पुनर्लधिघनं}${\rm J}$} शेषाङ्के योज्यं । पुनरत्र
द्द्वितीयांकौ\footnote{{\s  वितीयाङ्को }${\rm K}$}
यावत्तावता भाज्यः |
लब्धिपूर्वलब्धेरधस्थाप्यः | 
पुनर्लब्धिद्वययोगस्य घनः कार्यः । 
एवं तत्र प्रथमलब्धिघनः शोध्यः | 
शेषं द्वितीयलब्धिशेषे योज्यम् ||\footnote{{\s ज्योज्यं }${\rm J_1}$}
पुनस्तृतीयाङ्को यावत्तावता भाज्यः | 
इयं लब्धिः पूर्वलब्धिद्वयाधस्थाप्यः | 
पुनरस्य लब्धित्रययोगस्य घनः कार्यः । 
तत्र लब्धिद्वययोगस्य घनः शोध्यः | 
शेषं तृतीयलब्धिशेषे योज्यम् | 
पुनस्तत्रचतुर्थाङ्को यावत्तावता भाज्यः | 
एवमेवेष्टभागलब्धिपर्यन्तं विधिः कार्यः | 
एवं यमशैदेन\footnote{{\s यमशदैन }${\rm J_1}$}
यावत्तावत्प्रमाणो निष्काशितः २ | ५ | ३९ | २६ | २२ | २९ | २८ | ३२ | ५२ | ३३ ।
इयंमंशद्वयस्य\footnote{{\s इयमंश@ }${\rm K}$}
पूर्णज्यास्ति | 
अस्यार्धं एकांशज्या\footnote{{\s अस्यार्धमेकांशज्या }${\rm K}$} जाता १ | २ | ४९ | ४३ | ११ | १४ | ४४ | १६ | २६ | १७ । }


\newpage 
Now here, if the quantity with units [and] the cube of the \yavattavat\ is divided by three, \ya\ 3, 
then an amount of the \ya\ is obtained.    \todo{CHECK}
The method of the grasping of the part of it 
by the ancient teachers until now has not been obtained.

Here the rule is expounded by Yamashada. It is as [follows]:

The first digit of the unit [term] is to be divided by the \yavattavat\ [coefficient]. The quotient is to be set
[at] one end. Again, the quotient-cube is to be added to the remainder-digit;
here again, the second digit is to be divided by the \yavattavat\ [coefficient]; the quotient is to be set
below the previous quotient. Again, the cube of the sum of the two quotients
is to be made.  In this way there the cube of the first quotient is to be subtracted;
the remainder is to be added to the remainder of the second quotient. Again, the third digit is to be
divided by the \yavattavat\ [coefficient]. This quotient is to be set below the previous quotient. Again,
the cube of the sum of its three quotients is to be made. There the cube of
the sum of the two quotients is to be subtracted. 

$\marginnote{f.~4r J}$
The remainder is to be added to the remainder of the third quotient. Again there, the 
fourth digit is to be divided by the \yavattavat\ [coefficient]. In just this way the rule is to be done until either the the desired [number of] divisions or the desired quotient. In this way by Yama\'saida the amount of the \yavattavat\ is 
expanded:

2\danda 5\danda 39\danda 26\danda 22\danda 29\danda 28\danda 32\danda
52\danda 33\danda . This is the Chord of two degrees. Half of it produces the
Sine of one degree: 
1\danda 2\danda 49\danda 43\danda 11\danda 14\danda 44\danda 16\danda
26\danda 17 

\newpage
{\s अस्य भागहरणोपपत्तिः | 
तत्र पक्षद्वयमध्ये\footnote{{\s पक्षद्वयमध्य एक@ }${\rm K}$} एकपक्षे
यावत्तावदस्ति | 
द्विपक्षे यावद्यतः $?$ रूपराशिश्च | 
एवं तत्र यावत्तावद्ज्ञानं चेत्तदा यावत्तावद्घनं कृत्वा
रूपराशौ प्रक्षिप्य यावत्तावता भाज्यते । तत्र लब्धिर्यावत्तावन्मानं
स्यात् | 
अत्र तु यावत्तावद्ज्ञानं
नास्त्यतो रूपराशिरेव यावत्तावता भाज्यः । 
यल्लब्धं तत्सरूपयावघनस्य कोप्यंऽशो \footnote{{\s को .अप्यंशो}${\rm K}$} लब्धः । 
स चैकान्ते धृतः | 
पुनरस्य घनं कृत्वा $|$ \marginnote{f. 2v $J_2$}
शेषे जोतितं \footnote{{\s जोतितुं}${\rm K}$} पुनरत्र यावत्तावता द्वितीयांको भाजितः । 
एवं यल्लब्धं तत्तु\footnote{{\s त्तत्तु }${\rm J_1}$}
पूर्वलब्धेरथ स्थापितम् | 
एवं लघयावत्तावद्व्यनस्यासन्नो भागो
\footnote{{\s @स्वासन्नौ भोगो } with strokes stuck out $J_1$} %"{\\र्म् wइथ् स्त्रोकेस् स्त्रुच्क् ओउत्, J१}"}
लब्धः %\\(|\\)
अत्रासन्नता शेषावयवसत्वावात् निरवयवत्वे\footnote{{\s शेषावयवसत्वाद्निरवयवत्}${\rm K}$} स एव सूक्ष्मा
लब्धिः | 
अथलब्धिघने पूर्वघनं शोध्यं यतस्तदधिकं जातम् | 
एवमिष्टभागपर्यन्तं मुहुर्मुहुः कार्यम् | }
\newpage 

The demonstration with its \textit{bh\=agahara} [division of parts? maybe term for the long division?]. There in 
the middle of two sides (??), in one side is the \yavattavat. 
In the other side is the \ya-cube
and the amount of units. In this way there if the \ya\ is known, then having
made the \ya-cube and added it to the amount of units, [it] is divided by 
the \ya\ [coefficient]:  there the quotient should be the amount of the \ya. But here the
\ya\ is not known, hence the amount of units itself is to be divided by the \ya\
[coefficient]. 
Whatever is the quotient is obtained [as] some part of the \ya-cube-plus-units.
And it is kept at one end. Again, having made its cube, [it] is to be added
to the remainder. Again here, the second digit is divided by the \ya\ [coefficient]. In this way
whatever is the quotient is set below the previous quotient. In this way
the quotient is taken [as] the approximate-part of the \ya-cube. Here with the 
approximateness and part-of-syllogism-ness are just two part-of-syllogism-ness [???]:
the quotient is accurate. 

Now, having subtracted the previous cube from the cube of the quotient, since
the result [?] is greater than that, in this way up to the desired part 
\textit{mudgarb\=ara\*m v\=ara\*m} [???] is to be made.
\newpage 

{\s अथास्योद \\qउएर्य् ??? 
तत्र द्वितीयपक्षे रूपाणि ६ | १६ | ४९ | ७ | ५९ | ८ | ५६ | २९ | ४० | 
अत्र षट् ६ द्विपरिवर्ताः षष्टेरूर्ध्वमस्ति | 
यावत्तावत्त्रयं द्विपरिवर्ताः षष्टेरूर्ध्वमस्ति | 
अतो\footnote{{\s अत}${\rm K}$} एकजातौ 
भागे गृहीते लब्धमंशाः | 
पुनरस्य घने अंशाः ८ शेषां १६ शेषु ५९ योजितं
१६ | ५७  पुनः पूर्वलब्धे शेषं २८ तेनैव या ३
भक्तं लब्धकलाः ५ लब्धिः पूर्वलब्धेरधस्थोपिता अं २
क ५ पुनरत्र शिष्ठं १ | ५७ पुनर्लब्धिद्वयस्यघनः\\(|\\)
अं ९ | २ | ३२ | ५ अत्र प्रथमलब्धिघनः\\(|\\)
अं ८ शोधितः शेषं अं १| २| ३२| ५७|
इदं शेषे अं १| ५७| क ७| वि ५९| ८| ५६| २९| ४०\footnote{{\s १| ५७| %\\उपfतेर्{अं} ७|\\उपfतेर्{क} ५९\\उपfतेर्{वि}|
८| ५६| २९| ४० } ${\rm J_1}$}
अंशस्थाने योजितं \\(|\\) अं १ | ५८ | १० | ३१ | १३\footnote{{\s १| ५८| %\\उपfतेर्{अं}
 १०| ३१| १३} ${\rm J_1}$}
पुनर्हरेण\footnote{{\s पुनो हरेण} ${\rm K}$}
% या ३ अंशे ५८ भोगो\footnote{{\s अंशे \upbefore{५८}भोगो  }${\rm J_1}$}
गृहीतः \\(|\\)
लब्धिर्विकलात्मिका ३९\footnote{{\s लब्धिर्विकालात्मिका ३९ } ${\rm J_1}$}
शेषं अं १ | १० | ३१ | १३ पुनरियं लब्धिः पूर्वलब्धेरधः स्थाप्यः
अं २ | ५ | ३९
अस्य घनोंशादि ९ । ११ । २ । ३२ । २७ । ५३ । ३९ ।}
 %\query
\newpage 
Now, its example. There in the second side are units 

6\danda 16\danda 49\danda
7\danda 59\danda 8\danda 56\danda 29\danda 40. Here, six is turned around
[moved up] two [places]. [It] is above sixty [i.e., the sixties place]. 
Three times the \ya\ is turned around [moved up] two [places], above [the] sixty [place]. 
Hence when the divisor is taken of the same order [i.e., same 60'mal place],
the quotient is degrees [i.e., the
units place], 2.
Again, in its cube there are degrees [i.e., integer units], 8.
[They] are divided by sixty ``a pair of turns'' [i.e., twice: what's this in Arabic??]:
0\danda 0\danda 8. Added in the form of \vikalas\ to \vikalas,
16\danda 57.
Again,
$\marginnote{f.~4v J}$
the remainder of the first quotient 16 [is] divided by just that \ya\ 3; the quotient [is] \kalas, 5.
The quotient is put down below [i.e., to the right of] the first quotient, 2: digit 2\danda 5.
Again, the here-remaining 1\danda 57, again the cube of the two quotients: dig.\ [or deg.? \textit{a\*m}]
9\danda 2\danda 32\danda 5.  Here the cube of the first quotient  dig.\ [or deg.? \textit{a\*m}]
8 is subtracted; the remainder,dig.\ [or deg.? \textit{a\*m}]
1\danda 2\danda 32\danda 5. This in the remainder: dig.\ [or deg.? \textit{a\*m}]
1\danda 57\danda 7\danda 59\danda 8\danda 56\danda 29\danda 40. 
Added in the degrees place: 
1\danda 58\danda 10\danda 31\danda 13.
Again the portion taken with the divisor, \ya\ 3, in [the] degree [place], [1\danda] 58.
The quotient is in the form of \vikalas,
39, the remainder dig.\ [or deg.? \textit{a\*m}] 
1\danda 10\danda 31\danda 13. Again this quotient is to be set below [i.e., after] the 
previous quotient, 2\danda 5\danda 39.
Degrees etc.\ of its cube: 9\danda 11\danda 2\danda 32\danda 27\danda 43\danda 39.

\newpage 
{\s अत्र द्वितीयघनः ९ । २ । ३२ । ५ । शोधितः । शेषं कलादि ८ । २० । २७ । २७ । ४३ । ३९ । 
इदं शोषांकेषु अ १ । १० । ३१ । १ ? । योजितं । अं - - -  ४१ । २४ । १३ । १९ । 
पुनरत्र कलास्थाने भक्तं लब्धि पूर्वलव्धे स्थापिता । २ । ५ । ३९ । २६ ।
शेषं कलादि १ । १ । ४३ । २४ । १३ । १९ । पुनर्लब्धेर्धनः ९ । ११ । ८ । १४ । ३३ । १२ । २३ । २१ । ४ । ५६ । 
तृतीयधनोऽत्र शोधित । विकलावशिष्टं वि ५ । ४२ । ५ । २८ । ४४ । २१ । ४ । ५६ ।
पुनरिदं शेषांकेन योजितं क १ । २३ । २९ । ४३ । ३ । २१ । ४ । ५६ । 
पुनस्तेनैव भक्ते लब्धं २२ शेषं १ । २३ । २९ । ४२ । ३ । २१ । ४ । ५३ । 
पूर्वलब्धियोजितं  २ । ५ । ३९ । २६ । २२ । । 
अस्य घनः अं ९ । ११ । ८ । १९ । २२ । ४१ । ६ । ५२ । १५ । १ । ५३ । 
चतुर्थघनोत्र शोधेतः फोषं ५५ । ९ । २८ । ५३ । ३१ । १ । ५ । ५६ । 
इदं फोषे योजितं १ । २८ । १९ । १० । ४६ । ५२ । १५ । १ । ५३ । 
पुनर्बरभक्तः लब्धं २९ शेषं १ । १९ । १० । ४६ । ५२ । १५ । १ । ५६ । 
पुनर्लब्धिरधस्थापितः ३ । ५ । २९ । २६ । २२ । २९ । 
अस्य घनः ९  । ११ । ८ । १९ । २९ । २४ । ५८ । ५ । 
लब्धं शेषे योजितं - - - - २२ । १ । ३९ । ५० । ५२ । 
पुनरयं हर या ३ भक्तः ल ३ शेषं १ । २२ । २२ । - - - । १ । ३९ । ५० । ५२ । 
लब्धिः पूर्वलब्धेरधः स्थापितः २ । १ । ३९ । २६ । २२ । २९ । २८ । 
अस्य घनः शोधितः शेषं ६ । ८ । २५ । ४० । ८ । ५९ । 
शेषे योजितं ३८ । ३ । २७ । १९ । ५९ । ४३ । 
पुनहरभक्तः लं ३२ शे ३ । ३१  २७ । १५ । ५९ । ५३ । 
लब्धिः पूर्वलब्धेरध स्थापितः २ । ११ । ३९ । २६ । २२ । २९ । २८ । ३२ । 
अस्य घनः ९ । ११ । ८ । १९ । २९ । ८ । ५७ । २८ । २३ । ३७ । ९ । }
\newpage
Here, the second cube 9\danda 2\danda 32\danda 5 is subtracted; the remainder in \kalas etc.\ 
is 0 [?]\danda 30 27\danda 27\danda 43\danda 39. This is added to the remainder digits:
1\danda 10\danda 31\danda 13: [result] 1\danda 19\danda 1\danda 41\danda 24\danda 13\danda 19.
Again here, divided in the minutes place, quo.\ 26. [It] is set below the first quotient: 
2\danda 5\danda 39\danda 26. The remainder in \kalas\ etc.: 1\danda 1\danda 41\danda 24\danda
13\danda 19. Again, with the quotients, 2\danda 5\danda 39\danda 26, the cube: 9\danda 11\danda
8\danda 14\danda 23\danda 12\danda 23\danda 21\danda 4\danda 56.

The third cube is here subtracted, remaining in \vikalas: 

5\danda 42\danda 5\danda 28\danda 44\danda
21\danda 4\danda 56\danda 17 [?]. Again, this is added in the digits of the remainder: ka
1\danda 7\danda 23\danda 29\danda 42\danda 3\danda 21\danda 4\danda 56. Again, when divided with
just this the quotient is 22, remainder 1\danda 23\danda 29\danda 42\danda 3\danda 21\danda 4\danda
56.  Added to the first quotient: 2\danda 5\danda 39\danda 26\danda 22. Its cube:
9\danda 11\danda 8\danda 19\danda 22\danda 41\danda 6 52\danda 15\danda 16 56.

The fourth cube here is subtracted, the remainder 4\danda 49\danda 28 43\danda 31\danda 10\danda
5\danda 56. This is added to the remainder, 1\danda 28\danda 19\danda 10\danda 46\danda 52
15\danda 1\danda 56. Again, divided by the divisor, quotient 29, remainder 1\danda 19\danda 10\danda
46\danda 52 15\danda 2 [?] \danda 56. Again, the quotient is set below: 2\danda 5\danda 39\danda 26\danda
22\danda 29. Its cube: 9\danda 11\danda 8\danda 19\danda 29\danda 2\danda 42\danda 1\danda 39\danda
50\danda 52. 

Here, the fifth cube is subtracted, remainder 6\danda 21\danda 35\danda 9\danda 24\danda 48\danda
56. This is added to the remainder: 1\danda 25\danda 32\danda 22\danda 1\danda 39\danda 50\danda 52.
Again, this is divided by the divisor [?] 3, quotient 28, remainder 1\danda 32\danda 22\danda 1\danda 39\danda
50\danda 52. The quotient is set below the previous quotient, 2\danda 5\danda 39\danda 26\danda 22\danda
29 28. Its cube: 9\danda 11\danda 8\danda 19\danda 29\danda 8\danda 50\danda 27\danda 19\danda 59\danda
43. 

Here, the sixth cube is subtracted, remainder 6\danda 8\danda 25\danda 40\danda 8\danda
51. Added to the remainder: 1\danda 38\danda 30\danda 27\danda 19\danda 59\danda 43. Again, divided 
by the divisor, quotient 32,
$\marginnote{f.~5v J}$
remainder 2\danda 30\danda 27\danda 19\danda 59\danda 53. The quotient is set below the first
quotient, 2\danda 5\danda 39\danda 26\danda 22\danda 29\danda 28\danda 32. Its cube: 
9\danda 11 18 19\danda 29\danda 8\danda 57\danda 28\danda 23\danda 37\danda 1.
\newpage
{\s अत्र सप्तमघनः शोधितः शेषं ७ । १ । ३ । ३७ । १८ । 
इदं शेषे योजितं २ । ३७ । २८ । २३ । ३७ । १ । 
हरभक्तः ल ५२ शे १ । २८ । २३ । ३७ । १ । ल प्र ल २ । ५ । ३९ । २६ । २२। २९ । २८ । ३२ । १२ । 
अस्य घनः ९ । ११ । ८ । १९ । २९ । ८ । ५७ । ३९ । ४७ । ५० । १९ । 
पुनरत्राष्टमघनः शोधितः शेषं ११ । २५ । १३ । १८ । 
शेषे योजितं १ । ३९ । ४७ । ५० । १९ । 
हरभक्तः ल ३३ शे - - - - ५९ पू ल पूर्ववत् २ । ५ । ३९ । २६ । - - - - ५३ । ३३ । 
एतपर्यन्तं गृहीतं । अथात्र यावत्तावन्मानानयनार्थमुपायान्तरं आविदेन निष्काषितं । तद्यथा । 
रूपाणि यावत्तावता भाज्यानि । लब्धेर्घनः कार्याः । पुनर्घनोपि यावत्तावता भाज्यः यल्लभ्यते तं प्रथमलब्धौ योज्यं । 
पुनस्तस्य घनः कार्यः । पुन सः यावत्तावता भाज्यः यल्लब्धं तत्प्रथमलब्धौ योज्यं । पुनस्तस्य घनः कार्यः । 
एवमसकृत् । - - - - - पूर्ववद्यावत्तावता भक्ते रूपराशौ कश्चिद्यावत्तावतो भागो लब्ध
पुनस्तस्य घनं कृत्वा तद्योजनेन वास्तव घन स्यात् सन्तत्र जाता एवं मूहूं स्थिरीभूतं तदेका 
वास्तवघनरूपराशेर्यावत्तावतो भाग उपलब्धः । स एव यावत्तावन्मानमित्युपपन्नं । }
\newpage
Here the seventh cube is subtracted, remainder 7\danda 1\danda 3\danda 37\danda 18. 
This is added to the remainder, 2\danda 37\danda 28\danda 23\danda 37\danda 1.
When divided by the divisor, the quotient is 52, remainder 
1\danda 28\danda 23\danda 37\danda 1. The quotient 
is set below the first quotient: 2\danda 5\danda 39\danda 26\danda 22\danda 29\danda 28\danda 32\danda
52. Its cube: 9\danda 11\danda 8\danda 19\danda 29\danda 8\danda 57\danda 39\danda 47\danda
50\danda 1 [?]. 

Again, here the eighth cube is subtracted, remainder 11\danda 24\danda 13 [?] \danda 18. Added to the
remainder: 1 [?] \danda 39\danda 47\danda 50\danda 19. Divided by the divisor, quotient 33, 
remainder 57 [?]\danda 50\danda 19. Quo.\ as before: 2\danda [?] \danda 39\danda 26\danda 22\danda
29\danda 28\danda 32\danda 52\danda 33. Seized this far. 

Now here for the sake of the determination of the amount of the \ya, another approach is known 
[?? \textit{\=avideta}, finite verb??]. 
expounded. That is as [follows]: 

The units are divided by the \ya, the cube of the quotient is to be made. And again, the cube is
to be divided by the \ya. Whatever is the quotient, that is to be added to the first quotient. Again,
the cube of that is to be made, and so repeatedly.  

Here is the demonstration. In this case as before, when the amount of units is divided by the \ya, 
something is obtained [as?] the part [?] of the \ya. Again, having made its cube, that with addition
is fixed [? \textit{v\=astava\*m} ?]; the approximateness of the cube is produced. Thus the being
gone-out-fixed [???].  Just that from the amount of the units of its own cube, part with the \ya 
[is] conceived. That itself is obtained [as] the amount of the \ya. 
\newpage
{\s अत्रोदाहरणं । रू १६ । ४९ । - - ५६ । ८ । १६ । ६ । 
इदं यावत्तावता षष्टे द्विरूर्ध्व परिवर्तेन या ३ भक्ते ल २ । ५ । ३६ । २२ । ३९ । ४२ । ५८ । ५० । 
अस्य घनः ९ । १० । २ । ३ । ८ । ५२ । ५ । ३९ । 
पुनरयं तेनैव या ३ भक्तफलं वारद्वयं षष्ट्योर्ध्वी कृतः द्विपरिवर्ते योजिनार्थं ० । ०। २ । ३ । २९ । २ । ५७ । 
इदं प्रथमलब्धौ योजिते २ । ५ । ३९ । २६ ।  ९ । ४ । १ । ४७ । 
अस्य घनः - ११ । - । १६ । ३२ । ३० । ४८ । ९ । 
पुनरयं तेनैव भक्तः लब्धं - - - - ४२ । ४५ । ३० । ५० । 
इदं प्रथमलब्धौ योजितं जातं ५ । ३७ । २६ । २२ । १६ । २९ । ४० । 
अस्य घनः - - - - - - २८ । ३ । २३ । ५० । पुनस्तेनैव भक्तः लब्धं ० । ० । ३ । ४ । ४ । ४६ । २९ । 
प्रथमलब्धौ योजितं २ । ५ । ३९ । २६ । २२ । २९ । २८ । २९ । 
अस्य घनः ९ । ११ । -- । १९ । २९ । ८ । ५६ । ५१ पुनस्तेनैव या ३ भक्तः फ ० ।० । ३ । ३ । ४ । - - - । २९ । ५६ । 
इदं प्रथमलब्धौ योजितं २ । ५ । ३९ । २६ । २२ । २९ । २८ । ३२ । 
अस्य घनः - - - - - २९ । ८ । २७ । २८ । 
पुनस्तेनैव भक्तः लब्धं ० ।० । ३ । ३। ४२ । ३१ । २९ । ४२ । प्र ल योजितः २ । ५ । ३९ । २६ । २२ । २९ । २८ । ३२ । 
अयं स्थिरीभूतः । इयमशद्वयस्य पूर्णज्या जाता ।} 
\newpage
Here, an example: U.\ 6\danda 16  49\danda 7\danda 59\danda 8\danda 56\danda 30. 
This the \ya\ above of sixty twice
$\marginnote{f.~6r J} $
with moving back and forth [?] the \ya\ 3 when divided [???],
whatever is the quotient is the result 2\danda 5\danda 36\danda 22\danda 39\danda
42 58 \danda 50. Its cube: 9\danda 16 [?] \danda 28\danda 3\danda 8\danda 52\danda 5\danda 39.
Again, this is divided by that \ya\ 3, result 0\danda 0\danda 3\danda 3\danda 29\danda 21\danda 2\danda 
57. This is added to the first quotient: 2\danda 5\danda 39\danda 26\danda 9\danda 4\danda 0\danda 47.
Its cube, a pair of turns, divided by sixty, the result is in the form of \vikalas: 
0\danda 0\danda 9\danda 11\danda 8\danda 16 32\danda 30\danda 48\danda 9.
Again, this is divided by just that, quotient 0\danda 0\danda 3\danda 3\danda 42\danda 45 [?]
30\danda 50. This is added to the first quotient, result 5\danda 39\danda 26\danda 22\danda 28
29\danda 40. Its cube: 9\danda 11\danda 8\danda 19\danda 28\danda 36\danda 23\danda 50.
Again, divided by just that the quotient is: 0\danda 0\danda 3\danda 3\danda 42\danda 46\danda 29\danda
39. Added to the first quotient: 2\danda 5\danda 39\danda 26\danda 22\danda 29\danda 28\danda 29. 
Its cube: 

9\danda 11 8\danda 19\danda 29\danda 8\danda 56\danda 51. Again, divided by just that,
quotient 0\danda 0\danda 3\danda 3 42\danda 46\danda 29\danda 42. This is added to the first 
quotient: 2\danda 5\danda 39\danda 26\danda 22\danda 29\danda 28\danda 32. Its cube: 

9\danda 11\danda 8\danda 19\danda 29\danda 8\danda 57\danda 28. Again, divided by just that,
quotient 0\danda 0\danda 3\danda 3\danda 42\danda 46\danda 29\danda 42. 
Added to the first quotient: 2\danda 5\danda 39\danda 26\danda 22\danda 29\danda 28\danda 32.
This is being fixed. This occurs [as] the Chord of two degrees.

\newpage
{\s तत्र पूर्वोक्तमेवक्षेत्रं कार्यं । 
तत्र बचिह्नाद्वबतव अजरेखायां निष्काष्यते अथ मिर्वोलुग्वेगोक्तप्रकारेणांशद्वयस्य पूर्णज्या निश्काष्यते । 
अथ मिर्वोलुग्वेगोक्तप्रकारेणांशद्वयस्य पूर्णज्या निश्काष्यते । \todo{Why twice?}


तत्र अबहत्रिभुजे जवहत्रिभुजे हकोणः समकोणोऽस्ति । अबवर्गः अहहवर्गयोगतुल्योऽस्ति । 
पुनर्बजवर्गः जहहवयोर्वर्गयोगतुल्योऽस्ति । 
पुनरत्र अबबजौ मिथस्तुल्यौ कल्पितौ । वहमुभयोस्त्रिभुजयोरेकमेवास्ति । 
तस्मात् अहहजौ मिथः समानौ भविष्यतः । अबवर्गः वहस्य व्यासस्य च घातेन समानोऽस्ति । 
तस्माद्यद्याववर्गो व्यासेन भाज्यते तल्लब्धं बहप्रमाणं भवति ।
पुनर्यदि वहवर्गो अववर्गाच्छोध्यते तदा शेषं अहवर्गोऽवशिष्यते । }
\newpage
Now, by the method stated by Mirjolugvega [Ulugh Beg], the Chord of two degrees is expounded. 
Then just the previously stated figure is to be made. \\ 

\iffalse
\begin{center}
\includegraphics[width=2in]{6r.png}
\captionof{figure}{6r}
\end{center}
\fi 

There, having led out from point $V$ perpendicular $VH$ on line $KJ$, there in triangle
$FBH$ in triangle $JVH$ [there is?] angle $H$. [It] is a right angle. The square of $KB$ 

[Note: the scribe uses $V$ and $B$ more or less interchangeably to refer to this 
diagram letter.]
$\marginnote{f.~6v J}$
is equal to the sum of the squares of $KH$, $HV$. Again, the square of $BJ$ is equal to the
sum of the squares of $JH$, $HV$. Again, here the pair $KV$, $BJ$ are considered equal 
[to?] $VH$. Just one of both triangles is.  [?] From that, $KH$, $HJ$ will become an equal pair.
The square of $KV$ is equal to the product of $VH$ and of the diameter. From that, if the 
square of $KB$ is divided by the diameter, then the quotient becomes the amount of $BH$.
Again, if the square of $VH$ is subtracted from the square of $KV$, then the remainder
remains [as] the square of $KH$.  

\newpage
{\s अत्रांशद्वयस्य पूर्णज्यारूपा अबरेखा
यावत्तावन्मिता कल्पिता या १ अस्या वर्गः 
याव १ व्यासः १२० षष्ट्योर्ध्वीकृतः २ एकपरिवर्त्तः अनेन अबवर्गे याव १ षष्टेरेकपरिवर्त्तेन २ भक्ते लब्धं
यावद्वर्गादर्धकलालब्धाः । सा च यावद्वर्गस्य त्रिंशद्विकलात्मिका ३० ।
यावत्तावतः अंशात्मकत्वात् इदं वहप्रमाणं जातं । अस्य वर्गः यावद्वर्गवर्गस्य 
पञ्चदशप्रतिविकलाः जाताः १५ । इदं अबवर्गाच्छोधितं । 
शेषं अहवर्गः याव १ यावव ० १५ प्रतिविकला । तस्मात् अहवर्गः याव ४ यावव ${\dot{१}}$ विकला । 

अथ मिजिस्तीग्रंथस्य प्रथमध्यायस्य द्वितीयक्षेत्रे इदमुपपन्नं । अजवदघातः अजवर्गरूपः
अबजदघातस्य वजअदघातस्य च योगेन तुल्योऽस्ति । तत्र अदप्रमाणं ६ | १६ | ४९ | ७ | ६ | ५६ | ३० । 
बजं अबतुल्यं
यावन्मितमस्ति । तस्माद्वजं अदेन गुणितं सदेतावन्ति यावत्तावतानि यातानि या ६ | १६ | ४९ | ७ | ५९ | ८ | ५६ | ३० । 
अबवर्गरूप अबजदघातः याव १
तस्मादिदं याव ४ यवव १ विक अ अस्य याव १ या ६ | १६ | ४९ | ७ | ५९ | ८ | ५६ | ३० समानं जातं । 
पुनरेतौ समशोधितौ । तत्रैकपक्षे
याव ३ यावव १ विक या ६ | १६ | ४९ | ७ | ५९ | ८ | ५६ | ३० । एतौ समानौ स्तः |}
\newpage
Here the Chord of two degrees [in?] units [is] considered line $KV$, measured by the \ya. 
\ya\ [?? something erased?] its square \yava\ 1, diameter 120, divided by sixty, 2. 
One back-and-forth: with it the square of $KV$ \yava\ 1 from sixty with one back-and-forth 2
when divided, the quotient from the square of \ya\ half-\kala, and it has the form of thirty \vikalas\
of \yava\ 30. Because of the \ya\ having the form of degrees, this amount of 
$VH$ results. Its square,  of the \yavava\ fifteen like [\textit{prati}] arcminutes occur [??] 15. 
This is subtracted from the square of $KV$, the remainder is the square of $KH$: 
\yava\ 1 \yavava\ like [?] \vikalas [??] square of $KH$ is equal to a fourth part of the
square of $KJ$. From that, the square of $KJ$ \yava\ 4 \yavava\ 1 \vikalas.

Now, in the second figure of the first chapter of the book Mijast\={\i}, this is demonstrated.
The product of $KJ$, $VD$ has the form of the square of $KJ$; with
the sum of the product of $KB$, $JD$. 
$\marginnote{f.~7r J}$
and of the product of $VJ$, $KD$ it is equal.  Then the amound of $KD$ is 
6\danda 16\danda 49\danda 7\danda 59\danda 8\danda 56\danda 30.  $VJ$ is equal to 
$AV$ [? $A$?], measured by \ya.  From that, $VJ$ multiplied by $KD$ with really
that much [?] the \ya s result, \ya\ 6\danda 16\danda 49\danda 7 59\danda 8\danda 56\danda
30. 
The product of $KV$, $JD$ with the form of
the square of $KV$, \yava\ 1. From that, this \yava\ 4 \yavava\ 0, \vikalas, 
\textit{a\*m} [?]. Of it, \yava\ \ya\ 6 16\danda 49\danda 7\danda 59\danda 8\danda 56\danda 30
equal results. Again, the two equal-subtracted, then on one side \yava\ 3 \yava  [?] \textit{gha} [?]
\vikalas\ \ya\ 6\danda 16\danda 49 7\danda 59\danda 8\danda 56\danda 30. These two are equal.

\newpage
{\s अत्र समयोस् अंशौ च गृह्यते | तदा तावपि
समानौ भविष्यतः | अतः पक्षौ त्रिभक्तौ याव १ यावव २० 
या २ | ५ | ३६ | २२ | ३९ | ४२ | ५८ | ५० | 
पुनरेतौ यावत्तावतापवर्त्तितौ यथा या १
याव २० प्र वि रू २ | ५ | ३६ | २२ | ३९ | ४२ | ५८ । ५० । 
अथात्र तादृशस्येष्टांशस्य चिकीर्षास्ति यस्यांकस्य रूपराशि
स्थितांकस्य चांतरं तदंकघनस्य विंशतिं 
प्रतिविकलात्मकमन्तरं भवति । स एवांक इष्टो भवति । तस्योत्पादनेऽयमुपाय उपलब्धः । रूपाणां घनः कृतः ९ | १० | २८ | ३ | ८ | ५२ | ५ | ३९ ।
इदं विंशति प्रतिविकलाभिर्गुणितः ० | ० | ३ | ३ | २९ | २१ | २ | ५७ । 
रूपेषु योजितं २ | ५ | ३९ | २६ | ९ | ४ | १ | ४७ । अस्य घनः ९ | ११ | ८ | १६ | ३२ | ३० | ४८ | ९ । 
पुनरिदं विंशतिप्रतिविकलाभिर्गुणितं ० | ० | ३ | ३ | ४२ | ४५ | ३० | ५० । 
रूपेषु योजितं २ | ५ | ३९ | २६ | २२ | २८ | २९ | ४० । अस्य घनः 
९ | ११ | ८ | १९ | २८ |५६ | २३ | ५० पुनस्तेनैवप्र वि
२० गुणितः ० | ० | ३ | ३ | ४२ | ४६ | २९ | ३९
रू योजितं २ | ५ | ३९ | २६ | २२ | २९ | २८ | २९ । पुनर्घनः ९ | ११ | ८ | १९ | २९ | ८ | ५६ | ५१ । 
पुनस्तेनैव प्र वि २० गुणितः ० | ० | ३ | ३ | ४२ | ४६ | २९ | ४२ ।
रू योजितं २ | ५ | ३९ | २६ | २२ | २९ | २८ | ३२ ।
इदंमंशद्वयस्य पूर्णज्यारूपमिष्टांको जातः । पदास्य घनः । क्रियते विंशाति प्रतिकलाभिर्गुण्यते
रूपेषु योज्यते । तदा स एवांको भवति | }
\newpage
Here if two third parts of equals are taken then those two too will become equal.   Here the two sides
are divided by three, \yavava\ [? 0?] \ya\ \textit{ra} [?] 15\danda 32\danda 22\danda 39\danda 42\danda
[?] like \vikalas.
Again, those two reduced by \ya\ as: \ya\ 1 \yava [? 0?] un.\ 2\danda 5\danda 36\danda 22\danda 
39\danda 42\danda 58\danda 50. 

There, [there is] wish [for] the desired digit of sight [?].  The difference of whatever digit and the
digits standing in the amount of units, that digint twenty of the cube  becomes having the form like \vikalas. 
It becomes one desired digit. Its production [is?] conceived [as?] this approach: The cube of the 
units is made, 9\danda 10\danda 28\danda 3\danda 8\danda 52\danda 5\danda 39. 
This is multiplied by twenty-like-\vikalas [?], 0\danda 0\danda 3\danda 3\danda 29\danda 21\danda 2\danda
57. Added to the units: 2\danda 5\danda 39\danda 26\danda 9\danda 4\danda 1\danda 47.
Its cube: 9\danda 11\danda 8 16\danda 32\danda 30\danda 48\danda 9. Again, it is multiplied
by twenty-like-\vikalas, 0\danda 0\danda 3\danda 3\danda 42\danda 45\danda 30\danda 50.
Added to the units: 2\danda 5\danda 39\danda 26\danda 22\danda 28\danda 29\danda 40.
Its cube:
$\marginnote{f.~7v J} $
9\danda 11\danda 8\danda 19\danda 28\danda 46\danda 23\danda 50. 
Again, with just that pra-vi-20 multiplied [multiply by 20 and shift 3 places to the right, 
equivalent of dividing by $3,0,0$], 0\danda 0\danda 3\danda 3\danda 42\danda 46\danda
29\danda 39. Added to the units: 2\danda 5\danda 39 26\danda 22\danda 29\danda 
28\danda 29\danda. Again, the cube: 9\danda 11\danda 8\danda 19\danda 29\danda 8\danda
56 51. Again, with just that pra-vi-20 multiplied, 0\danda 0\danda 3\danda 3\danda 42\danda
46 29\danda 42. Added to the units: 2\danda 5\danda 39\danda 26\danda 22\danda 29\danda
28 32. This [is?] the Chord of two degrees, the unit [or number? having] the desired digit 
[or digits?] results. [I.e., this is the exact value.] 
When its cube is made, is multiplied with twenty-\pratikalas, is added [misspelled] to the units, then it
becomes the same digit. [I.e., further iteration doesn't change it.]

\newpage
{\s अथ मिर्जोलुग्वेगस्य द्वितीयप्रकारः । तत्र अबजदचापं षडंशानां कल्प्यं ।
यस्य वृत्तस्येदं चापं तद्वृत्तकेंद्रं तं कल्पनीयं । पुनः प्रत्येकं अचापं वजचापं जदचापमशद्वयं कल्पितं ।
अबअजअदवजजदपूर्णज्याः संयोज्याः । पुनः 
अवपूर्णज्या अजपूर्णज्यो चाह स्वबचिह्नेष्वर्धं कार्यं । पुनस्तहरेखातझतवरेखाः संयोज्याः । 
एताः रेखाः प्रत्येकं तासु पूर्णज्यासु लम्बरूपाः भविष्यन्ति । पुनः अतरेखाव्यासार्धं संयोज्यं । 
पुनः कचिह्नेर्धितं कार्यं । पुनः कचिह्नं केन्द्रं कृत्वा
कअ व्यासार्द्धेन वृत्तार्धं कार्यं । तत्र कहतकोणअझतकोणअबतकोणाः समकोणाः सन्ति । 
तदा वृत्तार्धमेतत् हझवचिह्नेषु गमिष्यति । पुनः हझझवबतहतरेखाः संयोज्याः । हचिह्नात् हललव अझपूर्णज्या सुनिष्काश्यः । 
तदा अझपूर्णज्या लचिह्नेष्वर्धिता भविष्यति । तत्र झवरेखया अजभुजअदभुजयोरर्धस्थाने कात्तितं । 
तदा झवरेखाजदरेखायाः समानान्तरा भविष्यति । तस्मात् अजदत्रिभुजअझवत्रिभुजे मिथः सजातीये भविष्यन्तः । अवं अदस्यार्धमस्ति ।  
झवं जदस्यार्धं भविष्यति । 
पुनरनेनैव प्रकारेण हझरेखावजस्यार्धं भविष्यति । अहं अवस्यार्धमत्स्येव । तस्मात् अहहझझवपूर्णज्याः मिथः समानाः भविष्यन्ति । }

\newpage
Now, the second method of Mirjolugvega. Then arc $KV1JD$ [is] considered [?] of six degrees.
Of whatever circle this arc [is part?], the center of that circle is to be considered $T$. Again, 
each one of arc $KV1$, arc $V1J$, arc $JD$ [is] considered two degrees. The Chords
of $TV1$ [should be $KV1$] $KJ$ $KD$ $V1J$ $V1D$ $JD$ are joined. Again, of the Chord of $KV1$,
the Chord of $KJ$, and the Chord of $KD$ in the points $H$, $Jh$, $V2$ half  is to be made.
Again, line $TH$, line $TJh$, line $TV2$ are  joined. Each one [of] those lines 
will become in the form of a perpendicular on those Chords. 
Again, line $KT$ is joined with [the form of?] the radius; again, that is to be made halved
in point $A$. Again, having made point $A$ the center, a circle is to be made with
radius $AK$. Then angle $KHT$, angle $KJhT$, angle $KV2T$ being right angles, 
then  that half circle will go through points $H$, $Jh$, $V2$. 
Again, lines $HJh$, $JhV2$, $V2T$, $HT$ are joined [Note: the latter two were already
drawn]. From 
point $H$ perpendicular $HL$ is extended on the Chords [? why plural?] $KJh$. 
$\marginnote{f.~8r J}$

Then the Chord of $KJh$ will become halved at points [? why plural] $L$. There with line $JhV2$ 
in the half-place [?] of side $KJ$ and side $KD$ is made [i.e., its endpoints Jh, V2 bisect those 
two segments]. Then line $JhV$ will
become differently-equal [i.e., a corresponding side in similar triangles]
to line $JD$. From that, triangle $KJD$ and triangle
$KJhV2$ pairwise will become together-born [i.e., similar]. 
$KV$ is half of $KD$, [so] $JhV2$ will be half of $JD$.
Again, in just this way line $HJh$ will become half of $V1 J$. $KH$ is half of just that $KV1$.
From that, the Chords $KH$ $HJh$ $JhV2$ will become pairwise equal. 
\newpage
{\s पुनः कचिह्नात् अललम्वः अझपूर्णज्यायां निष्काश्यः । अयं लंम्बो अझरेखां लचिह्नेऽर्द्धितं करिष्यति । 
पुनरयं वर्धितः सन् हचिह्ने लगिष्यति । तस्मात अबमंशत्रितयज्या एतावत्यस्ति ३ | ८ | २४ | ३३ | ५९ | ३४ | २८ | १५ | 

यदि अहवर्गः याव १ अतरेखया ६० भाज्यते । तदा लब्धिरेकाकला यावद्वर्गस्य भविष्यति । इदं हलप्रमाणमस्ति । 
यथा पूर्वमुपपन्नं । अस्यवर्गः यावद्वर्गवर्गस्यैका  
विकला भविष्यति । अयं अहवर्गे शोधितः । शेषं अलवर्गो भविष्यति । स च याव १ यावव १ विकला । अझवर्गस्य चतुर्थांश ४ 
अलवर्गोऽस्ति । तस्मात् अझवर्गः याव ४ यावव ४ विकला । पुनः अझहवघातरूप
अझवर्गः अझवहघातरूपवहवर्गस्य हझअवघातस्य च योगेन तुल्योऽस्ति ।
अहवर्गश्चैतावान् याव १ हझअवघातश्च ३ | ८ | २४ | ३३ | ५९ | ३४ | २८ | १५ |}  
\newpage
From point $A$ perpendicular $AL$ onto Chord $KJh$ is to be extended; this perpendicular 
will make line $KJh$ halved at point $L$. Or again, this [line] being halved [??] will fall on point $H$.
From that, the Sine of one degree $KH$ is to be considered the measure of the \ya. 

Again, $KV2$ [is] three degrees, its sine is so much: 3 8\danda 24\danda 33\danda 59\danda
34\danda 28\danda 15. \\ 

\iffalse 
\begin{center}
\includegraphics[width=1.5in]{8r.png}
\captionof{figure}{8r}
\end{center}
\fi 

If the square of $KH$ \yava\ is divided by line $KT$ [is this actually 
\textit{katar-rekh\=a} for Ar.\ \textit{qa\*tar}?] 60, then the quotient  will become
one \kala\ of \yava.   This is the $HL$-amount, as previously obtained. Its square
will become of the \yavava\ one
$\marginnote{f.~8v J}$
\vikala. This is subtracted from the square of $KH$, the remainder will become the
square of $KL$.  And it \yava\ 1  \yavava\ [1 ?] \vikalas. 
[$HL = \frac{x^2}{60} = \frac{KH^2}{KT}$, $HL^2 = \frac{x^4}{3600}$,
$KH^2 - HL^2 = KL^2$.]
A fourth part 4 of the square of $KJh$  is the square of $KL$. From that, 
the square of $KJh$ \yava\ 4 \yavava\ 4 \vikalas. 

Again, the number of the product of $KJh$, $HV2$ [is] the square of $KJh$; like [??]
the number of the product of $KJh$, $HV$. [It?] is equal with the sum of the square [of
\ya?] and of the
product of $HJh$, $KV2$. And the square of $KH$ is that much, \yava\ 1. And the
product of $HJh$, $KV2$ 3\danda 8\danda 24\danda 33\danda
59 34\danda 28\danda 25. 


\newpage
{\s अयं द्वितीयप्रकारेणाझवर्गः सिद्धः | एतौ समाविति शोधनार्थं न्यासः \\ 
याव ४ यावव $\dot{४}$ विकला \\
याव १ या ३ | ८ | २४ | ३३ | ५९ | ३४ | २८ | १५ | अत्र यावनीय 
संप्रदायेन शोधने कृते प्रथमपक्षे याव ३ द्वितीयपक्षे यावव ४ विकला या ३ | ८ | २४ |
३३ | ५९ | ३४ | २८ | १५ | अनयो त्रंशावपि समानौ ।
तस्मात् त्रिभिरपवर्त्तितौ याव १ यावव १ वि २ प्र वि १ । २ । ४८ | ११ |
१९ | ४१ । २१ | २५ । अवशिष्टावेतावपि समानौ । 
पुनरेतौ यावत्तावतापवर्त्तितौ जातौ या १ याघ १ वि २ प्र रू १ । २ । ४८ | ११ । १९ | ५१ | २९ | २५ । 
यद्यत्रतादृशोंऽको लभ्यते यस्याङ्कस्य पुनः रूपस्य चान्तरं अकधनस्यैककला विंशतिप्रतिविकला तुल्यमन्तरं भवेत् ।
सचानेन प्रकारेण निष्कांशितः । तत्र रूपाणां घनः अंशादि १ | ४८ | ४८ | ३० | २३ | ३६ | ३० | ४२ | १८ । 
अयमेकया कलया विंशतिप्रतिविकलाभिश्च गुणितः ० | ० | १ | ३१ | ४४ | ४० | ३१ | २८ | ४१ | 
अंशकलास्थानयोरभावात् स्थानद्वये शून्यं निवेशितं । 
रूपेषु संयोज्य जातं १ | २ | ४९ | ४३ | ४ | ३२ | ० | ५३ | ४१ । 
अस्य घनः १ | ८ | ५३ | ३२ | ४ | ३ | ५० | ५९ | १४ | ५७ । 
पुनरयमेतेन १ २० गुणितः ० | ० | १ | ३१ | ५१ | २२ | ४५ | २५ | ८ । 
रूपेषु योजितं १ | २ | ४९ | ४३ | ११ | १४ | १४ | ५० | ८ । 
अस्य घनः १ | ८ | ५३ | ३६ | २६ | ७ | १८ | ० | २२ | १० । 
पुनरयं तेनैव १ २० गुणितः ० | ० | १ | ३१ | ५१ | २३ | १४ | ४९ | २० । 
रूपेषु संयोज्य १ | २ | ४९ | ४३ | ११ | १४ | ४४ | १६ | ३० । 
अस्य घनः १ | ८ | ५३ | ३२ | २६ | ८ | ३७ | ५ | ३४ । 
पुनस्तेनैव १ २० गुणितः ० | ० | १ | ३१ | ५१ | २३ | १४ | ५१ | २९ ।
रूपेषु संयोज्यः १ | २ | ४९ | ४३ | ११ | १४ | ४४ | १६ | २६ । 
अयमिष्टांको जात । अत इयमेकांशज्या सिद्धा । कुतः । यतोऽस्य घने अनेन १ २० विक 
गुण्यते रूपेषु योज्यते तदायमेव भवति । तस्मादयमेवै त्रविकांशज्या जाता । इदमेवेष्टं |}
\newpage
Here with this approach, when the subtraction is done in the first side:
\yava\ 3 \yavava\ 4 \vikalas\ [=] \ya\ 3\danda 8\danda 24\danda 33\danda 59\danda 34\danda
28\danda 5. And of those two, two one-third parts are equal [?]. 
From that, the two are reduced by three. Those two remainders are equal:
\yava\ 1 [=?] \yavava\ 1 [?] \vikalas [?], un.\ 1\danda 2\danda 48\danda 11\danda 19
51\danda 29\danda 25.
Again, those two are reduced by the \ya, two sides result: 
\ya\ 1 \vikalas [?] un.\  1\danda 2\danda 48\danda 11\danda 19
51\danda 29\danda 25 [=?] \ya-cube $\frac{1}{2}$ \prativikalas.

If here \textit{t\=adi} [??] what degree is obtained?  [???] 
Again, the difference of whatever digit and of the units; one \kala\ of the cube of the digits
should be equal to twenty \prativikalas. [?]

And that by this method is laid out: there the cube of the units beginning with degrees [is]
1\danda 8\danda 48\danda 30\danda 23\danda 36\danda 30 42\danda 18.
This with one \kala
$\marginnote{f.~9r J}$
and with twenty \prativikalas\ is multiplied: 

0\danda 0\danda 1\danda 31\danda 44\danda
40\danda 3\danda 28\danda 41.
Because of the nonexistence of the two places of degrees and \kalas, zero is entered in 
[those] two places. Having added [the result?] to the units, result:
1\danda 2\danda 49\danda 43 4\danda 32\danda 0\danda 52\danda 41. Its cube:
1\danda 8\danda 53\danda 32\danda 4 3\danda 50\danda 59\danda 14\danda 57.
Again, this  is multiplied by that $\frac{1}{20}$: 


0\danda 0\danda 1\danda 31\danda 51\danda 22\danda 45\danda 25\danda 8.
Added to the units:
1\danda 2\danda 49\danda 43\danda 11\danda 14\danda 14\danda 50\danda 8.
Its cube: 1\danda 8\danda 53\danda 36\danda 26\danda 7\danda 18\danda 0\danda
22\danda 10.
Again, this is multiplied by just that $\frac{1}{20}$: 
0\danda 0\danda 1\danda 31\danda 51\danda 23\danda 14 49\danda 20.
Added to the units: 1\danda 2\danda 49\danda 43\danda 11\danda 14 44\danda 16\danda 30.
Its cube: 1\danda 8\danda 53\danda 32\danda 26\danda 8\danda 37 5\danda 34.
Again, this is multiplied by just that $\frac{1}{20}$: 
0\danda 0\danda 1\danda 31\danda 51\danda 23 14\danda 51\danda 29. 
Added to the units: 1\danda 2\danda 49\danda 43\danda 11 14\danda 44\danda 16\danda 26.
This results [as] the desired digit [desired sequence of digits, up to desired digit??]. 
Hence this Sine of one degree is attained.  Wherefore?  Because when its cube 
is multiplied by this $\frac{1}{20} \frac{\hbox{vi}}{\hbox{pra}}$ [and] added to the units,
then [it] becomes just this. From that, this results [as] the Sine of one degree; this is just
[what was] desired. 



\newpage
{\s अथात्र मिर्योलुग्वेगेन
यद्द्वितीयप्रकारेण क्षेत्रदःर्शितं । तत्र बहुरेखा संयोगः कृतः । 
अथात्र यथाल्परेखाभिरेवसिध्यति तथा पतितं महमद आविदसंज्ञः । 
तत्र तावदेव अबजदचापं पूर्वोक्ताः एव । पूर्णज्याः
संयोज्याः । अवपूर्णज्यार्धरूपां अहमेकांशस्य जीवास्ति । 
सा यावन्मिता कल्पिता या १ अवरेखा या ३ एव । वजं  या २ । जदं या २। यत एताः मिथः
समानाः सन्ति । पुनः अदरेखा ६ | १६ | ४९ | ७ | ५९ | ८ | १६ । अनया वजं या २
गुणितं अवजदघातयुतं जातं याव ४ या १२ | ३३ | ३८ | १५ | ५८ | १६ | ३३ | ० । अयं
जातः अजवर्गः |}
\newpage

Now here, [in] the figure shown by Mirjolugvega with the second method, there 
the contact of a \textit{vadgu}-line [???] is made. [?]  Now here, as [follows]:  
With \emph{small} lines 
\marginnote{f.~9v J}
it is effected, in that way by those known as \textit{yatitamavid} [``the knowers of
\textit{yatitama}??].  There just that much arc $VKVJD$ [sic], [by] just the previous
statement, the Chord is to be connected. $KH$ has the form of half Chord $KV$, 
[it] is the Sine of one degree. It is considered the measure of the \ya, \ya\ 1.
Line $KV$ \ya\ 2. In the same way $VJ$ \ya\ 2. $JD$ \ya\ 2. Since they are 
pairwise [?] equal, again line $KD$
6\danda 16\danda 49\danda 7\danda 59\danda 8\danda 16. 
With that $VJ$ \ya\ 2 [? illegible] is multiplied. Added to the product of
$KV$, $JD$, result \yava\ 4, \ya\ 12\danda 33\danda 38\danda 15\danda 58
16\danda 33\danda 0. This result is the square of $KJ$. \\

\iffalse 
\begin{center}
\includegraphics[width=2in]{9v.png}
\captionof{figure}{9v}
\end{center}
\fi 


\newpage
{\s पुनः प्रकारांतरेणाजवर्गः ५ साध्यते | अववर्गः याव ४
अयं व्यासेन २ षष्टेरेकोर्ध्वपरिवर्त्तेन भक्तः लब्धं
यावद्वर्गस्य कलाद्वयं २ इपमंशद्वयस्योत्क्रमज्या जाता । अस्याः वर्गः यावव ४
विकला अबवर्गे शोधितं शेषं याव ४ यावव $\dot{४}$ विकला अयमंशद्वयज्यावर्गो
जातः । अयं चतुर्गुणितः याव १६ यावव $\dot{१६}$ वि अयं
सिद्धः । कजवर्गः । एतौ पक्षौ समौ । समशोधनार्थं न्यासः याव ४ या १२ |
३३ | ३८ | १५ | ५८ | १६ | ३३ । ० । 
शोधिते शे याव १२  यावव १६  वि या १२ । ३३ । ३८ । १५ । ५८ । १६ । ३३ ।  
पुनर्द्वादश १२ यावत्तावता ४ पवर्तितः या १ । ११ । १९ । ५१ । २९ । २५ ।
इदमेव मिर्योलुग्वेगेन निष्काशितं बहुप्रयाशेन
या १ याघ १ २० वि रू १ | २ | ४८ प्र वि }
\newpage

Again, by another method: The square of $KJ$ is to be achieved. The square of $KV$
[is?] \yava\ 4. This is divided by the diameter 2 [which is] reduced one-above from sixty [?].
The quotient is \kalas\ of \yava, 2. This is known as the Versine of two degrees. 
Its square \yavava\ 4 \vikalas\ is subtracted from the square of $KV$. The remainder, 
\yava\ 4, \yavava\ 4 \vikalas.  This results [as] the square of the Sine of two degrees.
This 
\marginnote{f.~10r J}
multiplied by four, \yava\ 16 \yavava\ 16 \vikalas. This is the attained square of $KJ$. 
These two sides [are] equal; for the purpose of equals-subtraction, the statement:
\yava\ 4 \ya\ 12\danda 33\danda 38\danda 15\danda 58\danda 16\danda 33 
[=]
\yava\ 16 \yavava\ 16 \vikalas. When subtracted, the remainder again \textit{dv\=ada} [??]:
\yava\ 12 [=] \yavava 16 \ya\ 12\danda 33\danda 38\danda 15\danda 58\danda 16\danda 33.
Reduced by the \ya: 
\ya\ 1 [=] \yava\ $\frac{1}{2}$ \vikalas, un.\ 1\danda 2\danda 48\danda 11\danda 19\danda
51\danda 29 25. 
This is expounded by Mirjolugvega with the \textit{vadgupray\=a\'sa} [???]. 

\newpage
{\s यद्वा या ६० याघ ८ रू ५६ । अत्र याव भवति ख २ । 
अथ हि आनतिखानस्य प्रकारो लिख्यते । तत्र समशोधने याव १२ यावव १६ वि या १२ । ३३ । ३८ । १५ । ५८ । १६ । ३३ । 
एतौ या ४ पवर्तितौ या ३ याघ ४ वि रू ३ । ८ । २४ । ३३ । ५९ । ३४ । २८ । १५ । 
यद्यस्य रूपस्य घनो घनमूलं ग्राह्यते । परं च तादृशं घनमूलं ग्राह्यं कस्य वर्गः विकला चतुष्टयेन गुणितः याव ८ ? सौ ? 
तितशेषं उत्पादितां । केन गुणितः रूपराशि समो भवेत् \footnote{Crossed out शेषं केनाङ्केन गुणितः सन् रूपराशि तुल्यो भवति ।} 
एवेष्टाङ्कोऽस्ति । अत्रोपपत्तिः । इहैकपक्षे यावत्ताव अयं द्वितीयपक्षे यावद्घनविकलाशच । 
तस्रः रूपराशिश्च । तत्र प्रथमपक्षे राशिद्वये योगोस्त्यनूमूयते । यतो द्वितीयपक्षे राशिद्वयमस्त्यतो प्रथमपक्षेपि राशिद्वयहपेक्षितमस्य समत्वात् । 
तत्रैको राश्यंकस्तदन्तर्गतस्तादृशोऽस्ति । य अपेक्षिताङ्कवर्गस्य विकला चतुष्टयस्य च घाततुल्योऽस्ति । 
द्वितीयखण्डं तादृशास्ति । येन प्रथमाङ्कगुणितः सन् रूपराशितुल्यो भवति । 
अथ तदुत्पादनप्रकारः । तत्र रूपाणि स्थाप्यानि । पुनर्यावनीयसंप्रदायेन चिह्नानि कृत्वा मूलं ग्राह्यं । 
यथा प्रथमं रूपाणि स्थापितानि । $\dot{३} । ८ । २४ । \dot{३३} । ५९ । ३४ । \dot{२८} । १५ ।$ 
पुनः खण्डद्वयं कोष्टाकानकृतं । अत्रैकाङ्कघन १ उत्पादितः । स च तृतीयपङ्क्तौ स्थापितः । 
अस्य वर्गः १ विकला चतुष्टयेन गुणितः ४ द्वितीयपङ्क्तधो लिखितः । 
अयं यावत्तावतांशत्रये या ३ शोधितः शेषं २ । ५९ । ५६ । 
पुन लब्ध्या गुणितः रूपेषु शोधनार्थं प्रथमपन्तिसमो लिखितः २ । ५९ । ५६ । 
शोधितेऽवषिष्टं ८ । २८ । शेषं पूर्वस्थान्येव । अथ लब्धिः १ तृतीयपंक्तेस्थलब्धौ १ योजितः २ तृतीयपंक्तिसमो लिखितः । 
पुनरयं २ प्रथमलब्ध्या १ गुणितः २ विकलाभिः ४ गुणितश्च  वि ८ द्वितीयपंक्तिस्थविकलास्थानं समो न्यस्तः ८ । 
शोधितेऽवशिष्टमेककोष्टकमुत्तार्य लिखितं २ । ५९ । ४८ । पुन प्रथमलब्धिः १ निजाधस्था २ कि युता ३ इयमेककोष्टकमुत्तीर्य लिखितः ३ । 
पुनः प्रथमपंक्तौ द्वितीयचिह्ने ३३ एकांको उत्पादितः । स च तस्योपरिस्थः । तत्सन्मुखे तृतीयपंक्तौ च २ । 
पुनर्लब्धि २ तृतीयपंक्तिस्थिताङ्काभ्यां ३ । २ । गुणिता ६ । ४ । विकलाभि ४ र्गुणिताश्च २४ । १६ । 
एताः कला २ नां धातत्वा प्र विकलाः - - - जाताः । द्वितीयपंक्त्यां शोधनार्थं निजस्थाने स्थापिता । 
शोधितेऽवशिष्टं २ । ५९ । ४७ । ३५ । ४४ । पुनः शेषमिष्टलब्ध्या २ कला गुणितः ५ । ५९ । ३५ । ११ । २८ । 
प्रथमपंक्त्यधो यथास्थाने लिखितं । शोधितेऽवशिष्टं २ । २८ । ५८ । ४८ । ६ । 
पुनर्लब्धिः अधस्थिताङ्के २ युक्ता ४३ यं तदथोलिखिता ४ । पुनः पूर्वलब्धिः २ अनेनाङ्केन ३ । ४ । 
गुणिता विकलाभिश्च ४ गुणिता २४ । ३२ । एता प्रतिविकलाः स्वस्थाने द्वि प पङ्क्ता स्थापिताः । 
शोधिते शिष्टं २ । ५९ । ४७ । ११ । १२ । इदमेककोष्टकमुत्तार्य द्वितीयपंक्तौ लिखितं । 
पुनर्लब्धिः २ तृतीयपंक्तिस्थ कलासु ४ योजिताः ३ । ६ । इदं एककोष्टकमुत्तार्य तदधः ३ । २ । 
लिखितं । पुनस्तृतीयचिह्ने २८ शषाङ्क उत्पादितः । स च ४९ तदुपरिस्थः । 
तृतीयपंक्तौ तदधस्थश्च । एवमधोङ्कः ३ । ६ । ४९ । जाता । पुनर्लब्ध्या गुणिताः विकलाभि ४ र्गुणितश्च १० । १० । १६ । ४ । 
प्रतिविकलैताः जाताः । इदं द्वितीयपंक्त्यौ शोधतार्थं न्यस्तं । शोधिते शेषं २ । ५९ । ४७ । १ । ४३ । ५६ । 
पुनरिदं लब्ध्या ४९ गुणितः २ । २६ । ४९ । २३ । ५० । २४ । ५२ । ४४ । 
रूपेषु शोधिते शिष्टं २ । ९ । २४ । १६ । ३ । २२ । १६ । अस्य लब्धेः ४९ निजाधस्थतृतीयपंक्तिलब्धौ ४९ योजितः जातः ३ । ७ । ३८ । 
पुनरिदं लब्ध्या गुणितः विकला ४ भिश्च गुणितः १० । १२ । १६ । ८ । इदं द्वितीयपंक्त्यां न्यस्अं । 




पुनस्तत्र शोधिते शिष्टमेककोष्टकमुत्सार्य लिखितं २ । ५९ । ४६ । ५० । ४८ । ४७ । ४८ । 
अत्र लब्धिः ४९ तृतीयपंक्तिस्थ्ङ्के ३८ योजिता ३ । ८ । २७ । एककोष्टमुत्तार्य लिखिता । 
पुनश्चतुर्थाङ्के चिह्ने अङ्क उत्पादितः ४३ तदुपरि लिखितः तृतीयपंक्त्यधश्च । 
एवं अधस्याङ्कः ३ । ८ । २७ । ४३ । जातः । अयं लब्ध्या ४३ गुणितः विकलाभि ४ र्गुणितश्च । ९ । ० । १६ । 
द्वितीयपंक्त्यां शोधितः २ । ५९ । ४६ । ५० । ३९ । ४७ । ३२ । 
पुनरिदं लब्ध्या ४३ गुणितः २ । ८ । ५० । ३४ । १८ । ३७ । ४ । १७ । 
प्रथमपंक्तिस्थरूपेषु शोधितः शे ३३ । ४१ । ४४ । ५७ । ११ । ४३ । 
पुनर्ल. ४३ तृतीयपंक्त्यधस्थलब्धौ योजिता ३ । ८ । २८ । २६ । 
इदं लब्ध्यां गुणितं विकलाभिश्च  गुणितं ९ । ० । १७ । इदं शोधितं शेषं २ । ५९ । ४६ । ५० । ३० । ४७ । 
इदमेककोष्टमुत्तार्य लिखितं । पुनर्लब्धि ४३ तृतीयपंक्त्यधस्याङ्के २६ युक्ता जा. ३ । ८ । २९ । ?९ । 
अयमेककोष्टमुत्तार्य लिखितः । पुनः पञ्चमचिह्ने. इष्ट्ङ्क उत्पादितः १७ । 
अयं स्वोपरि । तथा तृतीयपंक्तौ च लिखितः ३ । ८ । २९ । ९ । ११ । 
लब्ध्या ११ गुणितः विकला ४ र्गुणितश्च २ । शोधिते शिष्टं २ । ५९ । ४६ । ५० । ३० । ४५ । 
पुनर्लब्ध्या १ । गुणितः ३२ । ५७ । ३५ । १५ । ३८ । १५ । रूपेषु शोधितं शे ४४ । ९ । ३५ । ३३ । ४९ । 
पुनर्लब्धि तृतीयस्य ११ योजिता ३ । ८ । २९ । ९ । २२ । लब्ध्या ११ गुणितः विक ४ गुणितश्चः २ शोधिते शिष्टं एककोष्टमुत्तार्य लिखितं २ । ५९ । ४६ । ५० । ३१ । 
अयमर्धाभ्यधिकः । पुनर्लब्धिः तृतीयस्य २२ योजिताः ३ । ८ । २९ । ९ । ३३ । 
अयमप्येकाङ्कमुत्तार्य लिखितः । पुनः षष्टचिह्ने ० श्व्यङ्क उत्पादितः १४ । तदुपरि तथा तृतीयपंक्त्यौ च लिखि ३ । ८ । २९ । ९ । ३३ । १४ । 
अयं लब्ध्या १४ गुणितः विकलाभिश्च ० । ० । \footnote{Crossed out तत्स्थानपर्यतपंक्तोन समागमः प्र तो ?न्यं} शिधितः शे २ । ५९ । ४६ । ५० । ३७ । 
लब्ध्या गुणितः ४१ । ५६ । ५५ । ४७ । १४ । इदं रूपे शो २ । १२ । ३९ । ४६ । ३५ । एवमग्रेपि ।}   










\end{document}
