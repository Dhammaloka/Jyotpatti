\documentclass[11pt,a5paper]{book}

%%%Font related packages
\usepackage[no-math]{fontspec}
\defaultfontfeatures{Ligatures=TeX} 
%\setmainfont{Linux Libertine O}
\newfontfamily\s[Scale=0.92, Script=Devanagari]{Shobhika-Regular}
\usepackage{color, dtk-logos}
 
 
 %%%Formatting related packages
\usepackage{graphicx}
\usepackage[hang,flushmargin]{footmisc} % For no indentation of footnotes.
\usepackage{array}
\usepackage{tabularx, multicol, vwcol}
\usepackage{fullpage}
\usepackage{marginnote}
\usepackage{bm} %making accents thicker

%Math related packages
\usepackage{amsmath} % āmerican Mathematical Society packages for metamathematical symbols.
%āpril 2017 āditya says install libertine math font package specially from website.

%Page formatting commands

%\newdimen\stdbaseline
%\stdbaseline = 13pt 
%\baselineskip = 2\baselineskip
\parskip = \baselineskip
\parindent = 0pt


%%%Critical editing packages
\usepackage[series={A,B,C,D},noend,noeledsec,nofamiliar,noledgroup]{reledmac}
\Xarrangement[B]{twocol}
\Xarrangement[C]{threecol}
\Xarrangement[D]{paragraph}



%% MāCROS for Diacriticals, symbols, math and text features

\def\elp{$\ldots\,$}
\def\degrees{$^\circ$}
\def\degree{$^\circ$}
\def\signs{$^s$}
\def\Sin{\mathop{\rm Sin}\nolimits}
\def\Cos{\mathop{\rm Cos}\nolimits}
\def\Versin{\mathop{\rm Vers}\nolimits}
\def\Coversin{\mathop{\rm Coversin}\nolimits}
\def\Crd{\mathop{\rm Crd}\nolimits}
\def\crd{\rm{crd}}
\newcommand\myatop[2]{\genfrac{}{}{0pt}{}{#1}{#2}}
%%%%%%%%%%%%%%%%%%%%%%%%

%% MāCROS for Sanskrit Names

\def\Ganesa{Ga\-\*ne\-\'sa}
\def\Bhaskara{Bh\=a\-ska\-ra}
%
\def\KKT{\textit{Kara\-\*na\-kut\=u\-hala\-\*t\={\i}\-k\=a}}
\def\KKe{\textit{Kara\-\*na\-ke\-\'sari}}
\def\KheSi{\textit{Khe\-cara\-siddhi}}
\def\JCU{\textit{Jy\=a\-c\=a\-potpa\-tti}}
\def\PhCC{\textit{Phira\.ngi\-candra\-cchedyayopa\-yogika}}
\def\Mahadevi{\textit{Mah\=a\-dev\={\i}}}
\def\RG{\textit{Rekh\=a\-ga\*ni\-ta}}
\def\Lil{\textit{L\={\i}l\=a\-va\-t\={\i}}}
%
\def\kalpa{\textit{ka\-lpa}}
\def\kala{\textit{ka\-l\=a}}
\def\kalas{\textit{ka\-l\=as}}
\def\ghatika{\textit{gha\*tik\=a}}
\def\jyotisa{\textit{jyo\-ti\-\*sa}}
\def\pratikalas{\textit{prati\-ka\-l\=as}}
\def\prativikalas{\textit{prati\-vi\-ka\-l\=as}}
\def\bijaganita{\textit{b\={\i}ja\-ga\*ni\-ta}}
\def\ya{\textit{y\=a}}
\def\yava{\textit{y\=ava}}
\def\yagha{\textit{y\=agha}}
\def\yavava{\textit{y\=avava}}
\def\rasi{\textit{rāśi}}
\def\ru{\textit{r\=upa}}
\def\vikala{\textit{vi\-ka\-l\=a}}
\def\vikalas{\textit{vi\-ka\-l\=as}}
%
\def\zij{\textit{z\={\i}j}}
\def\yavaniya{\textit{y\=avan\=iya}}
\def\bijaganita{\textit{b\=ijaga\*nita}}
%
\def\opcit{\textit{op.\ cit.}}
\def\loccit{\textit{loc.\ cit.}}
\def\elp{$\ldots$}
\def\danda{$|$}
\def\nl{\hfill\break}
%
\def\sri{\textit{\'sr\={\i}}}
%

% \newcommand{\JOne}[1]{\colchunk{{#1}}}
% \newcommand{\JTwo}[1]{\colchunk{{#1}%
%    }\colplacechunks}

%=============================================================%



\begin{document}

{\s  
श्रीगणेशाय नमः | \marginnote{f. 1r $J_2$}
अथैकांशजीवाविषये
उलुग्वेगीजीकस्य \\
शरहविर्जन्दीस्य\footnote{{\s @विर्जंदीस्थ }${\rm J_1}$}
व्याख्या लिख्यते |
तत्रैकांशजी\-वान\-यने नव्यतरं \\प्रकारद्वयमस्ति |
एकं यमशैदकाशीसंज्ञेन कृतम् |
द्वितीयं मिर्योलुग्वेगेन कृतम् |
परं चोलुग्वेगस्य यमसैदकाशीवेधप्रक्रियायां
साहाय्यकार्यस्ति\footnote{{\s सा हाथकार्यस्ति} $J$} |
तत्र प्रथमप्रकारः कथ्यते | सा\footnote{{\s सं} $J$} यथा |
अबजदं षडंश ६ चापं कल्पितम् |
पुनरस्य समानं भागत्रयं बचिह्नजचिह्नयोः कृतम् ।
तत्र अबमजमदं बजं बदं जदं चैताः पूर्णज्यारेखाः
योज्याः |
अथात्र पूर्वोक्तप्रकारेणांशत्रयस्य\footnote{{\s अथात्रपूर्वाक्त@}${\rm J_1}$}
जीवा ज्ञातास्ति | सा चेयं ३ | ८ | २४ | ३३ | ५९ | ३४ | २८ | ५४ | ५० |
इयं द्विगुणा ६ | १६ | ४९ | ७ | ५९ | ८ | ५६ | २९ | ४० | जाता अदपूर्णज्या\footnote{{\s मदपर्णज्या}${\rm J}$}|

अत्रांशद्वयस्याबपूर्णज्या ज्ञानमिष्टमस्ति |
तत्र मिजिस्तीग्रन्थस्य \\
प्रथमाध्यायस्थद्वितीयक्षेत्रे\footnote{{\s क्षेत्र} $J$} 
इदमुपपन्नम् \footnote{{\s @द्वितीयक्षेत्रयिदमुप@ }${\rm J_1}$}|
यच्चतुर्भुजवृत्तान्तः पतति \\
तत्सन्मुखस्थभुजानां\footnote{{\s तत्संन्मुख@ } ${\rm J_1}$}
घातयोगस्तच्चतुर्भुजान्तः पतितकर्णद्वयघातसमो भवति |
पुनस्तत्रैव प्रथमाध्यायस्य
चतुर्थक्षेत्रे\footnote{{\s क्षेत्र} $J$} इदमुपपन्नम् \footnote{{\s चतुर्थक्षेत्रे इदमुप@ }${\rm J_1}$}|| \\
इष्टचापार्धपूर्णज्यावर्गस्तेनाङ्केनसमो \footnote{{\s @बर्ग@}${\rm SB_J}$} भवति |
योऽङ्कः %\\(|\\)
 तच्चापोनभार्धांशानां या पूर्णज्या भवत्यस्या \footnote{{\s भवति । अस्याः}${\rm SB_J}$}
व्यासेन यदन्तरं तेन गुणित यो
व्यासार्धस्तदङ्कतुल्यो\footnote{{\s व्यासार्धः स्तदंक@ }${\rm J_1}$}
भवति |}


\newpage

Homage to Lord \Ganesa. \marginnote{f. 1r $J_2$} Now, with regard to the Sine of one degree, the commentary
\textit{\'Sarahavirja\*md\={\i}} of \textit{Ulugveg\={\i} j\={\i}ka} is written.
There in the determination of the Sine of one degree there is a newer pair of methods. 
One is made by [the one] named Yama\'saida K\=a\'s\={\i}; the second is made by
Miryolugvega. And furthermore of Ulugvega [?], in the procedure of opening [?]
of Yamasaida K\=a\'s\={\i}, that is to be made [?? \textit{h\=apya}?].
There the first method is stated. It [?] is as [follows]: Arc $ABJD$ is considered six degrees, 6. Again, equal division of it, this of [?]
point $B$ [and] point $J$, is made. There $AB$, $AJ$, $AD$, $VJ$ [$V$ and $B$ used
interchangeably for this point by the scribe], $BD$, and $JD$,
those lines are to be joined [as] Chords. 

Now here by the previously-stated method the Sine of three degrees is known, and it 
is this: 3\danda 8\danda 12 4\danda 33\danda 59\danda 34\danda 28\danda 54\danda 50.
This is multiplied by two: 6\danda 16\danda 49\danda 7\danda 59\danda 8\danda 56
[\elp illeg.] results, the Chord of $AD$. Here knowledge of the Chord $AB$ of two degrees
is desired. There in the second figure of the first chapter of the book Mijist\={\i} this
is explained. Of any quadrilateral that falls in a circle, the
sum of the products of sides standing opposite 
becomes equal to the product of the two diagonals falling in that quadrilateral.
Again just there in the fourth figure of the first chapter, this is explained. The square
of the Chord of half the desired arc [? the half-Chord of the desired
arc?] becomes equal with that number. [?] Whatever is the number, whatever becomes
the Chord of the degrees of half a circle diminished by that arc, whatever is the difference 
of that with the diameter, whatever is the half-diameter multiplied by that, becomes 
equal to that number. [?] 

\newpage
{\s एवमत्र अबचापजदचापे मिथः समाने स्तः\footnote{{\s स्थः} $J$} |
एवं अजबदचापे च समाने स्तः ||
तस्मात् अबजदघातो \footnote{{\s अवजदघातो@}${\rm SB_J}$} वर्गो भविष्यति ||
अत्र बजमज्ञातं तद्यावत्तावन्मितं कल्पितमिदमदेनगुणितं
जातं या\footnote{{\s वा} $J$} ६| १६|४९| ७| ५९| ८| ५६| २९| ४०|
एवं अबजदघातरूपवर्गराशेर्यावत्तावतश्च \footnote{{\s अवजद@}${\rm SB_J}$} योगः
अजवर्गतुल्य अजबदघा\footnote{{\s अनबदथ@}${\rm J}$} तेन समानोऽस्ति \footnote{{\s @ष्ति}${\rm J}$}|

अतोऽयं अजवर्गतुल्यो जातः ||
तस्मात् \\
अजवर्गराशौ एकोवर्गराशिरेभिर्यावत्तावद्भिरधिको
\footnote{{\s अजवर्गराशी एको@ }${\rm J_1}$}
\footnote{{\s अजवर्गराशौ एको@}${\rm SB_J}$}
जातः|
याव १ या ६| १६| ४९| ७| ५९| ८| ५६| २९| ४०}


\newpage
In this way here arc $AB$ [and] arc $JD$ pairwise become equal. 
And in this way arcs $AJ,$  $BD$ become equal.
From that, the product of $AB$, $JD$ will become a square. Here, $BJ$ is not known;
that is considered the measure of the \ya. This is multiplied by $AD$; and result,
6\danda 16\danda 49\danda 7\danda 59\danda 8\danda 56\danda 29\danda 40.
In this way the sum of the \ya\
and of the amount of the square of the number of the product of $AB$, $JD$
[is] equal to the square of $AJ$, moreover $AJ$ $BD$, is equal with that. [?]

%  \begin{figure}[h]
% \includegraphics[width=2in] %% 
% {JCUFig1bis}
% \centering
% \label{JCUFig1bis}
% \end{figure}

Here this result [is] equal to the square of $AJ$. From that, in the amount of the 
square of $AJ$ [is] one amount of the square; [?] with these unknowns the result 
is greater. \yava\ 1 \ya\ 6\danda 16\danda 49\danda 7\danda 59\danda 8\danda
56\danda 29\danda 40. 

\newpage
{\s अथ
प्रकारान्तरेण अजवर्गः\footnote{{\s प्रकारांतरेणाजवर्गः }${\rm J_1}$}
साध्यते |
तत्र अजचापं भार्धांशेभ्यः शोध्यं शेषस्य
पूर्णज्याया\footnote{{\s पूर्णज्यायाः}${\rm SB_J}$} व्यासस्य चान्तरं कार्यम् । तेन घ्नं षष्टितुल्यव्यासार्धं
अबवर्गतुल्यं भविष्यति
यतः अजचापं अबचापाद्द्विगुणमस्ति |
तस्माद्यदि अवबर्गः
याव १ षष्टि ६० तुल्यव्यासार्धेन [यदा] भाज्यते तदा यावद्वर्गस्य
षष्ट्यंशो लभ्यते तच्च अजचोपोनभार्धानां या पूर्णज्या
तदूनो यो व्यासस्तस्य प्रमाणमस्ति |
पुनरिदमन्तरं संपूर्णव्यासे १२० शोधितं
याव $\bm{\frac{\dot {1}}{60}}$ रू १२०
इदं अजचापोनभार्धांशानां पूर्णज्या जाता |
अस्याः वर्गः %\\(|\\)
यावव $\bm{\frac{1}{3600}}$ याव $\bm{\frac{\dot{240}}{60}}$
रू १४४००}


\newpage
Now by the other theorem, the square of $AJ$ is determined. There, arc $AJ$ is to be subtracted
from the degrees of half a circle; the difference of the Chord of the remainder 
and of the diameter is to be made. The half-diameter equal to sixty multiplied by that
will be equal to the square of $AB$ since arc $AB$ is doubled from arc $AB$. 
From that, if the square of $AB$ \yava\ 1, [when] it is divided by the half-diameter equal to sixty, 60, 
then a sixtieth part of the square of the \yava\ is obtained, that is, whatever is the Chord of half a 
circle diminished by arc $AJ$, whatever diameter is decreased by that, it is the amount of that.
Again, this difference is subtracted from the full diameter 120:
\yava\ $-\frac{1}{60}$, un.\ 120. This is the Chord of the
degrees of half a circle diminished by arc $AJ$. Its square: 
\yavava\ $\frac{1}{3600}$ \yava\ $\frac{\dot{240}}{60}$ un.\ 14400.

\newpage
{\s अथ च यदीष्टचापपूर्णज्यावर्गो \footnote{{\s यदीष्टचापपूर्णज्यावर्गः}${\rm SB_J}$} व्यासवर्गाच्छोध्यते
तत्र शेषं
$|$\marginnote{f. 1v $J_2$} \\
तच्चापोनभार्धांशपूर्णज्यावर्गो
भवति\footnote{{\s भावति }${\rm J_1}$}
यतो व्यासेन इ्ष्टचापे पूर्णज्यया इष्टचापोनभार्धांशपूर्णज्यया
चैकं समकोणत्रि\-भुजं भवति यतो वृत्तार्धे व्यासप्रान्तादुत्पन्नत्रिभुजस्य
पालिकोणः समकोणो
भवतीत्युक्लीदसस्य \footnote{OH MY GOD IT'S EUCLID}
तृतीयाध्याये
उपपन्न\-मस्ति 
तत्त्रिभुजे
समकोणसन्मुखभुजो व्यासोऽस्ति |
स च कर्णरूपः |
पुनः कर्णवर्गो \footnote{{\s कर्णवर्गः}${\rm SB_J}$} भुजकोट्योर्वर्गयोगेन समो
भवतीत्युक्लीदसस्य
प्रथमध्याये
उपपन्नम् |

अतोऽत्र व्यासवर्गः १४४०० कर्णवर्गरूपः |
अस्मिन्निष्टचापोनभार्धांशानां\footnote{{\s अस्मिनिष्टचा@} $J$} पूर्णज्यावर्गः
यावव $\bm{\frac{1}{3600}}$ याव $\bm{\frac{\dot{240}}{60}}$
रू १४४००
शोधितः | शेषं अजपूर्णज्यावर्गोऽवशिष्टः 
यावव %\\wहितेस्पचे\\wहितेस्पचे\\wहितेस्पचे
$\frac{\dot{1}}{3600}$ याव $\frac{240}{60}$ अयं
द्वितीयप्रकारेण अजवर्गः\footnote{{\s @प्रकारेणा .अजवर्गः }${\rm J_1}$}
सिद्धः |}

\newpage
And now, when the square of the Chord of the desired arc is subtracted from the square of the
diameter, there the remainder \marginnote{f. 1v $J_2$} is the square of the Chord of the degrees 
of half a circle diminished by that arc, since one right-angled triangle is [formed] by the diameter
and the Chord of the desired arc and the Chord of the degrees of half a circle diminished by the desired arc; 
and since ``the circumference-angle of a triangle produced from the end [actually ends] of the diameter 
in a half-circle is a right angle'' is demonstrated in the third book of Euclid (\textit{uklīdasa}), 
the side facing the right angle in that triangle is the diameter. And it 
has the form of a hypotenuse. Again ``the square of the hypotenuse
is equal to the sum of the squares of the arm and the upright'' is demonstrated in the first book of Euclid (\textit{uklīdasa}). 
So here the square of the diameter,14400, has the form of the square of a hypotenuse. From it, the square of the Chord of the 
degrees of half a circle diminished by the desired arc \yavava\ $\frac{1}{3600}$ \yava\ $\frac{\dot{240}}{60}$ un. 14400 is subtracted. 
The remainder is the square of the Chord AJ: 
remainder \yavava\ $\frac{\dot{1}}{3600}$ \yava\ $\frac{240}{60}$.
This by the second method is the established square of $AJ$.

\newpage

{\s 
अस्यप्रथमप्रकारागत अजवर्गः याव १ या ६| १६| ४९| ७| ५९| ८| ५६| २९| ४०
एतौ समावितिसमशोधनार्थं न्यासः
%"\nl"
%\Column{यावव \whitespace\whitespace\upbefore{०}१ \downafter{३६००}}
%"{८एम्}"
%\Column{याव \upbefore{२४०}६०}%"{६एम्}"
%\Column{या ०}%"{३एम्}"
%"\\न्ल्"
%\Column{यावव ०}%"{८एम्}"
%\Column{याव १}%"{६एम्}"
%\Column{या ६| १६| ४९| ७| ५९| ८| ५६| २९| ४०}%"{१६एम्}"

अथ यावनीयबीजगणिते समीकरणसंप्रदायस्त्वनया रीत्यास्ति
सा यथा ||
यदि समयोः पक्षयोर्मध्य एकाराशिरृण\-श्चेत्तद्राशितुल्यं
\footnote{{\s पक्षयोर्मध्ये एकाराशिः ऋणश्चेत् तद्राशि@ }${\rm J_1}$} \footnote{{\s पक्षयोर्मध्ये एकोराशिः ऋणश्चेत् तद्राशि@}${\rm SB_J}$}
पक्षद्वये योज्यं\footnote{{\s पक्षद्वयो जोज्यं }${\rm J_1}$}
तदापि पक्षद्वयं सममेव भवति ||
तस्मादत्र प्रथमपक्षे यावव %\\whitespace\\wहितेस्पचे\\wहितेस्पचे
$\frac{\dot{240}}{60}$
इदं पक्षद्वये योज्यं तत्र प्रथमपक्षे
धनर्णयोस्तुल्यत्वाद्यावद्वर्गवर्गस्य नाशो .अवशिष्टो \footnote{{\s नाशोः अवशिष्टः}${\rm SB_J}$}
याव $\frac{240}{60}$
अयं छेदभक्तो \footnote{{\s छेदभक्तः}${\rm SB_J}$} याव ४\\(|\\) अयं प्रथमपक्षः पुनर्द्वितीयपक्षे
यावव $\frac{1}{3600}$ योजितः यावव $\frac{1}{3600}$
याव १ या ६| १६| ४९| ७| ५९| ८| ५६| २९| ४०
एतौ पक्षौ पुनरपि समौ
%"\\न्ल्"
%\Column{याव ४}%"{८एम्}"
%\Column{या ०}%"{४एम्}"
%"\\न्ल्"
%\Column{यावव \upbefore{१}३६००}%"{८एम्}"
%\Column{याव १ या ६| १६| ४९| ७| ८| ५६| २९| ४०} \footnote{{\s याव १ या ६| १६| ४९| ७| ५९| ८| ५६| २९| ४०}${\rm SB_J}$}%"{२८एम्}"  
}
\newpage

Of it the first-method square
of AJ \yava\ 1 \ya\ 6|16|49|7|59|8|56|29|40. These two [are] equal: thus for the
purpose of subtraction of equals, the setting-out:
\yavava\ $\frac{-1}{3600}$ \yava\ $\frac{240}{60}$ \ya\ 0\\
\yavava\ 0 \yava\ 1 \ya\  6|16|49|7|59|8|56|29|40\\

Now, in western (\yavaniya) algebra (\bijaganita) the balancing (sam.pradaya) of equations is however by this method, as follows: If one quantity (\rasi) among the two equal sides, if negative, then an equal quantity is to be added in the two sides [?]. And then the pair of sides becomes just equal. From
that, here in the first side \yavava\ $\frac{-1}{3600}$ this is to be added to the second side, there
because of equality of the positive and negative in the first side [there is] elimination of the \yavava; remainder \yava\ $\frac{240}{60}$. This is divisor-divided: \yava\ 4. This first side again in the second side \yavava\ $\frac{1}{3600}$ is added: \yavava\  $\frac{1}{3600}$
\yava\ 1 \ya\ 6|16|49|7|59|8|56|29|40. These two sides again are equal:\\
\yava\ 4 \ya\ 0
\yavava\  $\frac{1}{3600}$ \ya\ 6|16|49|7|59|8|56|29|40.

\newpage
{\s अथयावनीयसंप्रदाये समपक्षयोर्मध्ये यौ
राश्येकजाती \footnote{{\s राशी एकजाती}${\rm SB_J}$} भवतः ||
तत्र लघुराशिर्महद्राशौ शोध्यते तदापि पक्षौ समावेदावशिष्टौ \footnote{{\s समावेवावशिष्टौ}${\rm SB_J}$}
भवतः ||
तस्मादत्र यावर्गराशिरेकजाती \footnote{{\s यावर्गराशिः एकजाती}${\rm SB_J}$} यो .अस्त्यतो \footnote{{\s योस्त्यतो}${\rm SB_J}$} लघुराशिर्याव \footnote{{\s लघुराशिः याव}${\rm SB_J}$} १
महद्राशौ याव ४ शोधितः शेषं प्रथमपक्षो \footnote{{\s प्रथमपक्षः}${\rm SB_J}$} याव ३
द्वितीयपक्षे च यावव $\frac{1}{3600}$
या ६| १६| ४९| ७| ५९| ८| ५६| २९| ४०
एतावपि समौ ||
पुनः\footnote{{\s पुन } ${\rm J_1}$}
पक्षद्वयमध्ये यद्येकोपक्षः खण्डितो \footnote{{\s खण्डितः}${\rm SB_J}$} द्वितीयपक्षः
संपूर्णश्चेत्तदा खण्डितराशिं प्रपूर्यं
तावद्गुणितं \footnote{{\s तावद्गुणं}${\rm SB_J}$} द्वितीयपक्षमपि
कुर्वति ||\footnote{{\s कुर्वंति }${\rm J_1}$}
अत्र सछेदराशिः खण्डितशब्देनोच्यते|
छेदरहितः संपूर्णोच्यते ||\footnote{{\s संपूर्ण उच्यते }${\rm J_1}$}
एवं कृते सति पक्षयोः समछेदत्वं\footnote{{\s पक्षयोसम@ }${\rm J_1}$}
विधाये छेदाय गम \footnote{{\s विधाय छेदापगम }${\rm J_1}$}
 एवोपपद्यते ||
तस्मादत्र द्वितीयप [क्षे \marginnote{f. 2r $J_2$}
खण्डितराशिर्यावव \footnote{{\s खण्डितराशिः यावव }${\rm J_1}$} $\frac{1}{3600}$}

\newpage
Now in the western balancing among the two equal sides [there] are [???] two [? du.?] quantities
of the same type. There the lesser quantity is subtracted from the greater quantity; and then the two
become just equal remainders. From that, here the \yava\ quantity, two [?] of one kind,
which[ever] is hence the small quantity,\yava\ 1, from the large quantity, \yava\
4, [is] subtracted. The remainder [is] the first side,\yava\ 3; and in the second side \yavava\  $\frac{1}{3600}$  \ya\ 6|16|49|7|59|8|56|29|40. 
And these two are equal. Again, among the two sides if one side broken, if second-side-full [?], then they make [?] the broken quantity filled
so-much-multiplier [??] second side. [???] Here a quantity with a divisor is stated by the word
“broken”. Devoid of divisor is called “filled”. When [it] is done thus, having established same-divisorness of the two sides, in the reduction [??] of the divisor [it] is just demonstrated. From that, here in the second side \marginnote{f. 2r $J_2$} 
the broken quantity is \yavava\ $\frac{1}{3600}$

\newpage 
{\s अयमर्थः %\\(|\\) 
अत्र यावद्वर्गवर्गराशे षट्शताधिकसहस्रत्रयमितोंशो
यावद्वर्गवर्गराशिरस्ति । \\(|\\)
तस्मादयं हरगुणितं क्रियते ताव १ "{\\र्म् ?}" द्यावद्वर्गवर्गराशिरेको
भवती । %\footnote{{\s भवती }"{\\र्म् wइथ्}" ति "{\\र्म् इन्सेर्तेद् इन् लेfत् मर्गिन् J१}"}
द्वितीयपक्षमप्येतावता गुणनीयम् | 
एवं कृते
रथमक्षे\footnote{{\s प्प्रथम\\(प\\)क्षे }${\rm K}$}
याव १०८०० द्वितीयपक्षे यावव १ अस्य प्रथमपक्षः\footnote{{\s प्रथमपक्षो }${\rm K}$}
वारद्वयं षष्टोध्व???  "{\\र्म् ?}" कृतः कृतो\footnote{{\s कृतो}${\rm K}$} याव ३\\(|\\)
अथ द्वितीयपक्षस्य\footnote{{\s प्द्वितीयपक्षस्थ}${\rm K}$} यावद्राशिस्तेनैव छेदेन ३६००
संगुण्य वारद्वयं षष्ट्यो उर्ध्व १\footnote{{\s ३ र्ध्व"\\र्म् ?" }${\rm K}$} कृतः\footnote{{\s कृतो}${\rm K}$}
या ६ | १६ | ४९ | ७ | ५९ | ८ | ५६ | २९ | ४० । 
एवम् जातौ पक्षौ समौ ।
\begin{tabular}{c | c }
{\s याव ३ द्विपरिवर्त्ताः }						\\ 
{\s यावव १ या ६ | १६ | ४९ | ७ | ५९ | ८ | ५६ | २९ | ४० |}	\\ 
\end{tabular}
%"\\न्ल्" 
%\Column{याव ३ द्विपरिवर्त्ताः}"{१२एम्}"
%"\\न्ल्"
%\Column{यावव १ या ६| १६| ४९| ७| ५९| ८| ५६| २९| ४०}"{३०एम्}"
एतौ यावत्तावता पवत्ति\footnote{{\s पन्नत्ति "{\\र्म् ?}" }${\rm K}$} द्विपरिवर्त्ता तौ
\begin{tabular}{c}
{\s या\footnote{{\s याव}${\rm K}$} ३ द्विपरिव@}"{१२एम्}"}				\\ 
{\s याद्य "{\\र्म् ?}" १ रू ६| १६| ४९| ७| ५९| ८| ५६| २९| ४० द्विपरिव@}"{३६एम्}" 	\\ 
\end{tabular} 
%"\\न्ल्"
%या\footnote{{\s याव}${\rm K}$} ३ द्विपरिव@}"{१२एम्}"
%"\\न्ल्"
%\Column{याद्य "{\\र्म् ?}" १ रू ६| १६| ४९| ७| ५९| ८| ५६| २९| ४० द्विपरिव@}"{३६एम्}" 
}

\newpage 


\newpage 
{\s अथात्र यावद्द्वयेन सरूपस्थराशिना
चेद्यावत्तावत्रयं\footnote{{\s चेद्यावत्तावत्त्रयं}${\rm K}$}
३ भाज्यते
तदा यावत्तावत्प्रमाणं लभ्यते |
अस्य भागग्रहणरीतिः पूर्वाचार्यैरेतावत्कालपर्यन्तं\footnote{{\s पूर्वाचार्पैरे "{\\र्म् ?}" तावत्कालपर्यन्तं}${\rm K}$}
न लब्धा |
अत्र यमशैदेन रीतिः प्रदर्शिता | 
सा यथा | 
रूपस्य प्रथमाङ्को यावत्तावता भाज्यः । लब्धेरेकान्ते
स्थाप्या | 
पुनर्लधिघनं\footnote{{\s पुनर्लाब्धिघनं}${\rm K}$} शेषाङ्के योज्यं । पुनरत्र
द्द्वितीयांकौ\footnote{{\s  वितीयाङ्को }${\rm K}$}
यावत्तावता भाज्यः |
लब्धिपूर्वलब्धेरधस्थाप्यः | 
पुनर्लब्धिद्वययोगस्य घनः कार्यः । 
एवं तत्र प्रथमलब्धिघनः शोध्यः | 
शेषं द्वितीयलब्धिशेषे योज्यम् ||\footnote{{\s ज्योज्यं }${\rm J_1}$}
पुनस्तृतीयाङ्को यावत्तावता भाज्यः | 
इयं लब्धिः पूर्वलब्धिद्वयाधस्थाप्यः | 
पुनरस्य लब्धित्रययोगस्य घनः कार्यः । 
तत्र लब्धिद्वययोगस्य घनः शोध्यः | 
शेषं तृतीयलब्धिशेषे योज्यम् | 
पुनस्तत्रचतुर्थाङ्को यावत्तावता भाज्यः | 
एवमेते "{\\र्म् ?}" ष्टभागलब्धिपर्यन्तं विधिः कार्यः | 
एवं यमशैदेन\footnote{{\s यमशदैन }${\rm J_1}$}
यावत्तावत्प्रमाणो निष्काशितः २ | ५ | ३९ | २६ | २२ | २९ | २८ | ३२ | ५२ | ३३ ।
इयंमंशद्वयस्य\footnote{{\s इयमंश@ }${\rm K}$}
पूर्णज्यास्ति | 
अस्यार्धं एकांशज्या\footnote{{\s अस्यार्धमेकांशज्या }${\rm K}$} जाता १ | २ | ४९ | ४३ | ११ | १४ | ४५ | १६ | १६ | १७ । 
अस्य भागहरणोपपत्तिः | 
तत्र पक्षद्वयमध्ये\footnote{{\s पक्षद्वयमध्य एक@ }${\rm K}$} एकपक्षे
यावत्तावदस्ति | 
द्विपक्षे यावद्यतः $?$ रूपराशिश्च | 
एवं तत्र यावत्तावद्ज्ञानं चेत्तदा यावत्तावद्घनं कृत्वा
रूपराशौ प्रक्षिप्य यावत्तावता भाज्यते । तत्र लब्धिर्यावत्तावन्मानं
स्यात् | 
अत्र तु यावत्तावद्ज्ञानं
नास्त्यतो रूपराशिरेव यावत्तावता भाज्यः । 
यल्लब्धं तत्सरूपयावघनस्य कोप्यंऽशो \footnote{{\s को .अप्यंशो}${\rm K}$} लब्धः । 
स चैकान्ते धृतः | 
पुनरस्य घनं कृत्वा $|$ \marginnote{f. 2v $J_2$}
शेषे जोतितं \footnote{{\s जोतितुं}${\rm K}$} पुनरत्र यावत्तावता द्वितीयांको भाजितः । 
एवं यल्लब्धं तत्तु\footnote{{\s त्तत्तु }${\rm J_1}$}
पूर्वलब्धेरथ स्थापितम् | 
एवं लघयावत्तावद्व्यनस्यासन्नो भागो
\footnote{{\s @स्वासन्नौ भोगो } with strokes stuck out $J_1$} %"{\\र्म् wइथ् स्त्रोकेस् स्त्रुच्क् ओउत्, J१}"}
लब्धः %\\(|\\)
अत्रासन्नता शेषावयवसत्वावात् निरवयवत्वे\footnote{{\s शेषावयवसत्वाद्निरवयवत्}${\rm K}$} स एव सूक्ष्मा
लब्धिः | 
अथलब्धिघने पूर्वघनं शोध्यं यतस्तदधिकं जातम् | 
एवमिष्टभागपर्यन्तं मुहुर्मुहुः कार्यम् | }


\newpage 


\newpage 
{\s अथास्योद \\qउएर्य् ??? 
तत्र द्वितीयपक्षे रूपाणि ६ | १६ | ४९ | ७ | ५९ | ८ | ५६ | २९ | ४० | 
अत्र षट् ६ द्विपरिवर्ताः षष्टेरूर्ध्वमस्ति | 
यावत्तावत्त्रयं द्विपरिवर्ताः षष्टेरूर्ध्वमस्ति | 
अतो\footnote{{\s अत}${\rm K}$} एकजातौ 
भागे गृहीते लब्धमंशाः | 
पुनरस्य घने अंशाः ८ शेषां १६ शेषु ५९ योजितं
१६ | ५७  पुनः पूर्वलब्धे शेषं २८ तेनैव या ३
भक्तं लब्धकलाः ५ लब्धिः पूर्वलब्धेरधस्थोपिता अं २
क ५ पुनरत्र शिष्ठं १ | ५७ पुनर्लब्धिद्वयस्यघनः\\(|\\)
अं ९ | २ | ३२ | ५ अत्र प्रथमलब्धिघनः\\(|\\)
अं ८ शोधितः शेषं अं १| २| ३२| ५७|
इदं शेषे अं १| ५७| क ७| वि ५९| ८| ५६| २९| ४०\footnote{{\s १| ५७| %\\उपfतेर्{अं} ७|\\उपfतेर्{क} ५९\\उपfतेर्{वि}|
८| ५६| २९| ४० } ${\rm J_1}$}
अंशस्थाने योजितं \\(|\\) अं १ | ५८ | १० | ३१ | १३\footnote{{\s १| ५८| %\\उपfतेर्{अं}
 १०| ३१| १३} ${\rm J_1}$}
पुनर्हरेण\footnote{{\s पुनो हरेण} ${\rm K}$}
या ३ अंशे ५८ भोगो\footnote{{\s अंशे \upbefore{५८}भोगो  }${\rm J_1}$}
गृहीतः \\(|\\)
लब्धिर्विकलात्मिका ३९\footnote{{\s लब्धिर्विकालात्मिका ३९ } ${\rm J_1}$}
शेषं अं १ | १० | ३१ | १३ पुनरियं लब्धिः पूर्वलब्धेरधः स्थाप्यः
अं २ | ५ | ३९
अस्य घनोंशादि ९ । ११ । २ । ३२ । २७ । ५३ । ३९ ।}
 %\query


\newpage 








\end{document}
