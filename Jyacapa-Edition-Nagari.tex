\documentclass[12pt]{article}

%%%Font related packages
\usepackage[no-math]{fontspec}
\defaultfontfeatures{Ligatures=TeX} 
%\setmainfont{Linux Libertine O}
\newfontfamily\s[Scale=0.92, Script=Devanagari]{Shobhika-Regular}
\usepackage{color, dtk-logos}
 
 
 %%%Formatting related packages
\usepackage{graphicx}
\usepackage[hang,flushmargin]{footmisc} % For no indentation of footnotes.
\usepackage{array}
\usepackage{tabularx, multicol, vwcol}
\usepackage{fullpage}
\usepackage{marginnote}


%Math related packages
\usepackage{amsmath} % āmerican Mathematical Society packages for metamathematical symbols.
%āpril 2017 āditya says install libertine math font package specially from website.

%Page formatting commands
%\newdimen\stdbaseline
%\stdbaseline = 13pt  
%\baselineskip = 2\baselineskip
\parskip = \baselineskip
\parindent = 0pt


%%%Critical editing packages
\usepackage[series={A,B,C,D},noend,noeledsec,nofamiliar,noledgroup]{reledmac}
\Xarrangement[B]{twocol}
\Xarrangement[C]{threecol}
\Xarrangement[D]{paragraph}

%=============================================================%


%% MāCROS for Diacriticals, symbols, math and text features
%%
\let\*=\d
\def\elp{$\ldots\,$}
\def\degrees{$^\circ$}
\def\degree{$^\circ$}
\def\signs{$^s$}
\def\Sin{\mathop{\rm Sin}\nolimits}
\def\Cos{\mathop{\rm Cos}\nolimits}
\def\Versin{\mathop{\rm Vers}\nolimits}
\def\Coversin{\mathop{\rm Coversin}\nolimits}
\def\Crd{\mathop{\rm Crd}\nolimits}
\def\crd{\rm{crd}}

%%%%%%%%%%%%%%%%%%%%%%%%


\begin{document}


\title{Jyacapa ūtpatti}
\author{Clemency Montelle, Kim Plofker, Glen Van Brummelen}

\maketitle

\begin{abstract}
We will write an abstract here
\end{abstract}


\vskip30pt

KEYWORDS: 

\newpage

\section{Introduction} 

Yadda yadda yadda...

\section{Critical Edition}




%%  jyācāpakī upapatti\nl


\beginnumbering
%\setcounter{stanzaindentsrepetition}{2}
\setstanzaindents{0,0,0,0,0}
%\setstanzaindents{8,0,1}

\pstart
{\s $|$\marginnote{f.~1r J}  श्रीगणेशाय नमः ||
अथांशत्रयस्य जीवाज्ञाने सत्येकांशजीवाज्ञानं
दोर्ज्यामितिश्लोकचतुष्टयेनाह ||
करसङ्गुणितां द्विगुणां भुजज्यां
त्रिभक्तां विधाय
गणकप्रवीणः फलमंशास्तान्पृथक्स्थापयेत् ||
तु परं तेषां फलरूपांशानां घने व्यासार्धवर्गभक्ते सति
यत्फलं तद्भागावशेषसहितं कृत्वा त्रिभक्तं कार्यं ||
त्रिभक्ते यत्फलं कलात्मकं तत्पूर्वलब्धांशरूपफलाधः स्थाप्यं सांशकलात्मिका पङ्क्तिर्जाता ||
पङ्क्तेर्घनः प्रथमघनेन हीनः व्यासार्धवर्गभक्तः फलं शेषे योज्यत्रिभिर्भजेत् ||
लब्धं फलं विकलां कलाधः स्थप्याः साम्शकलाविकलात्मिका
पङ्क्तिर्जाता ||
पङ्क्तेर्घनः कार्यः
अस्मिन्पूर्वघनः शोध्यः शेषे त्रिभज्यावर्गभक्ते यत्फलं तद्भाज्यावशेषसहितं
कृत्वा मुहुर्वारंवारं तदग्रे तु परं पुरोक्तो विधिः कार्यः एवं कृते फलं पङ्क्तिः स्यात् ||
पङ्क्त्यर्धं चापस्य तृतीयांशज्या स्यात् || अथ इत्यनन्तरं तां त्रिभागजीवां पुनः प्रकाराद्वच्मि ||}
\pend

\vskip15pt

\pstart
{\s यतः कारणादनेकैर्भेदैर्गणकस्य बुद्धिः स्वशास्त्रे प्रवीणतामेति || १६ ||}
\pend

\vskip15pt

\pstart
{\s पुनः प्रकारान्तरेण त्रिभागजीवामाह || स्वत्र्यंशहीनेति ||}
\pend

\vskip15pt

\stanza
{\s भुजज्या स्वतृतीयांशेन हीना पृथक्स्थाप्या ||\\
अस्याः घनस्त्रिगुणस्य कृत्या त्रिघ्न्या भक्तः (|)}&
{\s लब्धफलेन प्र्थक्स्था भुजज्या युता
कार्या \\
मुहूरनया क्रियया पङ्क्तिः साध्या
पङ्क्त्यर्धं चापगुणांशजीवा
धनुस्त्र्यंशज्या स्यात् || १७ ||}\&

\vskip15pt

\pstart
{\s पुनः प्रकारान्तरेणाह | भुजांशजीवेति ||}
\pend

\vskip15pt

%%%%%
\stanza
{\s 
 भु$|$\marginnote{f.~1v J}जांशज्यायास्तृतीयांशः पृथक्स्थाप्यः ||\\
तस्य घने स्वत्र्यंशयुते त्रिज्यावर्गभक्ते लब्धं} &
{\s यत्फलं तेन पृथक्स्थो युतः कार्यः ||\\
एवं मुहूर्वारंवारमनया क्रियया धनुस्त्रिभागजीवा स्यात् || १८ ||}\&

\vskip15pt

\pstart
{\s अस्योपपत्तिः || सा यथा ||
कबजदषडंशचापं कल्पितम् ||
पुनरस्य समानं भागत्रयं कल्पितं बचिह्नजचिह्नयोः ||
तत्र कबं कजं कदं बजं बदं जदं चैताः
पूर्णज्यारेखा योज्याः ||
अथात्र पूर्वोक्तप्रकारेणांशत्रयस्य जीवा ज्ञातास्ति साचेयं
३| ८| २४| ३३| ५९| ३४| २८| १४| ५०| इयं द्विगुणा
६| १६| ४९| ७| ५९| ८| ५६| २९| ४० जाता कदपूर्णज्या ||
अत्रांशद्वयस्य कबपूर्णज्याज्ञानमिष्टमस्ति ||
\edtext{अस्योपपत्त्यर्थं}{\Dfootnote{{\s अस्योपपत्य } $J$}}
किंचिदुच्यते ||
तत्र
\edtext{सिद्धान्तसम्राजिप्रथमाध्यायस्य}{{\lemma{\s -सम्राजि-}\Dfootnote{{\s -संम्राजि-} $J$}}} 
द्वितीय क्षेत्रे इदमुपपादितं ||
यच्चतुर्भुजं वृत्तान्तः पतति तत्सन्मुखस्थभुजानां घातयोगः
तच्चतुर्भुजान्तः
\edtext{पातिकर्णद्वयघातसमो}{\lemma{\s -कर्णद्वय-}\Dfootnote{CHECK: it was hardcoded $J$}}%"\\३७Fw\\३९२wय् {\\इत् x} आGआत्\\३७Fw {\\र्म् J}"}
%% KP: हर्द्चोदेद् वरिअन्त्
भवतीत्यस्योपपत्तिस्तु पूर्वं कथितैवास्ति ||
पुनस्तत्रैव प्रथ\-माध्यायस्य चतुर्थक्षेत्रे इदमुपपादितम् ||
\edtext{इष्टचापार्धपूर्णज्यावर्गस्तच्चापोनभार्धांशपूर्णज्या\-व्यासान्तर\-गुणित\-व्यासार्ध\-तुल्यो}%
{\lemma{\s गुणित\-व्यासार्ध\-तुल्यो}
\Dfootnote{{\s -गुणितद्व्यासार्धतुल्यो} $J$}} 
 भवति ||
अस्योपपत्तिः ||
कजव्यासे कबजदं वृत्तार्धं कार्यं ||
तत्र बदं \edtext{दजं}{\Dfootnote{inserted from left margin, $J$}}
सममस्ति || ||
कबरेखाकदरेखा\-बदरेखा\-जदरेखाः संयोज्याः ||
दचिह्नाद्दकरेखा कजव्यासे लम्बरूपा कार्या ||
अत्र जफरेखा कजव्यासस्य ज्ञातचापोनभार्द्धांशपूर्णज्यायाश्चान्तरार्धमितास्ति ||
कुतः ||
कबं कहतुल्यं कार्यं ||
दहरेखायोज्या ||
हकदत्रिभुज\-दकबत्रिभुजयोः ।
कहभुजकदभुजौ बकभुजक$|$%\\[
दभुजयोः
 \marginnote{f.~2r J}
क्रमेण समानौस्तः || पुनः हकदकोणः बफदकोणेन समोस्ति तस्मत् दहरेखा बदरेखा समाना जाता || दजरेखापि समानाजाता || तदा दजवर्गः दफवर्गजफवर्गयोर्योगेन तुल्यो भविष्यति ||
एवं दहवर्गदफवर्गफहवर्गयोगेन तुल्यो भविष्यति ||
तस्मात् फहं जफं च समानं जातं || अनयोर्योगः कजकबयोरन्तरमस्ति || पुनः कदजत्रिभुजे दसमकोणात् दफलम्बः कर्णोपर्यागतो स्ति ||
तस्मात् कजस्य निष्पत्तिर्जदेन तथास्ति
यथा जदनिष्पत्तिर्जफेन || तस्मात्कजजफघातः दजवर्गेण समानो जातः ||
तदेतदुपपन्नं इष्टचापार्धपूर्णज्यावर्गः चापोनभार्द्धांशपूर्णज्याव्यासान्तरगुणितव्यासार्धतुल्यो भवतीति ||
शकलं~||}
\pend

\vskip15pt

\pstart
{\s अथ प्रकृतमनुसरामः ||
एवमत्रकथितक्षेत्रे कबचापजदचापे मिथः समानेस्तः ||
एवं कजबदचापे चसमानेस्तः || तस्मात्कबजदरेखाघातो वर्गो भविष्यति ||
अत्र बजमज्ञातन्तघावत्तावन्मितं कल्पितं || इदं कदेन गुणितं जातं या
६ | १६ | ४९ | ७ | ५९ | ८ | ५६ | २९ | ४०}
\pend

\vskip15pt

\pstart
{\s [क्षेत्र]}
\pend

\vskip15pt

\pstart
{\s अस्य कबजदघातरूप याव$|$\marginnote{f.~2v J}%
द्वर्गस्य च योगः
कजवर्गतुल्यकजबदघातेन समानो स्ति || अतो यं कजवर्गतुल्यो जातस्तस्मात्कजवर्गराशौ एको
वर्गराशिरेभिर्यावत्तावद्भिरधिको जातः ||
या व १ या ६ | १६ | ४९ | ७ | ५९ | ८ | ५६ | २९ | ४० ||
अस्य
निदर्शनार्थं शकलं यथा || ||
प्रकारान्तरेण कजवर्गः साध्यते || तत्र कजचापं
भार्द्धांशेभ्यः शोध्यं शेषस्य पूर्णज्या या व्यासस्य चान्तरं कार्यं ||
तेन
गुणितं षष्टि तुल्यव्यासार्धं कबवर्गतुल्यं भविष्यति
यतः कजचापं
कबचापाद्विगुणमस्ति ||
तस्माद्यदि कबवर्गः याव १ षष्टि ६० तुल्यव्यासार्द्धेन यदा
भाज्यते तदा यावद्वर्गस्य षष्ट्यंशो लभ्यन्ते तच्च
कजचापोनभार्द्धांशपूर्णज्याव्यासान्तरस्य प्रमाणमस्ति || पुनरिदमन्तरं
संपूर्णव्यासे १२० शोधितं याव १ ६० रू १२० इदं
कजचापोनभार्धंशानां पूर्णज्या भवति ||
अस्याः वर्गः याव व १ ३६०० याव
२४० ६० रू १४४०० अथ च यदीष्टचापपूर्णज्यावर्गः व्यासवर्गा छोध्यते तत्र यछेषं तच्चापोनभार्द्धांशानां पूर्णज्यावर्गो भवति ||
यतो व्यासेन इष्टचापपूर्णज्यया
इष्टचापोन भार्द्धांशानां पूर्णज्यया चैकं समकोणत्रिभुजं भवति ||
यतो वृत्तार्द्धे
व्यासप्रांता उत्पन्नत्रिभुजस्य पालिकोणः समकोणो भवतीति रेखागणितस्य तृतीयाध्याये उपपन्नमस्ति ||
तत्त्रिभुजे समकोणसन्मुखभुजो व्यासो स्ति || सचकर्णरूपः || पुनः
कर्णवर्ग्गः भुजकोद्योवर्ग्गयोगेन समो $|$\marginnote{f.~3r J} भवतीतिरेखागणितस्य
प्रथमाध्यये उपपन्नं || अतो त्र व्यासवर्ग्गः १४४०० कर्णवर्ग्गरूपः ||
अस्मिन्निष्टचापोन\-भार्द्धांशानां पूर्णज्यावर्ग्गः यावव १ ३६०० याव २४० ६० रू १४४०० शोधितः शेषं कजपूर्णज्यावर्ग्गे ऽशिष्टः यावव १० ३६०० याव २४० ६०
अयं द्वितीय प्रकारेण कजवर्ग्गः सिद्धः~||}
\pend

{\s अथ प्रथमप्राकारागतकजवर्ग्गः याव १ या ६ | १६ | ४९ | ७ | ५९ | ८ | ५७ | ४९ | ४०
एतौ समावितिसमशोधनार्थं न्यासः यावव १० ३६०० याव २४० ६०
%\\NओLएम्म{या ०?}{६ | १६ | ४९ | ७ | ५९ | ८ | ५६ | २९ | ४० "{\\र्म् च्रोस्सेद् ओउत् J}}"
या ०?\footnote{{\s ६ | १६ | ४९ | ७ | ५९ | ८ | ५६ | २९ | ४० } Crossed out $J$}
यावव ० यव १ या
%\\NओLएम्म{६ | १६ | ४९ | ७ | ५९ | ८ |}{ ५६ | २९ | ४० "{\\र्म् Lअस्त् थ्रेए दिगित्स् च्रोस्सेद् ओउत् J}}"
६ | १६ | ४९ | ७ | ५९ | ८ |\footnote{ {\s ५६ | २९ | ४०}  Last three digits crossed out $J$}


अथ बीजगणिते समीकरणसंप्रदायस्त्वनयारीत्यास्ति ||
सायथा ||
यदि समयो पक्षयोर् मध्ये एको
राशोः ऋणश्चेत्तद्राशितुल्यं पक्षद्वये योज्यं तदापि ८ पक्षद्वयं सममेवभवति ||
तस्मादत्र प्रथमपक्षे यावव १ ३६०० इदं पक्षद्वयं योज्यं ||
तत्र प्रथमपक्षे
धनर्णयोस्तुल्यत्वाद्यावद्वर्ग्गवर्ग्गस्यनाशः अवशिष्टं याव २४० ६० अयं छेदभक्तः याव ४ अयं
प्रथमपक्षः ||
पुनर्द्वितीयपक्षे यावव १ ३६०० योजितः यावव १ ३६०० या व १ या ६ | १६ | ४९ | ७ | ५९ | ८ | ५६ | २९ |
४०
एतौ पक्षौ पुनरपि समानौ जातौ याव ४ या ० यावव १ ३६०० याव १ या ६ | १६ | ४९ | ७ | ५९ | ८ | ५७ | ४९ | ४० ||
समपक्षयोर्मध्ये यौ राशी एकजाती भवतः तत्र लघुराशिर्महद्राशौ शोध्यते तदापि पक्षौ
समावेवावशिष्टौ भवतः || तस्मादत्र यो वर्ग्गः राशिः एकजाती यो .अस्त्यतो लघुराशिः याव १ महद्राशौ
याव ४ शोधितः शेषं प्रथमपक्षः याव ३ द्वितीयपक्षे च यावव १ ३६०० या ६ | १६ | ४९ | ७ | ५९ | ८ | ५७ | ४९ | ४०
एतावपि समौ || पुनः पक्षद्वयमध्ये यद्येकः खण्डितोस्ति द्वितीयपक्षः | $|$
\marginnote{f.~3v J}
संपूर्णश्चेत्तदा खण्डितराशिं प्रपूर्यतावद्गुणं द्वितीयपक्षमपि कुर्वन्ति || अत्र सछेदराशिः खण्डित
उच्यते छेदरहितः संपुर्ण उच्यते || एवं कृते सति पक्षयोः समछेदत्वविधाय छेदायगम
एवोपपद्यते || तस्मादत्रद्वितीयपक्षे खण्डितराशिः यावव १ ३६०० अयमर्थः || अत्र
यावद्वर्गवर्गवर्गराशिरस्ति तस्मादयं हरं गुणितः क्रियते तावद्यावद्वर्गराशिरेकोभवतीति द्वितीय
पक्षोप्येतावती गुणनीयः || एवं कृते प्रथमपक्षे याव १०८०० द्वितीयपक्षे यावव १ अथप्रमपक्षः
वारद्वयं षष्ट्योर्ध्वाकृतः याव ३ अथ द्वितीयपक्षस्य यावद्राशिस्तेनैवछेदेन ३६०० संगुण्य
वारद्वयं षष्ट्योर्ध्वाकृतः या ६ | १६ | ४९ | ७ | ५९ | ८ | ५६ | २९ | ४० याव ३ द्विपरिवर्त्ताः यावव १ या ६ | १६ | ४९ | ७ |
५८ | ७ | ५८ | ५६ | २९ | ४० एवं जातौ पक्षौ समौ एतौ यावत्तावतापवर्त्तितौ || या ३ द्विपरिवर्राः याव १ रू ६ | १६ | ४९ | ७ |
५९ | ८ | ५७ | २९ | ४० अत्र यावत् घनः सरूपराशिश्चेद्यावत्तावत्त्रयेण या ३ भाज्यते तदायावत्तावत्प्रमाणं ल्
लभ्यते अस्य भागग्रहणरीतिः पूर्वाचार्यैः एतावत्कालपर्यन्तं न लब्धा || अत्र यमशदेन रीतिः प्रदर्शिता ||
सायथा रूपस्य प्रथमांके यावत्तावता भाज्यः लब्धिरेकाञ्कस्थाप्य || पुनर्लब्धिघनं शेषांके योज्य
पुनरत्रद्वितीयांको यावत्तावता भाज्यः लब्ध्$|$इः पूर्वलब्धेरधः स्थाप्या || पुनर्लब्धिद्वययोगस्य घनः
कार्यः || एवं तत्र प्रथमलब्धिघनः शोध्यः शेषं द्वितीयलब्धि\-शेषे योज्यं || पुनस्तृतीयांको
यावत्तावता भाज्यः इयं लब्धिः पूर्वलब्धिद्वयाधः स्थाप्या || पुनरस्यलब्धित्रययोगस्य घनः कार्यः
तत्र लब्धिद्वययोगस्य घनः शोध्यः | $|$ %NEW FOLIO
\marginnote{f.~4r J}
शेषं तृतीयलब्धिशेषे योज्यं || पुनस्तत्र चतुर्थां को यावत्तवता भाज्यः || एवमेवेष्टभागपर्यन्तं विधिः कार्यः ||
एवं यमशैदेन यावत्तावत्प्रमाणं निकाशिलं २ | ५ | ३९ | २६ | २२ | २९ | २८ | ३२ | ५२ | ३३ | इयं
अंशद्वयस्य पूर्णज्यास्ति अस्यार्धं एकांशज्याजाता | १ | २ | ४९ | ४३ | ११ | १४ | ४४ | १६ | २६ | १७
अस्यभागहरणोपपत्तिः ||
तत्र पक्षद्वयमध्ये एकपक्षे यावत्तावदस्ति ||
द्वितीयपक्षे यावत्तावद्घनः रूपराशिश्च ||
एवं तत्र यावत्तावद्ज्ञानं चेत्तदा यावत्तावद् घनं कृत्वा
रूपराशौ प्रक्षिप्य यावत्तावता भाज्यते तत्र लब्धिर्यावत्तावन्मानं स्यात् |

अत्र तु यावत्तावद्ज्ञानं नास्त्यतो रूपराशिरेवयावत्तावता भाज्यः ||
यल्लब्धं तत्सरूपयावद्घनस्य कोप्यंशो लब्धः |
सचैकान्ते धृतः || पुनरस्य घनं कृत्वा शेषे
योज्य पुनरत्रयावत्तावता द्वितीयांको भाजितः एवं यल्लब्धं तत्पूर्वलब्धेरधः स्थापितं एवं लब्धं यावत्तावद्धनस्यासन्नोभागोलब्धः
अत्रासन्नताशेषावयवसत्वाभिरवयवत्वे एव सूक्ष्मालब्धिः अथ लब्धिघने पूर्वघनं
संशोध्य यतस्तदधिकं जातं एवमिष्टभागपर्यन्तं मुहुर्वारं वारं कार्यं ||

अथास्योदाहरणं ||
तत्र द्वितीयपक्षे रूपाणि ६ | १६ | ४९ | ७ | ५९ | ८ | ५६ | २९ | ४४ अयत्र षद्
द्विपरिवर्त्ताः ||
षष्टेरूर्ध्वमस्ति ||
यावत्तावत्त्रयं द्विपरिवर्त्ताः षष्टेरूर्ध्वमस्ति ||

अत एकजातौ भागे गृहीते लब्धमंशाः | २ पुनरस्यघनेंशाः ८ वारद्वयं षष्ट्याभाक्ताः ० |
० | ८ विकलात्मकं विकलासु योजितं १६ | २७ | पुनः पू$|$\marginnote{f.~4v J}%
र्वलब्धिशेषं १६
तेनौव या ३ भक्तं लब्धाः कला ५ लब्धिः पूर्वलब्धेः २ अधः स्थापितां अञ्क २ | ५
पुनरत्रशिष्ट १ | ५७ पुनर्लब्धिद्वयस्य घनः अं ८ | २ | ३२ | ५
अत्र प्रथमलब्धिघनः अं ८
शोधितः शेषं अं १ | २ | ३२ | ५ इदं शेषे अं १ | ५७ | ७ | ५९ | ८ | ५६ | २९ | ४० अंशस्थाने योजितं
१ | ५८ | १० | ३१ | १३ पुनर्हरेणया ३ अंशे ५८ भागोगृहीतः लब्धिविकलात्मका ३९ शेषं अं १ | १० | ३ | १३ पुनरियंलब्धिः पूर्वलब्धेरधः स्थाप्य २ | ५ |
३९ | अस्य घनोंशादिः ९ | ११ | २ | ३२ | २७ | ४३ | ३९ || 
$|$ \marginnote{f.~5r J}

% "{\\र्म् रेस्त् ओf f.४व् ब्लन्क्}"

अत्रद्वितीयघनः ९ | २ | ३२ | ५ शोधितः शेषं कलादि ? | ३० | २७ | २७ | ४३ | ३९ इदं शेषाञ्केषु १ |
१० | ३१ | १३ योजितं १ | १९ | १ | ४१ | २४ | १३ | १९ पुनरत्रकलास्थाने भक्तः ल २६ पूर्वलब्धेरधः
स्थापिता २ | ५ | ३९ | २६ शेषं कलादि १ | १ | ४१ | २४ | १३ | १९ पुनर्लब्धैः २ | ५ | ३९ | २६ | घनः ९| ११ |
८ | १४ | ३३ | १२ | २३ | २१ | ४ | ५६ तृतीय घनोत्र शोधितः विकलास्ववशिष्टं ५ | ४२ | ५ | २८ | ४४ | २१ | ४ |
५६ | ७ | पुनरिदं शेषाञ्के योजितं क १ | ७ | २३ | २९ | ४२ | ३ | २१ | ४ | ५६ पुनर्तेनैवभक्ते लब्धं
२२ शेषं १ | २३ | २९ | ४२ |३ | २१ | ४ |५६ पूर्वलब्धौ योजितं २ | ५ | ३९ | २६ | २२ अस्य घनः ९| ११ | ८ |
१९ | २२ | ४१ | ६ | ५२ | १५ | १ | ५६ चतुर्थघनोत्र शोधितः शेषं ४ | ४९ | २८ | ४३ | ३१ | १० | ५ | ५६ इदं
शेषे योजितं १ | २८ | १९ | १० | ४६ | ५२ | १५ | १ | ५६ पुनर्हरभक्तः लभ्दं २९ शेषं १ | १९ | १० | ४६
| ५२ | १५ | १ | ५६ पुनर्लभिरधः स्थिता २ | ५ | ३९ | २६ | २२ | २९ | अस्यघनः ९ | ११ | ८ | १९ | २९ | २ | ४२ | १
| ३९ | ५० | ५२ अत्रपञ्चमघनः शोधितः शेषं ६ | २१ | ३५ | ९ | २४ | ४८ | ५६ इदं शेषं योजितं
१ | २५ | ३२ | २२ | १ |३९ | ५० | ५२ पुनरयं हरया ३ भक्तः लब्धं २८ शेषं १ | ३२ | २२ | १ |३९ | ५० |
५२ लब्धिः पूर्वलब्धेरधः स्थापिता २ | ५ | ३९ | २६ | २२ | २९ | २८ अस्यघनः ९ | ११ | ८ | १९ | २९ | ८ | ५०
| २७ | १९ | ५९ | ४३ अत्रषष्टघनः शोधितः शेषं ६ | ८ | २५ | ४० | ८ | ५१ शेषं योजितं १ | ३८ | ३० |
२७ | १९ | ५९ | ४३ पुनर्हरभक्तं लब्ध ३२$|$
\marginnote{f.~5v J}
शेषं २ | ३० | २७ | १९ |५९ | ४३ लब्धिः पूर्वलब्धेरधः
स्थापिता २ | ५ | ३९ | २६ | २२ | २९ | २८ | ३२ अस्यघनः ९ | ११ | १८ | १९ | २९ | ८ | ५७ | २८ | २३ | ३७ | १ अत्रसपमघनः शोधितः शेषं ७ | १ | ३ | ३७ | १८ इदं शेषे योजितं २ | ३७ | २८ | २३ | २७ | १ हरभक्ते लब्धं ५२ शेषं १ | २८ | २३ | ३७ | १ लब्धिः

पूर्वलब्धेरधः स्थापिता २|५|३९ | २६ | २२ | २९ | २८ | ३२ | ५२ अस्य घनः ९ | ११ | ८ | १९ | २९ | ८ | ५७ | ३९ | ४७ | ५० | १९ पुनरत्राष्टमघनः शोधितः शेषं ११ | २४ | १३ | १८
शेषे योजितं १ | ३९ | ४७ | ५० | १९ हरभक्तः लब्धं ३३ शेषं ० | ४७ |५० | १९ |
ल@
पूर्ववत् २ | ५ | ३९ | २६ | २२ | २९ |२८ | ३२ | ५२ | २२ एतपर्यन्तं गृहीतं अथात्र यावत्तावन्मानानयनार्थमुपायन्तरमाविदेन निष्कासितं तद्यथा रूपाणि यावत्तावता भाज्यानिलब्धेर्घनः कार्यः पुनर्घतो पि यावत्तावता भाज्यः यल्लब्धं तत्प्रथमलब्धौ योज्यं पुनस्वस्य्घनः कार्यः एवमसकृत्

अत्रोपपत्तिः
इह पूर्ववद्यावत्तावत भक्तेरूपराशौ कश्चिद्यावत्तावतो भागोलब्धः पुनस्तस्य घनं कृत्वा
तद्योजनेनवास्तं वघनस्यां स्वन्नताजाता एवमुहुः स्थिरी भूतं तदेव घनस्वरूपराशेर्यावत्तावतोभाग उपलब्धः स एवयावत्तावन्मा\-नमुपपन्नं
अत्रोदाहरणं रू ६ | १६ | ४९ | ७ | ५९ | ८ | ५६ | ३० इयं यावत्तावत् षष्टेत्द्विरूर्ध्$|$
\marginnote{f.~6r J}
परिवर्त्तेनं या ३ भक्ते यल्लब्धं फलं २ | ५ | ३६ | २२ | २९ | ४२ | ५८ | ५० अस्य घनं ९ | ? | २८ | ३ | ८ | ५२ | ५ | ३९
पुनरयं तेनैव या ३ भक्तः फलं ० | ० | ३ | ३ | २९ | २१ | २ | ५७ |
इदं प्रथमलब्धौ योजितं २ | ५ | ३९ | २६ | ९ | ४ | १ | ४७ अस्य घनं वारद्वयं षष्ट्या भक्तः जातो विकलात्मकः ० | ० | ९ | ११ | ८ | १६ | ३२ | ३० | ४८ | ९ पुनरयं तेनैव भक्तः लब्धं ० | ० | ३ | ३ | ४२ | ४१ | ३० | ५०
इदं प्रथमलब्धौ योजितं जातं ५ | ३९ | २६ | २२ | २८ | २९ | ४० अस्य घनः ९ | ११ | ८ | १९ | २८ | ३६ | २३ | ५० पुनस्तेनैव भक्तः लब्धं ० | ० | ३ | ३ | ४२ | ४६ | २९ | ३९ प्रथमलब्धौ योजितं २ | ५ ३९ | २६ | २२ | २९ | २८ | २९
अस्यघनः ९ | ११ | ८ | १९ | २९ | ८ | ५६ | ५१ पुनस्तेनैव भक्तः लभ्दं ० | ० | ३ | ३ | ४२ | ४६ | २९ | ४२ | इदं प्रथमलब्धौ योजितं २ | ५ | ३९ | २६ | १२ | २९ | २८ | ३२ अस्य घनः ९ | ११ | ८ | १९ | २९ | ८ | ५७ | २८
पुनस्तेनैव भक्तः लब्धं ० | ० | ३ | ३ | ४२ | ४६ | २९ | ४२ प्रथमलब्धौ योजितं २ | ५ | ३९ | २६ | २२ | २९ | २८ | ३२ अयं स्थिरीभूतः इयमंशद्वयस्य पूर्णज्याजाता

अथमिर्जोलुग्बेगोक्तप्रकारेणांशद्वयस्य पूर्णज्या निष्का ? ते तत्र पूर्वोक्तमेव क्षेत्रं
कार्यं तत्र वचिह्नात् वहलंबः कजरेखायां निष्काश्यः तत्र कबकत्रिभुजेन वहत्रिभुजे हकोणः समकोणो स्ति कबवर्गः कहहव$|$
\marginnote{f.~6v J}
योर्वर्गयोगतुल्यो स्ति पुनर्बजवर्गः जहहवयोर्वर्गयोगतुल्यो स्ति
पुनरत्रकवबजौ मिथस्तुल्यौ कल्पितौ वहमुभ\-योस्त्रिभुजयोरेकमेवास्ति तस्मात्कहहजौमिथः समानौ भविष्यतः कववर्गः वहस्य व्यासस्य च घातेन समानो स्ति तस्माद्यदिकबवर्गो व्यासेन भाज्यते तदा लब्धं बहप्रमाणं भवति ||
पुनर्यदि वहवर्गः कववर्ग्गाछोध्यते तदा शेषं कहवर्गो वशिष्यते अत्रांशद्वयस्य पूर्णज्यारूपाकवरेखा
यावत्तावन्मिताकल्पिता या १? अस्यवर्गः
%CM: Tहिस् लस्त् फ्रसे मय् हवे बेएन् इन्तेन्देद् तो बे च्रोस्सेद् ओउतित् इस् हर्द् तो तेल्ल्!
याव १ व्यासः १२० षष्ट्योर्ध्वा कृतः २ एकपरिवर्त्तः अनेन कववर्गो याव १ षष्टेरेकपरिवर्त्तेन २ भक्ते लब्धं
यावद्वर्गादर्धकलासाचयावद्वर्गस्य त्रिंशद्विकलात्मिका ३०
यावत्तावतः अंशात्मकत्वादिदं वहप्रमाणजातं अस्यवर्गः यावद्वर्गवर्गस्य
%CM: थे सेचोन्द् वर्ग इस् अ मर्गिनल् अद्दितिओन्/चोर्रेच्तिओन्
पंचदशप्रतिविकलाः जाताः १५ इदं कववर्ग्गाछोधितं शेषं अहवर्गः याव १ याववतिविकलात्मकहवर्गः कजवर्गस्य चतुर्थांशत्तुल्यो स्ति तस्मात्कजवर्ग्गः याव ४ यावव १
विकला अथमिज स्ती ग्रंथस्य प्रथमध्यायस्यद्व्हितीयक्षेत्रे इसमुपपन्नं कजवदघातः कजवर्गरूपः
कबजदघा$|$
\marginnote{f.~7r J}
तस्य वजकदघातस्य च योगेन तुल्यो स्ति तत्र कदप्रमाणं ६ | १६ | ४९ | ८ | ५६ | ३० वजं अवतुल्यं
यावन्मितमस्ति तस्माद्वजं कदेन गुणितं सदेतावन्ति यावत्तावतानिजातानि या ६ | १६ | ४९ | ७ | ५९ | ८ | ५६ | ३० कवजदघातयाव १
तस्मादिदं याव ४ यावव १ विक@ अं अस्य याव १ या ६ | १६ | ४९ | ७ | ५९ | ८ | ५६ | ३० समानं जातं पुनरेतौ समशोधितौ तत्रैकपक्षे
याव ३ ? याव ध १ विक@
या ६ | १६ | ४९ | ७ | ५९ | ८ | ५६ | ३० एतौ समानौ स्तः अत्र समयोसंशौ चेद्गृहेते तदा तावपि
समानौ भविष्यतः अत्र पक्षौ त्रिभक्तौ याव
%%CM: wहत् fओल्लोwस् हस् बेएन् अद्देद् इन्
याव व २ या २ | ५ | ३५ | २२ | ३९ | ४२ | ४८ | ५० प्रतिविकला
%%CM: एन्द् ओf अद्दितिओन्
पुनरेऔ यावत्तावताय वर्त्तितौ यथा या १
याव २ रू २ | ५ | ३६ | २२ | ३९ | ४२ | ५८ | ५०
तत्रादृष्टांकस्य चिकीर्षास्ति यस्यांकस्य रूप राशि
स्थितांकस्य चांतरं तदंक घतस्य चिंशति
प्रति विकलात्मकं भवति स एवांक इष्टो भवति तस्योत्पादनेयमुपाय उपलब्धः रूपाणां घनः कृतः ९ | १० | २८ | ३ | ८ | ५२ | ५ | ३९
इदं विंशति प्रतिविकलाभिर्गुणितं ० | ० | ३ | ३ | २९ | २१ | २ | ५७ रूपेषु योजितं २ | ५ | ३९ | २६ | ९ | ४ | १ | ४७ अस्य घनः ९ | ११ | ८ | १६ | ३२ | ३० | ४८ | ९ पुनरिदं विंशतिप्रतिविकलाभिर्गुणितं ० | ० | ३ | ३ | ४२ | ४५ | ३० | ५०
रूपेषु योजितं २ | ५ | ३९ | २६ | २२ | २८ | २९ | ४० अस्य घनः $|$\marginnote{f.~7v J}
९ | ११ | ८ | १९ | २८ |५६ | २३ | ५० पुनस्तेनैवप्र@ वि@
२० गुणितः ० | ० | ३ | ३ | ४२ | ४६ | २९ | ३९
रूपेषु योजितः २ | ५ | ३९ | २६ | २२ | २९ | २८ | २९ पुनर्घन ९ | ११ | ८ | १९ | २९ | ८ | ५६ | ५१ पुनस्तेनैव प्र@ वि@ २० गुणितः ० | ० | ३ | ३ | ४२ | ४६ | २९ | ४२
रूपेषु योजितं २ | ५ | ३९ | २६ | २२ | २९ | २८ | ३२
इदमंश द्वयस्य पूर्णज्यारूप इष्टांको जातः पदास्य घनः क्रियते विंशाति प्रतिकलाभिर्गुण्यते
रूपेषु ज्योजिते तदा स एवांको भवति अथमिर्जोलुग्वेगस्य द्वितीयः प्रकारः तत्र कवजदं चापं षडंशानां कल्पं
यस्य वृत्तस्येदं चापं तद्वृत्तकेंद्रं तं कल्पनीयं पुनः प्रयेकं कवचापं वजचापं जदचापमं:सद्वयं कल्पितं
तवकजकदवजवदजदपूर्णज्याः संयोज्याः पुनः कवपूर्णज्याकजपूर्णज्याकदपूर्णज्यानाह ?
वचिह्नेष्वर्धकार्यं पुनस्नहरेखातद्भतवरेखाः संयोज्याः एतारेखाः प्रतेकंतासु पूर्णज्या सुलंवरूपाभविष्यन्ति पुनः कतरेखाव्यासार्धसंयोज्यं पुनः तत् अचिह्नेर्धतं कार्यं पुनः अचिह्नंकेन्द्रं कृत्वा
अकव्यासार्द्धेन वृत्तं कार्यं तत्र कहतकोणकझतकोणकवतकोणाः समकोणाः सन्ति तदा वृत्तार्धमेतत् हझवचिह्नेषु गमिष्यति पुनः हझझबवतहतरेखाः संयोज्याः हचिह्नात् हललंवऽह् कझपूर्णज्या सुनिष्काश्यः $|$\marginnote{f.~8r J}
तदा कझपूर्णज्या लचिह्नेष्वर्धता भविष्यति तत्र झवरेखया कजभुजकदभुजयोरर्धस्याने कार्त्ततं तदा झवरेखा\-जदरेखायाः समानान्तरा भविष्यति तस्मात्कजदत्रिभुजकझवत्रिभुजे मिथः सजातीयेभविष्यन्त कवं कदस्यार्धमस्ति झवंजदस्यार्धं भविष्यति
पुनरनेनैव प्रकारेण हझरेखावजस्यार्धं भविष्यति कहंकवस्यार्धमस्येव तस्मात् कहहझझवपूर्णज्यामिथः समाना भविष्यति
पुनः अचिह्नात् अललंवः कझपूर्णज्यायां निष्काश्यः अयं लंवः कझरेखांलचिह्ने र्द्धितां करिष्यति
पुनरयं वार्धतः सन् हचिह्ने लगिष्यति तस्मात्कहमेकांशज्यायावत्तावन्मिता कल्प्या पुनः कवं अंशत्रितयंस्यज्या एतांवत्यस्ति ३ | ८ | २४ | ३३ | ५९ | ३४ | २८ | १५
यदि कहवर्गः यावकतरेखया ६० भाज्यते तदा लब्धिरेकाकलायावद्वर्गस्य भविष्यति इदं हलप्रमाणमस्ति यथापूर्वमुपपन्नं अस्यवर्गः यावद्वर्गः यावद्वर्गवर्गस्यैका $|$\marginnote{f.~8v J}
विकला भविष्यति अयं कहवर्गे शोधितः शेषं कलवर्गो भविष्यति सच या १ यावव विकला कझवर्गस्य चतुर्थांशः ४
कलवर्गो स्ति तस्मात् कझावर्गः याव ४ यावव ४ विकला पुनः कझहवघातरूपः
कझवर्गः कझहवघातरूपवत् वर्गस्य हझकवघातस्य च योगेत तुल्यो स्ति
कहवर्गश्चैतावान् याव १ हझकवघातश्च ३ | ८ | २४ | ३३ | ५९ | ३४ | २८ | १५ अयं
द्वितीय प्रकारेण कझवर्गः सिद्धः एतौ समाविति समशोधनार्थं न्यासः
यथा याव ४ यावव ४ विकला याव १ या ३ |८ | २४ | ३३ | ५९ | ३४ | २८ | १५ अत्र
संप्रदायेन शोधने कृते सति प्रथमपक्षे याव ३ यावव ४ विकला या ३ | ८ | २४ |
३३ | ५९ | ३४ | २८ | १५ धने कृते सति प्रथमपक्षे अनयोस्त्र्यंशावपि समारौ
तस्मात्त्रिभिरपवर्त्तितौ अवशिष्टावेतावपिसमानौ याव १ यावव १ विरू १ | २ | ४८ | ११ |
१९ |११ | २९ |२५
%%CM: थेरे अरे अ fएw fओर्मत्तिन्ग् इस्सुएस् हेरे अन्द् ईऽम् नोत् सुरे अल्ल् इस् इन् थे रिघ्त् ओर्देर्:
%% छेच्क्!
पुनरेतौयावतावतापवर्त्तितौ जातौ पक्षौ या १ विरू १ | २ | ४८ | ११ १९ | ५१ | २९ | २५ याघ १ २ प्र@ वि@
यद्यत्रताद्यशोंको लभ्यते यस्यां कस्य पुनः रूपस्य चांतरं अंक धनस्यैककलाविंशति प्रतिविकला तुल्यं भवेत्
सचानेन प्रकारेण निष्कांशितः तत्ररूपानां घनः अंशादि १ | ८ | ४८ | ३० | २३ | ३६ | ३० | ४२ | १८ अयमेकयाकलयाविंशतिप्रतिवि$|$\marginnote{f.~9r J} कलाभिश्च गुणितः ० | ० | ३१ | ४४ | ४० | ३ | २८ | ४१ | अंशकलास्थानयोरभावात् स्थानद्वयेशुन्यं निवेशितं
रूपेषु संयोज्यजातं १ | २ | ४९ | ४३ | ४ | ३२ | ० | ५३ | ४१ अस्यघन १ | ८ | ५३ | ३२ | ४ | ३ | ५० | ५९ | १४ | ५७ पुनरयमेतेन १ २० गुणितः ० | ० | १ | ३१ | ५१ | २२ | ४५ | २५ | ८ रूपेषु योजितं १ | २ | ४९ | ४३ | ११ | १४ | १८ | ५० | ८ अस्यघनः १ | ८ | ५३ | ३६ | २६ | ७ | १८ | ० | २२ | १० पुनरयंतेनैव १ २० गुणित ० | ० | १ | ३ | ५१ | २३ | ४ | ४९ | २० रूपेषु संयोज्य १ | २ | ४९ | ४३ | ११ | १४ | ४४ | १६ | ३० अस्यघनः १ | ८ | ५३ | ३२ | २६ | ८ | ३७ | ५ | ३४ पुनस्तेनैव १ २० गुणितः ० | ० | १ | ३१ | ५१ | २३ | १४ | ५१ | २९
रूपेषुसंयोज्य १ | २ | ४९ | ४३ | ११ | ४ | ४४ | १६ | ३६ अयमिष्टांको जाता अत इयमेकांशज्यासिद्धा कृतः त्यतो स्य घने अनेन १ २० वि प्र
%%fओर्मत्तिन्ग् इस्सुएस् वि इस् अबोवे प्र इन्लिने wइथ् १ अन्द् २०.
गुणिते रूपेषु योज्यते तदा यमेव भवति तस्मादयमेव
एकांशज्याजाता इदमेवेष्टं अथात्रमिर्जोलग्वेगेन
यत् द्वितीयप्रकारेण क्षेत्रं प्रदर्शितं तत्र वहुरेखा संयोगः कृतं २? अथात्र यथाल्परेखाभिरे$|$\marginnote{f.~9v J}
वसिध्यति तथा यतितमाविदसंज्ञै ?
तत्र तावदेवकवजदचापं पूर्वोक्ति एवपूर्णज्याः
संयोज्या कवपूर्णज्यार्धरूपं कहं एकांशज्या स्ति
सायावन्मिता कल्पिता या १ कवरेखा या २ एवं वजय २ जदया २ यत एतामिथः
समानाः संति पुनः कदरेखां ६ | १६ | ४९ | ७ | ५९ | ८ | १६ अनयावजंया २?
गुणितं कवजदघातयुतं जातं याव ? या १२ | ३३ | ३८ | १५ | ५८ | १६ | ३३ | ० अयं
जातः कजवर्गः पुनः प्रकारांतरेण कजवर्गः साध्यते ? कववर्गः याव ४
अयं व्यासेन २ षष्टेरेकोर्धपरिवर्त्तेन भक्तो लब्धं

क्षेत्र

यावद्वर्गस्य कला २ इदमंशद्वयस्य उत्क्रमज्या जाता अस्या वर्गः यावव ४
विकला कववर्गे शोधितः शेषं याव ४ यावव ४ विकला अयमंशद्वयज्यावर्गो
जातः अयं चतुर्गु$|$णितः
\marginnote{f.~10r J} याव १६ यावव १६ विक@ अयं
सिद्धः कजवर्गः एतौ पक्षौ समौ समशोधिअनार्थं न्यासः याव ४ या १२ |
३३ | ३८ | १५ | ५८ | १६ | ३३ शोधिते शेषं पुनर्द्धाद
%CM: सोमे fओर्मत्तिन्ग् इस्सुएस् तो रेसोल्वे
याव १ यावव १६ विकला

याव १२
यावव १६ या १२ | ३३ | ३८ | १५ | ५८ | १६ | ३३ यावत्तावतापवर्त्त्य इदं मिर्जोलुग्वेगेन निष्काशित वहुप्रयाशेन

या १ याव १ २० विरू १ | २ | ४८ | ११ | १९ | ५१ | २९ | २५}

 
%  ${\rm rest of f.4v blank}$
%\end{document}


{\s  
श्रीगणेशाय नमः || \marginnote{f. 1r $J_2$}
अथैकांशजीवाविषय\footnote{{\s @विषये } $J_1$}
उलुग्वेगीजीकस्य
शरहविर्जन्दीस्य\footnote{{\s @विर्जंदीस्थ }${\rm J_1}$}
व्याख्या लिख्यते ||
तत्रैकांशजी\-वान\-यनेन व्यतरं प्रकारद्वयमस्ति || 
एकं यमशैदकाशीसंज्ञेन कृतम् ||
द्वितीयं मिर्योलुग्वेगेन कृतम् ||
परंचोलुग्वेगस्य यमसैदकाशीवेधप्रक्रियायां
सा हाथकार्यस्ति|
तत्र प्रथमप्रकारः कथ्यते| सं यथा ||
अवजदं षडंश ६ चापं कल्पितम् ||
पुनरस्य समानं भागत्रयं बचिह्नजचिह्नयोः कृतं
तत्र अबमजमहं वजं बदं जदं चैताः पूर्णज्यारेखा
योज्याः ||
अथात्रपूर्वोक्तप्रकारेणांशत्रयस्य\footnote{{\s अथात्रपूर्वाक्त@}${\rm J_1}$}
जीवा ज्ञातास्ति सा चेयं ३| ८| २४| ३३| ५९| ३४| २८| ५४| ५०
इयं द्विगुणा ६| १६| ४९| ७| ५९| ८| ५६| २९| ४० जाता मदपूर्णज्या ||

अत्रांशद्वयस्याबपूर्णज्या ज्ञातमिष्टमस्ति ||
तत्र मिजिस्तीग्रन्थस्य
प्रथमाध्यायस्थद्वितीयक्षेत्र
इदमुपपन्नम् ||\footnote{{\s @द्वितीयक्षेत्रयिदमुप@ }${\rm J_1}$}
यच्चतुर्भुजवृत्तान्तः पतति
तत्सन्मुखस्थभुजानां\footnote{{\s तत्संन्मुख@ } ${\rm J_1}$}
घातयोगस्तच्चतुर्भुजान्तः पतितकर्णद्वयघातसमो भवति ||
पुनस्तत्रैव प्रथमाध्यायस्य
चतुर्थक्षेत्र इदमुपपन्नम् ||\footnote{{\s चतुर्थक्षेत्रे इदमुप@ }${\rm J_1}$}
इष्टचापार्धपूर्णज्यावर्गस्तेनाङ्केनसमो भवति ||
यो कः %\\(|\\)
 तच्चापोनभार्धांशानं या पूर्णज्या भवत्यस्या
व्यासेन यदन्तरं तेन गुणित यो
व्यासार्धस्तदङ्कतुल्यो\footnote{{\s व्यासार्धः स्तदंक@ }${\rm J_1}$}
भवति ||

एवमत्र अबचापजदचापे मिथः समाने स्थः|
एवं अजबदचापे च समाने स्तः ||
तस्मात् अबजदघातो वर्गो भविष्यति ||
अत्र बजमज्ञातं तद्यावत्तावन्मितं कल्पितमिदमदेनगुणितं
जातं वा ६| १६|४९| ७| ५९| ८| ५६| २९| ४०|
एवं अबजदघातरूपवर्गराशेर्यावत्तावतश्च योगः
अजवर्गतुल्य अनबदथा तेन समानो ष्ति|

अतो .अयं अजवर्गतुल्यो जातः ||
तस्मात्
अजवर्गराशिरेकोवर्गराशिरेभिर्यावत्तावद्भिरधिको
\footnote{{\s अजवर्गराशी एको@ }${\rm J_1}$}
जातः|
याव १ या ६| १६| ४९| ७| ५९| ८| ५६| २९| ४०

अथ
प्रकारान्तरेण अजवर्गः\footnote{{\s प्रकारांतरेणाजवर्गः }${\rm J_1}$}
साध्यते ||
तत्र अजचापं भार्धांशेभ्यः शोध्यं शेषस्य
पूर्णज्याया व्यासस्य चान्तरं कार्यं तेन घ्न षष्टितुल्यव्यासार्धं
अववर्गतुल्यं भविष्यति ||
यतः अजचापं अवचापाद्द्विगुणमस्ति ||
तस्माद्यदि अववर्गो\footnote{{\s अवबर्गः} ${\rm J_1}$}
याव १ षष्टि ६० तुल्यव्यासार्धेन भाज्यते तदा यावद्वर्गस्य
षष्ट्यंशो लभ्यते तच्च अजचापानभार्धानां या पूर्णज्या
तदूनो यो व्यासस्तस्य प्रमाणमस्ति ||
पुनरिदमन्तरं संपूर्णव्यासे १२० शोधितं
याव \upbefore{०}१ \downafter{ ६०} रू १२०
इदं अजचापोनभार्धांशानां पूर्णज्या जाता ||
अस्या वर्गः %\\(|\\)
यावव \upbefore{१}३६०० याव \upbefore{०}२४०\downafter{६०}
रू १४४००
अथ च यदीष्ष्तचापपूर्णज्यावर्गो व्यासवर्गाच्छोद्यते
तत्र शेषं
$|$\marginnote{f. 1v $J_2$}
तच्चापोनभार्धांशपूर्णज्यावर्गो
भवति ||\footnote{{\s भावति }${\rm J_1}$}
यतो व्यासेनेष्टचापे पूर्णज्यया इष्टचापोनभार्धांशपूर्णज्यया
चैकसमकोणत्रि\-भुजं भवति यतो वृत्तार्धे व्यासप्रान्तादुत्पन्नत्रिभुजस्य
पालिकोणः समकोणो
भवतीत्युक्लीदसस्य
तृतीयाध्याय\footnote{{\s तृतीयाध्याये उप@ }${\rm J_1}$}
उपपन्न\-मस्ति ||
तत्त्रिभुजे
समकोणसन्मुखभुजो व्यासो .अस्ति ||
स च कर्णरूपः ||
पुनः कर्णवर्गो भुजकोट्योर्वर्गयोगेन समो
भवति|\footnote{{\s भवती| }${\rm J_1}$}
भवतीत्युक्लीदसस्य
प्रथमाध्याय\footnote{{\s प्रथमध्याये उप@ }${\rm J_1}$}
उपपन्नम् ||

अतो .अत्र व्यासवर्गः १४४०० कर्णवर्गरूपः ||
अस्मिनिष्टचापोनभार्धांशानां पूर्णज्यावर्गः
यावव \upbefore{१}३६०० याव \upbefore{०}२४०\downafter{६०}
रू १४४०० %%%%% 
शोधितः शेषं अजपूर्णज्यावर्गो .अवशिष्टः ||
यावव \\wहितेस्पचे\\wहितेस्पचे\\wहितेस्पचे
\upbefore{०}१ \downafter{ ३६००} या \upbefore{२४०}६०
अयं
द्वितीयप्रकारेण अजवर्गः\footnote{{\s @प्रकारेणा .अजवर्गः }${\rm J_1}$}
सिद्धः ||
अस्यप्रथमप्रकारागत अजवर्गः याव १ या ६| १६| ४९| ७| ५९| ८| ५६| २९| ४०
एतौ समावितिसमशोधनार्थं न्यासः
"\\न्ल्"
\Column{यावव \\wहितेस्पचे\\wहितेस्पचे\upbefore{०}१ \downafter{३६००}}
"{८एम्}"
\Column{याव \upbefore{२४०}६०}"{६एम्}"
\Column{या ०}"{३एम्}"
%"\\न्ल्"
\Column{यावव ०}"{८एम्}"
\Column{याव १}"{६एम्}"
\Column{या ६| १६| ४९| ७| ५९| ८| ५६| २९| ४०}"{१६एम्}"

अथ यावनीयबीजगणिते समीकरणसंप्रदायस्त्वनया रीत्यास्ति
सा यथा ||
यदि समयोः पक्षयोर्मध्य एकाराशिरृण\-श्चेत्तद्राशितुल्यं
\footnote{{\s पक्षयोर्मध्ये एकाराशिः ऋणश्चेत् तद्राशि@ }${\rm J_1}$}
पक्षद्वये योज्यं\footnote{{\s पक्षद्वयो जोज्यं }${\rm J_1}$}
तदापि पक्षद्वयं सममेव भवति ||
तस्मादत्र प्रथमपक्षे यावव \\wहितेस्पचे\\wहितेस्पचे\\wहितेस्पचे
\upbefore{०}१ \downafter{ ३६००}
इदं पक्षद्वये योज्यं तत्र प्रथमपक्षे
धनर्णयोस्तुल्यत्वाद्यावद्वर्गवर्गस्य नाशो .अवशिष्टो
याव \upbefore{२४०}६०
अयं छेदभक्तो याव ४\\(|\\) अयं प्रथमपक्षः पुनर्द्वितीयपक्षे
यावव \upbefore{१}३६०० योजितः यावव \upbefore{१}३६००
याव १ या ६| १६| ४९| ७| ५९| ८| ५६| २९| ४०
एतौ पक्षौ पुनरपि समौ
"\\न्ल्"
\Column{याव ४}"{८एम्}"
\Column{या ०}"{४एम्}"
"\\न्ल्"
\Column{यावव \upbefore{१}३६००}"{८एम्}"
\Column{याव १ या ६| १६| ४९| ७| ५९| ८| ५६| २९| ४०}"{२८एम्}"

अथयावनीयसंप्रदाये समपक्षयोर्मध्ये यौ
राश्येकजाती भवतः ||
तत्र लघुराशिर्महद्राशौ शोध्यते तदापि पक्षौ समावेदावशिष्टौ
भवतः ||
तस्मादत्र यावर्गराशिरेकजाती यो .अस्त्यतो लघुराशिर्याव १
महद्राशौ याव ४ शोधितः शेषं प्रथमपक्षो याव ३
द्वितीयपक्षे च यावव \upbefore{१}३६००
या ६| १६| ४९| ७| ५९| ८| ५६| २९| ४०
एतावपि समौ ||
पुनः\footnote{{\s पुन } ${\rm J_1}$}
पक्षद्वयमध्ये यद्येकोपक्षः खण्डितो द्वितीयपक्षः
संपूर्णश्चेत्तदा खण्डितराशिं प्रपूर्यं
तावद्गुणितं द्वितीयपक्षमपि
कुर्वति ||\footnote{{\s कुर्वंति }${\rm J_1}$}
अत्र सछेदराशिः खण्डितशब्देनोच्यते|
छेदरहितः संपूर्णोच्यते ||\footnote{{\s संपूर्ण उच्यते }${\rm J_1}$}
एवं कृते सति पक्षयोः समछेदत्वं\footnote{{\s पक्षयोसम@ }${\rm J_1}$}
विधाये छेदाय ${\rm ?}$
गम एवोपपद्यते ||
तस्मादत्र द्वितीयप [क्षे \marginnote{f. 2r $J_2$} 
खण्डितराशिर्यावव उप्बेfओरे{१}३६००



अयमर्थः %\\(|\\)
अत्र यावद्वर्गवर्गराशेषट् शताधिकसहस्रत्रयमितोंशो
यावद्वर्गवर्गराशिरस्ति\\(|\\)
तस्मादयं हरगुणितं क्रियते ताव १ "{\\र्म् ?}" द्यावद्वर्गवर्गराशिरेको
भवतीति\footnote{{\s भवती }"{\\र्म् wइथ्}" ति "{\\र्म् इन्सेर्तेद् इन् लेfत् मर्गिन् J१}"}
द्वितीयपक्षमप्येतावता गुणनीयम् ||
एवं कृते
प्रथम\\(प\\)क्षे\footnote{{\s प्रथमक्षे }${\rm J_1}$}
याव १०८०० द्वितीयपक्षे यावव १ अस्य प्रथमपक्षो
वारद्वयं षष्टोध्व "{\\र्म् ?}" कृतो याव ३\\(|\\)
अथ द्वितीयपक्षस्थयावद्राशिस्तेनैव छेदेन ३६०० 
संगुण्य वारद्वयं षष्ट्यो ३ र्ध्व १ "{\\र्म् ?}" कृतो
या ६| १६| ४९| ७| ५९| ८| ५६| २९| ४०
एवम् जातौ पक्षौ समौ
"\\न्ल्"
\Column{याव ३ द्विपरिवर्त्ताः}"{१२एम्}"
"\\न्ल्"
\Column{यावव १ या ६| १६| ४९| ७| ५९| ८| ५६| २९| ४०}"{३०एम्}"
एतौ यावत्तावता पन्नत्ति "{\\र्म् ?}" द्विपरिवर्त्ता तौ
"\\न्ल्"
\Column{याव ३ द्विपरिव@}"{१२एम्}"
"\\न्ल्"
\Column{याद्य "{\\र्म् ?}" १ रू ६| १६| ४९| ७| ५९| ८| ५६| २९| ४० द्विपरिव@}"{३६एम्}"

अथात्रयावद्द्वयेन सरूपस्थराशिना
चेद्यावत्तावत्त्रयं\footnote{{\s चेद्यावत्तावत्रयं }${\rm J_1}$}
३ भाज्यते
तदा यावत्तावत्प्रमाणं लभ्यते|
अस्य भागग्रहणरीतिः पूर्वाचार्पैरे "{\\र्म् ?}" तावत्कालपर्यन्तं
न लब्धा|
अत्र यमशैदेन रीतिः प्रदर्शिता ||
सा यथा ||
रूपस्य प्रथमाङ्को यावत्तावता भाज्यः\\(|\\) लब्धेरेकान्ते
स्थाप्या ||
पुनर्लाब्धिघनं शेषाङ्के योज्यं पुनरत्र
द्वितीयाङ्को\footnote{{\s  द्वितीयांकौ }${\rm J_1}$}
यावत्तावता भाज्यः|
लब्धिपूर्वलब्धेरधस्\-थाप्यः ||
पुनर्लब्धिद्वययोगस्य घनः कार्यः\\(|\\)
एवं तत्र प्रथमलब्धिघनः शोध्यः ||
शेषं द्वितीयलब्धिशेषे योज्यम् ||\footnote{{\s ज्योज्यं }${\rm J_1}$}
पुनस्तृतीयाङ्को यावत्तावता भाज्यः ||
इयं लब्धिः पूर्वलब्धिद्वयाधस्थाप्यः ||
पुनरस्य लब्धित्रययोगस्य घनः कार्यः\\(|\\)
तत्र लब्धिद्वययोगस्य घनः शोध्यः ||
शेषं तृतीयलब्धिशेषे योज्यम् ||
पुनस्तत्रचतुर्थाङ्को यावत्तावता भाज्यः ||
एवमेते "{\\र्म् ?}" ष्टभागलब्धिपर्यन्तं विधिः कार्यः ||
एवं यमशैदेन\footnote{{\s यमशदैन }${\rm J_1}$}
यावत्तावत्प्रमाणो निष्काशितः २| ५| ३९| २६| २२| २९| २८| ३२| ५२| ३३
इयमंशद्वयस्य\footnote{{\s इयंमंश@ }${\rm J_1}$}
पूर्णज्यास्ति ||
अस्यार्धमेकांशज्या जाता १| २| ४९| ४३| ११| १४| ४५| १६| १६| १७
अस्य भागहरणोपपत्तिः ||
तत्र पक्षद्वयमध्य एकपक्षे\footnote{{\s पक्षद्वयमध्ये एक@ }${\rm J_1}$}
यावत्तावदस्ति ||
द्विपक्षे यावद्यतः $?$ रूपराशिश्च ||
एवं तत्र यावत्तावद्ज्ञानं चेत्तदा यावत्तावद्घनं कृत्वा
रूपराशौ प्रक्षिप्य यावत्तावता भाज्यते तत्र लब्धिर्यावत्तावन्मानं
स्यात् ||
अत्र तु यावत्तावद्ज्ञानं\footnote{{\s यावत्त्रवद्ज्ञानं }${\rm J_1}$}
नास्त्यतो रूपराशिरेव यावत्तावता भाज्यः\\(|\\)
यल्लब्धं तत्सरूपयावघनस्य को .अप्यंशो लब्धः
स चैकान्ते धृतः ||
पुनरस्य घनं कृत्वा $|$ \marginnote{f. 2v $J_2$}
शेषे जोतितुं पुनरत्र यावत्तावता द्वितीयां को भाजितः!!
एवं यल्लब्धं तत्तु\footnote{{\s त्तत्तु }${\rm J_1}$}
पूर्वलब्धेरथ स्थापितम् ||
एवं लघयावत्तावद्व्यनस्यासन्नो भागो
\footnote{{\s @स्वासन्नौ भोगो } with strokes stuck out $J_1$} %"{\\र्म् wइथ् स्त्रोकेस् स्त्रुच्क् ओउत्, J१}"}
लब्धः %\\(|\\)
अत्रासन्नता शेषावयवसत्वाद्निरवयवत्वे स एव सूक्ष्मा
लब्धिः ||
अथलब्धिघने पूर्वघनं शोध्यं यतस्तदधिकं जातम् ||
एवमिष्टभागपर्यन्तं मुहुर्मुहुः कार्यम् ||

अथास्योद \\qउएर्य्
तत्र द्वितीयपक्षे रूपाणि ६| १६| ४९| ७| ५९| ८| ५६| २९| ४० ||
अत्र षट् ६ द्विपरिवर्ताः षष्टेरूर्ध्वमस्ति ||
यावत्तावत्त्रयं द्विपरिवर्ताः षष्टेरूर्ध्वमस्ति ||
अत एकजातौ\footnote{{\s अतो एकजातौ }${\rm J_1}$}
भागे गृहीते लब्धमंशाः ||
पुनरस्य घने .अंशाः ८ शेषां १६ शेषु ५९ योजितं
१६| ५७ पुनः पूर्वलब्धे शेषं २८ तेनैव या ३
भक्तं लब्धकलाः ५ लब्धिः पूर्वलब्धेरदस्थोपित अं २
क ५ पुनरत्र शिष्ठं १| ५७ पुनर्लब्धिद्वयस्यघनः\\(|\\)
अं ९| २| ३२| ५ अत्र प्रथमलब्धिघनः\\(|\\)
अं ८ शोधितः शेषं अं १| २| ३२| ५७|
इदं शेषे अं १| ५७| क ७| वि ५९| ८| ५६| २९| ४०\footnote{{\s १| ५७| %\\उपfतेर्{अं} ७|\\उपfतेर्{क} ५९\\उपfतेर्{वि}|
८| ५६| २९| ४० } ${\rm J_1}$}
अंशस्थाने योजितं \\(|\\) अं १| ५८| १०| ३१| १३\footnote{{\s १| ५८| %\\उपfतेर्{अं}
 १०| ३१| १३} ${\rm J_1}$}
पुनो हरेण\footnote{{\s पुनः हरेण} ${\rm J_1}$}
या ३ अंशे ५८ भोगो\footnote{{\s अंशे \upbefore{५८}भोगो  }${\rm J_1}$}
गृहीतः \\(|\\)
लब्धिर्विकलात्मिका ३९\footnote{{\s लब्धिर्विकालात्मिका ३९ } ${\rm J_1}$}
शेषं अं १| १०| ३१| १३ पुनरियं लब्धिः पूर्वलब्धेरधःस्थाप्यः
अं २| ५| ३९
अस्य व्यनोशादि}
 %\query



 



\end{document}
