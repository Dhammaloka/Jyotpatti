\documentclass[12pt]{book}
\usepackage[a5paper]{geometry}
\usepackage{geometry}
%\usepackage[greek,french,latin,english]{babel}
\usepackage[no-math]{fontspec}
\usepackage[english]{babel}
\usepackage{polyglossia}
\setmainlanguage{english}
\setotherlanguage{sanskrit}
%\defaultfontfeeatures{Ligatures=TeX}
\setmainfont{Linux Libertine O}
\newfontfamily\s[Scale=1, Script=Devanagari]{Shobhika-Regular}
\usepackage{color}

%for comments and version control shouty notes
\usepackage{todonotes} %add comments in working doc for others to see
\newcommandx{\unsure}[2][1=]{\todo[linecolor=red,backgroundcolor=red!25,bordercolor=red,#1]{#2}}

%for images
\usepackage{graphicx}
\usepackage{caption}
\graphicspath{{./ksetra/}}



\usepackage{marginnote}


\parskip = \baselineskip
\parindent = 0pt

\let\*=\d

\def\Ganesa{Ga\-\*ne\-\'sa}
\def\Bhaskara{Bh\=a\-ska\-ra}
%
\def\KKT{\textit{Kara\-\*na\-kut\=u\-hala\-\*t\={\i}\-k\=a}}
\def\KKe{\textit{Kara\-\*na\-ke\-\'sari}}
\def\KheSi{\textit{Khe\-cara\-siddhi}}
\def\JCU{\textit{Jy\=a\-c\=a\-potpa\-tti}}
\def\PhCC{\textit{Phira\.ngi\-candra\-cchedyayopa\-yogika}}
\def\Mahadevi{\textit{Mah\=a\-dev\={\i}}}
\def\RG{\textit{Rekh\=a\-ga\*ni\-ta}}
\def\Lil{\textit{L\={\i}l\=a\-va\-t\={\i}}}
%
\def\kalpa{\textit{ka\-lpa}}
\def\kala{\textit{ka\-l\=a}}
\def\kalas{\textit{ka\-l\=as}}
\def\ghatika{\textit{gha\*tik\=a}}
\def\jyotisa{\textit{jyo\-ti\-\*sa}}
\def\purnajya{\textit{p\=ur\*najy\=a}}
\def\pratikalas{\textit{prati\-ka\-l\=as}}
\def\prativikalas{\textit{prati\-vi\-ka\-l\=as}}
\def\bijaganita{\textit{b\={\i}ja\-ga\*ni\-ta}}
\def\ya{\textit{y\=avat\-t\=avat}}
\def\yava{\textit{y\=avat\-t\=avat-varga}}
\def\yagha{\textit{y\=avat\-t\=avat-ghana}}
\def\yavava{\textit{y\=avat\-t\=avat-varga-varga}}
\def\ru{\textit{r\=upa}}
\def\vikala{\textit{vi\-ka\-l\=a}}
\def\vikalas{\textit{vi\-ka\-l\=as}}
%
\def\zij{\textit{z\={\i}j}}
%
\def\opcit{\textit{op.\ cit.}}
\def\loccit{\textit{loc.\ cit.}}
\def\elp{$\ldots$}
\def\danda{$|$}
\def\nl{\hfill\break}
%
\def\sri{\textit{\'sr\={\i}}}
%

\begin{document}

\begin{center}
{\s ज्याचापोत्पत्तिः |} \\
\end{center}
\bigskip

{\s $|$\marginnote{f.~1r J}  श्रीगणेशाय नमः |
अथांशत्रयस्य जीवाज्ञाने सत्येकांशजीवाज्ञानं
दोर्ज्यामिति-श्लोकचतुष्टयेनाह |
करसङ्गुणितां द्विगुणां भुजज्यां
त्रिभक्तां विधाय
गणकप्रवीणः फलमंशास्तान्पृथक्स्थापयेत् |
तु परं तेषां फलरूपांशानां घने व्यासार्धवर्गभक्ते सति
यत्फलं तद्भागावशेषसहितं कृत्वा त्रिभक्तं कार्यं |
त्रिभक्ते यत्फलं कलात्मकं तत्पूर्वलब्धांशरूपफलाधः स्थाप्यं सांशकलात्मिका पङ्क्तिर्जाता |
पङ्क्तेर्घनः प्रथमघनेन हीनः 

व्यासार्धवर्गभक्तः फलं शेषे योज्यत्रिभिर्भजेत् |
लब्धं फलं विकलां कलाधः स्थाप्याः साम्शकलाविकलात्मिका
पङ्क्तिर्जाता |
पङ्क्तेर्घनः कार्यः | 
अस्मिन्पूर्वघनः शोध्यः शेषे त्रिभज्यावर्गभक्ते यत्फलं तद्भाज्यावशेषसहितं
कृत्वा मुहुर्वारंवारं तदग्रे तु परं पुरोक्तो विधिः कार्यः | एवं कृते फलं पङ्क्तिः स्यात् |
पङ्क्त्यर्धं चापस्य तृतीयांशज्या स्यात् | }

\newpage


\begin{center}
{Jyācāpotpattiḥ}
\end{center}

[\kalas\ are sixtieths of a degree, \vikalas\ sixtieths of a \kala]
\medskip

$\marginnote{f.~1r J}$Homage to Lord \Ganesa. Now, ``When there is knowledge of the Sine of three degrees,
knowledge of the Sine of one degree'': thus he states the Sine with four verses.
Having set the Sine of the arc multiplied by two, multiplied by the kara [?] and divided by three, the 
clever mathematician should set those down separately (the result [is] degrees). 
But when the cube of the degrees of the number [?\unsure{Sine of one degree}] of the result is divided
by the square of the half diameter, whatever is the result, having added to the 
remainder [??] of the degrees of that, is to be divided by three. When divided by
three, whatever is the result in minutes, that is to be set down below the degree-number-result
of [?] the previous quotient. The row in degrees and minutes results. 

The cube of the row diminished by the first cube [is] divided by the square of the 
half-diameter. The result, to be added [?] to the remainder, one should divide by three.
Having set the quotient  result below the seconds-number-minutes, it results a row 
in degrees, minutes, seconds.  The cube of the row is to be made. In it, the cube of
the previous is to be subtracted. The remainder is [?] divided by three when divided
by the square, [???] whatever is the result, having increased that by the remainder
of the dividend, again turn by turn [??] in the tip of that but the previous rule is to be 
done. When done thus, [it] should be the result row.  The Sine of a third part of the 
arc  of half the row should be.

\newpage


{\s अथ इत्यनन्तरं तां त्रिभागजीवां पुनः प्रकाराद्वच्मि |}

\vskip15pt
{\s यतः कारणादनेकैर्भेदैर्गणकस्य बुद्धिः स्वशास्त्रे प्रवीणतामेति || १६ ||}

\vskip15pt
{\s पुनः प्रकारान्तरेण त्रिभागजीवामाह || स्वत्र्यंशहीनेति |}

\vskip15pt

\begin{verse}
{\s भुजज्या स्वतृतीयांशेन हीना पृथक्स्थाप्या |\\
अस्याः घनस्त्रिगुणस्य कृत्या त्रिघ्न्या भक्तः (|)}\&
{\s लब्धफलेन पृथक्स्था भुजज्या युता
कार्या \\
मुहूरनया क्रियया पङ्क्तिः साध्या
पङ्क्त्यर्धं चापगुणांशजीवा
धनुस्त्र्यंशज्या स्यात् || १७ ||}\&
\end{verse}
\newpage


Now, thus immediately that Sine of three degrees again from the method I will say.

``Since from the cause by many bhedas the wisdom of the mathematician in 
his own science cleverness'' \danda16\danda. Again, by another method he states the Sine
of three degrees.  ``Diminished by its own third part''. 
Diminished by its own third part of the Sine
of the arc, [it] is to be set down separately.  The cube of it [is] multiplied by three [\elp]
With the quotient-result the separate-standing Sine of the arc is to be added. Again
and again with that made a row is to be determined; the Sine of degrees multiplied
by the arc of half the row should be the Sine of a third part of the arc. \danda17\danda
\newpage 




\vskip15pt
{\s पुनः प्रकारान्तरेणाह | भुजांशजीवेति |}

{\s अस्योपपत्तिः | सा यथा |
कबजदषडंशचापं कल्पितम् | 
पुनरस्य समानं भागत्रयं कल्पितं बचिह्नजचिह्नयोः |
तत्र कबं कजं कदं बजं बदं जदं चैताः
पूर्णज्यारेखा योज्याः |
अथात्र पूर्वोक्तप्रकारेणांशत्रयस्य जीवा ज्ञातास्ति | सा चेयं । 
३| ८| २४| ३३| ५९| ३४| २८| १४| ५०| इयं द्विगुणा
६| १६| ४९| ७| ५९| ८| ५६| २९| ४० जाता कदपूर्णज्या |
अत्रांशद्वयस्य कबपूर्णज्याज्ञानमिष्टमस्ति |
\edtext{अस्योपपत्त्यर्थं}{\Dfootnote{{\s अस्योपपत्य } $J$}}
किंचिदुच्यते |
तत्र
\edtext{सिद्धान्तसम्राजिप्रथमाध्यायस्य}{{\lemma{\s -सम्राजि-}\Dfootnote{{\s -संम्राजि-} $J$}}} 
द्वितीयक्षेत्रे इदमुपपादितं |
यच्चतुर्भुजं वृत्तान्तः पतति तत्सन्मुखस्थभुजानां घातयोगः
तच्चतुर्भुजान्तः
\edtext{पतितकर्णद्वयघातसमो }{\lemma{\s पातिकर्णद्वय-}\Dfootnote{CHECK: it was hardcoded $J$}}%"\\३७Fw\\३९२wय् {\\इत् x} आGआत्\\३७Fw {\\र्म् J}"}
%% KP: हर्द्चोदेद् वरिअन्त्
भवतीत्यस्योपपत्तिस्तु पूर्वं कथितैवास्ति |}

\newpage


Again, he states by another method: ``The Sine of the degrees of arc\elp''. Its demonstration:

$\marginnote{f.~1v J}$A third part of the Sine of the degrees of arc is to be set down separately. Whatever
is the quotient when its cube
is added to its own degrees [?] [and] divided by the square of the Radius, the result
to  that is to be added separately. 

Thus again turn by turn, with that made, the Sine of a third part of the arc should be. \danda18\danda

Its rationale: it is as [follows]. $KBJD$ is considered an arc of six degrees. 
Again, three equal parts of this [arc] are made at points $B$ [and] $J$. \footnote{Kim reads: ``Again, its equal [?] three degrees is considered of point $B$ [and] point $J$''}.
Then $KB$, $KJ$, $KD$, $BJ$, $BD$, and $JD$: they are to be drawn as
Chord lines [\purnajya]. 

Now here by the previously stated method the Sine of three degrees is known, 
and it is this:  $3|8|24|33|59|34|28|14|50$. This is multiplied by two:
$6|16|49|7|59|8|56|29|40$ results the full Chord of $KD$. Here the knowledge
of the Chord $KB$ of two degrees is desired. \footnote{Kim reads: Here the knowledge is desired [of?] the Chord $KB$ of two degrees}

Something is said for the sake of its rationale. Then \unsure{there?} in the 
\textit{Siddh\=anta\-samr\=aj} [i.e., \textit{Siddh\=antasa\*mr\=at} translation
of the Almagest?] in the second figure of the first chapter this is
explained.  ``Whatever quadrilateral falls inside a circle, the sum of the 
products of the sides standing opposite \footnote{Kim reads: ``face to face''} becomes equal to the
product of the two diagonals falling \footnote{Kim reads: intersecting [?]} inside that quadrilateral''.
But its rationale is just explained previously. 

\newpage 


{\s पुनस्तत्रैव प्रथ\-माध्यायस्य चतुर्थक्षेत्रे इदमुपपादितम् |
\edtext{इष्टचापार्धपूर्णज्या-
वर्गस्तच्चापोनभार्धांशपूर्णज्या\-व्यासान्तर\-गुणित\-व्यासार्ध\-तुल्यो}%
{\lemma{\s गुणित\-व्यासार्ध\-तुल्यो}
\Dfootnote{{\s -गुणितद्व्यासार्धतुल्यो} $J$}} 
 भवति |
अस्योपपत्तिः |
कजव्यासे कबजदं वृत्तार्धं कार्यम् |
तत्र बदं \edtext{दजं}{\Dfootnote{inserted from left margin, $J$}}
सममस्ति |
कबरेखाकदरेखा\-बदरेखा\-जदरेखाः संयोज्याः |
दचिह्नाद्दकरेखा कजव्यासे लम्बरूपा कार्या |
अत्र जफरेखा कजव्यासस्य ज्ञातचापोन-भार्धांशपूर्णज्यायाश्चान्तरार्धमितास्ति |
कुतः |
कबं कहतुल्यं कार्यम् |
दहरेखायोज्या |
हकदत्रिभुज\-दकबत्रिभुजयोः ।
कहभुजकदभुजौ बकभुजक$|$%\\[
दभुजयोः
 \marginnote{f.~2r J}
क्रमेण समानौ | स्तः | पुनः हकदकोणः बफदकोणेन समोऽस्ति | तस्मात् दहरेखा बदरेखा समाना जाता | दजरेखापि समाना जाता | तदा दजवर्गः दफवर्गजफवर्गयोर्योगेन तुल्यो भविष्यति |
एवं दहवर्गदफवर्गफहवर्गयोगेन तुल्यो भविष्यति |
तस्मात् फहं जफं च समानं जातम् | अनयोर्योगः कजकबयोरन्तरमस्ति |} \\

\newpage

Then again, in the fourth figure \footnote{Kim reads: diagram} of the first chapter this is demonstrated [\texit{upap\=aditam}]. 
The square of the Chord of half the desired arc  becomes
equal to the half-diameter 
multiplied by the difference of the 
diameter and the Chord of the degrees of half a circle minus that arc .
Its rationale: On diameter $KJ$ 
half-circle $KBJD$ is to be made.  Then $BD$ is equal to $DJ$. From point $D$
line $DF$ is to be made perpendicular on diameter $KJ$.  Here line $JF$ is measured
by half the difference of 
diameter $KJ$ and of the Chord of the degrees of half a circle diminished by
the known arc.  Why?  [Just what I was wondering myself.] 
$KB$ is to be made equal to $KH$. Line $DH$ is joined.  [There is equality] of triangle
$HKD$ [and?] triangle $DKB$. Side $KH$ [and] side $KD$ to side $BK$
[and] side
$ \marginnote{f.~2r J}$
$KD$ are respectively equal. 
Again, angle $HKD$ is equal with angle $BKD$. From that, line $DH$ occurs equal
to line $BD$.  And line $DJ$ occurs equal. [?] Then the square of $DJ$ will become 
equal with the sum of the square of $DF$ [?] [and] the square of $JF$. In the 
same way  the 
square of $DH$ will become equal with the sum of the square of $DF$ [and]
the square of $FH$.  From that results equality of $FH$ and $JF$.  The sum of 
those two is the difference of $KJ$ and $KB$. 

\newpage

{\s पुनः कदजत्रिभुजे दसमकोणात् दफलम्बः कर्णोपर्यागतो स्ति ||
तस्मात् कजस्य निष्पत्तिर्जदेन तथास्ति
यथा जदनिष्पत्तिर्जफेन || तस्मात्कजजफघातः दजवर्गेण समानो जातः ||
तदेतदुपपन्नं इष्टचापार्धपूर्णज्यावर्गः चापोन-भार्द्धांशपूर्णज्याव्यासान्तरगुणितव्यासार्धतुल्यो भवतीति ||
शकलं~||}

\vskip15pt

{\s अथ प्रकृतमनुसरामः ||
एवमत्रकथितक्षेत्रे कबचापजदचापे मिथः समाने स्तः ||
एवं कजबदचापे च समाने स्तः || तस्मात्कबजदरेखाघातो वर्गो भविष्यति ||
अत्र बजमज्ञातन्तघावत्तावन्मितं कल्पितं || इदं कदेन गुणितं जातं या
६ | १६ | ४९ | ७ | ५९ | ८ | ५६ | २९ | ४०}

\vskip15pt

{\s [क्षेत्र]}

\vskip15pt

{\s अस्य कबजदघातरूप याव$|$\marginnote{f.~2v J}%
द्वर्गस्य च योगः
कजवर्गतुल्यकजबदघातेन समानो स्ति || अतो यं कजवर्गतुल्यो जातस्तस्मात्कजवर्गराशौ एको
वर्गराशिरेभि-र्यावत्तावद्भिरधिको जातः ||
या व १ या ६ | १६ | ४९ | ७ | ५९ | ८ | ५६ | २९ | ४० ||
अस्य
निदर्शनार्थं शकलं यथा || ||} \\


\newpage

Again, in triangle $KDJ$ from the right [?] angle $D$, perpendicular $DF$ 
is revolved hypotenuse [??].  From that, the coming out of $KJ$ is in the same
way with $JD$ as the coming out of $JD$ with $JF$. 
From that, the product of $KJ$ and $JF$ results equal with the square of 
$DJ$. That this [?] explained. ``The square of the Chord of half the desired arc
becomes equal to the half-diameter multiplied by the difference of the 
diameter and the Chord of the degrees of half a circle diminished by the arc.''
Figure [shakal]. 

Now, we pursue the nature.  [??] Thus here in the explained figure, arcs
$KB$ and $JD$ are equal together. In the same way, arcs $KJ$ and  $BD$ 
are equal.  From that, the product of lines $KB$, $JD$ will be a square.

Here, $BJ$ [is?] unknown, that is considered measured by \ya\ [variable]. 
This is multiplied by $KD$; result, \ya\ 

$6|16|49|7|59|8|56|29|40|$. 
Of it, and of the square of the \textit{y\=avat}---[that is,] 
the number [of the?] product of $KB$, $JD$---
$\marginnote{f.~2v J}$ 
the sum is equal to the product of $KJ$, $BD$ [or] equal to the square of $KJ$.

Here, this occurs equal to the square of $KJ$. From that, in the heap of the
square of $KJ$ one square heap [??], with these \ya s [it] occurs additive. [??]
\yava\ 1, \ya\ $6|16|49|7|59|8|56|29|40$. For the sake of its proof, a figure
[shakal] as [follows]: [?] 

\newpage

{\s प्रकारान्तरेण कजवर्गः साध्यते || तत्र कजचापं
भार्द्धांशेभ्यः शोध्यं शेषस्य पूर्णज्या या व्यासस्य चान्तरं कार्यं ||
तेन
गुणितं षष्टि तुल्यव्यासार्धं कबवर्गतुल्यं भविष्यति
यतः कजचापं
कबचापाद्विगुणमस्ति ||
तस्माद्यदि कबवर्गः याव १ षष्टि ६० तुल्यव्यासार्द्धेन यदा
भाज्यते तदा यावद्वर्गस्य षष्ट्यंशो लभ्यते तच्च
कजचापोनभार्द्धांशपूर्णज्याव्यासान्तरस्य प्रमाणमस्ति || पुनरिदमन्तरं
संपूर्णव्यासे १२० शोधितं याव १ ६० रू १२० इदं
कजचापोनभार्धंशानां पूर्णज्या भवति ||
अस्याः वर्गः याव व १ ३६०० याव
२४० ६० रू १४४०० |}

\newpage

By another method [??], the square of $KJ$ is determined.  There the arc of $KJ$ 
is subtracted from the degrees of half a circle. The difference of the Chord of the
remainder and the diameter is to be made. The half-diameter equal to sixty 
multiplied by that will be equal to the square of $KB$. Since the arc of $KJ$
is twice the arc of $KB$.  From that, if the square of $KB$ \yava\ 1 when it is
divided by the half-diameter equal to sixty 60, then a sixtieth part of the 
\yava\ remains, and that is the amount of the difference
of the diameter [and] the Chord of the degrees of a half-circle
diminished by the arc of $KJ$.  But this difference [is] \yava\ $\frac{1}{60}$, units 120
subtracted from the 
full diameter 120. This becomes the 
Chord of the degrees of half a circle diminished by the arc of $KJ$. 
Its square \yavava\ $\frac{1}{3600}$, \yava\ $\frac{240}{60}$, units 14400. 

\newpage

{\s अथ च यदीष्टचापपूर्णज्यावर्गः व्यासवर्गा छोध्यते तत्र यछेषं तच्चापोन-भार्द्धांशानां पूर्णज्यावर्गो भवति ||
यतो व्यासेन इष्टचापपूर्णज्यया
इष्टचापोन भार्द्धांशानां पूर्णज्यया चैकं समकोणत्रिभुजं भवति ||
यतो वृत्तार्द्धे
व्यासप्रांता उत्पन्नत्रिभुजस्य पालिकोणः समकोणो भवतीति रेखागणितस्य तृतीयाध्याये उपपन्नमस्ति ||
तत्त्रिभुजे समकोणसन्मुखभुजो व्यासो स्ति || सचकर्णरूपः || पुनः
कर्णवर्ग्गः भुजकोट्योवर्ग्गयोगेन समो $|$\marginnote{f.~3r J} भवतीतिरेखागणितस्य
प्रथमाध्यये उपपन्नं || अतो त्र व्यासवर्ग्गः १४४०० कर्णवर्ग्गरूपः ||
अस्मिन्निष्टचापोन\-भार्द्धांशानां पूर्णज्यावर्ग्गः यावव १ ३६०० याव २४० ६० रू १४४०० शोधितः शेषं कजपूर्णज्यावर्ग्गो ऽशिष्टः यावव १० ३६०० याव २४० ६०
अयं द्वितीय प्रकारेण कजवर्ग्गः सिद्धः~|| अथ प्रथमप्रकारागतकजवर्ग्गः याव १ या ६ | १६ | ४९ | ७ | ५९ | ८ | ५७ | ४९ | ४०
एतौ समावितिसमशोधनार्थं न्यासः यावव १० ३६०० याव २४० ६०
%\\NओLएम्म{या ०?}{६ | १६ | ४९ | ७ | ५९ | ८ | ५६ | २९ | ४० "{\\र्म् च्रोस्सेद् ओउत् J}}"
या ०?\footnote{{\s ६ | १६ | ४९ | ७ | ५९ | ८ | ५६ | २९ | ४० } Crossed out $J$}
यावव ० याव १ या
%\\NओLएम्म{६ | १६ | ४९ | ७ | ५९ | ८ |}{ ५६ | २९ | ४० "{\\र्म् Lअस्त् थ्रेए दिगित्स् च्रोस्सेद् ओउत् J}}"
६ | १६ | ४९ | ७ | ५९ | ८ |\footnote{ {\s ५६ | २९ | ४०}  Last three digits crossed out $J$}}

\newpage

And now, if the square of the Chord of the desired arc is subtracted from the 
square of the diameter, there whatever is the remainder is the square of 
the Chord of the degrees of a half-circle diminished by the arc; since one
right-angled triangle is [formed] by the diameter and the Chord of the 
desired arc and the Chord of the degrees of a half-circle diminished by the
desired arc; and since ``the circumference-angle 
of a triangle produced from the end [actually ends]
of the diameter in a half-circle is a 
right angle'' is demonstrated in the third chapter of \RG,
the side facing the right angle in that triangle is the diameter. And it 
has the form of [a] hypotenuse. But ``the square of  the hypotenuse
is equal to the sum of the squares of the arm and the upright''
$\marginnote{f.~3r J}$ [is] demonstrated in the first chapter  of \RG. So here the square of the diameter
14400 has the form of the square of a hypotenuse. From it, the square of the 
Chord of the degrees of a half-circle diminished by the desired arc,
\yavava\ $\frac{1}{3600}$, \yava\ $\frac{240}{60}$, units 14400 is subtracted.
The remainder is the square of the Chord of $KJ$. Not remaining [???]: 
\yavava\ $\frac{1}{3600}$, \yava $\frac{240}{60}$. This by a second method
is the established square of $KJ$.

Now the square of $KJ$ first derived, \yava\ 1, \ya\ $6|16|49|7|59|8|57|49|40$
These two are equal: thus for the sake of the same-investigating [?], the statement.
\yavava\ $-\frac{1}{3600}$, \ya\ $\frac{240}{60}$, \ya\ 0 [equals] 
\yavava\ 0, \yava\ 1,
\ya\ 

$6|16|49|7|59|8|57|49|40$. 

\newpage

{\s अथ बीजगणिते समीकरणसंप्रदायस्त्वनयारीत्यास्ति ||
सा यथा ||
यदि समयो पक्षयोर् मध्ये एको
राशिः ऋणश्चेत्तद्राशितुल्यं पक्षद्वये योज्यं तदापि ८ पक्षद्वयं सममेवभवति ||
तस्मादत्र प्रथमपक्षे यावव १ ३६०० इदं पक्षद्वयं योज्यं ||
तत्र प्रथमपक्षे
धनर्णयोस्तुल्यत्वा-द्यावद्वर्ग्गवर्ग्गस्यनाशः अवशिष्टं याव २४० ६० अयं छेदभक्तः याव ४ अयं
प्रथमपक्षः ||
पुनर्द्वितीयपक्षे यावव १ ३६०० योजितः यावव १ ३६०० या व १ या ६ | १६ | ४९ | ७ | ५९ | ८ | ५७ | ४९ |
४०
एतौ पक्षौ पुनरपि समानौ जातौ याव ४ या ० यावव १ ३६०० याव १ या ६ | १६ | ४९ | ७ | ५९ | ८ | ५७ | ४९ | ४० ||
समपक्षयोर्मध्ये यौ राशी एकजाती भवतः तत्र लघुराशिर्महद्राशौ शोध्यते तदापि पक्षौ
समावेवावशिष्टौ भवतः || तस्मादत्र यो वर्ग्गः राशिः एकजाती यो .अस्त्यतो लघुराशिः याव १ महद्राशौ
याव ४ शोधितः शेषं प्रथमपक्षः याव ३ द्वितीयपक्षे च यावव १ ३६०० या ६ | १६ | ४९ | ७ | ५९ | ८ | ५७ | ४९ | ४०
एतावपि समौ || पुनः पक्षद्वयमध्ये यद्येकः खण्डितोस्ति द्वितीयपक्षः | $|$
\marginnote{f.~3v J}
संपूर्णश्चेत्तदा खण्डितराशिं प्रपूर्यतावद्गुणं द्वितीयपक्षमपि कुर्वन्ति ||} 

\newpage

Now, in algebra (\bijaganita) the doctrine of the equation is by this method.
That is as [follows]:  When the middle [term?] of two equal sides 
if one amount is negative, [an amount] equal to that amount is to be added to
both sides.  And then the pair of sides become just equal. From that, here in 
the first side the \yavava\ $\frac{1}{3600}$, this is to be added to [?] both sides.
Then in the first side because of equality of positive and negative, there is
destruction of the \yavava. The remaining \yava\ $\frac{240}{60}$: this is 
divided by the divisor, $yava$ 4. This is the first side. But in the second side
\yavava\ $\frac{1}{3600}$ is added, \yavava\ $\frac{1}{3600}$. 
\yava\ 1 \ya\ $6|16|49|7|59|8|57|49|40$.  And again, the two sides occur
equal: \yava\ 4 \ya\ 0, \yavava\ $\frac{1}{3600}$ 

\yava\ 1 \ya\ 6 \danda 16 \danda 49 \danda 7 \danda 59 
\danda 8 \danda 57 \danda 49 \danda 40.   Whatever two quantities are in the middle 
of the two sides (being) equal, the two are produced eka-jAtI (??).  There the small 
quantity is subtracted from the large quantity, then also the  two sides are in the same 
way remain the same.  From that, here whatever is the square, the quantity (?) whatever 
is eka-jAtI (?) the small quantity, \yava\ 1, it is subtracted from the large quantity 
\yava\ 4.  The remainder is the 1st side and on the second side \yava\ 3 
and \yavava\ $\frac{1}{3600}$ (and) y\=a 6 \danda 16 \danda 49 \danda 7 \danda 59 \danda 
8 \danda 57 \danda 49 \danda 40.  These two are also equal.  Again, in the middle (?) 
of the two sides, if one is `broken' (\textit{kha\d n\d dita}), the second side 

$\marginnote{f.~3v J}$
is full (\textit{sa\d mpur\d na}).  

\newpage

{\s अत्र सछेदराशिः खण्डित
उच्यते छेदरहितः संपुर्ण उच्यते || एवं कृते सति पक्षयोः समछेदत्वविधाय छेदायगम
एवोपपद्यते || तस्मादत्रद्वितीयपक्षे खण्डितराशिः यावव १ ३६०० अयमर्थः || अत्र
यावद्वर्गवर्गवर्गराशिरस्ति तस्मादयं हरं गुणितः क्रियते तावद्यावद्वर्गराशिरेकोभवतीति 

द्वितीयपक्षोप्येतावती गुणनीयः || एवं कृते प्रथमपक्षे याव १०८०० द्वितीयपक्षे यावव १ अथप्रमपक्षः
वारद्वयं षष्ट्योर्ध्वाकृतः याव ३ अथ द्वितीयपक्षस्य यावद्राशिस्तेनैवछेदेन ३६०० संगुण्य
वारद्वयं षष्ट्योर्ध्वाकृतः या ६ | १६ | ४९ | ७ | ५९ | ८ | ५६ | २९ | ४० याव ३ द्विपरिवर्त्ताः यावव १ या ६ | १६ | ४९ | ७ |
५८ | ५६ | २९ | ४० एवं जातौ पक्षौ समौ एतौ यावत्तावतापवर्त्तितौ || या ३ द्विपरिवर्ताः याव १ रू ६ | १६ | ४९ | ७ |
५९ | ८ | ५७ | २९ | ४० अत्र यावत् घनः सरूपराशिश्चेद्यावत्तावत्त्रयेण या ३ भाज्यते तदायावत्तावत्प्रमाणं
लभ्यते अस्य भागग्रहणरीतिः पूर्वाचार्यैः एतावत्कालपर्यन्तं न लब्धा ||}

\newpage

If (this is the case), then, filling the broken quantity, one should make so much a quantity 
on the second side.  Here the quantity with divisor is called `\textit{kha\d n\d dita}'; 
`\textit{chedarahita} is called full (\textit{sa\d mp\=ur\d na}).  Computing thus, with there 
being two sides,  producing the sameness of divisors, a departure of divisors is produced.  
From that here on the second side the broken quantity is \yavava\  $\frac{1}{3600}$; 
This is the meaning.  Here the quantity of the square of the square of the square [sic!] 
of the \ya, is one, the second side is multiplied by t\=avat.  Computing in this way on 
the first side there is \yava\ 10800 (and) on the second side y\=a va va 1 Now the 
first side is made upwards (?) quantity-pair-60, \ya\ 3.  Now, the quantity of \ya s 
of the second side, multipling by that divisor, 3600, it is made upwards-quantity-pair-60. 
\ya\ 6 \danda 16 \danda 49 \danda 7 \danda 59 \danda 8 \danda 57 \danda 49 \danda 40 
\yava 3 changes to \yavava 1 \ya\ 6 \danda 16 \danda 49 \danda 7 \danda 59 \danda 
8 \danda 57 \danda 49 \danda 40.  In this way these two sides  produced equal are reduced 
to a common measure (\textit{aparvartita}) with respect to \ya\ \yava\ 1 r\=u 6 \danda 
16 \danda 49 \danda 7 \danda 59 \danda 8 \danda 57 \danda 49 \danda 40.

Now the cube of \ya.  If the quantity with units is divided by the three [of the] \ya,
\ya\ 3, 
then an amount of the \ya\ is obtained.    
The method of the grasping of the part of it 
by the ancient teachers \elp 

\newpage

{\s अत्र यमशदेन रीतिः प्रदर्शिता ||
सा यथा रूपस्य प्रथमाङ्को यावत्तावता भाज्यः लब्धिरेकान्तस्थाप्य || पुनर्लब्धिघनं शेषांके योज्य
पुनरत्रद्वितीयांको यावत्तावता भाज्यः लब्ध्$|$इः पूर्वलब्धेरधः स्थाप्या || पुनर्लब्धिद्वययोगस्य घनः
कार्यः || एवं तत्र प्रथमलब्धिघनः शोध्यः शेषं द्वितीयलब्धि\-शेषे योज्यं || पुनस्तृतीयांको
यावत्तावता भाज्यः इयं लब्धिः पूर्वलब्धिद्वयाधः स्थाप्या || पुनरस्यलब्धित्रययोगस्य घनः कार्यः
तत्र लब्धिद्वययोगस्य घनः शोध्यः | $|$ %NEW FOLIO
\marginnote{f.~4r J}
शेषं तृतीयलब्धिशेषे योज्यं || पुनस्तत्र चतुर्थां को यावत्तवता भाज्यः || एवमेवेष्टभागपर्यन्तं विधिः कार्यः ||
एवं यमशैदेन यावत्तावत्प्रमाणं निकाशिलं २ | ५ | ३९ | २६ | २२ | २९ | २८ | ३२ | ५२ | ३३ | इयं
अंशद्वयस्य पूर्णज्यास्ति अस्यार्धं एकांशज्याजाता | १ | २ | ४९ | ४३ | ११ | १४ | ४४ | १६ | २६ | १७} 

\newpage

Here the rule is expounded by yamashada. 
[= jamshad = Jamsh\={\i}d al-K\=ash\={\i}, although this method is actually
the ``RumiBeg'' alternative to al-K\=ash\={\i}'s]. It is as [follows]:

The first digit of the unit [term] is to be divided by the \ya\ [coefficient]. The quotient is to be set
[at] one end. Again, the quotient-cube is to be added to the remainder-digit;
here again, the second digit is to be divided by the \ya\ [coefficient]; the quotient is to be  set
below the previous quotient. Again, the cube of the sum of the two quotients
is to be made.  In this way there the cube of the first quotient is to be subtracted;
the remainder is to be added in the second quotient. Again, the third digit is to be
divided by the \ya. This quotient is to be set below the previous quotient. Again,
the cube of the sum of its three quotients is to be made. There the cube of
the sum of the two quotients is to be subtracted. 

$\marginnote{f.~4r J}$
The remainder is to be added in the remainder of the third quotient. Again there, the 
fourth digit is to be divided by the \ya. In just this way the rule is to be done as far
as the desired place. In this way by  Yama\'said the amount of the \ya\ is 
\textit{nik\=a\'sila\*m} [?].

2\danda 5\danda 39\danda 26\danda 22\danda 29\danda 28\danda 32\danda
52\danda 33\danda This is the Chord of two degrees. Half of it occurs ths
Sine of one degree: 
1\danda 2\danda 49\danda 43\danda 11\danda 14\danda 44\danda 16\danda
26\danda 17 

\newpage

{\s अस्यभागहरणोपपत्तिः ||
तत्र पक्षद्वयमध्ये एकपक्षे यावत्तावदस्ति ||
द्वितीयपक्षे यावत्तावद्घनः रूपराशिश्च ||
एवं तत्र यावत्तावद्ज्ञानं चेत्तदा यावत्तावद् घनं कृत्वा
रूपराशौ प्रक्षिप्य यावत्तावता भाज्यते तत्र लब्धिर्यावत्तावन्मानं स्यात् |

अत्र तु यावत्तावद्ज्ञानं नास्त्यतो रूपराशिरेवयावत्तावता भाज्यः ||
यल्लब्धं तत्सरूपयावद्घनस्य कोप्यंशो लब्धः |
सचैकान्ते धृतः || पुनरस्य घनं कृत्वा शेषे
योज्य पुनरत्रयावत्तावता द्वितीयांको भाजितः एवं यल्लब्धं तत्पूर्वलब्धेरधः स्थापितं एवं लब्धं यावत्तावद्धनस्या-सन्नोभागोलब्धः

अत्रासन्नताशेषावयवसत्वाभिरवयवत्वे एव सूक्ष्मालब्धिः अथ लब्धिघने पूर्वघनं
संशोध्य यतस्तदधिकं जातं एवमिष्टभागपर्यन्तं मुहुर्वारं वारं कार्यं ||}

\newpage

The demonstration with its \textit{bh\=agahara} [? co-heir, divisor?]. There in 
the middle of two sides [Is there an Arabic term for ``in an equation'' that this
is literally translating?  see expressions for ``in an equation'' such as 2 lines
above marginal (8) in Rumi-Beg facsimile], in one side is the \ya. 
In the other side is the \ya-cube
and the amount of units. In this way there if the \ya\ is known, then having
made the \ya-cube and added it to the amount of units, [it] is divided by 
the \ya\ [coefficient]:  there the quotient should be the amount of the \ya. But here the
\ya\ is not known, hence the amount of units itself is to be divided by the \ya\
[coefficient]. 
Whatever is the quotient is obtained [as] some part of the \ya-cube-plus-units.
And it is kept at one end. Again, having made its cube, [it] is to be added
to the remainder. Again here, the second digit is divided by the \ya\ [coefficient]. In this way
whatever is the quotient is set below the previous quotient. In this way
the quotient is taken [as] the approximate-part of the \ya-cube. Here with the 
approximateness and part-of-syllogism-ness are just two part-of-syllogism-ness [???]:
the quotient is accurate. 

Now, having subtracted the previous cube from the cube of the quotient, since
the result [?] is greater than that, in this way up to the desired part 
\textit{mudgarb\=ara\*m v\=ara\*m} [???] is to be made.

\newpage

{\s अथास्योदाहरणं ||
तत्र द्वितीयपक्षे रूपाणि ६ | १६ | ४९ | ७ | ५९ | ८ | ५६ | २९ | ४४ अयत्र षद्
द्विपरिवर्त्ताः ||
षष्टेरूर्ध्वमस्ति ||
यावत्तावत्त्रयं द्विपरिवर्त्ताः षष्टेरूर्ध्वमस्ति ||

अत एकजातौ भागे गृहीते लब्धमंशाः | २ पुनरस्यघनेंशाः ८ वारद्वयं षष्ट्याभाक्ताः ० |
० | ८ विकलात्मकं विकलासु योजितं १६ | २७ | पुनः पू$|$\marginnote{f.~4v J}%
र्वलब्धिशेषं १६
तेनौव या ३ भक्तं लब्धाः कला ५ लब्धिः पूर्वलब्धेः २ अधः स्थापितां अङ्क २ | ५
पुनरत्रशिष्ट १ | ५७ पुनर्लब्धिद्वयस्य घनः अं ८ | २ | ३२ | ५
अत्र प्रथमलब्धिघनः अं ८
शोधितः शेषं अं १ | २ | ३२ | ५ इदं शेषे अं १ | ५७ | ७ | ५९ | ८ | ५६ | २९ | ४० अंशस्थाने योजितं
१ | ५८ | १० | ३१ | १३ पुनर्हरेणया ३ अंशे ५८ भागोगृहीतः लब्धिर्विकलात्मिका ३९ शेषं अं १ | १० | ३१ | १३ पुनरियंलब्धिः पूर्वलब्धेरधः स्थाप्य २ | ५ |
३९ | अस्य घनोंशादिः ९ | ११ | २ | ३२ | २७ | ४३ | ३९ || }

\newpage 

Now, its example. There in the second side are units 

6\danda 16\danda 49\danda
7\danda 59\danda 8\danda 56\danda 29\danda 40. Here, six is turned around
[moved up] two [places]. [It] is above sixty [i.e., the sixties place]. 
Three times the \ya\ is turned around [moved up] two [places], above [the] sixty [place]. 
Hence when the divisor is taken of the same order [i.e., same 60'mal place],
the quotient is degrees [i.e., the
units place], 2.
Again, in its cube there are degrees [i.e., integer units], 8.
[They] are divided by sixty ``a pair of turns'' [i.e., twice: what's this in Arabic??]:
0\danda 0\danda 8. Added in the form of \vikalas\ to \vikalas,
16\danda 57.
Again,
$\marginnote{f.~4v J}$
the remainder of the first quotient 16 [is] divided by just that \ya\ 3; the quotient [is] \kalas, 5.
The quotient is put down below [i.e., to the right of] the first quotient, 2: digit 2\danda 5.
Again, the here-remaining 1\danda 57, again the cube of the two quotients: dig.\ [or deg.? \textit{a\*m}]
9\danda 2\danda 32\danda 5.  Here the cube of the first quotient  dig.\ [or deg.? \textit{a\*m}]
8 is subtracted; the remainder,dig.\ [or deg.? \textit{a\*m}]
1\danda 2\danda 32\danda 5. This in the remainder: dig.\ [or deg.? \textit{a\*m}]
1\danda 57\danda 7\danda 59\danda 8\danda 56\danda 29\danda 40. 
Added in the degrees place: 
1\danda 58\danda 10\danda 31\danda 13.
Again the portion taken with the divisor, \ya\ 3, in [the] degree [place], [1\danda] 58.
The quotient is in the form of \vikalas,
39, the remainder dig.\ [or deg.? \textit{a\*m}] 
1\danda 10\danda 31\danda 13. Again this quotient is to be set below [i.e., after] the 
previous quotient, 2\danda 5\danda 39.
Degrees etc.\ of its cube: 9\danda 11\danda 2\danda 32\danda 27\danda 43\danda 39.

\newpage

{\s $|$ \marginnote{f.~5r J}

% "{\\र्म् रेस्त् ओf f.४व् ब्लन्क्}"

अत्रद्वितीयघनः ९ | २ | ३२ | ५ शोधितः शेषं कलादि ०? | ३० | २७ | २७ | ४३ | ३९ इदं शेषाञ्केषु १ |
१० | ३१ | १३ योजितं १ | १९ | १ | ४१ | २४ | १३ | १९ पुनरत्रकलास्थाने भक्तं ल २६ पूर्वलब्धेरधः
स्थापिता २ | ५ | ३९ | २६ शेषं कलादि १ | १ | ४१ | २४ | १३ | १९ पुनर्लब्धैः २ | ५ | ३९ | २६ | घनः ९| ११ |
८ | १४ | ३३ | १२ | २३ | २१ | ४ | ५६ | 
तृतीय घनोत्र शोधितः विकलास्ववशिष्टं ५ | ४२ | ५ | २८ | ४४ | २१ | ४ |
५६ | ७ | पुनरिदं शेषाञ्के योजितं क १ | ७ | २३ | २९ | ४२ | ३ | २१ | ४ | ५६ पुनस्तेनैवभक्ते लब्धं
२२ शेषं १ | २३ | २९ | ४२ |३ | २१ | ४ |५६ पूर्वलब्धौ योजितं २ | ५ | ३९ | २६ | २२ अस्य घनः ९| ११ | ८ |
१९ | २२ | ४१ | ६ | ५२ | १५ | १ | ५६ चतुर्थघनोत्र शोधितः शेषं ४ | ४९ | २८ | ४३ | ३१ | १४? | ५ | ५६ इदं
शेषे योजितं १ | २८ | १९ | १० | ४६ | ५२ | १५ | १ | ५६ पुनर्हरभक्तः लभ्दं २९ शेषं १ | १९ | १० | ४६
| ५२ | १५ | १ | ५६ पुनर्लभिरधः स्थिता २ | ५ | ३९ | २६ | २२ | २९ | अस्यघनः ९ | ११ | ८ | १९ | २९ | २ | ४२ | १
| ३९ | ५० | ५२ अत्रपञ्चमघनः शोधितः शेषं ६ | २१ | ३५ | ९ | २४ | ४८ | ५६ इदं शेषे योजितं
१ | २५ | ३२ | २२ | १ |३९ | ५० | ५२ पुनरयं हरया ३ भक्तः लब्धं २८ शेषं १ | ३२ | २२ | १ |३९ | ५० |
५२ लब्धिः पूर्वलब्धेरधः स्थापिता २ | ५ | ३९ | २६ | २२ | २९ | २८ अस्यघनः ९ | ११ | ८ | १९ | २९ | ८ | ५०
| २७ | १९ | ५९ | ४३ अत्रषष्टघनः शोधितः शेषं ६ | ८ | २५ | ४० | ८ | ५१ शेषे योजितं १ | ३८ | ३० |
२७ | १९ | ५९ | ४३ पुनर्हरभक्तं लब्ध ३२$|$
\marginnote{f.~5v J}
शेषं २ | ३० | २७ | १९ |५९ | ४३ लब्धिः पूर्वलब्धेरधः
स्थापिता २ | ५ | ३९ | २६ | २२ | २९ | २८ | ३२ अस्यघनः ९ | ११ | १८ | १९ | २९ | ८ | ५७ | २८ | २३ | ३७ | १}

\newpage

$\marginnote{f.~5r J}$
Here, the second cube 9\danda 2\danda 32\danda 5 is subtracted; the remainder in \kalas etc.\ 
is 0 [?]\danda 30 27\danda 27\danda 43\danda 39. This is added to the remainder digits:
1\danda 10\danda 31\danda 13: [result] 1\danda 19\danda 1\danda 41\danda 24\danda 13\danda 19.
Again here, divided in the minutes place, quo.\ 26. [It] is set below the first quotient: 
2\danda 5\danda 39\danda 26. The remainder in \kalas\ etc.: 1\danda 1\danda 41\danda 24\danda
13\danda 19. Again, with the quotients, 2\danda 5\danda 39\danda 26, the cube: 9\danda 11\danda
8\danda 14\danda 23\danda 12\danda 23\danda 21\danda 4\danda 56.

The third cube is here subtracted, remaining in \vikalas: 

5\danda 42\danda 5\danda 28\danda 44\danda
21\danda 4\danda 56\danda 17 [?]. Again, this is added in the digits of the remainder: ka
1\danda 7\danda 23\danda 29\danda 42\danda 3\danda 21\danda 4\danda 56. Again, when divided with
just this the quotient is 22, remainder 1\danda 23\danda 29\danda 42\danda 3\danda 21\danda 4\danda
56.  Added to the first quotient: 2\danda 5\danda 39\danda 26\danda 22. Its cube:
9\danda 11\danda 8\danda 19\danda 22\danda 41\danda 6 52\danda 15\danda 16 56.

The fourth cube here is subtracted, the remainder 4\danda 49\danda 28 43\danda 31\danda 10\danda
5\danda 56. This is added to the remainder, 1\danda 28\danda 19\danda 10\danda 46\danda 52
15\danda 1\danda 56. Again, divided by the divisor, quotient 29, remainder 1\danda 19\danda 10\danda
46\danda 52 15\danda 2 [?] \danda 56. Again, the quotient is set below: 2\danda 5\danda 39\danda 26\danda
22\danda 29. Its cube: 9\danda 11\danda 8\danda 19\danda 29\danda 2\danda 42\danda 1\danda 39\danda
50\danda 52. 

Here, the fifth cube is subtracted, remainder 6\danda 21\danda 35\danda 9\danda 24\danda 48\danda
56. This is added to the remainder: 1\danda 25\danda 32\danda 22\danda 1\danda 39\danda 50\danda 52.
Again, this is divided by the divisor [?] 3, quotient 28, remainder 1\danda 32\danda 22\danda 1\danda 39\danda
50\danda 52. The quotient is set below the previous quotient, 2\danda 5\danda 39\danda 26\danda 22\danda
29 28. Its cube: 9\danda 11\danda 8\danda 19\danda 29\danda 8\danda 50\danda 27\danda 19\danda 59\danda
43. 

Here, the sixth cube is subtracted, remainder 6\danda 8\danda 25\danda 40\danda 8\danda
51. Added to the remainder: 1\danda 38\danda 30\danda 27\danda 19\danda 59\danda 43. Again, divided 
by the divisor, quotient 32,
$\marginnote{f.~5v J}$
remainder 2\danda 30\danda 27\danda 19\danda 59\danda 53. The quotient is set below the first
quotient, 2\danda 5\danda 39\danda 26\danda 22\danda 29\danda 28\danda 32. Its cube: 
9\danda 11 18 19\danda 29\danda 8\danda 57\danda 28\danda 23\danda 37\danda 1.

\newpage


{\s अत्रसप्तमघनः शोधितः शेषं ७ | १ | ३ | ३७ | १८ इदं शेषे योजितं २ | ३७ | २८ | २३ | ३७? | १ हरभक्ते लब्धं ५२ शेषं १ | २८ | २३ | ३७ | १ लब्धिः

पूर्वलब्धेरधः स्थापिता २|५|३९ | २६ | २२ | २९ | २८ | ३२ | ५२ अस्य घनः ९ | ११ | ८ | १९ | २९ | ८ | ५७ | ३९ | ४७ | ५० | १९ पुनरत्राष्टमघनः शोधितः शेषं ११ | ?४ | १३ | १८
शेषे योजितं १ | ३९ | ४७ | ५० | १९ हरभक्तः लब्धं ३३ शेषं ० | ४७ |५० | १९ | 
ल@
पूर्ववत् २ | ५ | ३९ | २६ | २२ | २९ |२८ | ३२ | ५२ | ३३? एतत्पर्यन्तं गृहीतं अथात्र यावत्तावन्मानानयनार्थमुपायन्तरमाविदेन निष्कासितं तद्यथा रूपाणि यावत्तावता भाज्यानिलब्धेर्घनः कार्यः पुनर्घनोऽपि यावत्तावता भाज्यः यल्लब्धं तत्प्रथमलब्धौ योज्यं पुनस्वस्य्घनः कार्यः एवमसकृत्

अत्रोपपत्तिः
इह पूर्ववद्यावत्तावता भक्तेरूपराशौ कश्चिद्यावत्तावतो भागोलब्धः पुनस्तस्य घनं कृत्वा
तद्योजनेनवास्तं वघनस्यां स्वन्नताजाता एवमुहुः स्थिरी भूतं तदेव घनस्वरूपराशेर्यावत्तावतोभाग उपलब्धः स एव-यावत्तावन्मानमुपपन्नं |}

\newpage


Here the seventh cube is subtracted, remainder 7\danda 1\danda 3\danda 37\danda 18. 
This is added to the remainder, 2\danda 37\danda 28\danda 23\danda 37\danda 1.
When divided by the divisor, the quotient is 52, remainder 
1\danda 28\danda 23\danda 37\danda 1. The quotient 
is set below the first quotient: 2\danda 5\danda 39\danda 26\danda 22\danda 29\danda 28\danda 32\danda
52. Its cube: 9\danda 11\danda 8\danda 19\danda 29\danda 8\danda 57\danda 39\danda 47\danda
50\danda 1 [?]. 

Again, here the eighth cube is subtracted, remainder 11\danda 24\danda 13 [?] \danda 18. Added to the
remainder: 1 [?] \danda 39\danda 47\danda 50\danda 19. Divided by the divisor, quotient 33, 
remainder 57 [?]\danda 50\danda 19. Quo.\ as before: 2\danda [?] \danda 39\danda 26\danda 22\danda
29\danda 28\danda 32\danda 52\danda 33. Seized this far. 

Now here for the sake of the determination of the amount of the \ya, another approach is known 
[?? \textit{\=avideta}, finite verb??]. 
expounded. That is as [follows]: 

The units are divided by the \ya, the cube of the quotient is to be made. And again, the cube is
to be divided by the \ya. Whatever is the quotient, that is to be added to the first quotient. Again,
the cube of that is to be made, and so repeatedly.  

Here is the demonstration. In this case as before, when the amount of units is divided by the \ya, 
something is obtained [as?] the part [?] of the \ya. Again, having made its cube, that with addition
is fixed [? \textit{v\=astava\*m} ?]; the approximateness of the cube is produced. Thus the being
gone-out-fixed [???].  Just that from the amount of the units of its own cube, part with the \ya 
[is] conceived. That itself is obtained [as] the amount of the \ya. 

\newpage

{\s अत्रोदाहरणं रू ६ | १६ | ४९ | ७ | ५९ | ८ | ५६ | ३० इयं यावत्तावत् षष्टेर्द्विरूर्ध्व$|$
\marginnote{f.~6r J} 
परिवर्त्तेन या ३ भक्ते यल्लब्धं फलं २ | ५ | ३६ | २२ | ३९? | ४२ | ५८ | ५० अस्य घनं ९ | १०? | २८ | ३ | ८ | ५२ | ५ | ३९
पुनरयं तेनैव या ३ भक्तः फलं ० | ० | ३ | ३ | २९ | २१ | २ | ५७ |
इदं प्रथमलब्धौ योजितं २ | ५ | ३९ | २६ | ९ | ४ | १ | ४७ अस्य घनः वारद्वयं षष्ट्या भक्तः जातो विकलात्मकः ० | ० | ९ | ११ | ०? | १६ | ३२ | ३० | ४८ | ९ पुनरयं तेनैव भक्तः लब्धं ० | ० | ३ | ३ | ४२ | ४३? | ३० | ५०
इदं प्रथमलब्धौ योजितं जातं ५ | ३९ | २६ | २२ | २८ | २९ | ४० अस्य घनः ९ | ११ | ८ | १९ | २८ | ३६ | २३ | ५० पुनस्तेनैव भक्तः लब्धं ० | ० | ३ | ३ | ४२ | ४६ | २९ | ३९ प्रथमलब्धौ योजितं २ | ५ | ३९ | २६ | २२? | २९ | २८ | ३२
अस्यघनः ९ | ११ | ८ | १९ | २९ | ८ | ५६ | ५१ पुनस्तेनैव भक्तः लब्धं ० | ० | ३ | ३ | ४२ | ४६ | २९ | ४२ | इदं प्रथमलब्धौ योजितं २ | ५ | ३९ | २६ | २२? | २९ | २८ | ३२ अस्य घनः ९ | ११ | ८ | १९ | २९ | ८ | ५७ | २८
पुनस्तेनैव भक्तः लब्धं ० | ० | ३ | ३ | ४२ | ४६ | २९ | ४२ प्रथमलब्धौ योजितं २ | ५ | ३९ | २६ | २२ | २९ | २८ | ३२ अयं स्थिरीभूतः इयमंशद्वयस्य पूर्णज्याजाता |} 

\newpage

Here, an example: U.\ 6\danda 16  49\danda 7\danda 59\danda 8\danda 56\danda 30. 
This the \ya\ above of sixty twice
$\marginnote{f.~6r J} $
with moving back and forth [?] the \ya\ 3 when divided [???],
whatever is the quotient is the result 2\danda 5\danda 36\danda 22\danda 39\danda
42 58 \danda 50. Its cube: 9\danda 16 [?] \danda 28\danda 3\danda 8\danda 52\danda 5\danda 39.
Again, this is divided by that \ya\ 3, result 0\danda 0\danda 3\danda 3\danda 29\danda 21\danda 2\danda 
57. This is added to the first quotient: 2\danda 5\danda 39\danda 26\danda 9\danda 4\danda 0\danda 47.
Its cube, a pair of turns, divided by sixty, the result is in the form of \vikalas: 
0\danda 0\danda 9\danda 11\danda 8\danda 16 32\danda 30\danda 48\danda 9.
Again, this is divided by just that, quotient 0\danda 0\danda 3\danda 3\danda 42\danda 45 [?]
30\danda 50. This is added to the first quotient, result 5\danda 39\danda 26\danda 22\danda 28
29\danda 40. Its cube: 9\danda 11\danda 8\danda 19\danda 28\danda 36\danda 23\danda 50.
Again, divided by just that the quotient is: 0\danda 0\danda 3\danda 3\danda 42\danda 46\danda 29\danda
39. Added to the first quotient: 2\danda 5\danda 39\danda 26\danda 22\danda 29\danda 28\danda 29. 
Its cube: 

9\danda 11 8\danda 19\danda 29\danda 8\danda 56\danda 51. Again, divided by just that,
quotient 0\danda 0\danda 3\danda 3 42\danda 46\danda 29\danda 42. This is added to the first 
quotient: 2\danda 5\danda 39\danda 26\danda 22\danda 29\danda 28\danda 32. Its cube: 

9\danda 11\danda 8\danda 19\danda 29\danda 8\danda 57\danda 28. Again, divided by just that,
quotient 0\danda 0\danda 3\danda 3\danda 42\danda 46\danda 29\danda 42. 
Added to the first quotient: 2\danda 5\danda 39\danda 26\danda 22\danda 29\danda 28\danda 32.
This is being fixed. This occurs [as] the Chord of two degrees.

\newpage

{\s अथमिर्जोलुग्बेगोक्तप्रकारेणांशद्वयस्य पूर्णज्या निष्काश्यते तत्र पूर्वोक्तमेव क्षेत्रं
कार्यं |} \\ 

\begin{center}
\includegraphics[width=2in]{6r.png}
\captionof{figure}{6r}
\end{center}
 
{\s तत्र वचिह्नात् वहलंबः कजरेखायां निष्काश्यः तत्र कबहत्रिभुजे जवहत्रिभुजे हकोणः समकोणो स्ति कबवर्गः कहहव$|$
\marginnote{f.~6v J}
योर्वर्गयोगतुल्यो स्ति पुनर्बजवर्गः जहहवयोर्वर्गयोगतुल्यो स्ति
पुनरत्रकवबजौ मिथस्तुल्यौ कल्पितौ वहमुभ\-योस्त्रिभुजयोरेकमेवास्ति तस्मात्कहहजौमिथः समानौ भविष्यतः कववर्गः वहस्य व्यासस्य च घातेन समानो स्ति तस्माद्यदिकबवर्गो व्यासेन भाज्यते तदा लब्धं बहप्रमाणं भवति ||
पुनर्यदि वहवर्गः कववर्ग्गाछोध्यते तदा शेषं कहवर्गो वशिष्यते } 

\newpage

Now, by the method stated by Mirjolugvega [Ulugh Beg], the Chord of two degrees is expounded. 
Then just the previously stated figure is to be made. \\ 

\iffalse
\begin{center}
\includegraphics[width=2in]{6r.png}
\captionof{figure}{6r}
\end{center}
\fi 

There, having led out from point $V$ perpendicular $VH$ on line $KJ$, there in triangle
$FBH$ in triangle $JVH$ [there is?] angle $H$. [It] is a right angle. The square of $KB$ 

[Note: the scribe uses $V$ and $B$ more or less interchangeably to refer to this 
diagram letter.]
$\marginnote{f.~6v J}$
is equal to the sum of the squares of $KH$, $HV$. Again, the square of $BJ$ is equal to the
sum of the squares of $JH$, $HV$. Again, here the pair $KV$, $BJ$ are considered equal 
[to?] $VH$. Just one of both triangles is.  [?] From that, $KH$, $HJ$ will become an equal pair.
The square of $KV$ is equal to the product of $VH$ and of the diameter. From that, if the 
square of $KB$ is divided by the diameter, then the quotient becomes the amount of $BH$.
Again, if the square of $VH$ is subtracted from the square of $KV$, then the remainder
remains [as] the square of $KH$.  


\newpage

{\s अत्रांशद्वयस्य पूर्णज्यारूपाकवरेखा
यावत्तावन्मिताकल्पिता या १? अस्यवर्गः
%CM: Tहिस् लस्त् फ्रसे मय् हवे बेएन् इन्तेन्देद् तो बे च्रोस्सेद् ओउतित् इस् हर्द् तो तेल्ल्!
याव १ व्यासः १२० षष्ट्योर्ध्वा कृतः २ एकपरिवर्त्तः अनेन कववर्गो याव १ षष्टेरेकपरिवर्त्तेन २ भक्ते लब्धं
यावद्वर्गादर्धकला-साचयावद्वर्गस्य 

त्रिंशद्विकलात्मिका ३०
यावत्तावतः अंशात्मकत्वादिदं वहप्रमाणजातं अस्यवर्गः यावद्वर्गवर्गस्य
%CM: थे सेचोन्द् वर्ग इस् अ मर्गिनल् अद्दितिओन्/चोर्रेच्तिओन्
पंचदशप्रतिविकलाः जाताः १५ इदं कववर्ग्गाछोधितं शेषं कहवर्गः याव १ याववतिविकला २ |? कहवर्गः कजवर्गस्य चतुर्थांशत्तुल्यो स्ति तस्मात्कजवर्ग्गः याव ४ यावव १
विकला |

अथमिज स्ती ग्रंथस्य प्रथमध्यायस्यद्वितीयक्षेत्रे इदमुपपन्नं कजवद्घातः कजवर्गरूपः
कबजदघा$|$
\marginnote{f.~7r J}
तस्य वजकदघातस्य च योगेन तुल्यो स्ति तत्र कदप्रमाणं ६ | १६ | ४९ | ७ | ५९ | ८ | ५६ | ३० वजं अवतुल्यं
यावन्मितमस्ति तस्माद्वजं कदेन गुणितं सदेतावन्ति यावत्तावतानिजातानि या ६ | १६ | ४९ | ७ | ५९ | ८ | ५६ | ३० कववर्गरूप कवजदघात याव १
तस्मादिदं याव ४ यावव १ विक@ अं अस्य याव १ या ६ | १६ | ४९ | ७ | ५९ | ८ | ५६ | ३० समानं जातं पुनरेतौ समशोधितौ तत्रैकपक्षे
याव ३ ? याव ध १ विक@ 
या ६ | १६ | ४९ | ७ | ५९ | ८ | ५६ | ३० एतौ समानौ स्तः |} 

\newpage

Here the Chord of two degrees [in?] units [is] considered line $KV$, measured by the \ya. 
\ya\ [?? something erased?] its square \yava\ 1, diameter 120, divided by sixty, 2. 
One back-and-forth: with it the square of $KV$ \yava\ 1 from sixty with one back-and-forth 2
when divided, the quotient from the square of \ya\ half-\kala, and it has the form of thirty \vikalas\
of \yava\ 30. Because of the \ya\ having the form of degrees, this amount of 
$VH$ results. Its square,  of the \yavava\ fifteen like [\textit{prati}] arcminutes occur [??] 15. 
This is subtracted from the square of $KV$, the remainder is the square of $KH$: 
\yava\ 1 \yavava\ like [?] \vikalas [??] square of $KH$ is equal to a fourth part of the
square of $KJ$. From that, the square of $KJ$ \yava\ 4 \yavava\ 1 \vikalas.

Now, in the second figure of the first chapter of the book Mijast\={\i}, this is demonstrated.
The product of $KJ$, $VD$ has the form of the square of $KJ$; with
the sum of the product of $KB$, $JD$. 
$\marginnote{f.~7r J}$
and of the product of $VJ$, $KD$ it is equal.  Then the amound of $KD$ is 
6\danda 16\danda 49\danda 7\danda 59\danda 8\danda 56\danda 30.  $VJ$ is equal to 
$AV$ [? $A$?], measured by \ya.  From that, $VJ$ multiplied by $KD$ with really
that much [?] the \ya s result, \ya\ 6\danda 16\danda 49\danda 7 59\danda 8\danda 56\danda
30. 
The product of $KV$, $JD$ with the form of
the square of $KV$, \yava\ 1. From that, this \yava\ 4 \yavava\ 0, \vikalas, 
\textit{a\*m} [?]. Of it, \yava\ \ya\ 6 16\danda 49\danda 7\danda 59\danda 8\danda 56\danda 30
equal results. Again, the two equal-subtracted, then on one side \yava\ 3 \yava  [?] \textit{gha} [?]
\vikalas\ \ya\ 6\danda 16\danda 49 7\danda 59\danda 8\danda 56\danda 30. These two are equal.

\newpage


{\s अत्र समयोसंशौ चेद्गृहेते तदा तावपि
समानौ भविष्यतः अत्र पक्षौ त्रिभक्तौ याव १ यावव २
%%CM: wहत् fओल्लोwस् हस् बेएन् अद्देद् इन्
या २ | ५ | ३५ | २२ | ३९ | ४२ | ४८? | ५० प्रतिविकला
%%CM: एन्द् ओf अद्दितिओन्
पुनरेतौ यावत्तावतापवर्त्तितौ यथा या १
याव २ रू २ | ५ | ३६ | २२ | ३९ | ४२ | ५८ | ५०
तत्रादृशस्येष्टांकस्य चिकीर्षास्ति यस्यांकस्य रूप राशि
स्थितांकस्य चांतरं तदंकं घनस्य विंशति
प्रति विकलात्मकं भवति स एवांक इष्टो भवति तस्योत्पादनेयमुपाय उपलब्धः रूपाणां घनः कृतः ९ | १० | २८ | ३ | ८ | ५२ | ५ | ३९
इदं विंशति प्रतिविकलाभिर्गुणितं ० | ० | ३ | ३ | २९ | २१ | २ | ५७ रूपेषु योजितं २ | ५ | ३९ | २६ | ९ | ४ | १ | ४७ अस्य घनः ९ | ११ | ८ | १६ | ३२ | ३० | ४८ | ९ पुनरिदं विंशतिप्रतिविकलाभिर्गुणितं ० | ० | ३ | ३ | ४२ | ४५ | ३० | ५०
रूपेषु योजितं २ | ५ | ३९ | २६ | २२ | २८ | २९ | ४० अस्य घनः $|$\marginnote{f.~7v J}  
९ | ११ | ८ | १९ | २८ |५६ | २३ | ५० पुनस्तेनैवप्र@ वि@
२० गुणितः ० | ० | ३ | ३ | ४२ | ४६ | २९ | ३९
रूपेषु योजितः २ | ५ | ३९ | २६ | २२ | २९ | २८ | २९ पुनर्घनः ९ | ११ | ८ | १९ | २९ | ८ | ५६ | ५१ पुनस्तेनैव प्र@ वि@ २० गुणितः ० | ० | ३ | ३ | ४२ | ४६ | २९ | ४२
रूपेषु योजितं २ | ५ | ३९ | २६ | २२ | २९ | २८ | ३२
इदमंश द्वयस्य पूर्णज्यारूप इष्टांको जातः पदास्य घनः क्रियते विंशाति प्रतिकलाभिर्गुण्यते
रूपेषु ज्योजिते तदा स एवांको भवति |} 

\newpage

Here if two third parts of equals are taken then those two too will become equal.   Here the two sides
are divided by three, \yavava\ [? 0?] \ya\ \textit{ra} [?] 15\danda 32\danda 22\danda 39\danda 42\danda
[?] like \vikalas.
Again, those two reduced by \ya\ as: \ya\ 1 \yava [? 0?] un.\ 2\danda 5\danda 36\danda 22\danda 
39\danda 42\danda 58\danda 50. 

There, [there is] wish [for] the desired digit of sight [?].  The difference of whatever digit and the
digits standing in the amount of units, that digint twenty of the cube  becomes having the form like \vikalas. 
It becomes one desired digit. Its production [is?] conceived [as?] this approach: The cube of the 
units is made, 9\danda 10\danda 28\danda 3\danda 8\danda 52\danda 5\danda 39. 
This is multiplied by twenty-like-\vikalas [?], 0\danda 0\danda 3\danda 3\danda 29\danda 21\danda 2\danda
57. Added to the units: 2\danda 5\danda 39\danda 26\danda 9\danda 4\danda 1\danda 47.
Its cube: 9\danda 11\danda 8 16\danda 32\danda 30\danda 48\danda 9. Again, it is multiplied
by twenty-like-\vikalas, 0\danda 0\danda 3\danda 3\danda 42\danda 45\danda 30\danda 50.
Added to the units: 2\danda 5\danda 39\danda 26\danda 22\danda 28\danda 29\danda 40.
Its cube:
$\marginnote{f.~7v J} $
9\danda 11\danda 8\danda 19\danda 28\danda 46\danda 23\danda 50. 
Again, with just that pra-vi-20 multiplied [multiply by 20 and shift 3 places to the right, 
equivalent of dividing by $3,0,0$], 0\danda 0\danda 3\danda 3\danda 42\danda 46\danda
29\danda 39. Added to the units: 2\danda 5\danda 39 26\danda 22\danda 29\danda 
28\danda 29\danda. Again, the cube: 9\danda 11\danda 8\danda 19\danda 29\danda 8\danda
56 51. Again, with just that pra-vi-20 multiplied, 0\danda 0\danda 3\danda 3\danda 42\danda
46 29\danda 42. Added to the units: 2\danda 5\danda 39\danda 26\danda 22\danda 29\danda
28 32. This [is?] the Chord of two degrees, the unit [or number? having] the desired digit 
[or digits?] results. [I.e., this is the exact value.] 
When its cube is made, is multiplied with twenty-\pratikalas, is added [misspelled] to the units, then it
becomes the same digit. [I.e., further iteration doesn't change it.]

\newpage

{\s अथमिर्जोलुग्वेगस्य द्वितीयः प्रकारः तत्र कवजदं चापं षडंशानां कल्पं
यस्य वृत्तस्येदं चापं तद्वृत्तकेंद्रं तं कल्पनीयं पुनः प्रत्येकं कवचापं वजचापं जदचापमंशद्वयं कल्पितं
तवकजकदवजवदजदपूर्णज्याः संयोज्याः पुनः 

कवपूर्णज्याकजपूर्णज्याकदपूर्णज्यानाह ?
स्वचिह्नेष्वर्धं कार्यं पुनस्नहरेखा-तद्भतवरेखाः संयोज्याः एतारेखाः प्रतेकं तासु पूर्णज्यासु लम्बरूपाभविष्यन्ति पुनः कतरेखाव्यासार्धं संयोज्यं पुनः तत् अचिह्नेरधितं कार्यं पुनः अचिह्नं केन्द्रं कृत्वा
अकव्यासार्द्धेन वृत्तं कार्यं तत्र कहतकोण-कझतकोणकवतकोणाः समकोणाः सन्ति तदा वृत्तार्धमेतत् हझवचिह्नेषु गमिष्यति पुनः हझझबवतहतरेखाः संयोज्याः हचिह्नात् हललम्बः कझपूर्णज्या सुनिष्काश्यः $|$\marginnote{f.~8r J}
तदा कझपूर्णज्या लचिह्नेष्वर्धिता भविष्यति तत्र झवरेखया कजभुजकदभुजयोरर्धस्थाने कार्त्ततं तदा झवरेखा\-जदरेखायाः समानान्तरा भविष्यति तस्मात्कजद-त्रिभुजकझवत्रिभुजे मिथः सजातीयेभविष्यन्तः? कवं कदस्यार्धमस्ति झवंजदस्यार्धं भविष्यति
पुनरनेनैव प्रकारेण हझरेखा-वजस्यार्धं भविष्यति कहंकवस्यार्धमस्येव तस्मात् कहहझझवपूर्णज्यामिथः समाना भविष्यति |}  

\newpage

Now, the second method of Mirjolugvega. Then arc $KV1JD$ [is] considered [?] of six degrees.
Of whatever circle this arc [is part?], the center of that circle is to be considered $T$. Again, 
each one of arc $KV1$, arc $V1J$, arc $JD$ [is] considered two degrees. The Chords
of $TV1$ [should be $KV1$] $KJ$ $KD$ $V1J$ $V1D$ $JD$ are joined. Again, of the Chord of $KV1$,
the Chord of $KJ$, and the Chord of $KD$ in the points $H$, $Jh$, $V2$ half  is to be made.
Again, line $TH$, line $TJh$, line $TV2$ are  joined. Each one [of] those lines 
will become in the form of a perpendicular on those Chords. 
Again, line $KT$ is joined with [the form of?] the radius; again, that is to be made halved
in point $A$. Again, having made point $A$ the center, a circle is to be made with
radius $AK$. Then angle $KHT$, angle $KJhT$, angle $KV2T$ being right angles, 
then  that half circle will go through points $H$, $Jh$, $V2$. 
Again, lines $HJh$, $JhV2$, $V2T$, $HT$ are joined [Note: the latter two were already
drawn]. From 
point $H$ perpendicular $HL$ is extended on the Chords [? why plural?] $KJh$. 
$\marginnote{f.~8r J}$

Then the Chord of $KJh$ will become halved at points [? why plural] $L$. There with line $JhV2$ 
in the half-place [?] of side $KJ$ and side $KD$ is made [i.e., its endpoints Jh, V2 bisect those 
two segments]. Then line $JhV$ will
become differently-equal [i.e., a corresponding side in similar triangles]
to line $JD$. From that, triangle $KJD$ and triangle
$KJhV2$ pairwise will become together-born [i.e., similar]. 
$KV$ is half of $KD$, [so] $JhV2$ will be half of $JD$.
Again, in just this way line $HJh$ will become half of $V1 J$. $KH$ is half of just that $KV1$.
From that, the Chords $KH$ $HJh$ $JhV2$ will become pairwise equal. 

\newpage

{\s पुनः अचिह्नात् अललम्बः कझपूर्णज्यायां निष्काश्यः अयं लंम्बः कझ-रेखांलचिह्नेऽर्द्धितां करिष्यति
पुनरयं वर्धितः सन् हचिह्ने लगिष्यति तस्मात्कहमेकांशज्या-यावत्तावन्मिता कल्प्या पुनः कवं अंशत्रितयंस्य?ज्या एतावत्यस्ति ३ | ८ | २४ | ३३ | ५९ | ३४ | २८ | १५ |} \\

\begin{center}
\includegraphics[width=1.5in]{8r.png}
\captionof{figure}{8r}
\end{center}

{\s यदि कहवर्गः याव १ कतरेखया ६० भाज्यते तदा लब्धिरेकाकलायावद्वर्गस्य भविष्यति इदं हलप्रमाणमस्ति यथापूर्वमुपपन्नं अस्यवर्गः यावद्वर्गः यावद्वर्गवर्गस्यैका $|$\marginnote{f.~8v J}
विकला भविष्यति अयं कहवर्गे शोधितः शेषं कलवर्गो भविष्यति सच याव १ यावव ? विकला कझवर्गस्य चतुर्थांशः ४ 
कलवर्गोऽस्ति तस्मात् कझवर्गः याव ४ यावव ४ विकला पुनः कझहवघातरूपः
कझवर्गः कझहवघातरूपवत् वर्गस्य हझकवघातस्य च योगेन तुल्योऽस्ति
कहवर्गश्चैतावान् याव १ हझकवघातश्च ३ | ८ | २४ | ३३ | ५९ | ३४ | २८ | १५ |}

\newpage

From point $A$ perpendicular $AL$ onto Chord $KJh$ is to be extended; this perpendicular 
will make line $KJh$ halved at point $L$. Or again, this [line] being halved [??] will fall on point $H$.
From that, the Sine of one degree $KH$ is to be considered the measure of the \ya. 

Again, $KV2$ [is] three degrees, its sine is so much: 3 8\danda 24\danda 33\danda 59\danda
34\danda 28\danda 15. \\ 

\iffalse 
\begin{center}
\includegraphics[width=1.5in]{8r.png}
\captionof{figure}{8r}
\end{center}
\fi 

If the square of $KH$ \yava\ is divided by line $KT$ [is this actually 
\textit{katar-rekh\=a} for Ar.\ \textit{qa\*tar}?] 60, then the quotient  will become
one \kala\ of \yava.   This is the $HL$-amount, as previously obtained. Its square
will become of the \yavava\ one
$\marginnote{f.~8v J}$
\vikala. This is subtracted from the square of $KH$, the remainder will become the
square of $KL$.  And it \yava\ 1  \yavava\ [1 ?] \vikalas. 
[$HL = \frac{x^2}{60} = \frac{KH^2}{KT}$, $HL^2 = \frac{x^4}{3600}$,
$KH^2 - HL^2 = KL^2$.]
A fourth part 4 of the square of $KJh$  is the square of $KL$. From that, 
the square of $KJh$ \yava\ 4 \yavava\ 4 \vikalas. 

Again, the number of the product of $KJh$, $HV2$ [is] the square of $KJh$; like [??]
the number of the product of $KJh$, $HV$. [It?] is equal with the sum of the square [of
\ya?] and of the
product of $HJh$, $KV2$. And the square of $KH$ is that much, \yava\ 1. And the
product of $HJh$, $KV2$ 3\danda 8\danda 24\danda 33\danda
59 34\danda 28\danda 25. 

\newpage

{\s अयं
द्वितीय प्रकारेण कझवर्गः सिद्धः एतौ समाविति समशोधनार्थं न्यासः
यथा याव ४ यावव ४ विकला याव १ या ३ |८ | २४ | ३३ | ५९ | ३४ | २८ | १५ अत्र
संप्रदायेन शोधने कृते सति प्रथमपक्षे याव ३ यावव ४ विकला या ३ | ८ | २४ |
३३ | ५९ | ३४ | २८ | १५ धने कृते सति प्रथमपक्षे अनयोस्त्र्यंशावपि समारौ
तस्मात्त्रिभिरपवर्त्तितौ अवशिष्टावेतावपिसमानौ याव १ यावव १ विरू १ | २ | ४८ | ११ |
१९ |११ | २९ |२५
%%CM: थेरे अरे अ fएw fओर्मत्तिन्ग् इस्सुएस् हेरे अन्द् ईऽम् नोत् सुरे अल्ल् इस् इन् थे रिघ्त् ओर्देर्:
%% छेच्क्!
पुनरेतौयावतावतापवर्त्तितौ जातौ पक्षौ या १ विरू १ | २ | ४८ | ११ १९ | ५१ | २९ | २५ याघ १ २ प्र@ वि@
यद्यत्रताद्यशोंको लभ्यते यस्यां कस्य पुनः रूपस्य चांतरं अंक धनस्यैककलाविंशति प्रतिविकला तुल्यं भवेत्
सचानेन प्रकारेण निष्कांशितः तत्ररूपानां घनः अंशादि १ | ८ | ४८ | ३० | २३ | ३६ | ३० | ४२ | १८ अयमेकयाकलयाविंशतिप्रतिवि$|$\marginnote{f.~9r J} कलाभिश्च गुणितः ० | ० | १ | ३१ | ४४ | ४० | ३ | २८ | ४१ | अंशकलास्थानयोरभावात् स्थानद्वयेशून्यं निवेशितं
रूपेषु संयोज्यजातं १ | २ | ४९ | ४३ | ४ | ३२ | ० | ५३ | ४१ अस्यघन १ | ८ | ५३ | ३२ | ४ | ३ | ५० | ५९ | १४ | ५७ पुनरयमेतेन १ २० गुणितः ० | ० | १ | ३१ | ५१ | २२ | ४५ | २५ | ८ रूपेषु योजितं १ | २ | ४९ | ४३ | ११ | १४ | १८ | ५० | ८ अस्यघनः १ | ८ | ५३ | ३६ | २६ | ७ | १८ | ० | २२ | १० पुनरयं तेनैव १ २० गुणित ० | ० | १ | ३१ | ५१ | २३ | ४ | ४९ | २० रूपेषु संयोज्य १ | २ | ४९ | ४३ | ११ | १४ | ४४ | १६ | ३० अस्यघनः १ | ८ | ५३ | ३२ | २६ | ८ | ३७ | ५ | ३४ पुनस्तेनैव १ २० गुणितः ० | ० | १ | ३१ | ५१ | २३ | १४ | ५१ | २९
रूपेषुसंयोज्य १ | २ | ४९ | ४३ | ११ | ४ | ४४ | १६ | ३६ अयमिष्टांको जाता अत इयमेकांशज्यासिद्धा कृतः यतोऽस्य घने अनेन १ २० वि प्र
%%fओर्मत्तिन्ग् इस्सुएस् वि इस् अबोवे प्र इन्लिने wइथ् १ अन्द् २०.
गुणिते रूपेषु योज्यते तदा यमेव भवति तस्मादयमेव
एकांशज्याजाता इदमेवेष्टं |}

\newpage

Here with this approach, when the subtraction is done in the first side:
\yava\ 3 \yavava\ 4 \vikalas\ [=] \ya\ 3\danda 8\danda 24\danda 33\danda 59\danda 34\danda
28\danda 5. And of those two, two one-third parts are equal [?]. 
From that, the two are reduced by three. Those two remainders are equal:
\yava\ 1 [=?] \yavava\ 1 [?] \vikalas [?], un.\ 1\danda 2\danda 48\danda 11\danda 19
51\danda 29\danda 25.
Again, those two are reduced by the \ya, two sides result: 
\ya\ 1 \vikalas [?] un.\  1\danda 2\danda 48\danda 11\danda 19
51\danda 29\danda 25 [=?] \ya-cube $\frac{1}{2}$ \prativikalas.

If here \textit{t\=adi} [??] what degree is obtained?  [???] 
Again, the difference of whatever digit and of the units; one \kala\ of the cube of the digits
should be equal to twenty \prativikalas. [?]

And that by this method is laid out: there the cube of the units beginning with degrees [is]
1\danda 8\danda 48\danda 30\danda 23\danda 36\danda 30 42\danda 18.
This with one \kala
$\marginnote{f.~9r J}$
and with twenty \prativikalas\ is multiplied: 

0\danda 0\danda 1\danda 31\danda 44\danda
40\danda 3\danda 28\danda 41.
Because of the nonexistence of the two places of degrees and \kalas, zero is entered in 
[those] two places. Having added [the result?] to the units, result:
1\danda 2\danda 49\danda 43 4\danda 32\danda 0\danda 52\danda 41. Its cube:
1\danda 8\danda 53\danda 32\danda 4 3\danda 50\danda 59\danda 14\danda 57.
Again, this  is multiplied by that $\frac{1}{20}$: 


0\danda 0\danda 1\danda 31\danda 51\danda 22\danda 45\danda 25\danda 8.
Added to the units:
1\danda 2\danda 49\danda 43\danda 11\danda 14\danda 14\danda 50\danda 8.
Its cube: 1\danda 8\danda 53\danda 36\danda 26\danda 7\danda 18\danda 0\danda
22\danda 10.
Again, this is multiplied by just that $\frac{1}{20}$: 
0\danda 0\danda 1\danda 31\danda 51\danda 23\danda 14 49\danda 20.
Added to the units: 1\danda 2\danda 49\danda 43\danda 11\danda 14 44\danda 16\danda 30.
Its cube: 1\danda 8\danda 53\danda 32\danda 26\danda 8\danda 37 5\danda 34.
Again, this is multiplied by just that $\frac{1}{20}$: 
0\danda 0\danda 1\danda 31\danda 51\danda 23 14\danda 51\danda 29. 
Added to the units: 1\danda 2\danda 49\danda 43\danda 11 14\danda 44\danda 16\danda 26.
This results [as] the desired digit [desired sequence of digits, up to desired digit??]. 
Hence this Sine of one degree is attained.  Wherefore?  Because when its cube 
is multiplied by this $\frac{1}{20} \frac{\hbox{vi}}{\hbox{pra}}$ [and] added to the units,
then [it] becomes just this. From that, this results [as] the Sine of one degree; this is just
[what was] desired. 

\newpage

{\s अथात्रमिर्जोलग्वेगेन
यत् द्वितीयप्रकारेण क्षेत्रं प्रदर्शितं तत्र बहुरेखा संयोगः कृतं २? अथात्र यथाल्परेखाभिरे$|$\marginnote{f.~9v J}
वसिध्यति तथा पतितमाविदसंज्ञै ?
तत्र तावदेवकवजदचापं पूर्वोक्ति एवपूर्णज्याः
संयोज्या कवपूर्णज्यार्धरूपं कहं एकांशज्या स्ति
सायावन्मिता कल्पिता या १ कवरेखा या २ एवं वजय २ जदया २ यत एतामिथः
समानाः संति पुनः कदरेखां ६ | १६ | ४९ | ७ | ५९ | ८ | १६ अनयावजंया २?
गुणितं कवजदघातयुतं जातं याव ४? या १२ | ३३ | ३८ | १५ | ५८ | १६ | ३३ | ० अयं
जातः कजवर्गः |} \\ 

\begin{center}
\includegraphics[width=2in]{9v.png}
\captionof{figure}{9v}
\end{center}

\newpage

Now here, [in] the figure shown by Mirjolugvega with the second method, there 
the contact of a \textit{vadgu}-line [???] is made. [?]  Now here, as [follows]:  
With \emph{small} lines 
\marginnote{f.~9v J}
it is effected, in that way by those known as \textit{yatitamavid} [``the knowers of
\textit{yatitama}??].  There just that much arc $VKVJD$ [sic], [by] just the previous
statement, the Chord is to be connected. $KH$ has the form of half Chord $KV$, 
[it] is the Sine of one degree. It is considered the measure of the \ya, \ya\ 1.
Line $KV$ \ya\ 2. In the same way $VJ$ \ya\ 2. $JD$ \ya\ 2. Since they are 
pairwise [?] equal, again line $KD$
6\danda 16\danda 49\danda 7\danda 59\danda 8\danda 16. 
With that $VJ$ \ya\ 2 [? illegible] is multiplied. Added to the product of
$KV$, $JD$, result \yava\ 4, \ya\ 12\danda 33\danda 38\danda 15\danda 58
16\danda 33\danda 0. This result is the square of $KJ$. \\

\iffalse 
\begin{center}
\includegraphics[width=2in]{9v.png}
\captionof{figure}{9v}
\end{center}
\fi 

\newpage

{\s पुनः प्रकारांतरेण कजवर्गः साध्यते ? कववर्गः याव ४
अयं व्यासेन २ षष्टेरेकोर्धपरिवर्त्तेन भक्तो लब्धं

क्षेत्र

यावद्वर्गस्य कला २ इदमंशद्वयस्य उत्क्रमज्या जाता अस्या वर्गः यावव ४
विकला कववर्गे शोधितः शेषं याव ४ यावव ४ विकला अयमंशद्वयज्यावर्गो
जातः अयं चतुर्गु$|$णितः
\marginnote{f.~10r J} याव १६ यावव १६ विक@ अयं
सिद्धः कजवर्गः एतौ पक्षौ समौ समशोधनार्थं न्यासः याव ४ या १२ |
३३ | ३८ | १५ | ५८ | १६ | ३३ शोधिते शेषं पुनर्द्धाद 
%CM: सोमे fओर्मत्तिन्ग् इस्सुएस् तो रेसोल्वे
याव १ यावव १६ विकला

याव १२
यावव १६ या १२ | ३३ | ३८ | १५ | ५८ | १६ | ३३ यावत्तावतापवर्त्त्य इदं मिर्जोलुग्वेगेन निष्काशित? बहुप्रयाशेन

या १ याव १ २० विरू १ | २ | ४८ | ११ | १९ | ५१ | २९ | २५ |} 

\newpage

Again, by another method: The square of $KJ$ is to be achieved. The square of $KV$
[is?] \yava\ 4. This is divided by the diameter 2 [which is] reduced one-above from sixty [?].
The quotient is \kalas\ of \yava, 2. This is known as the Versine of two degrees. 
Its square \yavava\ 4 \vikalas\ is subtracted from the square of $KV$. The remainder, 
\yava\ 4, \yavava\ 4 \vikalas.  This results [as] the square of the Sine of two degrees.
This 
\marginnote{f.~10r J}
multiplied by four, \yava\ 16 \yavava\ 16 \vikalas. This is the attained square of $KJ$. 
These two sides [are] equal; for the purpose of equals-subtraction, the statement:
\yava\ 4 \ya\ 12\danda 33\danda 38\danda 15\danda 58\danda 16\danda 33 
[=]
\yava\ 16 \yavava\ 16 \vikalas. When subtracted, the remainder again \textit{dv\=ada} [??]:
\yava\ 12 [=] \yavava 16 \ya\ 12\danda 33\danda 38\danda 15\danda 58\danda 16\danda 33.
Reduced by the \ya: 
\ya\ 1 [=] \yava\ $\frac{1}{2}$ \vikalas, un.\ 1\danda 2\danda 48\danda 11\danda 19\danda
51\danda 29 25. 
This is expounded by Mirjolugvega with the \textit{vadgupray\=a\'sa} [???]. 
\end{document}

\begin{pages}
\begin{Leftside}
\beginnumbering
\pstart
{\s श्रीगणेशाय नमः} ||
\pend
\endnumbering
\end{Leftside}

\begin{Rightside}
\beginnumbering
\pstart
Translation 
\pend
\endnumbering
\end{Rightside}
\end{pages}
\Pages




\begin{pages}
\begin{Leftside}
\beginnumbering
\pstart
{\s श्रीगणेशाय नमः} ||
\pend
\endnumbering
\end{Leftside}

\begin{Rightside}
\beginnumbering
\pstart
Translation 
\pend
\endnumbering
\end{Rightside}
\end{pages}
\Pages
