\documentclass[10pt]{article}

%%%Font related packages
\usepackage[no-math]{fontspec}
\defaultfontfeatures{Ligatures=TeX} 
%\setmainfont{Linux Libertine O}
\newfontfamily\s[Scale=0.92, Script=Devanagari]{Shobhika-Regular}
\usepackage{color, dtk-logos}
 
 
 %%%Formatting related packages
\usepackage{graphicx}
\usepackage[hang,flushmargin]{footmisc} % For no indentation of footnotes.
\usepackage{array}
\usepackage{tabularx, multicol, vwcol}
\usepackage{fullpage}


%Math related packages
\usepackage{amsmath} % āmerican Mathematical Society packages for metamathematical symbols.
%āpril 2017 āditya says install libertine math font package specially from website.

%Page formatting commands
%\newdimen\stdbaseline
%\stdbaseline = 13pt \baselineskip = \baselineskip
\parskip = \baselineskip
\parindent = 0pt


%%%Critical editing packages
\usepackage[series={ā,B,C,D},noend,noeledsec,nofamiliar,noledgroup]{reledmac}
\Xarrangement[B]{twocol}
\Xarrangement[C]{threecol}
\Xarrangement[D]{paragraph}

%=============================================================%


%% MāCROS for Diacriticals, symbols, math and text features
%%
\let\*=\d
\def\elp{$\ldots\,$}
\def\degrees{$^\circ$}
\def\degree{$^\circ$}
\def\signs{$^s$}
\def\Sin{\mathop{\rm Sin}\nolimits}
\def\Cos{\mathop{\rm Cos}\nolimits}
\def\Versin{\mathop{\rm Vers}\nolimits}
\def\Coversin{\mathop{\rm Coversin}\nolimits}
\def\Crd{\mathop{\rm Crd}\nolimits}
\def\crd{\rm{crd}}

%%%%%%%%%%%%%%%%%%%%%%%%


\begin{document}


\title{Jyacapa ūtpatti}
\author{Clemency Montelle, Kim Plofker, Glen Van Brummelen}

\maketitle

\begin{abstract}
We will write an abstract here
\end{abstract}


\vskip30pt

KEYWORDS: 

\newpage

\section{Introduction} 




%%  jyācāpakī upapatti\nl


\marginpar{f.~1r J}
śrīgaṇeśāya namaḥ ||  
athāṃśatrayasya jīvājñāne satyekāṃśajīvājñānaṃ 
dorjyāmitiślokacatuṣṭayenāha || 
karasaṅguṇitāṃ dviguṇāṃ bhujajyāṃ 
tribhaktāṃ vi\-dhāya 
gaṇakapravīṇaḥ phalamaṃśāstānpṛthaksthāpayet ||
tu paraṃ teṣāṃ phala\-rūpāṃśānāṃ ghane  vyāsārdhavargabhakte sati 
yatphalaṃ tadbhāgāvaśeṣasahitaṃ kṛtvā tribhaktaṃ kāryaṃ ||
tribhakte yatphalaṃ kalātmakaṃ tatpūrvalabdhāṃśarūpaphalādhaḥ sthāpyaṃ sāṃśakalātmikā paṅktirjātā ||
paṅkterghanaḥ prathamaghanena hīnaḥ vyā\-sārdhavargabhaktaḥ phalaṃ śeṣe yojyatribhirbhajet ||
labdhaṃ phalaṃ vikalāṃ kalādhaḥ sthapyāḥ sāmśakalāvikalātmikā
paṅktirjātā || 
paṅkterghanaḥ kāryaḥ 
 asminpūrvaghanaḥ śodhyaḥ śeṣe tribhajyāvargabhakte yatphalaṃ tadbhājyāvaśeṣasahitaṃ 
 kṛtvā muhurvāraṃvāraṃ tadagre tu paraṃ purokto vidhiḥ kāryaḥ evaṃ kṛte phalaṃ paṅktiḥ syāt || 
paṅktyardhaṃ cāpasya tṛtīyāṃśajyā syāt || 

atha ityanantaraṃ tāṃ tribhāgajīvāṃ punaḥ prakārādvacmi || 
yataḥ kāra\-ṇādanekairbhedairgaṇakasya buddhiḥ svaśāstre 
pravīṇatāmeti  || 16 || 

punaḥ prakārāntareṇa  tribhāgajīvāmāha || 
svatryaṃśahīneti || 
bhujajyā sva\-tṛtīyāṃśena hīnā pṛthaksthāpyā || 
asyāḥ  ghanastriguṇasya kṛtyā trighnyā bhaktaḥ\(|\) 
labdhaphalena prthaksthā bhujajyā yutā 
 kāryā muhūranayā kriyayā paṅktiḥ sādhyā
 paṅktyardhaṃ cāpaguṇāṃśajīvā
 dhanustryaṃśajyā syāt  || 17 || 
 
punaḥ prakārāntareṇāha | bhujāṃśajīveti || 
\marginpar{f.~1v J}
bhu%\[
jāṃśa\-jyāyāstṛtīyāṃśaḥ pṛthaksthāpyaḥ || 
tasya ghane svatryaṃśayute trijyāvargabhakte labdhaṃ 
yatphalaṃ tena pṛthakstho yutaḥ kāryaḥ || 
evaṃ muhūrvāraṃvāramanayā kriyayā dhanustribhāgajīvā syāt  || 18 || 
 
asyopapattiḥ || sā yathā || 
kabajadaṣaḍaṃśacāpaṃ kalpitam || 
punarasya samānaṃ bhāgatrayaṃ kalpitaṃ bacihnajacihnayoḥ || 
tatra  kabaṃ kajaṃ kadaṃ bajaṃ badaṃ jadaṃ caitāḥ 
pūrṇajyārekhā yojyāḥ || 
athātra  pūrvoktaprakāreṇāṃśatrayasya jīvā jñātāsti sāceyaṃ 
3| 8| 24| 33| 59| 34| 28| 14| 50| iyaṃ dviguṇā 
6| 16| 49| 7| 59| 8| 56| 29| 40 jātā kadapūrṇajyā || 
atrāṃśadvayasya kabapūrṇajyājñānamiṣṭamasti || 
asyopapattyarthaṃ\footnote{asyopapatya ${\rm J}$} 
kiṃciducyate  || 
tatra 
siddhāntasamrājiprathamādhyāyasya\footnote{@saṃmrāji@ ${\rm J}$}
dvitīya kṣetre idamupapāditaṃ || 
yaccaturbhujaṃ vṛttāntaḥ patati tatsanmukhasthabhujānāṃ ghātayogaḥ
taccaturbhujāntaḥ 
pātikarṇadvayaghātasamo\footnote{Hardcoded}%$\37Fw\392wy {\it x} āGāt\37Fw {\rm J}$}
%% KP: hardcoded variant
bhavatītyasyopapattistu pūrvaṃ kathitaivāsti || 
punastatraiva prathamādhyāyasya  caturthakṣetre idamupapāditam || 
iṣṭacāpārdha\-pūrṇajyā\-vargastaccāpona\-bhārdhāṃśapūrṇa\-jyāvyāsāntaraguṇitavyāsārdha\-tulyo\footnote{@guṇitadvyāsārdhatulyo@ ${\rm J}$}
bhavati || 
asyopapattiḥ || 
kajavyāse kabajadaṃ vṛttārdhaṃ kāryaṃ || 
tatra badaṃ dajaṃ\footnote{dajaṃ ${\rm inserted from left margin, J}$}
samamasti   ||  || 
kabarekhākadarekhābadarekhājadarekhāḥ saṃyojyāḥ  || 
dacihnāddakarekhā kajavyāse lambarūpā kāryā || 
atra  japharekhā kajavyāsasya jñātacāponabhārddhāṃśapūrṇajyāyāścāntarārdhamitāsti || 
kutaḥ || 
kabaṃ  kahatulyaṃ kāryaṃ || 
daharekhāyojyā || 
hakadatribhujadakabatribhujayoḥ | 
kaha\-bhuja\-kada\-bhu\-jau bakabhujaka%\[
dabhujayoḥ 
 \marginpar{f.~2r J}
krameṇa samānaustaḥ || punaḥ hakadakoṇaḥ bapha\-dakoṇena samosti tasmat daharekhā badarekhā samānā jātā || dajarekhāpi samānājātā || tadā dajavargaḥ daphavargajaphavargayoryogena tulyo bhaviṣyati || 
 evaṃ dahavargadaphavargaphahavargayogena tulyo bhaviṣyati || 
 tasmāt phahaṃ japhaṃ ca samānaṃ jātaṃ || anayoryogaḥ kajakabayorantaramasti || punaḥ kadajatribhuje dasamakoṇāt daphalambaḥ karṇoparyāgato sti || 
 tasmāt kajasya niṣpattirjadena tathāsti 
 yathā jadaniṣpattirjaphena || tasmātkajajaphaghātaḥ dajavargeṇa samāno jātaḥ || 
 tadetadupapannaṃ iṣṭacāpārdhapūrṇajyāvargaḥ cāponabhārddhāṃśapūrṇajyāvyāsāntaraguṇitavyāsārdhatulyo bhavatīti || 
 śakalaṃ || 
 
 atha prakṛtamanusarāmaḥ || 
 evamatrakathitakṣetre kabacāpajadacāpe mithaḥ samānestaḥ || 
 evaṃ kajabadacāpe casamānestaḥ || tasmātkabajadarekhāghāto vargo bhaviṣyati || 
atra bajamajñātantaghāvattāvanmitaṃ kalpitaṃ || idaṃ kadena guṇitaṃ jātaṃ yā 
6 |  16 | 49 | 7 | 59 | 8 | 56 | 29 | 40
 
 [kṣetra]
 
asya kabajadaghātarūpa yāva $\|$ NEW FOLIO
 \marginpar{f.~2v J}
dvargasya ca yogaḥ 
 kajavargatulyakajabadaghātena samāno sti || ato yaṃ kajavargatulyo jātastasmātkajavargarāśau eko 
  vargarāśirebhiryāvattāvadbhiradhiko jātaḥ || 
  yā va 1 yā 6 |  16 | 49 | 7 | 59 | 8 | 56 | 29 | 40 || 
asya 
  nidarśanārthaṃ śakalaṃ yathā || || 
prakārāntareṇa kajavargaḥ sādhyate || tatra kajacāpaṃ 
  bhārddhāṃśebhyaḥ śodhyaṃ śeṣasya pūrṇajyā yā vyāsasya cāntaraṃ kāryaṃ || 
 tena 
  guṇitaṃ ṣaṣṭi tulyavyāsārdhaṃ kabavargatulyaṃ bhaviṣyati 
yataḥ kajacāpaṃ 
  kabacāpādviguṇamasti || 
tasmādyadi kabavargaḥ yāva 1 ṣaṣṭi 60 tulyavyāsārddhena yadā 
  bhājyate tadā yāvadvargasya ṣaṣṭyaṃśo labhyante tacca 
  kajacāponabhārddhāṃśapūrṇajyāvyāsāntarasya pramāṇamasti || punaridamantaraṃ 
  saṃpūrṇavyāse 120 śodhitaṃ yāva 1 60 rū 120 idaṃ 
  kajacāponabhārdhaṃśānāṃ pūrṇajyā bhavati || 
asyāḥ vargaḥ yāva va 1 3600 yāva
  240 60 rū 14400 atha ca yadīṣṭacāpapūrṇajyāvargaḥ vyāsavargā chodhyate tatra yacheṣaṃ   taccāponabhārddhāṃśānāṃ pūrṇajyāvargo bhavati || 
yato vyāsena iṣṭacāpapūrṇajyayā 
  iṣṭacāpona bhārddhāṃśānāṃ pūrṇajyayā caikaṃ samakoṇatribhujaṃ bhavati || 
yato vṛttārddhe 
  vyāsaprāṃtā utpannatribhujasya pālikoṇaḥ samakoṇo bhavatīti rekhāgaṇitasya tṛtīyādhyāye   upapannamasti || 
tattribhuje samakoṇasanmukhabhujo vyāso sti || sacakarṇarūpaḥ || punaḥ 
  karṇavarggaḥ bhujakodyovarggayogena samo $\|$ 
  \marginpar{f.~3r J} bhavatītirekhāgaṇitasya 
  prathamādhyaye upapannaṃ || ato tra vyāsavarggaḥ 14400 karṇavarggarūpaḥ || 
asminniṣṭacāponabhārddhāṃśānāṃ pūrṇajyāvarggaḥ yāvava 1 3600 yāva 240 60  rū 14400 śodhitaḥ śeṣaṃ kajapūrṇajyāvargge 'śiṣṭaḥ yāvava 10 3600 yāva 240 60 
ayaṃ dvitīya prakāreṇa   kajavarggaḥ siddhaḥ || 

atha prathamaprākārāgatakajavarggaḥ yāva 1 yā 6 |  16 | 49 | 7 | 59 | 8 | 57 | 49 | 40 
etau   samāvitisamaśodhanārthaṃ nyāsaḥ yāvava 10 3600 yāva 240 60  
%\NoLemma{yā 0?}{6 |  16 | 49 | 7 | 59 | 8 | 56 | 29 | 40  ${\rm crossed  out J}}$  
yā 0?\footnote{6 |  16 | 49 | 7 | 59 | 8 | 56 | 29 | 40  ${\rm crossed  out J}$}
yāvava 0 yava 1 yā 
%\NoLemma{6 |  16 | 49 | 7 | 59 | 8 |}{ 56 | 29 | 40 ${\rm Last three digits crossed out J}}$ 
6 |  16 | 49 | 7 | 59 | 8 |\footnote{ 56 | 29 | 40 ${\rm Last three digits crossed out J}$}


atha   bījagaṇite samīkaraṇasaṃpradāyastvanayārītyāsti || 
sāyathā || 
yadi samayo pakṣayor madhye eko 
  rāśoḥ ṛṇaścettadrāśitulyaṃ pakṣadvaye yojyaṃ tadāpi 8 pakṣadvayaṃ samamevabhavati || 
tasmādatra  prathamapakṣe yāvava 1 3600 idaṃ pakṣadvayaṃ yojyaṃ || 
tatra prathamapakṣe 
  dhanarṇayostulyatvādyāvadvarggavarggasyanāśaḥ avaśiṣṭaṃ yāva 240 60 ayaṃ chedabhaktaḥ yāva 4 ayaṃ 
  prathamapakṣaḥ || 
punardvitīyapakṣe yāvava 1 3600 yojitaḥ yāvava 1 3600 yā va 1 yā 6 |  16 | 49 | 7 | 59 | 8 | 56 | 29 | 
  40 
etau pakṣau punarapi samānau jātau yāva 4 yā 0 yāvava 1 3600 yāva 1 yā 6 |  16 | 49 | 7 | 59 | 8 | 57 | 49 | 40 || 
samapakṣayormadhye yau rāśī ekajātī bhavataḥ  tatra laghurāśirmahadrāśau śodhyate tadāpi pakṣau 
  samāvevāvaśiṣṭau bhavataḥ || tasmādatra yo varggaḥ rāśiḥ ekajātī yo .astyato laghurāśiḥ yāva 1 mahadrāśau 
  yāva 4 śodhitaḥ śeṣaṃ prathamapakṣaḥ yāva 3 dvitīyapakṣe ca yāvava 1 3600 yā 6 |  16 | 49 | 7 | 59 | 8 | 57 | 49 | 40 
  etāvapi samau || punaḥ pakṣadvayamadhye yadyekaḥ khaṇḍitosti dvitīyapakṣaḥ | $\|$ 
\marginpar{f.~3v J} 
  saṃpūrṇaścettadā khaṇḍitarāśiṃ prapūryatāvadguṇaṃ dvitīyapakṣamapi kurvanti || atra sachedarāśiḥ khaṇḍita 
  ucyate chedarahitaḥ saṃpurṇa ucyate || evaṃ kṛte sati pakṣayoḥ samachedatvavidhāya chedāyagama 
  evopapadyate || tasmādatradvitīyapakṣe khaṇḍitarāśiḥ yāvava 1 3600 ayamarthaḥ || atra 
  yāvadvargavargavargarāśirasti tasmādayaṃ haraṃ guṇitaḥ kriyate tāvadyāvadvargarāśirekobhavatīti dvitīya 
  pakṣopyetāvatī guṇanīyaḥ || evaṃ kṛte prathamapakṣe yāva 10800 dvitīyapakṣe yāvava 1 athapramapakṣaḥ 
  vāradvayaṃ ṣaṣṭyordhvākṛtaḥ yāva 3 atha dvitīyapakṣasya yāvadrāśistenaivachedena 3600 saṃguṇya 
  vāradvayaṃ ṣaṣṭyordhvākṛtaḥ yā 6 |  16 | 49 | 7 | 59 | 8 | 56 | 29 | 40  yāva 3 dviparivarttāḥ yāvava 1 yā 6 |  16 | 49 | 7 | 
  58 | 7 | 58 | 56 | 29 | 40  evaṃ jātau pakṣau samau etau yāvattāvatāpavarttitau || yā 3 dviparivarrāḥ yāva 1 rū 6 |  16 | 49 | 7 | 
  59 | 8 | 57 | 29 | 40  atra yāvat ghanaḥ sarūparāśiścedyāvattāvattrayeṇa yā 3 bhājyate tadāyāvattāvatpramāṇaṃ l
  labhyate asya bhāgagrahaṇarītiḥ pūrvācāryaiḥ etāvatkālaparyantaṃ na labdhā || atra yamaśadena rītiḥ pradarśitā || 
  sāyathā rūpasya prathamāṃke yāvattāvatā bhājyaḥ labdhirekāñkasthāpya || punarlabdhighanaṃ śeṣāṃke yojya 
  punaratradvitīyāṃko yāvattāvatā bhājyaḥ labdh$\|$iḥ pūrvalabdheradhaḥ sthāpyā || punarlabdhidvayayogasya ghanaḥ 
  kāryaḥ || evaṃ tatra prathamalabdhighanaḥ śodhyaḥ śeṣaṃ dvitīyalabdhiśeṣe yojyaṃ || punastṛtīyāṃko 
  yāvattāvatā bhājyaḥ iyaṃ labdhiḥ pūrvalabdhidvayādhaḥ sthāpyā || punarasyalabdhitrayayogasya ghanaḥ kāryaḥ 
  tatra labdhidvayayogasya ghanaḥ śodhyaḥ | $\|$ %NEW FOLIO
\marginpar{f.~4r J} 
śeṣaṃ tṛtīyalabdhiśeṣe  yojyaṃ || punastatra caturthāṃ ko yāvattavatā bhājyaḥ || evameveṣṭabhāgaparyantaṃ vidhiḥ kāryaḥ || 
evaṃ yamaśaidena  yāvattāvatpramāṇaṃ nikāśilaṃ  2 | 5 | 39 | 26 | 22 | 29 | 28 | 32 | 52 | 33 | iyaṃ 
  aṃśadvayasya pūrṇajyāsti asyārdhaṃ ekāṃśajyājātā | 1 | 2 | 49 | 43 | 11 | 14 | 44 | 16 | 26 | 17 
asyabhāgaharaṇopapattiḥ  || 
tatra pakṣadvayamadhye ekapakṣe yāvattāvadasti || 
dvitīyapakṣe   yāvattāvadghanaḥ rūparāśiśca || 
evaṃ tatra yāvattāvadjñānaṃ cettadā yāvattāvad ghanaṃ kṛtvā 
  rūparāśau prakṣipya yāvattāvatā bhājyate tatra labdhiryāvattāvanmānaṃ syāt | 
  
atra tu   yāvattāvadjñānaṃ nāstyato rūparāśirevayāvattāvatā bhājyaḥ || 
yallabdhaṃ   tatsarūpayāvadghanasya kopyaṃśo labdhaḥ | 
sacaikānte dhṛtaḥ || punarasya ghanaṃ kṛtvā śeṣe 
  yojya punaratrayāvattāvatā dvitīyāṃko bhājitaḥ evaṃ yallabdhaṃ tatpūrvalabdheradhaḥ sthāpitaṃ evaṃ labdhaṃ yāvattāvaddhanasyāsannobhāgolabdhaḥ 
  atrāsannatāśeṣāvayavasatvābhiravayavatve evasūkṣmālabdhiḥ atha labdhighane pūrvaghanaṃ 
  saṃśodhya yatastadadhikaṃ jātaṃ evamiṣṭabhāgaparyantaṃ muhurvāraṃ vāraṃ kāryaṃ || 

athāsyodāharaṇaṃ || 
tatra dvitīyapakṣe rūpāṇi 6 |  16 | 49 | 7 | 59 | 8 | 56 | 29 | 44 ayatra ṣad 
  dviparivarttāḥ || 
ṣaṣṭerūrdhvamasti || 
yāvattāvattrayaṃ dviparivarttāḥ ṣaṣṭerūrdhvamasti || 

ata  ekajātau bhāge gṛhīte labdhamaṃśāḥ | 2 punarasyaghaneṃśāḥ 8 vāradvayaṃ ṣaṣṭyābhāktāḥ 0 | 
  0 | 8 vikalātmakaṃ vikalāsu yojitaṃ 16 | 27 | punaḥ pū$\|$
\marginpar{f.~4v J}
rvalabdhiśeṣaṃ 16 
  tenauva yā 3 bhaktaṃ labdhāḥ kalā 5 labdhiḥ pūrvalabdheḥ 2 adhaḥ sthāpitāṃ añka 2 | 5 
  punaratraśiṣṭa 1 | 57 punarlabdhidvayasya ghanaḥ aṃ 8 | 2 | 32 | 5 
atra prathamalabdhighanaḥ aṃ 8 
  śodhitaḥ śeṣaṃ aṃ 1 | 2 | 32 | 5 idaṃ śeṣe aṃ 1 | 57 | 7 | 59 | 8 | 56 | 29 | 40 aṃśasthāne yojitaṃ 
  1 | 58 | 10 | 31 | 13  punarhareṇayā 3 aṃśe 58 bhāgogṛhītaḥ labdhivikalātmakā 39 śeṣaṃ aṃ 1 | 10 | 3 | 13  punariyaṃlabdhiḥ pūrvalabdheradhaḥ sthāpya 2 | 5 | 
  39 | asya ghanoṃśādiḥ 9 | 11 | 2 | 32 | 27 | 43 | 39 ||$\|$
\marginpar{f.~5r J}
  
%  ${\rm rest of f.4v blank}$

atradvitīyaghanaḥ 9 | 2 | 32 | 5 śodhitaḥ śeṣaṃ kalādi ? | 30 | 27 | 27 | 43 | 39  idaṃ śeṣāñkeṣu 1 | 
 10 | 31 | 13 yojitaṃ 1 | 19 | 1 | 41 | 24 | 13 | 19 punaratrakalāsthāne bhaktaḥ la 26 pūrvalabdheradhaḥ 
 sthāpitā 2 | 5 | 39 | 26 śeṣaṃ kalādi 1 | 1 | 41 | 24 | 13 | 19 punarlabdhaiḥ 2 | 5 | 39 | 26 | ghanaḥ 9| 11 |
  8 | 14 | 33 | 12 | 23 | 21 | 4 | 56 tṛtīya ghanotra śodhitaḥ vikalāsvavaśiṣṭaṃ 5 | 42 | 5 | 28 | 44 | 21 | 4 |
  56 | 7 | punaridaṃ śeṣāñke yojitaṃ ka 1 | 7 | 23 | 29 | 42 | 3 | 21 | 4 | 56 punartenaivabhakte labdhaṃ 
  22 śeṣaṃ 1 | 23 | 29 | 42 |3 | 21 | 4 |56 pūrvalabdhau yojitaṃ 2 | 5 | 39 | 26 | 22 asya ghanaḥ 9| 11 | 8 | 
  19 | 22 | 41 | 6 | 52 | 15 | 1 | 56 caturthaghanotra śodhitaḥ śeṣaṃ 4 | 49 | 28 | 43 | 31 | 10 | 5 | 56 idaṃ 
  śeṣe yojitaṃ 1 | 28 | 19 | 10 | 46 | 52 | 15 | 1 | 56 punarharabhaktaḥ labhdaṃ 29 śeṣaṃ 1 | 19 | 10 | 46 
  | 52 | 15 | 1 | 56 punarlabhiradhaḥ sthitā 2 | 5 | 39 | 26 | 22 | 29 | asyaghanaḥ 9 | 11 | 8 | 19 | 29 | 2 | 42 | 1 
  | 39 | 50 | 52  atrapañcamaghanaḥ śodhitaḥ śeṣaṃ 6 | 21 | 35 | 9 | 24 | 48 | 56 idaṃ śeṣaṃ yojitaṃ 
  1 | 25 | 32 | 22 | 1 |39 | 50 | 52  punarayaṃ harayā 3 bhaktaḥ labdhaṃ 28 śeṣaṃ 1 | 32 | 22 | 1 |39 | 50 | 
  52 labdhiḥ pūrvalabdheradhaḥ sthāpitā 2 | 5 | 39 | 26 | 22 | 29 | 28 asyaghanaḥ 9 | 11 | 8 | 19 | 29 | 8 | 50 
  | 27 | 19 | 59 | 43 atraṣaṣṭaghanaḥ śodhitaḥ śeṣaṃ 6 | 8 | 25 | 40 | 8 | 51 śeṣaṃ yojitaṃ 1 | 38 | 30 | 
  27 | 19  | 59 | 43 punarharabhaktaṃ labdha 32$\|$
\marginpar{f.~5v J} 
śeṣaṃ 2 | 30 | 27 | 19 |59 | 43 labdhiḥ pūrvalabdheradhaḥ 
  sthāpitā 2 | 5 | 39 | 26 | 22 | 29 | 28 | 32 asyaghanaḥ 9 | 11 | 18 | 19 | 29 | 8 | 57 | 28 | 23 | 37 | 1 atrasapamaghanaḥ śodhitaḥ śeṣaṃ 7 | 1 | 3 | 37 | 18  idaṃ śeṣe yojitaṃ 2 | 37 | 28 | 23 |  27 | 1  harabhakte labdhaṃ 52 śeṣaṃ 1 | 28 | 23 | 37 | 1 labdhiḥ 
 
pūrvalabdheradhaḥ sthāpitā 2|5|39 | 26 | 22 | 29 | 28 | 32 | 52 asya ghanaḥ 9 | 11 | 8 | 19 | 29 | 8 | 57 | 39 | 47 | 50 | 19 punaratrāṣṭamaghanaḥ śodhitaḥ śeṣaṃ 11 | 24 | 13 | 18 
  śeṣe yojitaṃ 1 | 39 | 47 | 50 | 19 harabhaktaḥ labdhaṃ 33 śeṣaṃ 0 | 47 |50 | 19 | 
la@  
pūrvavat 2 | 5 | 39 | 26 | 22 | 29 |28 | 32 | 52 | 22 etaparyantaṃ gṛhītaṃ athātra yāvattāvanmānānayanārthamupāyantaramāvidena niṣkāsitaṃ tadyathā rūpāṇi yāvattāvatā bhājyānilabdherghanaḥ kāryaḥ punarghato pi yāvattāvatā bhājyaḥ yallabdhaṃ tatprathamalabdhau yojyaṃ punasvasyghanaḥ kāryaḥ evamasakṛt 

atropapattiḥ
 iha pūrvavadyāvattāvata bhakterūparāśau kaścidyāvattāvato bhāgolabdhaḥ punastasya ghanaṃ kṛtvā 
 tadyojanenavāstaṃ vaghanasyāṃ svannatājātā evamuhuḥ sthirī bhūtaṃ tadeva ghanasvarūparāśeryāvattāvatobhāga upalabdhaḥ sa evayāvattāvanmānamupapannaṃ 
atrodāharaṇaṃ rū 6 | 16 | 49 | 7 | 59 | 8 | 56 | 30 iyaṃ yāvattāvat ṣaṣṭetdvirūrdh$\|$
\marginpar{f.~6r J}
parivarttenaṃ yā 3 bhakte yallabdhaṃ phalaṃ 2 | 5 | 36 | 22 | 29 | 42 | 58 | 50 asya ghanaṃ 9 | ? | 28 | 3 | 8 | 52 | 5 | 39 
punarayaṃ tenaiva yā 3 bhaktaḥ phalaṃ 0 | 0 | 3 | 3 | 29 | 21 | 2 | 57 | 
idaṃ prathamalabdhau yojitaṃ 2 | 5 | 39 | 26 | 9 | 4 | 1 | 47 asya ghanaṃ vāradvayaṃ ṣaṣṭyā bhaktaḥ jāto vikalātmakaḥ 0 | 0 | 9 | 11 | 8 | 16 | 32 | 30 | 48 | 9 punarayaṃ tenaiva bhaktaḥ labdhaṃ 0 | 0 | 3 | 3 | 42 | 41 | 30 | 50 
idaṃ prathamalabdhau yojitaṃ jātaṃ 5 | 39 | 26 | 22 | 28 | 29 | 40 asya ghanaḥ 9 | 11 | 8 | 19 | 28 | 36 | 23 | 50 punastenaiva bhaktaḥ labdhaṃ 0 | 0 | 3 | 3 | 42 | 46 | 29 | 39  prathamalabdhau yojitaṃ 2 | 5  39 | 26 | 22 | 29 | 28 | 29 
asyaghanaḥ 9 | 11 | 8 | 19 | 29 | 8 | 56 | 51 punastenaiva bhaktaḥ labhdaṃ 0 | 0 | 3 | 3 | 42 | 46 | 29 | 42 | idaṃ prathamalabdhau yojitaṃ 2 | 5 | 39 | 26 | 12 | 29 | 28 | 32 asya ghanaḥ 9 | 11 | 8 | 19 | 29 | 8 | 57 | 28 
punastenaiva bhaktaḥ labdhaṃ 0 | 0 | 3 | 3 | 42 | 46 | 29 | 42 prathamalabdhau yojitaṃ  2 | 5 | 39 | 26 | 22 | 29 | 28 | 32 ayaṃ sthirībhūtaḥ iyamaṃśadvayasya pūrṇajyājātā 

athamirjolugbegoktaprakāreṇāṃśadvayasya pūrṇajyā niṣkā ? te tatra pūrvoktameva kṣetraṃ 
kāryaṃ tatra vacihnāt vahalaṃbaḥ kajarekhāyāṃ niṣkāśyaḥ tatra kabakatribhujena vahatribhuje hakoṇaḥ samakoṇo sti kabavargaḥ kahahava$\|$
\marginpar{f.~6v J}
yorvargayogatulyo sti punarbajavargaḥ jahahavayorvargayogatulyo sti 
punaratrakavabajau mithastulyau kalpitau vahamubhayostribhujayorekamevāsti tasmātkahahajaumithaḥ samānau bhaviṣyataḥ kavavargaḥ vahasya vyāsasya ca ghātena samāno sti tasmādyadikabavargo vyāsena bhājyate tadā labdhaṃ bahapramāṇaṃ bhavati || 
punaryadi vahavargaḥ kavavarggāchodhyate tadā śeṣaṃ kahavargo vaśiṣyate atrāṃśadvayasya pūrṇajyārūpākavarekhā 
yāvattāvanmitākalpitā yā 1? asyavargaḥ
%CM: This last phrase may have been intended to be crossed out-it is hard to tell!
 yāva 1 vyāsaḥ 120 ṣaṣṭyordhvā kṛtaḥ 2 ekaparivarttaḥ anena  kavavargo yāva 1 ṣaṣṭerekaparivarttena 2 bhakte labdhaṃ 
 yāvadvargādardhakalāsācayāvadvargasya triṃśadvikalātmikā 30 
 yāvattāvataḥ aṃśātmakatvādidaṃ vahapramāṇajātaṃ asyavargaḥ yāvadvargavargasya
 %CM: the second varga is a marginal addition/correction
paṃcadaśaprativikalāḥ jātāḥ 15 idaṃ kavavarggāchodhitaṃ śeṣaṃ ahavargaḥ yāva 1 yāvavativikalātmakahavargaḥ kajavargasya caturthāṃśattulyo sti tasmātkajavarggaḥ yāva 4 yāvava 1 
vikalā athamija stī graṃthasya prathamadhyāyasyadvhitīyakṣetre isamupapannaṃ kajavadaghātaḥ kajavargarūpaḥ 
kabajadaghā$\|$ 
\marginpar{f.~7r J}
tasya vajakadaghātasya ca yogena tulyo sti tatra kadapramāṇaṃ 6 | 16 | 49 | 8 | 56 | 30 vajaṃ avatulyaṃ 
yāvanmitamasti tasmādvajaṃ kadena guṇitaṃ sadetāvanti yāvattāvatānijātāni yā 6 | 16 | 49 | 7 | 59 | 8 | 56 | 30 kavajadaghātayāva 1 
tasmādidaṃ yāva 4 yāvava 1 vika@ aṃ asya yāva 1 yā 6 | 16 | 49 | 7 | 59 | 8 | 56 | 30 samānaṃ jātaṃ punaretau samaśodhitau tatraikapakṣe 
yāva 3 ? yāva dha 1 vika@ 
yā 6 | 16 | 49 | 7 | 59 | 8 | 56 | 30 etau samānau staḥ atra samayosaṃśau cedgṛhete tadā tāvapi 
samānau bhaviṣyataḥ atra pakṣau tribhaktau yāva 
%%CM: what follows has been added in
yāva va 2 yā 2 | 5 | 35 | 22 | 39 | 42 | 48 | 50 prativikalā
%%CM: end of addition
punareau yāvattāvatāya varttitau yathā yā 1 
yāva 2 rū 2 | 5 | 36 | 22 | 39 | 42 | 58 | 50 
tatrādṛṣṭāṃkasya cikīrṣāsti yasyāṃkasya rūpa rāśi 
sthitāṃkasya cāṃtaraṃ tadaṃka ghatasya ciṃśati 
prati vikalātmakaṃ bhavati sa evāṃka iṣṭo bhavati tasyotpādaneyamupāya upalabdhaḥ rūpāṇāṃ ghanaḥ kṛtaḥ 9 | 10 | 28 | 3 | 8 | 52 | 5 | 39 
idaṃ viṃśati prativikalābhirguṇitaṃ 0 | 0 | 3 | 3 | 29 | 21 | 2 | 57 rūpeṣu yojitaṃ 2 | 5 | 39 | 26 | 9 | 4 | 1 | 47 asya ghanaḥ 9 | 11 | 8 | 16 | 32 | 30 | 48 | 9 punaridaṃ viṃśatiprativikalābhirguṇitaṃ 0 | 0 | 3 | 3 | 42 | 45 | 30 | 50 
rūpeṣu yojitaṃ 2 | 5 | 39 | 26 | 22 | 28 | 29 | 40 asya ghanaḥ   $\|$\marginpar{f.~7v J} 
9 | 11 | 8 | 19 | 28 |56 | 23 | 50 punastenaivapra@ vi@ 
20 guṇitaḥ 0 | 0 | 3 | 3 | 42 | 46 | 29 | 39  
rūpeṣu yojitaḥ 2 | 5 | 39 | 26 | 22 | 29 | 28 | 29 punarghana 9 | 11 | 8 | 19 | 29 | 8 | 56 | 51 punastenaiva pra@  vi@ 20 guṇitaḥ 0 | 0 | 3 | 3 | 42 | 46 | 29 | 42 
rūpeṣu yojitaṃ 2 | 5 | 39 | 26 | 22 | 29 | 28 | 32 
idamaṃśa dvayasya pūrṇajyārūpa iṣṭāṃko jātaḥ padāsya ghanaḥ kriyate viṃśāti pratikalābhirguṇyate 
rūpeṣu  jyojite tadā sa evāṃko bhavati athamirjolugvegasya dvitīyaḥ prakāraḥ tatra kavajadaṃ cāpaṃ ṣaḍaṃśānāṃ kalpaṃ 
yasya vṛttasyedaṃ cāpaṃ tadvṛttakeṃdraṃ taṃ kalpanīyaṃ punaḥ prayekaṃ kavacāpaṃ vajacāpaṃ jadacāpamaṃ:sadvayaṃ kalpitaṃ 
tavakajakadavajavadajadapūrṇajyāḥ saṃyojyāḥ punaḥ kavapūrṇajyākajapūrṇajyākadapūrṇajyānāha ? 
vacihneṣvardhakāryaṃ punasnaharekhātadbhatavarekhāḥ saṃyojyāḥ etārekhāḥ  pratekaṃtāsu pūrṇajyā sulaṃvarūpābhaviṣyanti punaḥ katarekhāvyāsārdhasaṃyojyaṃ punaḥ tat acihnerdhataṃ kāryaṃ punaḥ acihnaṃkendraṃ kṛtvā 
akavyāsārddhena vṛttaṃ kāryaṃ tatra kahatakoṇakajhatakoṇakavatakoṇāḥ samakoṇāḥ santi tadā vṛttārdhametat hajhavacihneṣu gamiṣyati punaḥ hajhajhabavatahatarekhāḥ saṃyojyāḥ hacihnāt halalaṃva'h kajhapūrṇajyā suniṣkāśyaḥ $\|$\marginpar{f.~8r J} 
tadā kajhapūrṇajyā lacihneṣvardhatā bhaviṣyati tatra jhavarekhayā kajabhujakadabhujayorardhasyāne kārttataṃ tadā jhavarekhājadarekhāyāḥ samānāntarā bhaviṣyati tasmātkajadatribhujakajhavatribhuje mithaḥ sajātīyebhaviṣyanta kavaṃ kadasyārdhamasti jhavaṃjadasyārdhaṃ bhaviṣyati 
punaranenaiva prakāreṇa hajharekhāvajasyārdhaṃ bhaviṣyati kahaṃkavasyārdhamasyeva tasmāt kahahajhajhavapūrṇajyāmithaḥ samānā bhaviṣyati 
punaḥ acihnāt alalaṃvaḥ kajhapūrṇajyāyāṃ niṣkāśyaḥ ayaṃ laṃvaḥ kajharekhāṃlacihne rddhitāṃ kariṣyati 
punarayaṃ vārdhataḥ san hacihne lagiṣyati tasmātkahamekāṃśajyāyāvattāvanmitā kalpyā punaḥ kavaṃ aṃśatritayaṃsyajyā etāṃvatyasti 3 | 8 | 24 | 33 | 59 | 34 | 28 | 15 
yadi kahavargaḥ yāvakatarekhayā 60 bhājyate tadā labdhirekākalāyāvadvargasya bhaviṣyati idaṃ halapramāṇamasti yathāpūrvamupapannaṃ asyavargaḥ yāvadvargaḥ yāvadvargavargasyaikā $\|$\marginpar{f.~8v J} 
vikalā bhaviṣyati ayaṃ kahavarge śodhitaḥ śeṣaṃ kalavargo bhaviṣyati saca yā 1 yāvava vikalā kajhavargasya caturthāṃśaḥ 4 
kalavargo sti tasmāt kajhāvargaḥ yāva 4 yāvava 4 vikalā punaḥ kajhahavaghātarūpaḥ 
kajhavargaḥ kajhahavaghātarūpavat vargasya  hajhakavaghātasya ca yogeta tulyo sti 
kahavargaścaitāvān yāva 1 hajhakavaghātaśca 3 | 8 | 24 | 33 | 59 | 34 | 28 | 15 ayaṃ 
dvitīya prakāreṇa kajhavargaḥ siddhaḥ etau samāviti samaśodhanārthaṃ nyāsaḥ  
yathā yāva 4 yāvava 4 vikalā yāva 1 yā 3 |8 | 24 | 33 | 59 | 34 | 28 | 15 atra 
saṃpradāyena śodhane kṛte sati prathamapakṣe yāva 3  yāvava 4 vikalā yā 3 | 8 | 24 | 
33 | 59 | 34 | 28 | 15 dhane kṛte sati prathamapakṣe anayostryaṃśāvapi samārau 
tasmāttribhirapavarttitau avaśiṣṭāvetāvapisamānau yāva 1 yāvava 1 virū 1 | 2 | 48 | 11 | 
19 |11 | 29 |25
%%CM: there are a few formatting issues here and ī'm not sure all is in the right order:
%% check!
punaretauyāvatāvatāpavarttitau jātau pakṣau yā 1 virū 1 | 2 | 48 | 11 19 | 51 | 29 | 25 yāgha 1 2 pra@ vi@
yadyatratādyaśoṃko labhyate yasyāṃ kasya punaḥ rūpasya cāṃtaraṃ aṃka dhanasyaikakalāviṃśati prativikalā tulyaṃ bhavet 
 sacānena prakāreṇa niṣkāṃśitaḥ tatrarūpānāṃ ghanaḥ aṃśādi 1 | 8 | 48 | 30 | 23 | 36 | 30 | 42 | 18 ayamekayākalayāviṃśatiprativi$\|$\marginpar{f.~9r J} kalābhiśca guṇitaḥ 0 | 0 | 31 | 44 | 40 | 3 | 28 | 41 | aṃśakalāsthānayorabhāvāt sthānadvayeśunyaṃ niveśitaṃ 
 rūpeṣu saṃyojyajātaṃ 1 | 2 | 49 | 43 | 4 | 32 | 0 | 53 | 41 asyaghana 1 | 8 | 53 | 32 | 4 | 3 | 50 | 59 | 14 | 57 punarayametena 1 20 guṇitaḥ 0 | 0 | 1 | 31 | 51 | 22 | 45 | 25 | 8 rūpeṣu yojitaṃ 1 | 2 | 49 | 43 | 11 | 14 | 18 | 50 | 8 asyaghanaḥ 1 | 8 | 53 | 36 | 26 | 7 | 18 | 0 | 22 | 10 punarayaṃtenaiva 1 20 guṇita 0 | 0 | 1 | 3 | 51 | 23 | 4 | 49 | 20 rūpeṣu saṃyojya 1 | 2 | 49 | 43 | 11 | 14 | 44 | 16 | 30 asyaghanaḥ 1 | 8 | 53 | 32 | 26 | 8 | 37 | 5 | 34 punastenaiva 1 20 guṇitaḥ 0 | 0 | 1 | 31 | 51 | 23 | 14 | 51 | 29 
 rūpeṣusaṃyojya 1 | 2 | 49 | 43 | 11 | 4 | 44 | 16 | 36 ayamiṣṭāṃko jātā ata iyamekāṃśajyāsiddhā kṛtaḥ tyato sya ghane anena 1 20 vi pra 
 %%formatting issues vi is above pra inline with 1 and 20.
 guṇite rūpeṣu yojyate tadā yameva bhavati tasmādayameva 
 ekāṃśajyājātā idameveṣṭaṃ athātramirjolagvegena 
 yat dvitīyaprakāreṇa kṣetraṃ pradarśitaṃ tatra vahurekhā saṃyogaḥ kṛtaṃ 2? athātra yathālparekhābhire$\|$\marginpar{f.~9v J}
 vasidhyati tathā yatitamāvidasaṃjñai ? 
 tatra tāvadevakavajadacāpaṃ pūrvokti evapūrṇajyāḥ 
 saṃyojyā kavapūrṇajyārdharūpaṃ kahaṃ ekāṃśajyā sti 
sāyāvanmitā kalpitā yā 1 kavarekhā yā 2 evaṃ vajaya 2 jadayā 2 yata etāmithaḥ 
samānāḥ saṃti punaḥ kadarekhāṃ 6 | 16 | 49 | 7 | 59 | 8 | 16 anayāvajaṃyā 2? 
guṇitaṃ kavajadaghātayutaṃ jātaṃ yāva ? yā 12 | 33 | 38 | 15 | 58 | 16 | 33 | 0 ayaṃ 
jātaḥ kajavargaḥ punaḥ prakārāṃtareṇa kajavargaḥ sādhyate ? kavavargaḥ yāva 4 
ayaṃ vyāsena 2 ṣaṣṭerekordhaparivarttena bhakto labdhaṃ 
 
 kṣetra
 
yāvadvargasya kalā 2 idamaṃśadvayasya utkramajyā jātā asyā vargaḥ yāvava 4 
vikalā kavavarge śodhitaḥ śeṣaṃ yāva 4 yāvava 4 vikalā ayamaṃśadvayajyāvargo 
jātaḥ ayaṃ caturgu$\|$ṇitaḥ
\marginpar{f.~10r J} yāva 16 yāvava 16 vika@ ayaṃ 
siddhaḥ kajavargaḥ etau pakṣau samau samaśodhianārthaṃ nyāsaḥ yāva 4 yā 12 | 
33 | 38 | 15 | 58 | 16 | 33 śodhite śeṣaṃ punarddhāda
 %CM: some formatting issues to resolve  
 yāva 1 yāvava 16 vikalā
 
 yāva 12
 yāvava 16 yā 12 | 33 | 38 | 15 | 58 | 16 | 33 yāvattāvatāpavarttya idaṃ mirjolugvegena niṣkāśita vahuprayāśena 
 
 yā 1 yāva 1 20 virū 1 | 2 | 48 | 11 | 19 | 51 | 29 | 25
 
  ${\rm rest of f.4v blank}$
 
 
 
  śrīgaṇeśāya namaḥ || \marginpar{f. 1r J2}
athaikāṃśajīvāviṣaya\footnote{@viṣaye ${\rm J1}$}
ulugvegījīkasya 
śarahavirjandīsya\footnote{@virjaṃdīstha ${\rm J1}$}
vyākhyā likhyate ||
tatraikāṃśajīvānayanena vyataraṃ prakāradvayamasti ||
ekaṃ yamaśaidakāśīsaṃjñena kṛtam ||
dvitīyaṃ miryolugvegena kṛtam ||
paraṃcolugvegasya yamasaidakāśīvedhaprakriyāyāṃ 
sā hāthakāryasti|
tatra prathamaprakāraḥ kathyate| saṃ yathā ||
avajadaṃ ṣaḍaṃśa 6 cāpaṃ kalpitam ||
punarasya samānaṃ bhāgatrayaṃ bacihnajacihnayoḥ kṛtaṃ
tatra abamajamahaṃ vajaṃ badaṃ jadaṃ caitāḥ pūrṇajyārekhā 
yojyāḥ || 
athātrapūrvoktaprakāreṇāṃśatrayasya\footnote{athātrapūrvākta@
${\rm J1}$}
jīvā jñātāsti sā ceyaṃ 3| 8| 24| 33| 59| 34| 28| 54| 50 
iyaṃ dviguṇā 6| 16| 49| 7| 59| 8| 56| 29| 40 jātā madapūrṇajyā ||

atrāṃśadvayasyābapūrṇajyā jñātamiṣṭamasti ||
tatra mijistīgranthasya 
prathamādhyāyasthadvitīyakṣetra 
 idamupapannam ||\footnote{@dvitīyakṣetrayidamupa@ ${\rm J1}$}
yaccaturbhujavṛttāntaḥ patati 
tatsanmukhasthabhujānāṃ\footnote{tatsaṃnmukha@ ${\rm J1}$}
ghātayogastaccaturbhujāntaḥ patitakarṇadvayaghātasamo bhavati ||
punastatraiva prathamādhyāyasya 
caturthakṣetra idamupapannam ||\footnote{caturthakṣetre idamupa@ ${\rm J1}$}
iṣṭacāpārdhapūrṇajyāvargastenāṅkenasamo bhavati ||
yo kaḥ\(|\) taccāponabhārdhāṃśānaṃ yā pūrṇajyā bhavatyasyā
vyāsena yadantaraṃ tena guṇita yo 
vyāsārdhastadaṅkatulyo\footnote{vyāsārdhaḥ stadaṃka@ ${\rm J1}$}
 bhavati ||

evamatra abacāpajadacāpe mithaḥ samāne sthaḥ|
evaṃ ajabadacāpe ca samāne staḥ ||
tasmāt abajadaghāto vargo bhaviṣyati ||
atra bajamajñātaṃ tadyāvattāvanmitaṃ kalpitamidamadenaguṇitaṃ
jātaṃ vā 6| 16|49| 7| 59| 8| 56| 29| 40|
evaṃ abajadaghātarūpavargarāśeryāvattāvataśca yogaḥ
ajavargatulya anabadathā tena samāno ṣti|

ato .ayaṃ ajavargatulyo jātaḥ || 
tasmāt 
ajavargarāśirekovargarāśirebhiryāvattāvadbhiradhiko
\footnote{ajavargarāśī eko@ ${\rm J1}$}
jātaḥ|
yāva 1 yā 6| 16| 49| 7| 59| 8| 56| 29| 40 

atha 
prakārāntareṇa ajavargaḥ\footnote{prakārāṃtareṇājavargaḥ ${\rm J1}$}
 sādhyate ||
tatra ajacāpaṃ bhārdhāṃśebhyaḥ śodhyaṃ śeṣasya
pūrṇajyāyā vyāsasya cāntaraṃ kāryaṃ tena ghna ṣaṣṭitulyavyāsārdhaṃ
avavargatulyaṃ bhaviṣyati ||
yataḥ ajacāpaṃ avacāpāddviguṇamasti ||
tasmādyadi avavargo\footnote{avabargaḥ ${\rm J1}$}
yāva 1 ṣaṣṭi 60 tulyavyāsārdhena bhājyate tadā yāvadvargasya
ṣaṣṭyaṃśo labhyate tacca ajacāpānabhārdhānāṃ yā pūrṇajyā
tadūno yo vyāsastasya pramāṇamasti ||
punaridamantaraṃ saṃpūrṇavyāse 120 śodhitaṃ
yāva \upbefore{0}1 \downafter{ 60} rū 120
idaṃ ajacāponabhārdhāṃśānāṃ pūrṇajyā jātā ||
asyā vargaḥ\(|\) 
yāvava \upbefore{1}3600 yāva \upbefore{0}240\downafter{60}
rū 14400
atha ca yadīṣṣtacāpapūrṇajyāvargo vyāsavargācchodyate 
tatra śeṣaṃ %\[ 
\marginpar{f. 1v J2}
taccāponabhārdhāṃśapūrṇajyāvargo 
bhavati ||\footnote{bhāvati ${\rm J1}$}
yato vyāseneṣṭacāpe pūrṇajyayā iṣṭacāponabhārdhāṃśapūrṇajyayā
caikasamakoṇatribhujaṃ bhavati yato vṛttārdhe vyāsaprāntādutpannatribhujasya
pālikoṇaḥ samakoṇo 
bhavatītyuklīdasasya 
tṛtīyādhyāya\footnote{tṛtīyādhyāye upa@ ${\rm J1}$}
 upapannamasti ||
tattribhuje 
samakoṇasanmukhabhujo vyāso .asti ||
sa ca karṇarūpaḥ ||
punaḥ karṇavargo bhujakoṭyorvargayogena samo 
bhavati|\footnote{bhavatī| ${\rm J1}$}
bhavatītyuklīdasasya 
prathamādhyāya\footnote{prathamadhyāye upa@ ${\rm J1}$}
upapannam ||

ato .atra vyāsavargaḥ 14400 karṇavargarūpaḥ ||
asminiṣṭacāponabhārdhāṃśānāṃ pūrṇajyāvargaḥ
yāvava \upbefore{1}3600 yāva \upbefore{0}240\downafter{60}
rū 14400
śodhitaḥ śeṣaṃ ajapūrṇajyāvargo .avaśiṣṭaḥ ||
yāvava \whitespace\whitespace\whitespace 
\upbefore{0}1 \downafter{ 3600} yā \upbefore{240}60
ayaṃ 
dvitīyaprakāreṇa ajavargaḥ\footnote{@prakāreṇā .ajavargaḥ ${\rm J1}$}
siddhaḥ ||
asyaprathamaprakārāgata ajavargaḥ yāva 1 yā 6| 16| 49| 7| 59| 8| 56| 29| 40 
etau samāvitisamaśodhanārthaṃ nyāsaḥ
$\nl$
\Column{yāvava \whitespace\whitespace\upbefore{0}1 \downafter{3600}}
${8em}$
\Column{yāva \upbefore{240}60}${6em}$
\Column{yā 0}${3em}$
%$\nl$
\Column{yāvava 0}${8em}$
\Column{yāva 1}${6em}$
\Column{yā 6| 16| 49| 7| 59| 8| 56| 29| 40}${16em}$

atha yāvanīyabījagaṇite samīkaraṇasaṃpradāyastvanayā rītyāsti
sā yathā ||
yadi samayoḥ pakṣayormadhya ekārāśirṛṇaścettadrāśitulyaṃ
 \footnote{pakṣayormadhye ekārāśiḥ ṛṇaścet tadrāśi@ ${\rm J1}$}
pakṣadvaye yojyaṃ\footnote{pakṣadvayo jojyaṃ ${\rm J1}$}
tadāpi pakṣadvayaṃ samameva bhavati ||
tasmādatra prathamapakṣe yāvava \whitespace\whitespace\whitespace 
\upbefore{0}1 \downafter{ 3600} 
idaṃ pakṣadvaye yojyaṃ tatra prathamapakṣe 
dhanarṇayostulyatvādyāvadvargavargasya nāśo .avaśiṣṭo
yāva  \upbefore{240}60
ayaṃ chedabhakto yāva 4\(|\) ayaṃ prathamapakṣaḥ punardvitīyapakṣe
yāvava \upbefore{1}3600  yojitaḥ yāvava \upbefore{1}3600 
yāva 1 yā 6| 16| 49| 7| 59| 8| 56| 29| 40
etau pakṣau punarapi samau
$\nl$
\Column{yāva 4}${8em}$
\Column{yā 0}${4em}$
$\nl$
\Column{yāvava \upbefore{1}3600}${8em}$
\Column{yāva 1 yā 6| 16| 49| 7| 59| 8| 56| 29| 40}${28em}$

athayāvanīyasaṃpradāye samapakṣayormadhye yau
rāśyekajātī bhavataḥ ||
tatra laghurāśirmahadrāśau śodhyate tadāpi pakṣau samāvedāvaśiṣṭau
bhavataḥ ||
tasmādatra yāvargarāśirekajātī yo .astyato laghurāśiryāva 1 
mahadrāśau yāva 4 śodhitaḥ śeṣaṃ prathamapakṣo yāva 3
dvitīyapakṣe ca yāvava \upbefore{1}3600 
yā 6| 16| 49| 7| 59| 8| 56| 29| 40
etāvapi samau ||
punaḥ\footnote{puna ${\rm J1}$}
pakṣadvayamadhye yadyekopakṣaḥ khaṇḍito dvitīyapakṣaḥ
saṃpūrṇaścettadā khaṇḍitarāśiṃ prapūryaṃ 
tāvadguṇitaṃ dvitīyapakṣamapi 
kurvati ||\footnote{kurvaṃti ${\rm J1}$}
atra sachedarāśiḥ khaṇḍitaśabdenocyate|
chedarahitaḥ saṃpūrṇocyate ||\footnote{saṃpūrṇa ucyate ${\rm J1}$}
evaṃ kṛte sati pakṣayoḥ samachedatvaṃ\footnote{pakṣayosama@ ${\rm J1}$}
vidhāye chedāya ${\rm ?}$
gama evopapadyate ||
tasmādatra dvitīyapa\[kṣe \marginpar{f. 2r J2}
khaṇḍitarāśiryāvava \upbefore{1}3600 
ayamarthaḥ\(|\)

atra yāvadvargavargarāśeṣaṭ śatādhikasahasratrayamitoṃśo
yāvadvargavargarāśirasti\(|\)
tasmādayaṃ haraguṇitaṃ kriyate tāva 1 ${\rm ?}$ dyāvadvargavargarāśireko
bhavatīti\footnote{bhavatī ${\rm with}$ ti ${\rm inserted in left margin J1}$}
dvitīyapakṣamapyetāvatā guṇanīyam || 
evaṃ kṛte 
prathama\(pa\)kṣe\footnote{prathamakṣe ${\rm J1}$}
yāva 10800 dvitīyapakṣe yāvava 1 asya prathamapakṣo 
vāradvayaṃ ṣaṣṭodhva ${\rm ?}$ kṛto yāva 3\(|\)
atha dvitīyapakṣasthayāvadrāśistenaiva chedena 3600
saṃguṇya vāradvayaṃ ṣaṣṭyo 3 rdhva 1 ${\rm ?}$ kṛto
yā 6| 16| 49| 7| 59| 8| 56| 29| 40
evam jātau pakṣau samau 
$\nl$
\Column{yāva 3 dviparivarttāḥ}${12em}$
$\nl$
\Column{yāvava 1 yā 6| 16| 49| 7| 59| 8| 56| 29| 40}${30em}$
etau yāvattāvatā pannatti ${\rm ?}$ dviparivarttā tau
$\nl$
\Column{yāva 3 dvipariva@}${12em}$
$\nl$
\Column{yādya ${\rm ?}$ 1 rū 6| 16| 49| 7| 59| 8| 56| 29| 40 dvipariva@}${36em}$

athātrayāvaddvayena sarūpastharāśinā 
cedyāvattāvattrayaṃ\footnote{cedyāvattāvatrayaṃ ${\rm J1}$}
 3 bhājyate
tadā yāvattāvatpramāṇaṃ labhyate| 
asya bhāgagrahaṇarītiḥ pūrvācārpaire ${\rm ?}$ tāvatkālaparyantaṃ
na labdhā|
atra yamaśaidena rītiḥ pradarśitā ||
sā yathā ||
rūpasya prathamāṅko yāvattāvatā bhājyaḥ\(|\) labdherekānte
sthāpyā ||
punarlābdhighanaṃ śeṣāṅke yojyaṃ punaratra 
dvitīyāṅko\footnote{dvitīyāṃkau ${\rm J1}$} 
yāvattāvatā bhājyaḥ|
labdhipūrvalabdheradhasthāpyaḥ ||
punarlabdhidvayayogasya ghanaḥ kāryaḥ\(|\)
evaṃ tatra prathamalabdhighanaḥ śodhyaḥ ||
śeṣaṃ dvitīyalabdhiśeṣe yojyam ||\footnote{jyojyaṃ ${\rm J1}$} 
punastṛtīyāṅko yāvattāvatā bhājyaḥ ||
iyaṃ labdhiḥ pūrvalabdhidvayādhasthāpyaḥ ||
punarasya labdhitrayayogasya ghanaḥ kāryaḥ\(|\)
tatra labdhidvayayogasya ghanaḥ śodhyaḥ ||
śeṣaṃ tṛtīyalabdhiśeṣe yojyam ||
punastatracaturthāṅko yāvattāvatā bhājyaḥ ||
evamete ${\rm ?}$  ṣṭabhāgalabdhiparyantaṃ vidhiḥ kāryaḥ ||
evaṃ yamaśaidena\footnote{yamaśadaina ${\rm J1}$} 
yāvattāvatpramāṇo niṣkāśitaḥ 2| 5| 39| 26| 22| 29| 28| 32| 52| 33
iyamaṃśadvayasya\footnote{iyaṃmaṃśa@ ${\rm J1}$}
pūrṇajyāsti ||
asyārdhamekāṃśajyā jātā 1| 2| 49| 43| 11| 14| 45| 16| 16| 17
asya bhāgaharaṇopapattiḥ ||
tatra pakṣadvayamadhya ekapakṣe\footnote{pakṣadvayamadhye eka@ ${\rm J1}$}
yāvattāvadasti ||
dvipakṣe yāvadyataḥ ${\rm ?}$ rūparāśiśca ||
evaṃ tatra yāvattāvadjñānaṃ cettadā yāvattāvadghanaṃ kṛtvā
rūparāśau prakṣipya yāvattāvatā bhājyate tatra labdhiryāvattāvanmānaṃ
syāt ||
atra tu yāvattāvadjñānaṃ\footnote{yāvattravadjñānaṃ ${\rm J1}$}
nāstyato rūparāśireva yāvattāvatā bhājyaḥ\(|\)
yallabdhaṃ tatsarūpayāvaghanasya ko .apyaṃśo labdhaḥ
sa caikānte dhṛtaḥ ||
punarasya ghanaṃ kṛtvā \[ \marginpar{f. 2v J2}
śeṣe jotituṃ punaratra yāvattāvatā dvitīyāṃ ko bhājitaḥ!!
evaṃ yallabdhaṃ tattu\footnote{ttattu ${\rm J1}$}
pūrvalabdheratha sthāpitam ||
evaṃ laghayāvattāvadvyanasyāsanno bhāgo
 \footnote{@svāsannau bhogo ${\rm with strokes struck out, J1}$}
labdhaḥ %\(|\) 
atrāsannatā śeṣāvayavasatvādniravayavatve sa eva sūkṣmā
labdhiḥ ||
athalabdhighane pūrvaghanaṃ śodhyaṃ yatastadadhikaṃ jātam ||
evamiṣṭabhāgaparyantaṃ muhurmuhuḥ kāryam ||

athāsyoda \query
tatra dvitīyapakṣe rūpāṇi 6| 16| 49| 7| 59| 8| 56| 29| 40 ||
atra ṣaṭ 6 dviparivartāḥ ṣaṣṭerūrdhvamasti ||
yāvattāvattrayaṃ dviparivartāḥ ṣaṣṭerūrdhvamasti ||
ata ekajātau\footnote{ato ekajātau ${\rm J1}$}
bhāge gṛhīte labdhamaṃśāḥ || 
punarasya ghane .aṃśāḥ 8 śeṣāṃ 16 śeṣu 59 yojitaṃ 
16| 57 punaḥ pūrvalabdhe śeṣaṃ 28 tenaiva yā 3
bhaktaṃ labdhakalāḥ 5 labdhiḥ pūrvalabdheradasthopita aṃ 2
ka 5 punaratra śiṣṭhaṃ 1| 57 punarlabdhidvayasyaghanaḥ\(|\)
aṃ 9| 2| 32| 5 atra prathamalabdhighanaḥ\(|\)
aṃ 8 śodhitaḥ śeṣaṃ aṃ 1| 2| 32| 57|
idaṃ śeṣe aṃ 1| 57| ka 7| vi 59| 8| 56| 29| 40\footnote{1| 57|\upafter{aṃ} 7|\upafter{ka} 59\upafter{vi}| 
 8| 56| 29| 40 ${\rm J1}$}
aṃśasthāne yojitaṃ \(|\) aṃ 1| 58| 10| 31| 13\footnote{1| 58|\upafter{aṃ} 10| 31| 13 ${\rm J1}$}
puno hareṇa\footnote{punaḥ hareṇa ${\rm J1}$}
yā 3 aṃśe 58 bhogo\footnote{aṃśe \upbefore{58}bhogo ${\rm J1}$} 
gṛhītaḥ \(|\)
labdhirvikalātmikā 39\footnote{labdhirvikālātmikā 39 ${\rm J1}$}  
śeṣaṃ aṃ 1| 10| 31| 13 punariyaṃ labdhiḥ pūrvalabdheradhaḥsthāpyaḥ 
aṃ 2| 5| 39 
asya vyanośādi %\query



 



\end{document}
