%&LaTeX
\documentclass[12pt]{article}
\usepackage{graphicx, times} %% , parcolumns} % , lineno
\input epsf%
%%% for one-inch margins all around:
%\topmargin 0.0in
\oddsidemargin 0.0in
\evensidemargin 0.0in
\textheight 8.25in
\textwidth 6.75in

\let\*=\d

\def\Ganesa{Ga\-\*ne\-\'sa}
\def\Bhaskara{Bh\=a\-ska\-ra}
%
\def\KKT{\textit{Kara\-\*na\-kut\=u\-hala\-\*t\={\i}\-k\=a}}
\def\KKe{\textit{Kara\-\*na\-ke\-\'sari}}
\def\KheSi{\textit{Khe\-cara\-siddhi}}
\def\JCU{\textit{Jy\=a\-c\=a\-potpa\-tti}}
\def\PhCC{\textit{Phira\.ngi\-candra\-cchedyayopa\-yogika}}
\def\Mahadevi{\textit{Mah\=a\-dev\={\i}}}
\def\RG{\textit{Rekh\=a\-ga\*ni\-ta}}
\def\Lil{\textit{L\={\i}l\=a\-va\-t\={\i}}}
%
\def\kalpa{\textit{ka\-lpa}}
\def\kala{\textit{ka\-l\=a}}
\def\kalas{\textit{ka\-l\=as}}
\def\ghatika{\textit{gha\*tik\=a}}
\def\jyotisa{\textit{jyo\-ti\-\*sa}}
\def\pratikalas{\textit{prati\-ka\-l\=as}}
\def\prativikalas{\textit{prati\-vi\-ka\-l\=as}}
\def\bijaganita{\textit{b\={\i}ja\-ga\*ni\-ta}}
\def\ya{\textit{y\=avat\-t\=avat}}
\def\yava{\textit{y\=avat\-t\=avat-varga}}
\def\yagha{\textit{y\=avat\-t\=avat-ghana}}
\def\yavava{\textit{y\=avat\-t\=avat-varga-varga}}
\def\ru{\textit{r\=upa}}
\def\vikala{\textit{vi\-ka\-l\=a}}
\def\vikalas{\textit{vi\-ka\-l\=as}}
%
\def\zij{\textit{z\={\i}j}}
%
\def\opcit{\textit{op.\ cit.}}
\def\loccit{\textit{loc.\ cit.}}
\def\elp{$\ldots$}
\def\danda{$|$}
\def\nl{\hfill\break}
%
\def\sri{\textit{\'sr\={\i}}}
%

% \newcommand{\JOne}[1]{\colchunk{{#1}}}
% \newcommand{\JTwo}[1]{\colchunk{{#1}%
%    }\colplacechunks}


% \linenumbers

\begin{document}
% \begin{parcolumns}[colwidths={1=3 in},nofirstindent]{2}

\title{\JCU:  SRCMI 269.I.13.i.3}
\author{Kim Plofker, Clemency Montelle, and Glen Van Brummelen}
\date{\today \quad [PRELIMINARY DRAFT]}
 \maketitle

\section*{f.~1r}

[\kalas\ are sixtieths of a degree, \vikalas\ sixtieths of a \kala]
\medskip

Homage to Lord \Ganesa. Now, ``When there is knowledge of the Sine of three degrees,
knowledge of the Sine of one degree'': thus he states the Sine with four verses.
Having set the Sine of the arc multiplied by two, multiplied by the kara [?], the 
clever mathematician should set those down separately (the result [is] degrees). 
But when the cube of the degrees of the number [?] of the result is divided
by the square of the half diameter, whatever is the result, having added to the 
remainder [??] of the degrees of that, is to be divided by three. When divided by
three, whatever is the result in minutes, that is to be set down below the degree-number-result
of [?] the previous quotient. The row in degrees and minutes results. 

The cube of the row diminished by the first cube [is] divided by the square of the 
half-diameter. The result, to be added [?] to the remainder, one should divide by three.
Having set the quotient  result below the seconds-number-minutes, it results a row 
in degrees, minutes, seconds.  The cube of the row is to be made. In it, the cube of
the previous is to be subtracted. The remainder is [?] divided by three when divided
by the square, [???] whatever is the result, having increased that by the remainder
of the dividend, again turn by turn [??] in the tip of that but the previous rule is to be 
done. When done thus, [it] should be the result row.  The Sine of a third part of the 
arc  of half the row should be.

Now, thus immediately that Sine of three degrees again from the method I will say.

``Since from the cause by many bhedas the wisdom of the mathematician in 
his own science cleverness'' thus 16. Again, by another method he states the Sine
of three degrees.  ``Diminished by its own third part''. 
Diminished by its own third part of the Sine
of the arc, [it] is to be set down separately.  The cube of it [is] multiplied by three [\elp]
With the quotient-result the separate-standing Sine of the arc is to be added. Again
and again with that made a row is to be determined; the Sine of degrees multiplied
by the arc of half the row should be the Sine of a third part of the arc. 17

Again, he states by another method: ``The Sine of the degrees of arc\elp'':

\section*{f.~1v}
A third part of the Sine of the degrees of arc is to be set down separately. Whatever
is the quotient when its cube
is added to its own degrees [?] [and] divided by the square of the Radius, the result
to  that is to be added separately. 

Thus again turn by turn, with that made, the Sine of a third part of the arc should be. 18

Its rationale: it is as [follows]. $KBJD$ is considered an arc of six degrees. 
Again, its equal [?] three degrees is considered of point $B$ [and] point $J$.
Then $KB$, $KJ$, $KD$, $BJ$, $BD$, and $JD$: they are to be drawn as
lines of Chord. 

Now  here by the previously stated method the Sine of three degrees is known, 
and it is this:  $3|8|24|33|59|34|28|14|50$. This is multiplied by two:
$6|16|49|7|59|8|56|29|40$ results the full Chord of $KD$. Here the knowledge
is desired [of?] the Chord $KB$ of two  degrees. 

Something is said for the sake of its rationale. Then in the 
\textit{Siddh\=anta\-samr\=aj} [i.e., \textit{Siddh\=antasa\*mr\=at} translation
of the Almagest?] in the second figure of the first chapter this is
explained.  ``Whatever quadrilateral falls inside a circle, the sum of the 
products of the sides standing face to face becomes equal to the
product of the two diagonals intersecting [?] inside that quadrilateral''.
But its rationale is just explained previously. 

Then again, in the fourth diagram of the first chapter this is demonstrated. 
The square of the Chord of half the desired arc  becomes
equal to the half-diameter 
multiplied by the difference of the 
diameter and the Chord of the degrees of half a circle minus that arc .
Its rationale: On diameter $KJ$ 
half-circle $KBJD$ is to be made.  Then $BD$ is equal to $DJ$. From point $D$
line $DF$ is to be made perpendicular on diameter $KJ$.  Here line $JF$ is measured
by half the difference of 
diameter $KJ$ and of the Chord of the degrees of half a circle diminished by
the known arc.  Why?  [Just what I was wondering myself.] 
$KB$ is to be made equal to $KH$. Line $DH$ is joined.  [There is equality] of triangle
$HKD$ [and?] triangle $DKB$. Side $KH$ [and] side $KD$ to side $BK$
[and] side

\section*{f.~2r}

$KD$ are respectively equal. 
Again, angle $HKD$ is equal with angle $BKD$. From that, line $DH$ occurs equal
to line $BD$.  And line $DJ$ occurs equal. [?] Then the square of $DJ$ will become 
equal with the sum of the square of $DF$ [?] [and] the square of $JK$. In the 
same way  the 
square of $DH$ will become equal with the sum of the square of $DF$ [and]
the square of $FH$.  From that results equality of $FH$ and $JF$.  The sum of 
those two is the difference of $KJ$ and $KB$. 


\begin{figure}[h]
\includegraphics[width=1.5in] %% 
{JCUFig1}
% \centering
\label{JCUFig1}
\end{figure}

Again, in triangle $KDJ$ from the right [?] angle $D$, perpendicular $DF$ 
is revolved hypotenuse [??].  From that, the coming out of $KJ$ is in the same
way with $JD$ as the coming out of $JD$ with $JF$. 
From that, the product of $KJ$ and $JF$ results equal with the square of 
$DJ$. That this [?] explained. ``The square of the Chord of half the desired arc
becomes equal to the half-diameter multiplied by the difference of the 
diameter and the Chord of the degrees of half a circle diminished by the arc.''
Figure [shakal]. 

Now, we pursue the nature.  [??] Thus here in the explained figure, arcs
$KB$ and $JD$ are equal together. In the same way, arcs $KJ$ and  $BD$ 
are equal.  From that, the product of lines $KB$, $JD$ will be a square.

Here, $BJ$ [is?] unknown, that is considered measured by \ya\ [variable]. 
This is multiplied by $KD$; result, \ya\ $6|16|49|7|59|8|56|29|40|$. 
Of it, and of the square of the \textit{y\=avat}---[that is,] 
the number [of the?] product of $KB$, $JD$---

\section*{f.~2v}

the sum is equal to the product of $KJ$, $BD$ [or] equal to the square of $KJ$.

Here, this occurs equal to the square of $KJ$. From that, in the heap of the
square of $KJ$ one square heap [??], with these \ya s [it] occurs additive. [??]
\yava\ 1, \ya\ $6|16|49|7|59|8|56|29|40$. For the sake of its proof, a figure
[shakal] as [follows]: [?]

\begin{figure}[h]
\includegraphics[width=1.75in] %% 
{JCUFig2}
% \centering
\label{JCUFig2}
\end{figure}

By another method [??], the square of $KJ$ is determined.  There the arc of $KJ$ 
is subtracted from the degrees of half a circle. The difference of the Chord of the
remainder and the diameter is to be made. The half-diameter equal to sixty 
multiplied by that will be equal to the square of $KB$. Since the arc of $KJ$
is twice the arc of $KB$.  From that, if the square of $KB$ \yava\ 1 when it is
divided by the half-diameter equal to sixty 60, then a sixtieth part of the 
\yava\ remains, and that is the amount of the difference
of the diameter [and] the Chord of the degrees of a half-circle
diminished by the arc of $KJ$.  But this difference [is] \yava\ $\frac{1}{60}$, units 120
subtracted from the 
full diameter 120. This becomes the 
Chord of the degrees of half a circle diminished by the arc of $KJ$. 
Its square \yavava\ $\frac{1}{3600}$, \yava\ $\frac{240}{60}$, units 14400. 

And now, if the square of the Chord of the desired arc is subtracted from the 
square of the diameter, there whatever is the remainder is the square of 
the Chord of the degrees of a half-circle diminished by the arc; since one
right-angled triangle is [formed] by the diameter and the Chord of the 
desired arc and the Chord of the degrees of a half-circle diminished by the
desired arc; and since ``the circumference-angle 
of a triangle produced from the end [actually ends]
of the diameter in a half-circle is a 
right angle'' is demonstrated in the third chapter of \RG,
the side facing the right angle in that triangle is the diameter. And it 
has the form of [a] hypotenuse. But ``the square of  the hypotenuse
is equal to the sum of the squares of the arm and the upright''

\section*{f.~3r}

[is] demonstrated in the first chapter  of \RG. So here the square of the diameter
14400 has the form of the square of a hypotenuse. From it, the square of the 
Chord of the degrees of a half-circle diminished by the desired arc,
\yavava\ $\frac{1}{3600}$, \yava\ $\frac{240}{60}$, units 14400 is subtracted.
The remainder is the square of the Chord of $KJ$. Not remaining [???]: 
\yavava\ $\frac{1}{3600}$, \yava $\frac{240}{60}$. This by a second method
is the established square of $KJ$. 

Now the square of $KJ$ first derived, \yava\ 1, \ya\ $6|16|49|7|59|8|57|49|40$
These two are equal: thus for the sake of the same-investigating [?], the statement.
\yavava\ $-\frac{1}{3600}$, \ya\ $\frac{240}{60}$, \ya\ 0 [equals] 
\yavava\ 0, \yava\ 1,
\ya\ $6|16|49|7|59|8|57|49|40$.

Now, in algebra (\bijaganita) the doctrine of the equation is by this method.
That is as [follows]:  When the middle [term?] of two equal sides 
if one amount is negative, [an amount] equal to that amount is to be added to
both sides.  And then the pair of sides become just equal. From that, here in 
the first side the \yavava\ $\frac{1}{3600}$, this is to be added to [?] both sides.
Then in the first side because of equality of positive and negative, there is
destruction of the \yavava. The remaining \yava\ $\frac{240}{60}$: this is 
divided by the divisor, $yava$ 4. This is the first side. But in the second side
\yavava\ $\frac{1}{3600}$ is added, \yavava\ $\frac{1}{3600}$. 
\yava\ 1 \ya\ $6|16|49|7|59|8|57|49|40$.  And again, the two sides occur
equal: \yava\ 4 \ya\ 0, \yavava\ $\frac{1}{3600}$ 

\yava\ 1 \ya\ 6 \danda 16 \danda 49 \danda 7 \danda 59 
\danda 8 \danda 57 \danda 49 \danda 40.   Whatever two quantities are in the middle 
of the two sides (being) equal, the two are produced eka-jAtI (??).  There the small 
quantity is subtracted from the large quantity, then also the  two sides are in the same 
way remain the same.  From that, here whatever is the square, the quantity (?) whatever 
is eka-jAtI (?) the small quantity, \yava\ 1, it is subtracted from the large quantity 
\yava\ 4.  The remainder is the 1st side and on the second side \yava\ 3 
and \yavava\ $\frac{1}{3600}$ (and) y\=a 6 \danda 16 \danda 49 \danda 7 \danda 59 \danda 
8 \danda 57 \danda 49 \danda 40.  These two are also equal.  Again, in the middle (?) 
of the two sides, if one is `broken' (\textit{kha\d n\d dita}), the second side 

\section*{f.~3v}
is full (\textit{sa\d mpur\d na}).  
If (this is the case), then, filling the broken quantity, one should make so much a quantity 
on the second side.  Here the quantity with divisor is called `\textit{kha\d n\d dita}'; 
`\textit{chedarahita} is called full (\textit{sa\d mp\=ur\d na}).  Computing thus, with there 
being two sides,  producing the sameness of divisors, a departure of divisors is produced.  
From that here on the second side the broken quantity is \yavava\  $\frac{1}{3600}$; 
This is the meaning.  Here the quantity of the square of the square of the square [sic!] 
of the \ya, is one, the second side is multiplied by t\=avat.  Computing in this way on 
the first side there is \yava\ 10800 (and) on the second side y\=a va va 1 Now the 
first side is made upwards (?) quantity-pair-60, \ya\ 3.  Now, the quantity of \ya s 
of the second side, multipling by that divisor, 3600, it is made upwards-quantity-pair-60. 
\ya\ 6 \danda 16 \danda 49 \danda 7 \danda 59 \danda 8 \danda 57 \danda 49 \danda 40 
\yava 3 changes to \yavava 1 \ya\ 6 \danda 16 \danda 49 \danda 7 \danda 59 \danda 
8 \danda 57 \danda 49 \danda 40.  In this way these two sides  produced equal are reduced 
to a common measure (\textit{aparvartita}) with respect to \ya\ \yava\ 1 r\=u 6 \danda 
16 \danda 49 \danda 7 \danda 59 \danda 8 \danda 57 \danda 49 \danda 40.

Now the cube of \ya.  If the quantity with units is divided by the three [of the] \ya,
\ya\ 3, 
then an amount of the \ya\ is obtained.    The method of the grasping of the part of it 
by the ancient teachers \elp  Here the rule is expounded by yamashada. 
[= jamshad = Jamsh\={\i}d al-K\=ash\={\i}, although this method is actually
the ``RumiBeg'' alternative to al-K\=ash\={\i}'s]. It is as [follows]:

The first digit of the unit [term] is to be divided by the \ya\ [coefficient]. The quotient is to be set
[at] one end. Again, the quotient-cube is to be added to the remainder-digit;
here again, the second digit is to be divided by the \ya\ [coefficient]; the quotient is to be  set
below the previous quotient. Again, the cube of the sum of the two quotients
is to be made.  In this way there the cube of the first quotient is to be subtracted;
the remainder is to be added in the second quotient. Again, the third digit is to be
divided by the \ya. This quotient is to be set below the previous quotient. Again,
the cube of the sum of its three quotients is to be made. There the cube of
the sum of the two quotients is to be subtracted. 

\section*{f.~4r}

The remainder is to be added in the remainder of the third quotient. Again there, the 
fourth digit is to be divided by the \ya. In just this way the rule is to be done as far
as the desired place. In this way by  Yama\'said the amount of the \ya\ is 
\textit{nik\=a\'sila\*m} [?].
2\danda 5\danda 39\danda 26\danda 22\danda 29\danda 28\danda 32\danda
52\danda 33\danda This is the Chord of two degrees. Half of it occurs ths
Sine of one degree: 
1\danda 2\danda 49\danda 43\danda 11\danda 14\danda 44\danda 16\danda
26\danda 17 

The demonstration with its \textit{bh\=agahara} [? co-heir, divisor?]. There in 
the middle of two sides [Is there an Arabic term for ``in an equation'' that this
is literally translating?  see expressions for ``in an equation'' such as 2 lines
above marginal (8) in Rumi-Beg facsimile], in one side is the \ya. 
In the other side is the \ya-cube
and the amount of units. In this way there if the \ya\ is known, then having
made the \ya-cube and added it to the amount of units, [it] is divided by 
the \ya\ [coefficient]:  there the quotient should be the amount of the \ya. But here the
\ya\ is not known, hence the amount of units itself is to be divided by the \ya\
[coefficient]. 
Whatever is the quotient is obtained [as] some part of the \ya-cube-plus-units.
And it is kept at one end. Again, having made its cube, [it] is to be added
to the remainder. Again here, the second digit is divided by the \ya\ [coefficient]. In this way
whatever is the quotient is set below the previous quotient. In this way
the quotient is taken [as] the approximate-part of the \ya-cube. Here with the 
approximateness and part-of-syllogism-ness are just two part-of-syllogism-ness [???]:
the quotient is accurate. 

Now, having subtracted the previous cube from the cube of the quotient, since
the result [?] is greater than that, in this way up to the desired part 
\textit{mudgarb\=ara\*m v\=ara\*m} [???] is to be made. 

Now, its example. There in the second side are units 6\danda 16\danda 49\danda
7\danda 59\danda 8\danda 56\danda 29\danda 40. Here, six is turned around
[moved up] two [places]. [It] is above sixty [i.e., the sixties place]. 
Three times the \ya\ is turned around [moved up] two [places], above [the] sixty [place]. 
Hence when the divisor is taken of the same order [i.e., same 60'mal place],
the quotient is degrees [i.e., the
units place], 2.
Again, in its cube there are degrees [i.e., integer units], 8.
[They] are divided by sixty ``a pair of turns'' [i.e., twice: what's this in Arabic??]:
0\danda 0\danda 8. Added in the form of \vikalas\ to \vikalas,
16\danda 57.
Again,

\section*{f.~4v}
the remainder of the first quotient 16 [is] divided by just that \ya\ 3; the quotient [is] \kalas, 5.
The quotient is put down below [i.e., to the right of] the first quotient, 2: digit 2\danda 5.
Again, the here-remaining 1\danda 57, again the cube of the two quotients: dig.\ [or deg.? \textit{a\*m}]
9\danda 2\danda 32\danda 5.  Here the cube of the first quotient  dig.\ [or deg.? \textit{a\*m}]
8 is subtracted; the remainder,dig.\ [or deg.? \textit{a\*m}]
1\danda 2\danda 32\danda 5. This in the remainder: dig.\ [or deg.? \textit{a\*m}]
1\danda 57\danda 7\danda 59\danda 8\danda 56\danda 29\danda 40. 
Added in the degrees place: 
1\danda 58\danda 10\danda 31\danda 13.
Again the portion taken with the divisor, \ya\ 3, in [the] degree [place], [1\danda] 58.
The quotient is in the form of \vikalas,
39, the remainder dig.\ [or deg.? \textit{a\*m}] 
1\danda 10\danda 31\danda 13. Again this quotient is to be set below [i.e., after] the 
previous quotient, 2\danda 5\danda 39.
Degrees etc.\ of its cube: 9\danda 11\danda 2\danda 32\danda 27\danda 43\danda 39.

\section*{f.~5r}

Here, the second cube 9\danda 2\danda 32\danda 5 is subtracted; the remainder in \kalas etc.\ 
is 0 [?]\danda 30 27\danda 27\danda 43\danda 39. This is added to the remainder digits:
1\danda 10\danda 31\danda 13: [result] 1\danda 19\danda 1\danda 41\danda 24\danda 13\danda 19.
Again here, divided in the minutes place, quo.\ 26. [It] is set below the first quotient: 
2\danda 5\danda 39\danda 26. The remainder in \kalas\ etc.: 1\danda 1\danda 41\danda 24\danda
13\danda 19. Again, with the quotients, 2\danda 5\danda 39\danda 26, the cube: 9\danda 11\danda
8\danda 14\danda 23\danda 12\danda 23\danda 21\danda 4\danda 56.

The third cube is here subtracted, remaining in \vikalas: 5\danda 42\danda 5\danda 28\danda 44\danda
21\danda 4\danda 56\danda 17 [?]. Again, this is added in the digits of the remainder: ka
1\danda 7\danda 23\danda 29\danda 42\danda 3\danda 21\danda 4\danda 56. Again, when divided with
just this the quotient is 22, remainder 1\danda 23\danda 29\danda 42\danda 3\danda 21\danda 4\danda
56.  Added to the first quotient: 2\danda 5\danda 39\danda 26\danda 22. Its cube:
9\danda 11\danda 8\danda 19\danda 22\danda 41\danda 6 52\danda 15\danda 16 56.

The fourth cube here is subtracted, the remainder 4\danda 49\danda 28 43\danda 31\danda 10\danda
5\danda 56. This is added to the remainder, 1\danda 28\danda 19\danda 10\danda 46\danda 52
15\danda 1\danda 56. Again, divided by the divisor, quotient 29, remainder 1\danda 19\danda 10\danda
46\danda 52 15\danda 2 [?] \danda 56. Again, the quotient is set below: 2\danda 5\danda 39\danda 26\danda
22\danda 29. Its cube: 9\danda 11\danda 8\danda 19\danda 29\danda 2\danda 42\danda 1\danda 39\danda
50\danda 52. 

Here, the fifth cube is subtracted, remainder 6\danda 21\danda 35\danda 9\danda 24\danda 48\danda
56. This is added to the remainder: 1\danda 25\danda 32\danda 22\danda 1\danda 39\danda 50\danda 52.
Again, this is divided by the divisor [?] 3, quotient 28, remainder 1\danda 32\danda 22\danda 1\danda 39\danda
50\danda 52. The quotient is set below the previous quotient, 2\danda 5\danda 39\danda 26\danda 22\danda
29 28. Its cube: 9\danda 11\danda 8\danda 19\danda 29\danda 8\danda 50\danda 27\danda 19\danda 59\danda
43. 

Here, the sixth cube is subtracted, remainder 6\danda 8\danda 25\danda 40\danda 8\danda
51. Added to the remainder: 1\danda 38\danda 30\danda 27\danda 19\danda 59\danda 43. Again, divided 
by the divisor, quotient 32,

\section*{f.~5v}

remainder 2\danda 30\danda 27\danda 19\danda 59\danda 53. The quotient is set below the first
quotient, 2\danda 5\danda 39\danda 26\danda 22\danda 29\danda 28\danda 32. Its cube: 
9\danda 11 18 19\danda 29\danda 8\danda 57\danda 28\danda 23\danda 37\danda 1.

Here the seventh cube is subtracted, remainder 7\danda 1\danda 3\danda 37\danda 18. 
This is added to the remainder, 2\danda 37\danda 28\danda 23\danda 37\danda 1.
When divided by the divisor, the quotient is 52, remainder 
1\danda 28\danda 23\danda 37\danda 1. The quotient 
is set below the first quotient: 2\danda 5\danda 39\danda 26\danda 22\danda 29\danda 28\danda 32\danda
52. Its cube: 9\danda 11\danda 8\danda 19\danda 29\danda 8\danda 57\danda 39\danda 47\danda
50\danda 1 [?]. 

Again, here the eighth cube is subtracted, remainder 11\danda 24\danda 13 [?] \danda 18. Added to the
remainder: 1 [?] \danda 39\danda 47\danda 50\danda 19. Divided by the divisor, quotient 33, 
remainder 57 [?]\danda 50\danda 19. Quo.\ as before: 2\danda [?] \danda 39\danda 26\danda 22\danda
29\danda 28\danda 32\danda 52\danda 33. Seized this far. 

Now here for the sake of the determination of the amount of the \ya, another approach is known 
[?? \textit{\=avideta}, finite verb??]. 
expounded. That is as [follows]: 

The units are divided by the \ya, the cube of the quotient is to be made. And again, the cube is
to be divided by the \ya. Whatever is the quotient, that is to be added to the first quotient. Again,
the cube of that is to be made, and so repeatedly.  

Here is the demonstration. In this case as before, when the amount of units is divided by the \ya, 
something is obtained [as?] the part [?] of the \ya. Again, having made its cube, that with addition
is fixed [? \textit{v\=astava\*m} ?]; the approximateness of the cube is produced. Thus the being
gone-out-fixed [???].  Just that from the amount of the units of its own cube, part with the \ya 
[is] conceived. That itself is obtained [as] the amount of the \ya. 

Here, an example: U.\ 6\danda 16  49\danda 7\danda 59\danda 8\danda 56\danda 30. 
This the \ya\ above of sixty twice

\section*{f.~6r}

with moving back and forth [?] the \ya\ 3 when divided [???],
whatever is the quotient is the result 2\danda 5\danda 36\danda 22\danda 39\danda
42 58 \danda 50. Its cube: 9\danda 16 [?] \danda 28\danda 3\danda 8\danda 52\danda 5\danda 39.
Again, this is divided by that \ya\ 3, result 0\danda 0\danda 3\danda 3\danda 29\danda 21\danda 2\danda 
57. This is added to the first quotient: 2\danda 5\danda 39\danda 26\danda 9\danda 4\danda 0\danda 47.
Its cube, a pair of turns, divided by sixty, the result is in the form of \vikalas: 
0\danda 0\danda 9\danda 11\danda 8\danda 16 32\danda 30\danda 48\danda 9.
Again, this is divided by just that, quotient 0\danda 0\danda 3\danda 3\danda 42\danda 45 [?]
30\danda 50. This is added to the first quotient, result 5\danda 39\danda 26\danda 22\danda 28
29\danda 40. Its cube: 9\danda 11\danda 8\danda 19\danda 28\danda 36\danda 23\danda 50.
Again, divided by just that the quotient is: 0\danda 0\danda 3\danda 3\danda 42\danda 46\danda 29\danda
39. Added to the first quotient: 2\danda 5\danda 39\danda 26\danda 22\danda 29\danda 28\danda 29. 
Its cube: 9\danda 11 8\danda 19\danda 29\danda 8\danda 56\danda 51. Again, divided by just that,
quotient 0\danda 0\danda 3\danda 3 42\danda 46\danda 29\danda 42. This is added to the first 
quotient: 2\danda 5\danda 39\danda 26\danda 22\danda 29\danda 28\danda 32. Its cube: 
9\danda 11\danda 8\danda 19\danda 29\danda 8\danda 57\danda 28. Again, divided by just that,
quotient 0\danda 0\danda 3\danda 3\danda 42\danda 46\danda 29\danda 42. 
Added to the first quotient: 2\danda 5\danda 39\danda 26\danda 22\danda 29\danda 28\danda 32.
This is being fixed. This occurs [as] the Chord of two degrees.

Now, by the method stated by Mirjolugvega [Ulugh Beg], the Chord of two degrees is expounded. 
Then just the previously stated figure is to be made. 

\begin{figure}[h]
\includegraphics[width=2in] %% 
{JCUFig3}
% \centering
\label{JCUFig3}
\end{figure}

There, having led out from point $V$ perpendicular $VH$ on line $KJ$, there in triangle
$FBH$ in triangle $JVH$ [there is?] angle $H$. [It] is a right angle. The square of $KB$ 

[Note: the scribe uses $V$ and $B$ more or less interchangeably to refer to this 
diagram letter.]

\section*{f.~6v}

is equal to the sum of the squares of $KH$, $HV$. Again, the square of $BJ$ is equal to the
sum of the squares of $JH$, $HV$. Again, here the pair $KV$, $BJ$ are considered equal 
[to?] $VH$. Just one of both triangles is.  [?] From that, $KH$, $HJ$ will become an equal pair.
The square of $KV$ is equal to the product of $VH$ and of the diameter. From that, if the 
square of $KB$ is divided by the diameter, then the quotient becomes the amount of $BH$.
Again, if the square of $VH$ is subtracted from the square of $KV$, then the remainder
remains [as] the square of $KH$. 

Here the Chord of two degrees [in?] units [is] considered line $KV$, measured by the \ya. 
\ya\ [?? something erased?] its square \yava\ 1, diameter 120, divided by sixty, 2. 
One back-and-forth: with it the square of $KV$ \yava\ 1 from sixty with one back-and-forth 2
when divided, the quotient from the square of \ya\ half-\kala, and it has the form of thirty \vikalas\
of \yava\ 30. Because of the \ya\ having the form of degrees, this amount of 
$VH$ results. Its square,  of the \yavava\ fifteen like [\textit{prati}] arcminutes occur [??] 15. 
This is subtracted from the square of $KV$, the remainder is the square of $KH$: 
\yava\ 1 \yavava\ like [?] \vikalas [??] square of $KH$ is equal to a fourth part of the
square of $KJ$. From that, the square of $KJ$ \yava\ 4 \yavava\ 1 \vikalas. 

Now, in the second figure of the first chapter of the book Mijast\={\i}, this is demonstrated.
The product of $KJ$, $VD$ has the form of the square of $KJ$; with
the sum of the product of $KB$, $JD$. 

\section*{f.~7r}

and of the product of $VJ$, $KD$ it is equal.  Then the amound of $KD$ is 
6\danda 16\danda 49\danda 7\danda 59\danda 8\danda 56\danda 30.  $VJ$ is equal to 
$AV$ [? $A$?], measured by \ya.  From that, $VJ$ multiplied by $KD$ with really
that much [?] the \ya s result, \ya\ 6\danda 16\danda 49\danda 7 59\danda 8\danda 56\danda
30. 
The product of $KV$, $JD$ with the form of
the square of $KV$, \yava\ 1. From that, this \yava\ 4 \yavava\ 0, \vikalas, 
\textit{a\*m} [?]. Of it, \yava\ \ya\ 6 16\danda 49\danda 7\danda 59\danda 8\danda 56\danda 30
equal results. Again, the two equal-subtracted, then on one side \yava\ 3 \yava  [?] \textit{gha} [?]
\vikalas\ \ya\ 6\danda 16\danda 49 7\danda 59\danda 8\danda 56\danda 30. These two are equal.

Here if two third parts of equals are taken then those two too will become equal.   Here the two sides
are divided by three, \yavava\ [? 0?] \ya\ \textit{ra} [?] 15\danda 32\danda 22\danda 39\danda 42\danda
[?] like \vikalas.
Again, those two reduced by \ya\ as: \ya\ 1 \yava [? 0?] un.\ 2\danda 5\danda 36\danda 22\danda 
39\danda 42\danda 58\danda 50. 

There, [there is] wish [for] the desired digit of sight [?].  The difference of whatever digit and the
digits standing in the amount of units, that digint twenty of the cube  becomes having the form like \vikalas. 
It becomes one desired digit. Its production [is?] conceived [as?] this approach: The cube of the 
units is made, 9\danda 10\danda 28\danda 3\danda 8\danda 52\danda 5\danda 39. 
This is multiplied by twenty-like-\vikalas [?], 0\danda 0\danda 3\danda 3\danda 29\danda 21\danda 2\danda
57. Added to the units: 2\danda 5\danda 39\danda 26\danda 9\danda 4\danda 1\danda 47.
Its cube: 9\danda 11\danda 8 16\danda 32\danda 30\danda 48\danda 9. Again, it is multiplied
by twenty-like-\vikalas, 0\danda 0\danda 3\danda 3\danda 42\danda 45\danda 30\danda 50.
Added to the units: 2\danda 5\danda 39\danda 26\danda 22\danda 28\danda 29\danda 40.
Its cube:

\section*{f.~7v}

9\danda 11\danda 8\danda 19\danda 28\danda 46\danda 23\danda 50. 
Again, with just that pra-vi-20 multiplied [multiply by 20 and shift 3 places to the right, 
equivalent of dividing by $3,0,0$], 0\danda 0\danda 3\danda 3\danda 42\danda 46\danda
29\danda 39. Added to the units: 2\danda 5\danda 39 26\danda 22\danda 29\danda 
28\danda 29\danda. Again, the cube: 9\danda 11\danda 8\danda 19\danda 29\danda 8\danda
56 51. Again, with just that pra-vi-20 multiplied, 0\danda 0\danda 3\danda 3\danda 42\danda
46 29\danda 42. Added to the units: 2\danda 5\danda 39\danda 26\danda 22\danda 29\danda
28 32. This [is?] the Chord of two degrees, the unit [or number? having] the desired digit 
[or digits?] results. [I.e., this is the exact value.] 
When its cube is made, is multiplied with twenty-\pratikalas, is added [misspelled] to the units, then it
becomes the same digit. [I.e., further iteration doesn't change it.]

Now, the second method of Mirjolugvega. Then arc $KV1JD$ [is] considered [?] of six degrees.
Of whatever circle this arc [is part?], the center of that circle is to be considered $T$. Again, 
each one of arc $KV1$, arc $V1J$, arc $JD$ [is] considered two degrees. The Chords
of $TV1$ [should be $KV1$] $KJ$ $KD$ $V1J$ $V1D$ $JD$ are joined. Again, of the Chord of $KV1$,
the Chord of $KJ$, and the Chord of $KD$ in the points $H$, $Jh$, $V2$ half  is to be made.
Again, line $TH$, line $TJh$, line $TV2$ are  joined. Each one [of] those lines 
will become in the form of a perpendicular on those Chords. 
Again, line $KT$ is joined with [the form of?] the radius; again, that is to be made halved
in point $A$. Again, having made point $A$ the center, a circle is to be made with
radius $AK$. Then angle $KHT$, angle $KJhT$, angle $KV2T$ being right angles, 
then  that half circle will go through points $H$, $Jh$, $V2$. 
Again, lines $HJh$, $JhV2$, $V2T$, $HT$ are joined [Note: the latter two were already
drawn]. From 
point $H$ perpendicular $HL$ is extended on the Chords [? why plural?] $KJh$. 

\section*{f.~8r}

Then the Chord of $KJh$ will become halved at points [? why plural] $L$. There with line $JhV2$ 
in the half-place [?] of side $KJ$ and side $KD$ is made [i.e., its endpoints Jh, V2 bisect those 
two segments]. Then line $JhV$ will
become differently-equal [i.e., a corresponding side in similar triangles]
to line $JD$. From that, triangle $KJD$ and triangle
$KJhV2$ pairwise will become together-born [i.e., similar]. 
$KV$ is half of $KD$, [so] $JhV2$ will be half of $JD$.
Again, in just this way line $HJh$ will become half of $V1 J$. $KH$ is half of just that $KV1$.
From that, the Chords $KH$ $HJh$ $JhV2$ will become pairwise equal. 

From point $A$ perpendicular $AL$ onto Chord $KJh$ is to be extended; this perpendicular 
will make line $KJh$ halved at point $L$. Or again, this [line] being halved [??] will fall on point $H$.
From that, the Sine of one degree $KH$ is to be considered the measure of the \ya. 

Again, $KV2$ [is] three degrees, its sine is so much: 3 8\danda 24\danda 33\danda 59\danda
34\danda 28\danda 15. 

\begin{figure}[h]
\includegraphics[width=2.5in] %% 
{JCUFig4}
% \centering
\label{JCUFig4}
\end{figure}

If the square of $KH$ \yava\ is divided by line $KT$ [is this actually 
\textit{katar-rekh\=a} for Ar.\ \textit{qa\*tar}?] 60, then the quotient  will become
one \kala\ of \yava.   This is the $HL$-amount, as previously obtained. Its square
will become of the \yavava\ one

\section*{f.~8v}

\vikala. This is subtracted from the square of $KH$, the remainder will become the
square of $KL$.  And it \yava\ 1  \yavava\ [1 ?] \vikalas. 
[$HL = \frac{x^2}{60} = \frac{KH^2}{KT}$, $HL^2 = \frac{x^4}{3600}$,
$KH^2 - HL^2 = KL^2$.]
A fourth part 4 of the square of $KJh$  is the square of $KL$. From that, 
the square of $KJh$ \yava\ 4 \yavava\ 4 \vikalas. 

Again, the number of the product of $KJh$, $HV2$ [is] the square of $KJh$; like [??]
the number of the product of $KJh$, $HV$. [It?] is equal with the sum of the square [of
\ya?] and of the
product of $HJh$, $KV2$. And the square of $KH$ is that much, \yava\ 1. And the
product of $HJh$, $KV2$ 3\danda 8\danda 24\danda 33\danda
59 34\danda 28\danda 25. 

Here with this approach, when the subtraction is done in the first side:
\yava\ 3 \yavava\ 4 \vikalas\ [=] \ya\ 3\danda 8\danda 24\danda 33\danda 59\danda 34\danda
28\danda 5. And of those two, two one-third parts are equal [?]. 
From that, the two are reduced by three. Those two remainders are equal:
\yava\ 1 [=?] \yavava\ 1 [?] \vikalas [?], un.\ 1\danda 2\danda 48\danda 11\danda 19
51\danda 29\danda 25.
Again, those two are reduced by the \ya, two sides result: 
\ya\ 1 \vikalas [?] un.\  1\danda 2\danda 48\danda 11\danda 19
51\danda 29\danda 25 [=?] \ya-cube $\frac{1}{2}$ \prativikalas.

If here \textit{t\=adi} [??] what degree is obtained?  [???] 
Again, the difference of whatever digit and of the units; one \kala\ of the cube of the digits
should be equal to twenty \prativikalas. [?]

And that by this method is laid out: there the cube of the units beginning with degrees [is]
1\danda 8\danda 48\danda 30\danda 23\danda 36\danda 30 42\danda 18.
This with one \kala

\section*{f.~9r}

and with twenty \prativikalas\ is multiplied: 0\danda 0\danda 1\danda 31\danda 44\danda
40\danda 3\danda 28\danda 41.
Because of the nonexistence of the two places of degrees and \kalas, zero is entered in 
[those] two places. Having added [the result?] to the units, result:
1\danda 2\danda 49\danda 43 4\danda 32\danda 0\danda 52\danda 41. Its cube:
1\danda 8\danda 53\danda 32\danda 4 3\danda 50\danda 59\danda 14\danda 57.
Again, this  is multiplied by that $\frac{1}{20}$: 
0\danda 0\danda 1\danda 31\danda 51\danda 22\danda 45\danda 25\danda 8.
Added to the units:
1\danda 2\danda 49\danda 43\danda 11\danda 14\danda 14\danda 50\danda 8.
Its cube: 1\danda 8\danda 53\danda 36\danda 26\danda 7\danda 18\danda 0\danda
22\danda 10.
Again, this is multiplied by just that $\frac{1}{20}$: 
0\danda 0\danda 1\danda 31\danda 51\danda 23\danda 14 49\danda 20.
Added to the units: 1\danda 2\danda 49\danda 43\danda 11\danda 14 44\danda 16\danda 30.
Its cube: 1\danda 8\danda 53\danda 32\danda 26\danda 8\danda 37 5\danda 34.
Again, this is multiplied by just that $\frac{1}{20}$: 
0\danda 0\danda 1\danda 31\danda 51\danda 23 14\danda 51\danda 29. 
Added to the units: 1\danda 2\danda 49\danda 43\danda 11 14\danda 44\danda 16\danda 26.
This results [as] the desired digit [desired sequence of digits, up to desired digit??]. 
Hence this Sine of one degree is attained.  Wherefore?  Because when its cube 
is multiplied by this $\frac{1}{20} \frac{\hbox{vi}}{\hbox{pra}}$ [and] added to the units,
then [it] becomes just this. From that, this results [as] the Sine of one degree; this is just
[what was] desired. 

Now here, [in] the figure shown by Mirjolugvega with the second method, there 
the contact of a \textit{vadgu}-line [???] is made. [?]  Now here, as [follows]:  
With \emph{small} lines 

\section*{f.~9v}

it is effected, in that way by those known as \textit{yatitamavid} [``the knowers of
\textit{yatitama}??].  There just that much arc $VKVJD$ [sic], [by] just the previous
statement, the Chord is to be connected. $KH$ has the form of half Chord $KV$, 
[it] is the Sine of one degree. It is considered the measure of the \ya, \ya\ 1.
Line $KV$ \ya\ 2. In the same way $VJ$ \ya\ 2. $JD$ \ya\ 2. Since they are 
pairwise [?] equal, again line $KD$
6\danda 16\danda 49\danda 7\danda 59\danda 8\danda 16. 
With that $VJ$ \ya\ 2 [? illegible] is multiplied. Added to the product of
$KV$, $JD$, result \yava\ 4, \ya\ 12\danda 33\danda 38\danda 15\danda 58
16\danda 33\danda 0. This result is the square of $KJ$. 

\begin{figure}[h]
\includegraphics[width=2in] %% 
{JCUFig5}
% \centering
\label{JCUFig5}
\end{figure}

Again, by another method: The square of $KJ$ is to be achieved. The square of $KV$
[is?] \yava\ 4. This is divided by the diameter 2 [which is] reduced one-above from sixty [?].
The quotient is \kalas\ of \yava, 2. This is known as the Versine of two degrees. 
Its square \yavava\ 4 \vikalas\ is subtracted from the square of $KV$. The remainder, 
\yava\ 4, \yavava\ 4 \vikalas.  This results [as] the square of the Sine of two degrees.
This 

\section*{f.~10r}

multiplied by four, \yava\ 16 \yavava\ 16 \vikalas. This is the attained square of $KJ$. 
These two sides [are] equal; for the purpose of equals-subtraction, the statement:
\yava\ 4 \ya\ 12\danda 33\danda 38\danda 15\danda 58\danda 16\danda 33 
[=]
\yava\ 16 \yavava\ 16 \vikalas. When subtracted, the remainder again \textit{dv\=ada} [??]:
\yava\ 12 [=] \yavava 16 \ya\ 12\danda 33\danda 38\danda 15\danda 58\danda 16\danda 33.
Reduced by the \ya: 
\ya\ 1 [=] \yava\ $\frac{1}{2}$ \vikalas, un.\ 1\danda 2\danda 48\danda 11\danda 19\danda
51\danda 29 25. 
This is expounded by Mirjolugvega with the \textit{vadgupray\=a\'sa} [???]. 

\section*{f.~1r bis}

Homage to Lord \Ganesa. Now, with regard to the Sine of one degree, the commentary
\textit{\'Sarahavirja\*md\={\i}} of \textit{Ulugveg\={\i} j\={\i}ka} is written.
There in the determination of the Sine of one degree there is a newer pair of methods. 
One is made by [the one] named Yama\'saida K\=a\'s\={\i}; the second is made by
Miryolugvega. And furthermore of Ulugvega [?], in the procedure of opening [?]
of Yamasaida K\=a\'s\={\i}, that is to be made [?? \textit{h\=apya}?].
There the first method is stated. It [?] is as [follows]:

Arc $ABJD$ is considered six degrees, 6. Again, equal division of it, this of [?]
point $B$ [and] point $J$, is made. There $AB$, $AJ$, $AD$, $VJ$ [$V$ and $B$ used
interchangeably for this point by the scribe], $BD$, and $JD$,
those lines are to be joined [as] Chords. 

Now here by the previously-stated method the Sine of three degrees is known, and it 
is this: 3\danda 8\danda 12 4\danda 33\danda 59\danda 34\danda 28\danda 54\danda 50.
This is multiplied by two: 6\danda 16\danda 49\danda 7\danda 59\danda 8\danda 56
[\elp illeg.] results, the Chord of $AD$. Here knowledge of the Chord $AB$ of two degrees
is desired. There in the second figure of the first chapter of the book Mijist\={\i} this
is explained. Of any quadrilateral that falls in a circle, the
sum of the products of sides standing opposite 
becomes equal to the product of the two diagonals falling in that quadrilateral.
Again just there in the fourth figure of the first chapter, this is explained. The square
of the Chord of half the desired arc [? the half-Chord of the desired
arc?] becomes equal with that number. [?] Whatever is the number, whatever becomes
the Chord of the degrees of half a circle diminished by that arc, whatever is the difference 
of that with the diameter, whatever is the half-diameter multiplied by that, becomes 
equal to that number. [?] 

In this way here arc $AB$ [and] arc $JD$ pairwise become equal. 
And in this way arcs $AJ,$  $BD$ become equal.
From that, the product of $AB$, $JD$ will become a square. Here, $BJ$ is not known;
that is considered the measure of the \ya. This is multiplied by $AD$; and result,
6\danda 16\danda 49\danda 7\danda 59\danda 8\danda 56\danda 29\danda 40.
In this way the sum of the \ya\
and of the amount of the square of the number of the product of $AB$, $JD$
[is] equal to the square of $AJ$, moreover $AJ$ $BD$, is equal with that. [?]

 \begin{figure}[h]
\includegraphics[width=2in] %% 
{JCUFig1bis}
% \centering
\label{JCUFig1bis}
\end{figure}

Here this result [is] equal to the square of $AJ$. From that, in the amount of the 
square of $AJ$ [is] one amount of the square; [?] with these unknowns the result 
is greater. \yava\ 1 \ya\ 6\danda 16\danda 49\danda 7\danda 59\danda 8\danda
56\danda 29\danda 40. 

Now by another method, the square of $AJ$ is determined. There, arc $AJ$ is to be subtracted
from the degrees of half a circle; the difference of the Chord of the remainder 
and of the diameter is to be made. The half-diameter equal to sixty multiplied by that
will become equal to the square of $AV$. Since arc $AJ$ is doubled from arc $AB$. 
From that, if the square of $AB$ \yava\ 1 is divided by the half-diameter equal to sixty 60, 
then a sixtieth part of the \yava\ is obtained; and that of the half-circles [?] diminished by
arc $AJ$, \ya\ Chord [?], whatever is the diameter diminished by that, is the amount of that [?].
Again, this difference is subtracted from the full diameter 120:
\yava\ $-\frac{1}{60}$ [is this dot actually a negative sign?], un.\ 120. This, resulting Chord of the
degrees of half a circle diminished by arc $AJ$. Its square: 
\yavava\ $\frac{1}{3600}$, \yava\ $\frac{240}{60}$, un.\ 14400.

And now, when the square of the  Chord of the desired arc is subtracted from the 
square of the diameter, there the remainder

\section*{f.~1v bis}

becomes the square of the Chord of the degrees of half a circle diminished by that arc. 
Since [?] by the diameter, the Chord of the desired arc,
and the Chord of the degrees of half a circle diminished by the desired arc
a triangle with one right angle is [formed]; since in half a circle from the pr\=anta [??] of 
the diameter the p\=ali-angle [?] of the occurring triangle is a right angle---thus in the third
book of Ukl\={\i}dasa [it] is demonstrated. In that triangle the side opposite the right angle 
is the diameter. And it [has] the form [? number?]  of a hypotenuse. Again, the square of the hypotenuse
is equal to the sum of the squares of the arm and upright---thus in the first book of 
Ukl\={\i}dasa [it] is demonstrated. 
Whence here the square of the diameter 14400 is the number [? form?] of the square of the 
hypotenuse.  In [from?] this, the square of the Chord of the degrees of half a circle 
\yavava\ $\frac{1}{3600}$, \yava\ $\frac{240}{60}$, r\=u 14400
diminished by the 
desired arc is subtracted. The remainder the square of the Chord $AJ$: remainder [??], 
\yavava\ $-\frac{1}{3600}$, \yava\ $\frac{240}{60}$

This by the second method: The square of $AJ$ [is] determined [?].  Of it 
the first-method square of $AJ$ \yava\ 1 \ya\ 
6\danda 16\danda 49\danda 7\danda 59\danda 8\danda 56\danda 29\danda 40.
These two [are] equal: thus for the purpose of subtraction of equals, the setting-out:

\yavava\ $-\frac{1}{3600}$ \yava\ $\frac{240}{60}$ \ya\ 0

\yavava\ 0 \yava\ 1 \ya\ 6\danda 16\danda 49\danda 7\danda 59\danda 8\danda 56\danda 29\danda 40

Now, in western (y\=avan\={\i}ya) algebra (\bijaganita) the balancing (sa\*mprad\=aya) of
equations is however by this method, as follows: If one quantity (r\=a\'si) among the
two equal sides, if negative, then an equal quantity is to be added in the two sides [?]. 
And then the pair of sides becomes just equal. From that, here in the first side 
\yavava\ $-\frac{1}{3600}$ 
this is to be added to the second side, there because of equality of the positive and negative 
in the first side [there is] elimination of the \yavava; remainder \yava\ $\frac{240}{60}$. 
This is divisor-divided: \yava\ 4. This first side again in the second side 
\yavava\ $\frac{1}{3600}$ is added: 
\yavava\ $\frac{1}{3600}$ \yava\ 1 \ya\ 6\danda 16\danda 49\danda 7\danda 59\danda 8\danda 56\danda 29\danda 40.
These two sides again are equal: 

\yava\ 4 \ya\ 0

\yavava\ $\frac{1}{3600}$ \yava\ 1 \ya\ 6\danda 16\danda 49\danda 7\danda 59\danda 8\danda 56\danda 29\danda 40

Now in the western balancing among the two equal sides [there] are [???] two [? du.?] quantities of
the same type. There the lesser quantity is subtracted from the greater quantity; and then the two
become just equal remainders. From that, here the \yava\ quantity, two [?] of one kind, which[ever]
is hence the small quantity, \yava\ 1, from the large quantity, \yava\ 4, [is] subtracted. The remainder
[is] the first side, \yava\ 3; and in the second side 
\yavava\ $\frac{1}{3600}$ \ya\ 6\danda 16\danda 49\danda 7\danda 59\danda 8\danda 56\danda 29\danda 40.
And these two are equal. Again, among the two sides if one side broken,  
if second-side-full [?], then they make [?] the broken quantity filled so-much-multiplier [??] second side. [???]
Here a quantity with a divisor is stated by the word ``broken''. Devoid of divisor is called ``filled''. When
[it] is done thus, having established same-divisor-ness of the two sides, in the reduction [??] of the divisor
[it] is just demonstrated. From that, here in the second side

\section*{f.~2r bis}

the broken quantity is \yavava\ $\frac{1}{3600}$. This is the meaning: Here the \yavava\ 
\textit{r\=a\'se\*sa} [??] 8.  This a part equal to three thousands greater [?] is the \yavava\ quantity.
From that this [?] is made multiplied by the divisor [?], so much the \yavava\ quantity is one. Thus 
the second side too by so much is to be multiplied. When it is done in this way, in the first [side], 
\yava\ 10800; in the second side, \yavava\ 1. Now the first side: two turns, 
sixty above made, \yava\ 3. Now the \ya\ quantity standing in the second side: with 
that divisor 3600 multiplied two turns sixty above made. 
\ya\ 6\danda 16\danda 49\danda 7\danda 59\danda 8\danda 56\danda 29\danda 40. 
The two sides thus produced are equal. 

\yava\ 3 turned twice

\yavava\ 1  \ya\ 6\danda 16\danda 49\danda 7\danda 59\danda 8\danda 56\danda 29\danda 40

Both are turned twice: 

\yava\ 3 turned twice

\yagha\ 1  \ru\ 6\danda 16\danda 49\danda 7\danda 59\danda 8\danda 56\danda 29\danda 40

Now here 

[\elp]

Having again made its cube,

\section*{f.~2v bis}

in the remainder

[\elp]

Here the cube of sixty [?] is subtracted; 

\section*{f.~3r bis}

the remainder [\elp]

\begin{figure}[h]
\includegraphics[width=2in] %% 
{JCUFig2bis}
% \centering
\label{JCUFig2bis}
\end{figure}

[\elp]

Again here,

\section*{f.~3v bis}

a-ba-va

[\elp]

When 

\section*{f.~4r bis}

its cube; 

[\elp]

\begin{figure}[h]
\includegraphics[width=2.25in] %% 
{JCUFig3bis}
% \centering
\label{JCUFig3bis}
\end{figure}

[\elp]

And the square of a-ha [is] as much as \yava\ 1; and the cube of ha-\*r\-a-va

\section*{f.~4v bis}

\ya\

[\elp]

\begin{figure}[h]
\includegraphics[width=2in] %% 
{JCUFig4bis}
% \centering
\label{JCUFig4bis}
\end{figure}

[\elp]

\yava\ 4; this 

\section*{f.~5r bis}

with the diameter 2 

[\elp]

Its square 1

\section*{f.~5v bis}

with four arcminutes

[\elp]

This is to be set in the second row. 

\section*{f.~6r bis}

Again then,

[\elp]

Thus in the beginning too.

\section*{f.~6v bis}



\end{document}

\JOne{f.~1r J1

Homage to Lord \Ganesa. Now, ``When there is knowledge of the Sine of three degrees,
knowledge of the Sine of one degree'': thus he states the Sine with four verses.
Having set the Sine of the arc multiplied by two, multiplied by the kara [?], the 
clever mathematician should set those down separately (the result [is] degrees). 
But when the cube of the degrees of the number [?] of the result is divided
by the square of the half diameter, whatever is the result, having added to the 
remainder [??] of the degrees of that, is to be divided by three. When divided by
three, whatever is the result in minutes, that is to be set down below the degree-number-result
of [?] the previous quotient. The row in degrees and minutes results. 

The cube of the row diminished by the first cube [is] divided by the square of the 
half-diameter. The result, to be added [?] to the remainder, one should divide by three.
Having set the quotient  result below the seconds-number-minutes, it results a row 
in degrees, minutes, seconds.  The cube of the row is to be made. In it, the cube of
the previous is to be subtracted. The remainder is [?] divided by three when divided
by the square, [???] whatever is the result, having increased that by the remainder
of the dividend, again turn by turn [??] in the tip of that but the previous rule is to be 
done. When done thus, [it] should be the result row.  The Sine of a third part of the 
arc  of half the row should be.

Now, thus immediately that Sine of three degrees again from the method I will say.

``Since from the cause by many bhedas the wisdom of the mathematician in 
his own science cleverness'' thus 16. Again, by another method he states the Sine
of three degrees.  ``Diminished by its own third part''. 
Diminished by its own third part of the Sine
of the arc, [it] is to be set down separately.  The cube of it [is] multiplied by three [\elp]
With the quotient-result the separate-standing Sine of the arc is to be added. Again
and again with that made a row is to be determined; the Sine of degrees multiplied
by the arc of half the row should be the Sine of a third part of the arc. 17

Again, he states by another method: ``The Sine of the degrees of arc\elp'':

\bigskip
{f.~1v J1}

A third part of the Sine of the degrees of arc is to be set down separately. Whatever
is the quotient when its cube
is added to its own degrees [?] [and] divided by the square of the Radius, the result
to  that is to be added separately. 

Thus again turn by turn, with that made, the Sine of a third part of the arc should be. 18

Its rationale: it is as [follows]. $KBJD$ is considered an arc of six degrees. 
Again, its equal [?] three degrees is considered of point $B$ [and] point $J$.
Then $KB$, $KJ$, $KD$, $BJ$, $BD$, and $JD$: they are to be drawn as
lines of Chord. 

Now  here by the previously stated method the Sine of three degrees is known, 
and it is this:  $3|8|24|33|59|34|28|14|50$. This is multiplied by two:
$6|16|49|7|59|8|56|29|40$ results the full Chord of $KD$. Here the knowledge
is desired [of?] the Chord $KB$ of two  degrees. 

Something is said for the sake of its rationale. Then in the 
\textit{Siddh\=anta\-samr\=aj} [i.e., \textit{Siddh\=antasa\*mr\=at} translation
of the Almagest?] in the second figure of the first chapter this is
explained.  ``Whatever quadrilateral falls inside a circle, the sum of the 
products of the sides standing face to face becomes equal to the
product of the two diagonals intersecting [?] inside that quadrilateral''.
But its rationale is just explained previously. 

Then again, in the fourth diagram of the first chapter this is demonstrated. 
The square of the Chord of half the desired arc  becomes
equal to the half-diameter 
multiplied by the difference of the 
diameter and the Chord of the degrees of half a circle minus that arc .
Its rationale: On diameter $KJ$ 
half-circle $KBJD$ is to be made.  Then $BD$ is equal to $DJ$. From point $D$
line $DF$ is to be made perpendicular on diameter $KJ$.  Here line $JF$ is measured
by half the difference of 
diameter $KJ$ and of the Chord of the degrees of half a circle diminished by
the known arc.  Why?  [Just what I was wondering myself.] 
$KB$ is to be made equal to $KH$. Line $DH$ is joined.  [There is equality] of triangle
$HKD$ [and?] triangle $DKB$. Side $KH$ [and] side $KD$ to side $BK$
[and] side
}  %% JOne

\JTwo{f.~1r J2

Homage to Lord \Ganesa. Now, with regard to the Sine of one degree, the commentary
\textit{\'Sarahavirja\*md\={\i}} of \textit{Ulugveg\={\i} j\={\i}ka} is written.
There in the determination of the Sine of one degree there is a newer pair of methods. 
One is made by [the one] named Yama\'saida K\=a\'s\={\i}; the second is made by
Miryolugvega. And furthermore of Ulugvega [?], in the procedure of opening [?]
of Yamasaida K\=a\'s\={\i}, that is to be made [?? \textit{h\=apya}?].
There the first method is stated. It [?] is as [follows]:

Arc $ABJD$ is considered six degrees, 6. Again, equal division of it, this of [?]
point $B$ [and] point $J$, is made. There $AB$, $AJ$, $AD$, $VJ$ [$V$ and $B$ used
interchangeably for this point by the scribe], $BD$, and $JD$,
those lines are to be joined [as] Chords. 

Now here by the previously-stated method the Sine of three degrees is known, and it 
is this: 3\danda 8\danda 12 4\danda 33\danda 59\danda 34\danda 28\danda 54\danda 50.
This is multiplied by two: 6\danda 16\danda 49\danda 7\danda 59\danda 8\danda 56
[\elp illeg.] results, the Chord of $AD$. Here knowledge of the Chord $AB$ of two degrees
is desired. There in the second figure of the first chapter of the book Mijist\={\i} this
is explained. Of any quadrilateral that falls in a circle, the
sum of the products of sides standing opposite 
becomes equal to the product of the two diagonals falling in that quadrilateral.
Again just there in the fourth figure of the first chapter, this is explained. The square
of the Chord of half the desired arc [? the half-Chord of the desired
arc?] becomes equal with that number. [?] Whatever is the number, whatever becomes
the Chord of the degrees of half a circle diminished by that arc, whatever is the difference 
of that with the diameter, whatever is the half-diameter multiplied by that, becomes 
equal to that number. [?] 

In this way here arc $AB$ [and] arc $JD$ pairwise become equal. 
And in this way arcs $AJ,$  $BD$ become equal.
From that, the product of $AB$, $JD$ will become a square. Here, $BJ$ is not known;
that is considered the measure of the \ya. This is multiplied by $AD$; and result,
6\danda 16\danda 49\danda 7\danda 59\danda 8\danda 56\danda 29\danda 40.
In this way the sum of the \ya\
and of the amount of the square of the number of the product of $AB$, $JD$
[is] equal to the square of $AJ$, moreover $AJ$ $BD$, is equal with that. [?]

%  \begin{figure}[h]
% \includegraphics[width=2in] %% 
% {JCUFig1bis}
% \centering
% \label{JCUFig1bis}
% \end{figure}

Here this result [is] equal to the square of $AJ$. From that, in the amount of the 
square of $AJ$ [is] one amount of the square; [?] with these unknowns the result 
is greater. \yava\ 1 \ya\ 6\danda 16\danda 49\danda 7\danda 59\danda 8\danda
56\danda 29\danda 40. 

Now by another method, the square of $AJ$ is determined. There, arc $AJ$ is to be subtracted
from the degrees of half a circle; the difference of the Chord of the remainder 
and of the diameter is to be made. The half-diameter equal to sixty multiplied by that
will become equal to the square of $AV$. Since arc $AJ$ is doubled from arc $AB$. 
From that, if the square of $AB$ \yava\ 1 is divided by the half-diameter equal to sixty 60, 
then a sixtieth part of the \yava\ is obtained; and that of the half-circles [?] diminished by
arc $AJ$, \ya\ Chord [?], whatever is the diameter diminished by that, is the amount of that [?].
Again, this difference is subtracted from the full diameter 120:
\yava\ $-\frac{1}{60}$ [is this dot actually a negative sign?], un.\ 120. This, resulting Chord of the
degrees of half a circle diminished by arc $AJ$. Its square: 
\yavava\ $\frac{1}{3600}$, \yava\ $\frac{240}{60}$, un.\ 14400.

And now, when the square of the  Chord of the desired arc is subtracted from the 
square of the diameter, there the remainder

\medskip
{f.~1v J2}

becomes the square of the Chord of the degrees of half a circle diminished by that arc. 


} %% J2

\end{parcolumns}
\end{document}
